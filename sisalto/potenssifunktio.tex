\chapter{Potenssifunktio}

\laatikko{Potenssifunktio: 
$ f(x) = a \cdot x^n $, jossa $a \neq 0$ }

Eksponentti $n$ on potenssin $x^n$ aste. Se voi olla mikä tahansa reaaliluku. Rajoitumme käsittelemään tapauksia, jossa $n = 1, 2, 3\ldots $

\begin{esimerkki}
Neliön sivun pituus särmän pituus on $x$ senttimetriä. Tällöin neliön pinta-ala on $A(x)=x^2$ senttimetriä. Jos neliön sijaan tarkasteltava esine olisi kuutio, jossa $x$ kuvaa kuution särmän pituutta, ilmaisisi funktio $V(x)=x^3$ kuution tilavuutta.
\end{esimerkki}

Potenssin asteella on suuri merkitys funktion kuvaajan muotoon. Funktiot, joiden aste on parillinen, muodostavat U:n muotoisia kuvaajia ja saavat ($a$:n merkistä riippuen) vain positiivisia tai negatiivisia arvoja. Parittomat potenssifunktiot muodostavat sen sijaan "kaksoismutkan", ja saavat sekä positiivisia että negatiivisia arvoja.

\missingfigure{potenssifunktioiden kuvaajien kuvat - yksi parillisilla potensseilla ja toinen parittomilla (kerroin a positiivinen)}

Mikäli eksponentti $n$ on negatiivinen, funktion muoto muuttuu merkittävästi. Funktio ei ole enää määritelty kohdassa $x=0$, jossa funktion arvot näyttävät "räjähtävän".

Tällaiset funktiot ratkaistamaan samalla lailla kuin positiivieksponenttiset tapauksetkin. Huomaa kuitenkin, että $frac{1}{x^n} \neq 0 $ kaikilla $x$:n arvoilla!  

\missingfigure{negatiivipotenssifunktioiden kuvaajien kuvat, parilliset ja parittomat tapaukset}


%Tästä alaspäin on potenssiyhtälöitä, jotka varmaan menee jo päälle aiemmin käydyn asian kanssa. Sekaannus meikäläisen osalta... -Matti

Yhtälöt, jotka ovat muotoa $a\cdot x^n = b$ (joskus myös muotoa  ($a\cdot x^n - b = 0$), ratkaistavuus riippuu useammasta seikasta. Mikäli $n$ on pariton, yhtälö on aina ratkaistavissa (eli sillä on reaalinen juuri):
\begin{align*}
a\cdot x^n &= b \\
x^n &= \frac{b}{a} \\
x &= \sqrt[n]{\frac{b}{a}}
\end{align*}

\begin{esimerkki}
$2x^3 + 16 = 0 \Leftrightarrow 2x^3 = -16 \Leftrightarrow x^3 = -8  \Leftrightarrow x = sqrt[3]{-8} = -2 $
\end{esimerkki}

Mikäli $n$ on parillinen, yhtälö on ratkaistavissa jos ja vain jos $\frac{b}{a} \geq 0 $. Parillisella potenssifunktiolla voi olla yksi, kaksi tai ei yhtään ratkaisua.

\begin{esimerkki}
$x^2 = 0 \Leftrightarrow x = 0 \\
\Rightarrow$ Yhtälöllä on yksi ratkaisu.
\end{esimerkki}

\begin{esimerkki}
$x^2 - 9 = 0 \Leftrightarrow x^2 = 9 \Leftrightarrow x = \pm 3 \\
\Rightarrow$ Yhtälöllä on kaksi ratkaisua, sillä $3^2 = 9$ ja $(-3)^2 = 9$.
\end{esimerkki}

\begin{esimerkki}
$x^2 + 9 = 0 \Leftrightarrow x^2 = 9 \Leftrightarrow x = \sqrt{-9} \\
\Rightarrow$ Yhtälöllä ei ole reaalista ratkaisua.
\end{esimerkki}

\begin{tehtava}
Ratkaise: \\
a) $ x^2 = 4 $ \qquad
b) $ x^3 = 27 $ \qquad
c) $ x^5 = -1 $ \qquad
d) $ x^2 - 3 = 0 $ \qquad
e) $ x^3 + 125 = 0 $
\begin{vastaus}
a) $ x = 2 $ \qquad
b) $ x = 3 $ \qquad
c) $ x = -1 $ \qquad
d) $ x = \sqrt{3} $ \qquad
e) $ x = -5 $ 
\end{vastaus}
\end{tehtava}

\begin{tehtava}
Ratkaise: \\
a) $ 5x^2 = 25 $ \qquad
b) $ (2x)^3 = 8 $ \qquad
c) $ x^4 = \frac{1}{4} $ \qquad
d) $ (3x)^2 = 36 $ \qquad
3) $ (4x)^2 - 16 = 0 $ 
\begin{vastaus}
a) $ x = \sqrt{5} $ \qquad
b) $ x = 1 $ \qquad
c) $ x = -1 $ \qquad
d) $ x = 2 $ \qquad
e) Ei ratkaisua. 
\end{vastaus}
\end{tehtava}