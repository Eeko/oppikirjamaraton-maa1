\chapter{Funktio}
Matematiikassa tutkitaan paljon suureiden välisiä riippuvuuksia. Tällaiset riippuvuudet voidaan muotoilla funktioiden avulla. Esimerkiksi tuotteen arvonlisäveroprosentti riippuu tuotteen tyypistä. Tämä riippuvuus voidaan kirjoittaa funktiona eri tuotetyyppien joukolta $A$ reaalilukujen joukolle $\mathbb{R}$, missä funktio liittää jokaiseen tuotteeseen sen arvonlisäveroprosentin.

[Esimerkki, kuva arvonlisäverofunktiosta, missä \[A = \{\text{ahvenfilee}, \text{AIV-rehu}, \text{auto}, \text{runokirja}, \text{ravintola-ateria}, \text{särkylääke}, \text{televisio}\},\]$f(\text{ahvenfilee}) = 13$, $f(\text{AIV-rehu}) = 13$, $f(\text{auto}) = 23$, $f(\text{runokirja}) = 9$, $f(\text{ravintola-ateria}) = 13$, $f(\text{särkylääke}) = 9$, $f(\text{televisio}) = 23$]

\laatikko{Funktio $f$ joukosta $A$ joukkoon $B$ on sääntö, joka liittää $A$:n jokaiseen alkioon täsmälleen yhden $B$:n alkion. $A$ on tällöin $f$:n määrittelyjoukko ja $B$ sen maalijoukko. Funktion arvojoukko on kaikkien sen saamien arvojen joukko.}

Funktioita kutsutaan myös kuvauksiksi. Monesti funktion määrittely- ja maalijoukot jätetään merkitsemättä. Tällöin 
