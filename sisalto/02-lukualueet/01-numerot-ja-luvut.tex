\chapter{Numerot ja luvut}

(Joonas jatkaa tästä vielä!)

Matematiikka tarjoaa työkaluja asioiden jäsentämiseen, päättelyyn ja mallintamiseen. Alasta riippuen käsittelemme matematiikassa erilaisia \textbf{objekteja}: Geometriassa tarkastelemme tasokuvioita ja kolmiulotteisia rakenteita. Algebrassa tutkii lukujen ja funktioiden ominaisuuksia. Todennäköisyyslaskenta arvioi erilaisten tapausten ja tilanteiden mahdollisuuksia ja riskejä. Matemaattinen analyysi (kurssit 7,8 ja 10) tutkii funktioita ja niiden muuttumista.

Jokaiseen tarkastelukohteeseen liitetään myös niille ominaisia \textbf{operaatioita}. Tämä kurssi käsittelee lähinnä lukuja ja niiden operaatioita, joita \textbf{laskutoimituksiksi} kutsutaan. Aloitetaan yksinkertaisista määritelmistä: mitä tarkoittavat \textbf{numero} ja \textbf{luku}?

\laatikko{Länsimaisessa traditiossa käytössämme on kymmenen numeromerkkiä: 0, 1, 2, 3, 4, 5, 6, 7, 8 ja 9. Näitä kutsutaan hindu-arabialaisiksi numeroiksi.  }

Sanalla numero voidaan siis viitata yksittäiseen kirjoitettuun merkkiin. 

\laatikko{Luvut koostuvat numeroista.}
\begin{esimerkki}
Luku \[715531\] koostuu numeroista 7, 1, 5, 5, 3 ja 1.
\end{esimerkki}



Olennaista on myös...
lukujärjestelmä, paikkajärjestelmä

MIKSI KÄYTÄMME KIRJAIMIA?

suuruus, yhtäsuuruus, eri suuret

, ja Erilaisilla luvuilla voidaan suorittaa erilaisia laskutoimituksia. Seuraavissa luvuissa esitellään ja käydään läpi lukiomatematiikassa ja mahdollisissa jatko-opinnoissa käytettäviä lukujoukkoja ja tavallisimmat laskutoimitukset.

