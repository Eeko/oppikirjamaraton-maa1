\chapter{Potenssien laskusääntöjen todistuksia}
\label{pot_todistukset}
Tähän tulee potenssien laskusääntöjen todistuksia, kunhan joku laatii ne.

\section*{Murtolausekkeiden sieventäminen}

\laatikko{
Jos murtoluvun osoittajassa tai nimittäjässä on summa, jonka osilla on yhteinen tekijä, sen voi ottaa \emph{yhteiseksi tekijäksi} sulkujen eteen. Jos osoittajassa ja nimittäjässä on sen jälkeen sama kerroin, sen voi jakaa pois molemmista eli \emph{supistaa} pois.
\begin{equation}
\frac{ac+bc}{c} = \frac{ \cancel{c} (a+b)}{\cancel{c}} = a+b
\end{equation}


Joskus murtolauseke sieventyy, jos sen esittääkin kahden murtoluvun summana.
\begin{equation}
\frac{ca+b}{c} = \frac{ca}{c} + \frac{b}{c} = a + \frac{b}{c}
\end{equation}
}

Kun jakaa kolme erikokoista nallekarkkipussia ($a$, $b$ ja $c$) tasan kolmen ihmisen kesken, on sama, laittaako kaikki ensin samaan kulhoon ja jakaa ne sitten ($\frac{a+b+c}{3}$) vai jakaako jokaisen pussin erikseen ($ \frac{a}{3} + \frac{b}{3} + \frac{c}{3}$).

Jos taas samat kolme henkilöä jakavat keskenään pussin tikkareita ($6$ kpl) ja yhden pussin nallekarkkeja ($n$ kpl), niin saadaan seuraavanlainen lasku: $ \frac{6\text{ tikkaria}+n\text{ nallekarkkia}}{3} = \frac{6\text{ tikkaria}}{3} + \frac{n\text{ nallekarkkia}}{3} = \frac{\cancel{3} \cdot 2\text{ tikkaria}}{\cancel{3}} + \frac{n\text{ nallekarkkia}}{3} = 2\text{ tikkaria} + \frac{n\text{ nallekarkkia}}{3}$. Toisin sanoen, kukin saa kaksi tikkaria ja kuinka paljon ikinä onkaan kolmasosa kaikista nallekarkeista.

\laatikko{
Samantyyppiset asiat voidaan laskea yhteen tai \emph{ryhmitellä}.
\begin{equation}
ax^2 + bx + cx^2 + dy + ex = (a+c)x^2 + (b+e)x + dy
\end{equation}
}

\begin{esimerkki}

$ \frac{1}{6} + \frac{3}{2} = \frac{1}{2\cdot 3} + \frac{3}{2} = \frac{1}{2 \cdot 3} + \frac{3 \cdot 3}{2 \cdot 3} = \frac{1}{6} + \frac{9}{6} = \frac{10}{6} = \frac{\cancel{2} \cdot 5}{\cancel{2} \cdot 3} = \frac{5}{3}$

\end{esimerkki}