


%\part{Alku}
\chapter{Esipuhe}

%%%%%%%%%%%%%%%%%%%%%%%%%%%%%%%%%%%%%%%%%%%%%%%%%%%%%%%%%%%%%%%%%%%%%%%%%%%%%%%%
%%%%  /usr/share/doc/texlive-fonts-extra-doc/fonts/arev/mathtesty.tex

% mathtesty.tex, by Stephen Hartke 20050522
% based on mathtestx.tex in the mathptmx package
% and symbols.tex by David Carlisle

Lorem ipsum\ldots

\laatikko{Tässä on ältsin hieno teoriaboksi. Tänne voi laittaa myös kaavoja
\begin{equation}
(a+b)^2=a^2+2ab+b^2
\end{equation}
ja toimii kuin junan vessa.
}

\begin{tehtava}
Esimerkkikysymys.
\begin{vastaus}	
Esimerkkivastaus.
\end{vastaus}
\end{tehtava}

\begin{tehtava}
Laske: $1+2$
\begin{vastaus}
$3$
\end{vastaus}
\end{tehtava}

\begin{tehtava}
Ratkaise:
\[
x=2x+1
\]


\begin{vastaus}
$x=-1$
\end{vastaus}
\end{tehtava}


\begin{theorem}[Residue Theorem]
Let $f$ be analytic in the region $G$ except for the isolated singularities $a_1,a_2,\ldots,a_m$. If $\gamma$ is a closed rectifiable curve in $G$ which does not pass through any of the points $a_k$ and if $\gamma\approx 0$ in $G$ then
\[
\frac{1}{2\pi i}\int_\gamma f = \sum_{k=1}^m n(\gamma;a_k) \text{Res}(f;a_k).
\]
\end{theorem}

\begin{esimerkki}[Leivän paino]
Leipä painaa kilon ja puolet leivästä. Painavako oli leipä?\\
{\bf Ratkaisu.} Merkitään leivän painoa $x$:llä. Puolet leivästä on matemaattisesti ilmaistuna $\frac{x}{2}$ ja kun siihen lisätään kilogramma, saadaan leivän paino, joten saamme yhtälön
\begin{equation}
\frac{x}{2}+1=x
\end{equation}
josta ratkeaa
\begin{equation}
x=2.
\end{equation}
Leipä painaa siis 2 kilogrammaa.
\end{esimerkki}

Another nice theorem from complex analysis is

\begin{theorem}[Maximum Modulus]
Let $G$ be a bounded open set in $\mathbb{C}$ and suppose that $f$ is a continuous function on $G^-$ which is analytic in $G$. Then
\[
\max\{|f(z)|:z\in G^-\}=\max \{|f(z)|:z\in \partial G \}.
\]
\end{theorem}

\newcommand{\abc}{abcdefgh\hbar\hslash i\imath j\jmath klmnopqrstuvwxyz}
\newcommand{\ABC}{ABCDEFGHIJKLMNOPQRSTUVWXYZ}
\newcommand{\alphabeta}{\alpha\beta\varbeta\gamma\delta\epsilon\varepsilon\zeta\eta\theta\vartheta\iota\kappa\varkappa\lambda\mu\nu\xi o\pi\varpi\rho\varrho\sigma\varsigma\tau\upsilon\phi\varphi\chi\psi\omega}
\newcommand{\AlphaBeta}{\Gamma\Delta\Theta\Lambda\Xi\Pi\Sigma\Upsilon\Phi\Psi\Omega}



%%%%%%%%%%%%%%%%%%%%%%%%%%%%%%%%%%%%%%%%%%%%%%%%%%%%%%%%%%%%%%%%%%%%%%%%%%%%%%%%
%%%% /usr/share/doc/texlive-doc-en/fonts/free-math-font-survey/source/textfragment.tex







%%% Local Variables: 
%%% mode: latex
%%% End: 
