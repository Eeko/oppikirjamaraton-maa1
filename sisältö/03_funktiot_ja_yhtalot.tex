%
\part{Yhtälöt}
\chapter{Yhtälöiden teoriaa}
Monissa käytännön tilanteissa saamme samalle asialle kaksi erilaista esitystapaa.

\begin{esimerkki}
Meillä on orsivaaka, joka on tasapainossa. (kuva!) Toisessa vaakakupissa on kahden kilon siika ja toisessa puolen kilon ahven sekä tuntematon määrä lakritsia. Kuinka paljon vaakakupissa on lakritsia? (Ratkaistaan...) (Muita esimerkkejä, vähitellen vaikeutuvia (1. asteen) yhtälöitä)
\end{esimerkki}

Määritelmä: Yhtälöksi kutsutaan kahden lausekkeen merkittyä yhtäsuuruutta. Siis mielivaltaisille lausekkeille $A$ ja $B$ merkitään $A=B$. (Esim. $A=3x+5$ ja $B=7x+7$). Jos yhtälön puolien lausekkeiden arvot ovat samat, sanotaan että yhtälö pätee. 

Yhtälössä voi esiintyä myös muuttujia, eli symboleja joiden arvoa ei ole etukäteen määritelty. Muuttujia merkitään usein kirjaimilla $x$, $y$ ja $z$. Niitä muuttujien arvoja, joilla yhtälö pätee, kutsutaan yhtälön ratkaisuiksi. Yhtälön ratkaisemisella tarkoitetaan kaikkien yhtälön ratkaisujen selvittämistä.

Eräs tapa ratkaista yhtälöitä on muokata niitä niin, että muokattu yhtälö pätee täsmälleen silloin kun alkuperäinen yhtälö pätee. Tällaisia sallittuja muunnoksia ovat esimerkiksi:
\begin{itemize}
\item Yhtälön molemmat puolet voidaan kertoa nollasta poikkeavalla luvulla $m$. Muutos tehdään aina molemmille puolille. Tällöin saadaan yhtälö $mA = mB$.
\item Yhtälön molemmille puolille voidaan lisätä tai molemmilta puolilta vähentää luku $n$.
Tällöin saadaan yhtälö $A+n = B+n$.
\end{itemize}
[pitäskö perustella?] JOOOO T. ANNIKA

Monet yhtälöt ratkeavat toistamalla tällaisia muunnoksia kunnes yhtälö on niin yksinkertaisessa muodossa, että ratkaisu on helppo nähdä. Koska jokaisessa muokkausjonon yhtälössä ratkaisut ovat samat, näin saadaan alkuperäisen yhtälön ratkaisut.

[joku esimerkki tähän?]

Yhtälöt voidaan ratkaisujensa perusteella jakaa kolmeen tyyppiin:
\begin{enumerate}
\item Yhtälö, joka on aina tosi. Esimerkiksi yhtälöt $8=8$ ja $x=x$.
\item Yhtälö, joka on joskus tosi. Esimerkiksi yhtälö $x+2=3$ on tosi jos ja vain jos $x=1$.
\item Yhtälö, joka ei ole koskaan tosi. Esimerkiksi yhtälö $0=1$.
\end{enumerate}
Tämän kurssin ja ylipäätään matematiikan kannalta selvästi tärkein yhtälötyyppi on 2. Siirrymme nyt tarkastelemaan tärkeää erikoistapausta yhtälöistä, ensimmäisen asteen yhtälöitä.

\chapter{Ensimmäisen asteen yhtälö}
Ensimmäisen asteen yhtälö on yhtälö, joka on esitettävissä muodossa $ax+b=0$, jossa $a \neq 0$.

\begin{theorem}
Kaikki muotoa $ax+b=cx+d$ olevat yhtälöt, joissa $a \neq c$, ovat ensimmäisen asteen yhtälöitä.
\end{theorem}

\begin{proof}
\begin{align*}
ax+b &= cx+d & &| \, \textbf{Vähennetään molemmilta puolilta $cx+d$.} \\
ax+b - (cx+d) &= 0 & &| \, 
\end{align*}
\end{proof}

\begin{theorem}
Yleinen lähemistymistapa muotoa $ax+b = cx+d$ olevien yhtälöiden ratkaisuun on: \\
(1) Vähennä molemmilta puolilta $cx$. Saat yhtälön $(a-c)x + b = d$. \\
(2) Vähennä molemmilta puolita $b$. Saat yhtälön $(a-c)x = d-b$. \\
(3) Jaa $(A-C)$:llä. Saat yhtälön ratkaistuun muotoon $x = \frac{d-b}{a-c}$.
\end{theorem}

Esimerkki. Yhtälön $7x+4=4x+7$ ratkaisu saadaan seuraavasti:
\begin{align*}
7x+4 &= 4x+7 & &| \, \text{Vähennetään molemmilta puolilta 4x.} \\
3x+4 &= 7 & &| \, \text{Vähennetään molemmilta puolilta 4.} \\
3x &= 3 & &| \, \text{Jaetaan molemmat puolet kolmella eli kerrotaan $\frac{1}{3}$:lla.} \\
x &= 1 & &| \, \text{Saimme yhtälön ratkaistuun muotoon. $x=1$ on siis yhtälön ratkaisu.} \\
\end{align*}

\begin{tehtava}
%
Ratkaise:
\begin{enumerate}
\item $x + 4 = 5$
\item $1 - x = -3$
\item $7x = 35$
\item $-2x = 4$
\item $10 - 2x = x$
\item $9x + 4 = 6 - x$
\item $\frac{2x}{5} = 4$
\item $\frac{x}{3} + 1 = \frac{5}{6} - x$
\end{enumerate}
\begin{vastaus}
\begin{enumerate}
\item $x=1$
\item $x=4$
\item $x=35/7$
\item $x=-2$
\item $x=10/3$
\item $x=1/5$
\item $x=10$
\item $x=-1/8$
\end{enumerate}
\end{vastaus}
\end{tehtava}

\begin{tehtava}
Ratkaise kysytty muuttuja yhtälöstä
\begin{enumerate}
\item $F=ma$, $m=?$ "voima on massa kertaa kiihtyvyys"
\item $p=\frac{F}{A}$, $F=?$ "paine on voima jaettuna alalla"
\item $A=\pi r^2$, $r=?$ "(pallon) pinta-ala on pii kertaa säde toiseen"
\item $V=\frac{1}{3} \pi r^2 h$, $h=?$ "(kartion) tilavuus on kolmasosa pii kertaa säde toiseen kertaa korkeus"
\end{enumerate}
\begin{vastaus}
\begin{enumerate}
\item $m=\frac{F}{a}$
\item $F=\frac{p}{a}$
\item $r=\sqrt{\frac{A}{\pi}}$
\item $h=\frac{V}{ \frac{1}{3} \pi r^2 h}$
\end{enumerate}
\end{vastaus}
\end{tehtava}

\chapter{Yhtälöpari}


\chapter{Yleinen potenssi ja potenssiyhtälö}
%pitää esitellä mitä on mega, milli, sentti jne.

\begin{tehtava}
Esitä luku ilman kymmenpotenssia.
\begin{enumerate}
\item $3,2 * 10^4$
\item $-7,03 * 10^{-5}$
\item $10,005 * 10^{-2}$
\end{enumerate}
\begin{vastaus}
Vastaus
\begin{enumerate}
\item $32000$
\item $-0,0000703$
\item $0,10005$
\end{enumerate}
\end{vastaus}
\end{tehtava}

\begin{tehtava}
Esitä luku ilman etuliitettä.
\begin{enumerate}
\item $0,5 dl$
\item $233 mm$
\item $33 cm$
\item $16 kg$
\item $2 MJ$
%\item %megatavu, mibitavu jne.
%\item 
\end{enumerate}
\begin{vastaus}
Vastaus:
\begin{enumerate}
\item $0,05 l$
\item $0,233 m$
\item $0,33 m$
\item $16 000 g$
\item $2 000 000 J$
%\item $ $
%\item $ $
\end{enumerate}
\end{vastaus}
\end{tehtava}


\chapter{Kertaustiivistelmä}

\part{Funktiot}
\chapter{Funktio}
Matematiikassa tutkitaan paljon suureiden välisiä riippuvuuksia. Tällaiset riippuvuudet voidaan muotoilla funktioiden avulla. Esimerkiksi tuotteen arvonlisäveroprosentti riippuu tuotteen tyypistä. Tämä riippuvuus voidaan kirjoittaa funktiona eri tuotetyyppien joukolta $A$ reaalilukujen joukolle $\mathbb{R}$, missä funktio liittää jokaiseen tuotteeseen sen arvonlisäveroprosentin.

[Esimerkki, kuva arvonlisäverofunktiosta, missä \[A = \{\text{ahvenfilee}, \text{AIV-rehu}, \text{auto}, \text{runokirja}, \text{ravintola-ateria}, \text{särkylääke}, \text{televisio}\},\]$f(\text{ahvenfilee}) = 13$, $f(\text{AIV-rehu}) = 13$, $f(\text{auto}) = 23$, $f(\text{runokirja}) = 9$, $f(\text{ravintola-ateria}) = 13$, $f(\text{särkylääke}) = 9$, $f(\text{televisio}) = 23$]
%\input{kuvat/funktio.tex}


\laatikko{Funktio $f$ joukosta $A$ joukkoon $B$ on sääntö, joka liittää $A$:n jokaiseen alkioon täsmälleen yhden $B$:n alkion. $A$ on tällöin $f$:n määrittelyjoukko ja $B$ sen maalijoukko. Funktion arvojoukko on kaikkien sen saamien arvojen joukko.}

Funktioita kutsutaan myös kuvauksiksi. Monesti funktion määrittely- ja maalijoukot jätetään merkitsemättä. Tällöin 

\chapter{Erilaisia funktioita}
