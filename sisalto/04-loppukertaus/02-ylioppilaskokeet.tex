\chapter{Ylioppilaskoetehtäviä}

Ylioppilaskokeissa on yleensä tehtäviä tai tehtävien alakohtia, joiden ratkaiseminen on mahdollista ensimmäisen kurssin tiedoin.

\begin{description}

    \item[(K2012/1b)]  Ratkaise yhtälö
                        \[\frac{x}{6} - \frac{x-3}{2} - \frac{7}{9} = 0 \]
    \item[(K2012/2a)]  Laske lausekkeen $ \frac{15}{4} - \left( \frac{6}{3} \right)^2 $ arvo.
    \item[(S2011/1b)]  Ratkaise yhtälö
                        \[ \frac{4x - 1}{5} = \frac{x + 1}{2} + \frac{3 - x}{4} \]
%    \item[(S2011/2a)]  Sievennä välivaiheet esittäen lauseke
%                       \[ \frac{1}{\sqrt{2} + \frac{1}{2 + \sqrt{2}}} \]
    \item[(K2011/1a)]  Ratkaise yhtälö
                        \[ \frac{2}{x} = \frac{3}{x - 2} \]
    \item[(K2011/2a)]  Osakkeen arvo oli 35,50\,euroa. Se nousi ensin 12\,\%,
                        mutta laski seuraavana päivänä 10\%. Kuinka monta prosenttia
                        arvo nousi yhteensä näiden muutosten jälkeen?
    \item[(S2010/1a)]  Sievennä lauseke $ (a + b)^2 - (a - b)^2 $
    \item[(K2010/1b)]  Sievennä lauseke $ (\sqrt{a} + 1)^2 - a - 1 $
    \item[(S2009/1c)]  Osoita, että $ \sqrt{27 - 10 \sqrt{ 2} } = 5 - \sqrt{2} $
    \item[(S2009/2b)]  Ratkaise yhtälö $ \sqrt{x + 2 } = 3  $.
    \item[(K2009/1a)]  Sievennä $ \frac{a^2}{3} - \left( \frac{-a}{3} \right)^2 $
    \item[(S2008/1b)]  Sievennä lauseke
                        \[ \frac{1}{x} - \frac{1}{x^2} + \frac{1 + x}{x^2} \]
    \item[(S2009/2b)]  Ratkaise yhtälö
                        \[ \frac{x}{6} - \frac{x - 2}{3} = \frac{5}{12} \]
    \item[(K2008/4)]   Vuonna 2007 alennettiin parturimaksujen arvonlisäveroa 22
                        prosentista 8 prosenttiin. Jos alennus olisi siirtynyt
                        täysimääräisenä parturimaksuihin, kuinka monta prosenttia
                        ne olisivat alentuneet? Arvonlisävero ilmoitetaan prosentteina
                        verottomasta hinnasta ja se on osa tuotteen tai palvelun hintaa.
    \item[(S2007/1c)]  Ratkaise $L$ yhtälöstä
                        \[ t = \frac{1}{2\pi\sqrt{LC}} \]
    \item[(S2007/4)]   Tuotteen hintaa korotettiin $p$ prosenttia, jolloin menekki väheni.
                        Tämän johdosta hinta päätettiin alentaa takaisin alkuperäiseksi.
                        Kuinka monta prosenttia korotetusta hinnasta alennus oli?
    \item[(K2007/1c)]  Sievennä lauseke $ \sqrt[3]{a \sqrt{a}} \quad (a > 0) $.
    \item[(K2007/3a)]  Merivettä, jossa on 4,0 painoprosenttia suolaa, haihdutetaan
                        altaassa, kunnes sen massa on vähentynyt 28\,\%. Mikä on
                        suolapitoisuus haihduttamisen jälkeen? Anna vastaus prosentin
                        kymmenesosan tarkkuudella. 
    \item[(K2007/3b)]  Mikä on vuotuinen korkoprosentti, jos tilille talletettu rahamäärä
                        kasvaa korkoa korolle 1,5--kertaiseksi 10 vuodessa. Lähdeveroa
                        ei otetan huomioon. Anna vastaus prosentin sadasosan 
                        tarkkuudella.
    \item[(S2006/5)]   Hopean ja kuparin seoksesta tehty esine painaa 150\,g, ja sen
                        tiheys on 10,1\,kg/dm\(^3\). Kuinka monta painoprosenttia
                        esineessä on hopeaa ja kuinka monta kuparia, kun hopean tiheys on 
                        10,5\,kg/dm\(^3\) ja kuparin 9,0\,kg/dm\(^3\)?
    \item[(K2006/1a)]  Ratkaise $x$ yhtälöstä $4x + 2 =  3 - 2(x + 4)$.
    \item[(K2006/1c)]  Sievennä lauseke 
                        \[ \frac{1}{a - 1} \left( a - \frac{1}{a} \right) \]
    \item[(K2006/4)]   Kesämökin rakentaminen tuli 25\,\% arvioitua kalliimmaksi.
                        Rakennustarvikkeet olivat 19\,\% ja muut kustannukset 28\,\%
                        arvioitua kalliimpia. Mikä oli rakennustarvikkeiden arvioitu osuus ja 
                        mikä lopullinenosuus kokonaiskustannuksista?
    \item[(S2005/1a)]  Ratkaise reaalilukualueella yhtälö 
                        \[ 2(x - 1) + 3(x + 1 ) = -x \]
    \item[(S2005/1c)]  Ratkaise reaalilukualueella yhtälö $ x^{16} = 256 $.
    \item[(K2005/1a)]  Sievennä lauseke
                        \[ \frac{x}{1 - x} + \frac{x}{1 + x} \]
    \item[(K2005/2a)]  Ratkaise yhtälöryhmä
                        \[
                         \left\{
                         \begin{aligned}
                              x + y &= a \\
                              x - y &= 2a
                         \end{aligned}
                         \right.
                        \]
    \item[(K2005/3)]   Asuinrakennuksesta saadut vuokrat ovat 12\,\% pienemmät kuin
                        ylläpitokustannukset. Kuinka monta prosenttia vuokria olisi
                        korotettava, jotta ne tulisivat 10\,\% suuremmiksi kuin 
                        ylläpitokustannukset, jotka samanaikaisesti kohoavat 4\,\%?
    \item[(K2004/3)]   Perheen vuokramenot olivat 25\,\% tuloista. Vuokramenot nousivat
                        15\,\%. Montako prosenttia vähemmän rahaa riitti muuhun
                        käyttöön korotuksen jälkeen?
    \item[(S2003/5)]   Päärynämehusta ja omenamehusta tehdyn sekamehun sokeripitoisuus
                        on 11\,\%. Määritä mehujen sekoitussuhde, kun päärynämehun
                        sokeripitoisuus on 14\,\% ja omenamehun 7\,\%.
    \item[(S2003/12)]  Isä tallettaa poikansa tilille joka kuukauden alussa 200\,\euro \;
                        vuodenvaihteessa tapahtuneesta syntymästä alkaen. Tilille
                        maksetaan 1,5\,\% vuotuista korkoa, joka liitetään pääomaan aina 
                        vuoden lopussa.
                       
                        \begin{enumerate}[(a)]
                           \item Kuinka paljon rahaa tilillä on, kun poika täyttää
                                18 vuotta? 
                           \item Kuinka kauan isän olisi talletettava, jotta tilillä
                                olisi rahaa kaksiota varten, kun kaksion hinnaksi
                                oletetaan 135 000\,\euro ?
                        \end{enumerate}
                       
    \item[(K2003/1)]   Sievennä lausekkeet
        \begin{enumerate}[(a)]
            \item $ \sqrt{3\frac{3}{4}} \big/ \sqrt{1\frac{2}{3}} $
            \item $ \left( \frac{x}{y} + \frac{y}{x} -
                    2 \right) \big/ \left( \frac{x}{y} - \frac{y}{x} \right) $.
        \end{enumerate}
    \item[(S2002/2)]   Vuoden 1960 jälkeen on nopeimman junayhteyden matka-aika
                        Helsingin ja Lappeenrannan välillä lyhentynyt 37 prosenttia.
                        Laske, kuinka monta prosenttia keskinopeus on tällöin noussut.
                        Oletetaan, että radan pituus ei ole muuttunut.
    \item[(S2002/4a)]  Olkoon $ a \neq 0$ ja $b \neq 0 $. Sievennä lauseke
                        \[
                            \frac{a + \frac{b^2}{a} } {b + \frac{a^2}{b} }
                        \]
    \item[(K2002/3)]   Vuonna 2001 erään liikeyrityksen ulkomaille suuntautuvan
                        myynnin arvo kasvoi 10\,\% vuoteen 2000 verrattuna. Samaan
                        aikaan myynnin arvo kotimaassa väheni 5\,\%. Tällöin koko
                        myynnin arvo kasvoi 6\,\%. Laske, kuinka monta prosenttia
                        myynnistä meni vuonna 2000 ulkomaille.
    \item[(S2001/1)]   Ratkaise lineaarinen yhtälöryhmä
                       \[
                         \left\{
                          \begin{aligned}
                             3x - 2y &= 1 \\
                             4x + 5y &= 2                      
                         \end{aligned}
                         \right.
                       \]
    \item[(S2001/3)]   Juna lähtee Tampereelta klo 8.06 ja saapuu Helsinkiin klo 9.58.
                        Vastakkaiseen suuntaan kulkeva juna lähtee Helsingistä klo 8.58
                        ja saapuu Tampereelle klo 11.02. Matkan pituus on 187 kilometriä.
                        Oletetaan, että junat kulkevat tasaisella nopeudella, eikä
                        pysähdyksiin kuluvia aikoja oteta huomioon. Laske kummankin
                        junan keskinopeus. Millä etäisyydellä Helsingistä junat
                        kohtaavat, ja paljonko kello tällöin on? 
    \item[(K2001/4)]   Säiliö sisältää 2,3\,kg ilmaa, ja pumppu poistaa jokaisella
                        vedolla 5\,\% säiliössä olevasta ilmasta. Kunka monen vedon
                        jälkeen säiliössä on vähemmän kuin 0,2\,kg ilmaa?
    \item[(S2000/1)]   Sievennä seuraavat lausekkeet:
                        \begin{enumerate}[(a)]
                            \item $ \left( x^{n - 1} \right)^{n - 1} \cdot
                                \left( x^{n} \right)^{2 - n} $
                            \item $ \sqrt[3]{a} \; ( \sqrt[3]{a^2} - \sqrt[3]{a^5}) $
                        \end{enumerate}
    \item[(S2000/3)]   Matkaa kuljetaan tasaisella nopeudella. Kun matkasta on
                        jäljellä 40\,\%, nopeutta lisätään 20\,\%. Kuinka monta
                        prosenttia koko matkaan kuluva aika tällöin lyhenee?

\end{description}
