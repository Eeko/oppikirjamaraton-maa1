


\part{Lukualueet}
\chapter{Luonnolliset luvut}

Tähän tekstiä luonnollisista luvuista.

\chapter{Joukko-oppia}
\chapter{Logiikkaa}
\chapter{Kokonaisluvut}
\chapter{Kokonaislukujen aritmetiikkaa}

Tiivistelmä...

Kysymys: Mitä saadaan, kun luvusta $5$ vähennetään luku $-8$?

Negatiivisten ja positiivisten lukujen yhteen- ja vähennyslaskut on helppoa ymmärtää lukusuoran avulla.

Allaolevien esimerkkien yhteyteen kuvat lukusuoralla!

$5+8$ "viiteen lisätään $8$"

$5+(+8)$ "viiteen lisätään $+8$" $+8$ tarkoittaa samaa kuin $8$. '$+$'-merkkiä käytetään luvun edessä silloin, kun halutaan korostaa, että kyseessä on nimenomaan positiivinen luku.

$5-(+8)$ "viidestä vähennetään $+8$" Tämä tarkoittaa samaa kuin 5-8. Lukusuoralla siis liikutaan 8 pykälää taaksepäin.

$5+(-8)$  "viiteen lisätään $-8$" Mitä tapahtuu, kun lisätään negatiivinen luku? Kun lukuun lisätään 1, se kasvaa yhdellä. Kun lukuun lisätään 0, se ei kasva lainkaan. Eikö tällöin ole luonnollista ajatella, että kun lisätään luku, joka on pienempi kuin nolla, täytyisi lopputuloksesta tulla vielää pienempi. Tällä logiikalla negatiivisen luvun lisäämisen pitäisi siis pienentää alkuperäistä lukua. Siksi on sovittu, että $5+(-8)$ on yhtä suuri kuin $5-8$.

5-(-8) "viidestä vähennetään $-8$" Negatiivisen luvun lisääminen on vastakohtainen positiivisen luvun lisäämiselle. Tällöin olisi luonnillista, että negatiivisen luvun vähentäminen olisi myös vastakohtaista positiivisen luvun vähentämiselle. Kun positiivisen luvun vähentäminen pienentää lukua, pitäisi negatiivisen luvun vähentämisen siis kasvattaa lukua. Tämän vuoksi onkin sovittu, että $5-(-8)$ tarkoittaa samaa kuin $5+8$. Usein on myös tapana sanoa, että kaksi miinusmerkkiä kumoavat toisensa, jolloin lopputulos on positiivinen.

Samaan logiikkaan perustuen on sovittu myös merkkisäännöt positiivisten ja negatiivisten lukujen kertolaskuissa. Kun negatiivinen ja positiivinen luku kerrotaan keskenään, saadaan negatiivinen luku, mutta kun kaksi negatiivista lukua kerrotaan keskenään, saadaan positiivinen luku.

Seuraavista kuvat lukusuoralle:

$3*4$ "kolme kappaletta nelosia"

$3*(-4)$ "kolme kappaletta miinus-nelosia"

$-3*4$ "miinus-kolme kappaletta nelosia"

$-3*(-4)$ "miinus-kolme kappaletta miinus-nelosia"

\chapter{Jaollisuus \& tekijät}
\chapter{Rationaaliluvut ja laskusäännöt}

Laske %aika randomit luvut
a) $\frac{6}{2} + \frac{3}{5}$
b) $\frac{7}{8} - \frac{1}{4}$
c) $2 \frac{1}{3} + \frac{4}{6}$
Vastaus:
a) $\frac{18}{5}$
b) $\frac{5}{8}$
c) $3$

\chapter{Potenssisäännöt \& murtolausekkeiden sieventämistä}
\chapter{Juuret}

\section{Neliöjuuri}

\laatikko{Luvun $a$ neliöjuuri on ei-negatiivinen luku, jonka neliö on $a$. Tämä voidaan ilmaista lyhyemmin $\sqrt{b^2}=b$.}

Neliöjuuren määritteleminen $\sqrt{a}^2=a$ ei johda samaan lopputulokseen. Pohdi, miksi näin on.
%%vai parempi antaa suoraan $\sqrt{a}^2=a$, kun $a \ge 0$
Jatkossa tälaisia määritelmän pieniä muokkauksia ja niistä aiheutuvia muutoksia olisi aina hyvä pohdiskella -- saattavat jopa auttaa muistamaan määritelmän oikean muodon.
%%%%%%%%%%%%%%% ONKO ITSEISARVO KÄSITELTY!!!!! %%%%%%%%%%%%%%%%%%%%%%%%%
%Määtitelmäksi ei kelpaisi tämäkään. $\sqrt{a^2}=|a|$
%%%%%%%%%%%%%%% ONKO ITSEISARVO KÄSITELTY!!!!! %%%%%%%%%%%%%%%%%%%%%%%%%

Neliöjuurta ei siis nyt määritelty ollenkaan negatiivisille luvuille.

%yhtälöt tulevat vasta myöhemin, siksi esimerkit köyhiä

Esimerkki
\begin{align*}
\sqrt{4} = 2\ qquad\textrm{, koska $2>0$ ja $2^2 =4$} 
\end{align*}

%pythagoraan lause on pitänyt käydä ennen tätä!!
Taulutelevision kooksi on ilmoitettu mainoksessa $46''$ ja kuvasuhteeksi 19:6. Kuinka leveä televisio on arviolta?
Vastaus: $44''$ tai 111 cm

\section{Kuutiojuuri}

\laatikko{Luvun $a$ kuutiojuuri on luku, jonka kuutio on $a$. Tämä voidaan ilmaista lyhyemmin $\sqrt[3]{b^3}=b$.
Määritelmäksi voisi ottaa myös $\sqrt[3]{b^3}=b$.}

\section{n.s juuri}
Toista juurta $\sqrt[2]{a}$ merkitään $\sqrt{a}$


%\begin{Ex}
%Tehtävänanto
%\begin{solution}
%Ratkaisu
%\end{solution}
%\end{Ex}

\chapter{Murtopotenssi}
\chapter{Irrationaaliluvut}
\chapter{Reaaliluvut}
\chapter{Kompleksiluvut}
\chapter{Kertaustiivistelmä}
