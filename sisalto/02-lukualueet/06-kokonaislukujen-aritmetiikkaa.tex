\chapter{Kokonaislukujen aritmetiikkaa}

\section{Tiivistelmä}

$a-(-b)=a+b$

$-a\cdot (-b)=a\cdot b=ab$

$-a:(-b)=a\cdot b=ab$

\section{Yhteen- ja vähennyslasku}

Kysymys: Mitä saadaan, kun luvusta $5$ vähennetään luku $-8$?

Negatiivisten ja positiivisten lukujen yhteen- ja vähennyslaskut on helppoa ymmärtää lukusuoran avulla.

$5+8$ "viiteen lisätään $8$"

\missingfigure{$5+8$ lukusuoralla}

$5+(+8)$ "viiteen lisätään $+8$"

$+8$ tarkoittaa samaa kuin $8$. '$+$'-merkkiä käytetään luvun edessä silloin, kun halutaan korostaa, että kyseessä on nimenomaan positiivinen luku.

\missingfigure{$5+(+8)$ lukusuoralla}

$5-(+8)$ "viidestä vähennetään $+8$"

Tämä tarkoittaa samaa kuin 5-8. Lukusuoralla siis liikutaan 8 pykälää taaksepäin.

\missingfigure{$5-(+8)$ lukusuoralla}

$5+(-8)$ "viiteen lisätään $-8$"

Mitä tapahtuu, kun lisätään negatiivinen luku? Kun lukuun lisätään 1, se kasvaa yhdellä. Kun lukuun lisätään 0, se ei kasva lainkaan. Eikö tällöin ole luonnollista ajatella, että kun lisätään luku, joka on pienempi kuin nolla, täytyisi lopputuloksesta tulla vielää pienempi. Tällä logiikalla negatiivisen luvun lisäämisen pitäisi siis pienentää alkuperäistä lukua. Siksi on sovittu, että $5+(-8)$ on yhtä suuri kuin $5-8$.

\missingfigure{$5+(-8)$ lukusuoralla}

$5-(-8)$ "viidestä vähennetään $-8$"

Negatiivisen luvun lisääminen on vastakohtainen positiivisen luvun lisäämiselle. Tällöin olisi luonnillista, että negatiivisen luvun vähentäminen olisi myös vastakohtaista positiivisen luvun vähentämiselle. Kun positiivisen luvun vähentäminen pienentää lukua, pitäisi negatiivisen luvun vähentämisen siis kasvattaa lukua. Tämän vuoksi onkin sovittu, että $5-(-8)$ tarkoittaa samaa kuin $5+8$. Usein on myös tapana sanoa, että kaksi miinusmerkkiä kumoavat toisensa, jolloin lopputulos on positiivinen.

\missingfigure{$5-(-8)$ lukusuoralla}

Samaan logiikkaan perustuen on sovittu myös merkkisäännöt positiivisten ja negatiivisten lukujen kertolaskuissa. Kun negatiivinen ja positiivinen luku kerrotaan keskenään, saadaan negatiivinen luku, mutta kun kaksi negatiivista lukua kerrotaan keskenään, saadaan positiivinen luku.

\section{Kertolasku}

$3 \cdot 4$ "kolme kappaletta nelosia"

\missingfigure{$3 \cdot 4$ lukusuoralla}

$3 \cdot (-4)$ "kolme kappaletta miinus-nelosia"

\missingfigure{$3 \cdot (-4)$ lukusuoralla}

$-3 \cdot 4$ "miinus-kolme kappaletta nelosia"

\missingfigure{$-3 \cdot 4$ lukusuoralla}

$-3 \cdot (-4)$ "miinus-kolme kappaletta miinus-nelosia"

\missingfigure{$-3 \cdot (-4)$ lukusuoralla}

\section{Jakolasku}

Jakolaskua sanotaan kertolaskun käänteistoimitukseksi, koska se tekee tekee saman toimituksen kuin kertolasku, mutta vastakkaiseen suuntaan. Esimerkiksi $100:5\cdot 5=100$ ja $792\cdot 132:132=792$. Kun ensin kerrotaan jollain luvulla, ja sitten jaetaan samalla luvulla, päädytään takaisin samaan, mistä lähdettiin. Luonnollisesti olisi mukavaa, jos tämä ominaisuus säilyisi myös silloin, kun jaetaan ja kerrotaan negatiivisia lukuja. Tämän vuoksi jakolaskulle on sovittu samat merkkisäännöt kuin kertolaskulle, eli kaksi miinusmerkkiä kumoavat toisensa.

Esimerkiksi haluamme, että $(-12):(-3)\cdot (-3)=-12$. Nyt voimme kysyä, mitä laskun $(-12):(-3)$ tulokseksi pitäisi tulla, jotta jakolasku ja kertolasku säilyvät toisilleen käänteisinä, eli mikä luku kerrottuna $-3$:lla on $-12$. Kertolaskun merkkisäännöistä nähdään helposti, että tämän luvun täytyy olla $+4$ eli $4$. Niinpä on sovittu, että $(-12):(-3)=12:3=4$.

Laske

\begin{tehtava}
    a) $1+1$
    b) $11+(-14)$
    c) $-8-(-4)$
    d) $-9-(+7)$
    e) $-(-8)+(5)-(-(-11))$
    f) $-8:(-4)$
    g) $(-8):(-4)$
    h) $(-5)\cdot 12$


    \begin{vastaus}
        a) $2$
        b) $-3$
        c) $-4$
        d) $-16$
        e) $2$
        f) $2$
        g) $2$
        h) $-60$
    \end{vastaus}
\end{tehtava}
