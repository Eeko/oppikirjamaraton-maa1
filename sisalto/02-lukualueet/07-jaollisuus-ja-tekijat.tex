\section{Jaollisuus ja tekijät}

\laatikko{
Määritelmä

Kokonaisluvun $a$ sanotaan olevan jaollinen kokonaisluvulla $b$, jos on olemassa sellainen kokonaisluku $c$, että $a=b\cdot c$. Tällöin sanotaan myös, että $b$ on $a$:n tekijä ja $c$ on $a$:n tekijä.
}

Esimerkiksi luku $-12$ on jaollinen luvulla $3$, koska $-12=3\cdot (-4)$. Toisaalta luku $-12$ ei ole jaollinen luvulla $5$, koska ei ole mitään kokonaislukua, joka kerrottuna viidellä olisi $12$.

Jaollisuuden voi ajatella myös jakolaskun avulla niin, että esim. luku $12$ on jaollinen luvulla $3$, koska luku $12$ voidaan jakaa $3$ yhtäsuureen osaan niin, että jokaisen osan koko on kokonaisluku -- tässä tapauksessa $4$.

Kaikki luvut ovat jaollisia itsellään ja luvulla $1$. Esimerkiksi $7=7 \cdot 1=1 \cdot 7$, joten $7$ on jaollinen $1$:llä ja $37$:llä.

\laatikko{
Määritelmä

Ykköstä suurempaa kokonaislukua sanotaan alkuluvuksi, jos sillä ei ole muita tekijöitä kuin $1$ ja luku itse.
}

Esimerkiksi luvut 2, 3, 5, 7, 11, 13, 17 ja 19 ovat alkulukuja. 

\laatikko{
Aritmetiikan peruslause

Jokainen ykköstä suurempi kokonaisluku voidaan esittää yksikäsitteisesti alkulukujen tulona.
}

Esimerkiksi luku $84$ voidaan kirjoittaa muodossa $2\cdot 2\cdot 3\cdot 7$. Kokeilemalla huomataan helposti, että 2, 3, ja 7 ovat kaikki alkulukuja. Aritmetiikan peruslauseen nojalla tiedetään, että tämä on ainoa tapa kirjoittaa $84$ alkulukujen tulona - mahdollista kertolaskujärjestyksen vaihtoa lukuunottamatta. Kun luku $84$ esitetään muodossa $2\cdot 2\cdot 3\cdot 7$ on tapana sanoa, että se on \emph{jaettu alkutekijöihin}. Alkutekjät esitetään yleensä kasvavassa numerojärjestyksessä. Jos sama luku esiintyy tekijöissä useampaan kertaan, on se yleensä yleensä tapana merkitä potenssina. Tällöin luku $84$ voitaisiin kirjoittaa tekijöihin jaettuna $2^2\cdot 3\cdot 7$ ja luku $96$ muodossa $2\cdot 2\cdot 2\cdot 2\cdot 2\cdot 3=2^5\cdot 3$.

Luvun alkutekijät voi löytää etsimällä luvulle ensin jonkin esityksen kahden luvun tulona. Näiden kahden luvun ei tarvitse olla alkulukuja. Sen jälkeen sama toistetaan näille kahdelle luvulle ja edelleen aina uusille luvuille, kunnes tulossa on jäljellä vain alkulukuja. Esimerkiksi luvun $96$ alkutekijät voi löytää vaikkapa seuraavanlaisella ketjulla: $96 = 2 \cdot 48 = 2 \cdot (2 \cdot 24) = 2 \cdot 2 \cdot (6 \cdot 4) = 2 \cdot 2 \cdot (2 \cdot 3) \cdot (2 \cdot 2)$. Nyt jäljellä on vain alkulukuja ja saatu tulo voidaan kirjoittaa lyhennettynä $96 = 2^5 \cdot 3$.

\begin{tehtava}
Mitkä seuraavista luvuista ovat jaollisia luvulla $4$? Jos luku $a$ on jaollinen luvulla $4$, kerro, millä kokonaisluvulla $b$ pätee $a = 4 \cdot b$.\\
a) 1 \quad b) 12  \quad c) 13 \quad d) 2 \quad e) -20 \quad f) 0

\begin{vastaus}
\begin{enumerate}
	\item Ei ole jaollinen luvulla 4
	\item On jaollinen luvulla 4, $12 = 4 \cdot 3$
	\item Ei ole jaollinen luvulla 4
	\item Ei ole jaollinen luvulla 4
	\item On jaollinen luvulla 4, $-20 = 4 \cdot (-5)$
	\item On jaollinen luvulla 4, $0 = 4 \cdot 0$
\end{enumerate}
\end{vastaus}
\end{tehtava}

\begin{tehtava}
Mitkä seuraavista luvuista ovat alkulukuja? Jos luku ei ole alkuluku, esitä se joidenkin kahden kokonaisluvun (jotka eivät ole ykkönen ja luku itse) tulona.\\
a) 6 \quad b) 11 \quad c) 29 \quad d) -27 \quad e) -11 \quad f) 0

\begin{vastaus}
\begin{enumerate}
	\item Ei ole alkuluku, esim. $6 = 2 \cdot 3$
	\item On alkuluku
	\item On alkuluku
	\item Ei ole alkuluku, esim. $27 = 3 \cdot (-9)$
	\item Ei ole alkuluku, esim. $-11 = (-1) \cdot 11$ Huom. alkuluvut ovat suurempia kuin yksi (ja siis positiivisia)
	\item Ei ole alkuluku, esim. $0 = 6 \cdot 0$
\end{enumerate}
\end{vastaus}
\end{tehtava}

\begin{tehtava}
Jaa seuraavat luvut alkutekijöihin.\\
a) 12 \quad b) 15 \quad c) 28 \quad d) 30 \quad e) 64 \quad f) 90 \quad g) 100

\begin{vastaus}
\begin{enumerate}
	\item $12 = 2^2 \cdot 3$
	\item $15 = 3 \cdot 5$
	\item $28 = 2^2 \cdot 7$
	\item $30 = 2 \cdot 3 \cdot 5$
	\item $64 = 2^6$
	\item $90 = 2 \cdot 3^2 \cdot 5$
	\item $100 = 2^2 \cdot 5^2$
\end{enumerate}
\end{vastaus}
\end{tehtava}