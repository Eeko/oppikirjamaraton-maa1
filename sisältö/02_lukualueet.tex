
\part{Luvut ja laskutoimitukset}
\chapter{Lähtötasotesti}

(Tää tulee oikeasti ennen tätä chapteria ja osaa)

\begin{tehtava}
\begin{enumerate}
\item Laske $2^2+2 \cdot 2+2$
\item sasdas
\item 
\end{enumerate}

\begin{vastaus}
\begin{enumerate}
\item 
\item
\item

\end{enumerate}
\end{vastaus}
\end{tehtava}
\chapter{Numerot ja luvut}

(Joonas jatkaa tästä vielä!)

Matematiikka tarjoaa työkaluja asioiden jäsentämiseen, päättelyyn ja mallintamiseen. Alasta riippuen käsittelemme matematiikassa erilaisia \textbf{objekteja}: Geometriassa tarkastelemme tasokuvioita ja kolmiulotteisia rakenteita. Algebrassa tutkii lukujen ja funktioiden ominaisuuksia. Todennäköisyyslaskenta arvioi erilaisten tapausten ja tilanteiden mahdollisuuksia ja riskejä. Matemaattinen analyysi (kurssit 7,8 ja 10) tutkii funktioita ja niiden muuttumista.

Jokaiseen tarkastelukohteeseen liitetään myös niille ominaisia \textbf{operaatioita}. Tämä kurssi käsittelee lähinnä lukuja ja niiden operaatioita, joita \textbf{laskutoimituksiksi} kutsutaan. Aloitetaan yksinkertaisista määritelmistä: mitä tarkoittavat \textbf{numero} ja \textbf{luku}?

\laatikko{Länsimaisessa ... on käytössämme kymmenen numeromerkkiä: 0, 1, 2, 3, 4, 5, 6, 7, 8 ja 9. Näitä kutsutaan hindu-arabialaisiksi numeroiksi.  }

Sanalla numero voidaan siis viitata yksittäiseen kirjoitettuun merkkiin. 

\begin{esimerkki}
Luku \[715531\] koostuu numeroista 7, 1, 5, 5, 3 ja 1.
\end{esimerkki}



Olennaista on myös...
lukujärjestelmä, paikkajärjestelmä

MIKSI KÄYTÄMME KIRJAIMIA?

suuruus, yhtäsuuruus, eri suuret

, ja Erilaisilla luvuilla voidaan suorittaa erilaisia laskutoimituksia. Seuraavissa luvuissa esitellään ja käydään läpi lukiomatematiikassa ja mahdollisissa jatko-opinnoissa käytettäviä lukujoukkoja ja tavallisimmat laskutoimitukset.

\chapter{Luonnolliset luvut}

(Joonas jatkaa tästä vielä!)

Suomen kielen verbi 'laskea' voi tarkoittaa matematiikassa kahta eri asiaa: lukumäärien laskemista ja laskutoimitusten suorittamista.

\laatikko{laskea (lukumäärä) englanti count ruotsi \_ /n
laskea (laskutoimitus) englanti calculate , ruotsi \_}

Ihmisellä ja muilla eläimillä on luonnostaan matemaattisia taitoja. Monet niistä, esimerkiksi lukumäärien laskeminen, ovat yllättävän monimutkaisia kognitiivisia prosesseja, jotka kehittyvät lapsuudessa – toisilla aiemmin, toisilla myöhemmin. Kaikki koulussa opeteltava peruslaskento ja myös matematiikka tieteen alana rakentavat tämän biologisen osaamisen päälle. Laskeminen itsessään on vain yksi matematiikan osa-alue, eikä kaikki matematiikka ole laskemista. Huomaa, että suomen kielen verbillä laskea tarkoitetaan sekä lukumäärien laskemista (engl. counting) että lukujen laskutoimitusten suorittamista (engl. calculating).

Hyvin olennaisena kehitysaskeleena niin yksilön matemaattiselle ajattelulle kuin yhteiskunnallekin on ollut luonnollisen kielen tavoin kyky merkitä lukumäärien laskemista ja muuta matemaattista pohdintaa kirjalliseen muotoon.  On olemassa hyvin monia erilaisia tapoja merkitä lukumääriä. Helpoin tapa ja yksinkertaisin tapa on käyttää vain yhtä samaa merkkiä ja toistaa sitä. Jos

käytettävissä olevien merkintöjä määrää voidaan lisätä, jolloin suuria lukuja voidaan kirjoittaa lyhyemmin. Tämä vastaa myös luonnollisten kielten tilannetta: Suomen kielen aakkosiin kuuluu 29 kirjainta, joista sanat muodostetaan. Sanat voivat olla kuinka pitkiä vain kahdesta kirjaimesta ylöspäin. Kiinassa sen sijaan käytetään omaa piirrosmerkkiä jokaiselle sanalle. Merkkejä täytyy osata 29 sijaan tuhansia, mutta jokaisen sanan voi kirjoittaa lyhyesti. 
Matematiikassa erilaisista numeromerkeistä tai yksinkertaisesti numeroista muodostetaan lukuja yhdistelemällä niitä sopivasti erilaisten paikkajärjestelmien mukaan. Esimerkiksi antiikin Roomassa käytössä olivat numeromerkit I, V, X, L, C , D ja M. Niiden numeroiden vastaavuudet meidän käyttämiimme lukuarvoihin ovat seuraavat:
I=1
V=5
X=10
L=50
C=100
D=500
M=1 000
Huomaa, että suuri osa roomalaisista numeromerkeistä ovat jo itsessään arvoltaan niin suuria, että me tarvitsemme niiden nykyilmaisuun monta merkkiä! Nollaa roomalaisissa numeroissa ei ole, ja tiettävästi tuhatta suurempia arvoja esittäviä numeromerkkejä merkkejä otettiin käyttöön vasta keskiajalla. 
Lukuja koostetaan näistä merkeistä siten, että merkit kirjoitetaan peräkkäin pääasiassa laskevassa järjestyksessä ja niiden numeroarvot lasketaan yhteen. Jos arvoltaan pienempi numeromerkki (korkeintaan yksi) edeltää suurempaa, pienempi vähennetään suuremmasta ennen yhteenlaskun jatkamista. 

\chapter{Joukko-oppia}

(voisi integroida lukuoppiin) T: Joonas
\chapter{Logiikkaa}

(voisi integroida yhtälöiden teoriaan, sinne saa hyvin ekvivalenssin, implikaation ja disjunktion ja nepä ovat ne, mitä juuri tarvitaan) T: Joonas
\chapter{Kokonaisluvut}
\chapter{Kokonaislukujen aritmetiikkaa}
\chapter{Jaollisuus \& tekijät}
\chapter{Rationaaliluvut ja laskusäännöt}

Laske %aika randomit luvut
a) $\frac{6}{2} + \frac{3}{5}$
b) $\frac{7}{8} - \frac{1}{4}$
c) $2 \frac{1}{3} + \frac{4}{6}$
Vastaus:
<<<<<<< HEAD
a) $\frac{18}{5}$
b) $\frac{5}{8}$
c) $3$
=======
\begin{enumerate}
\item $x^9$
\item $a^6$
\item $a^4$
\item $1$
\item $1000$
\end{enumerate}
\end{vastaus}
\end{tehtava}

\section{Murtolausekkeiden sieventäminen}
%tässä pitää opettaa binomin neliösäännöt ja ne (3kpl)
%Kaavojen johtaminen
\laatikko{
\begin{enumerate}
\item $(a+b)^2 = a \cdot a + a \cdot b + b \cdot a + b \cdot b = a^2 + ab + ba + b^2 = a^2 + 2ab + b^2 $
\item $(a-b)^2 = a \cdot a + a \cdot -b + (-b) \cdot a + (-b) \cdot (-b) = a^2 - ab - ba + b^2 = a^2 - 2ab + b^2 $
\item $(a+b)(a-b) = a \cdot a + a \cdot (-b) + b \cdot a + b \cdot (-b) = a^2 - ab + ba - b^2 = a^2 - b^2 $
\end{enumerate}}

\begin{tehtava}
%tää voi olla eka tehtävä
Sievennä
\begin{enumerate}
\item $\frac{a^2+2ab+b^2}{a+b}$
\item $\frac{a^2-2ab+b^2}{a-b}$
\item $\frac{a^2-b^2}{a+b}$
\end{enumerate}
\begin{vastaus}
Vastaus:
\begin{enumerate}
\item $a-b$
\item $a+b$
\item $a-b$
\end{enumerate}
\end{vastaus}
\end{tehtava}
>>>>>>> 5e28041e61842d3cd8b8fe172f1cb301f14ce64a

\chapter{Potenssisäännöt \& murtolausekkeiden sieventämistä}
\chapter{Juuret}

\section{Neliöjuuri}

\laatikko{Luvun $a$ neliöjuuri on ei-negatiivinen luku, jonka neliö on $a$. Tämä voidaan ilmaista lyhyemmin $\sqrt{b^2}=b$.}

Neliöjuuren määritteleminen $\sqrt{a}^2=a$ ei johda samaan lopputulokseen. Pohdi, miksi näin on.
%%vai parempi antaa suoraan $\sqrt{a}^2=a$, kun $a \ge 0$
Jatkossa tälaisia määritelmän pieniä muokkauksia ja niistä aiheutuvia muutoksia olisi aina hyvä pohdiskella -- saattavat jopa auttaa muistamaan määritelmän oikean muodon.
%%%%%%%%%%%%%%% ONKO ITSEISARVO KÄSITELTY!!!!! %%%%%%%%%%%%%%%%%%%%%%%%%
<<<<<<< HEAD
%Määtitelmäksi ei kelpaisi tämäkään. $\sqrt{a^2}=|a|$
=======

%Määritelmäksi ei kelpaisi tämäkään. $\sqrt{a^2}=|a|$ EI OLE KÄSITELTY. Tulee esimerkkinä funktiosta funktioaiheen jälkeen.

>>>>>>> 38cece91be982ab0a64e0d881624a60dc5308e2d
%%%%%%%%%%%%%%% ONKO ITSEISARVO KÄSITELTY!!!!! %%%%%%%%%%%%%%%%%%%%%%%%%

Neliöjuurta ei siis nyt määritelty ollenkaan negatiivisille luvuille.

%yhtälöt tulevat vasta myöhemin, siksi esimerkit köyhiä

Esimerkki
\begin{align*}
\sqrt{4} = 2\ qquad\textrm{, koska $2>0$ ja $2^2 =4$} 
\end{align*}

%pythagoraan lause on pitänyt käydä ennen tätä!!
Taulutelevision kooksi on ilmoitettu mainoksessa $46''$ ja kuvasuhteeksi 19:6. Kuinka leveä televisio on arviolta?
Vastaus: $44''$ tai 111 cm

\section{Kuutiojuuri}

\laatikko{Luvun $a$ kuutiojuuri on luku, jonka kuutio on $a$. Tämä voidaan ilmaista lyhyemmin $\sqrt[3]{b^3}=b$.
Määritelmäksi voisi ottaa myös $\sqrt[3]{b^3}=b$.}
%tämä jälkimmäinen on ehkä järkevämpi määritelmä.
%Olisi varmaan hyvä ottaa samanlainen määritelmä neliöjuuren tapauksessakin
%jolloin neliöjuureen määritelmään tulisi 2 ehtoa.
Kuutiojuuren voi siis ottaa mistä tahansa luvusta.
%vai reaaliluvusta?
%Reaaliluvuista puhutaan kuitenkin vasta myöhemmin, niin olkoon näin.

<<<<<<< HEAD
\section{n.s juuri}
=======
<<<<<<< HEAD
\section{n:s juuri}
=======
\section{n. juuri}
>>>>>>> e4d568f631b907029aaa13a7ccf75b37f49df6ae
>>>>>>> 38cece91be982ab0a64e0d881624a60dc5308e2d
Kaikkia juuria ei kuitenkaan kannata määritellä yksitellen. Tehdään siis mahdollisimman paljon kerralla. Edeltä kuitenkin voi huomata, että kuutiojuuri on määritelty kaikille luvuille, mutta neliöjuuri vain ei-negatiivisille luvuille. Tämä toistuu myös muissa juurissa. Määritellään siis parilliset ja parittomat juuret erikseen.

%on ehkä parempi esittää nämä molemma samalla eikä kuten subsections
%sama määritelmä, mutta todetaan parillisilla vaadittavan >= 0.
%%%RISTIRIITA ED. KANSSA
Juurimerkinnällä $\sqrt[n]{a}=b$ (luetaan \emph{ännäs juuri aasta on bee} tarkoitetaan lukua, joka toteuttaa ehdon $b^n = a$. Jotta juuri olisi ykskäsitteinen, on parillisilla juurilla ($\sqrt{a}, \sqrt[4]{a}, \sqrt[6]{a}$\ldots) vaadittava, että $b\ge0$.

\subsection{parilliset juuret}

\laatikko{Luvun $a$ $n$.s juuri (luetaan \emph{ännäs juuri}) on ei-negatiivinen luku, jonka neliö on $a$. Tämä voidaan ilmaista lyhyemmin $\sqrt[n]{b^n}=b$.}

\subsection{parittomat juuret}
\laatikko{Luvun $a$ n.s juuri on ei-negatiivinen luku, jonka neliö on $a$. Tämä voidaan ilmaista lyhyemmin $\sqrt[n]{b^n}=b$.}

Nyt on paikallaan todeta, että toista juurta $\sqrt[2]{a}$ merkitään $\sqrt{a}$.

\begin{tabular}{c|c}
parillinen juuri & pariton juuri\\
\hline
$\sqrt[n]{a}^n=a$, $a\ge0$ & $\sqrt[n]{a}^n$, kaikilla $a$
\end{tabular}

Esimerkiksi $\sqrt[3]{-8}=-2$ koska $(-2)^2=-8$, mutta $\sqrt[4]{-8}$ ei ole määritelty, koska minkään luvun neljäs potenssi ei ole negatiivinen.

%Mitä näille kahdelle seuraavalle tehdään?
%$\sqrt[n]{ab}=\sqrt[n]{a}\sqrt[n]{b}$
%Jos n on parillinen, niin on lisäksi vaadittava, että $a\ge0$ ja $b\ge0$.
%
%$\sqrt[n]{\frac{a}{b}}=\frac{\sqrt[n]{a}}{\sqrt[n]{b}}$
%Jos n on parillinen, niin on lisäksi vaadittava, että $a\ge0$ ja $b\ge0$.

\input{Murtopotenssi}
\chapter{Irrationaaliluvut}
\chapter{Reaaliluvut}
\chapter{Kompleksiluvut}
\chapter{Kertaustiivistelmä}
