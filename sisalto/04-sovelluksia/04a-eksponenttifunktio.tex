\chapter{Eksponenttifunktio}

\laatikko{Muotoa $f(x) = a^x$ olevia funktioita kutsutaan eksponenttifunktioiksi.}

Eksponenttifunktio eroaa aiemmin esitellystä potenssifunktiosta siinä, että
eksponenttifunktiossa muuttuja $x$ on eksponentissa, kun potenssifunktiossa
se on kantalukuna. Tästä seuraa se, että eksponenttifunktio eroaa luonteeltaan
oleellisesti potenssifunktiosta.

Eksponenttifunktioita on kahta tyyppiä: kasvavia ja väheneviä.
Kasvavilla eksponenttifunktioilla $a>1$, esimerkiksi

\missingfigure{Kuva, jossa joukko kuvaajia $f(x) = a^x$, joille $a>1$.}

Vähenevillä eksponenttifunktioilla $0<a<1$, esimerkiksi

\missingfigure{Kuva, jossa joukko kuvaajia $f(x) = a^x$, joille $0<a<1$.}

Kun $a<0$, eksponenttifunktiota ei ole määritelty, koska negatiiviselle
kantaluvulle ei ole määritelty ei-kokonaislukupotenssia.

Kun $a=0$ tai $a=1$, eksponenttifunktio pelkistyy vakiofunktioksi.
Lisäksi $0^0$ ei ole määritelty, joten on turvallisinta vaatia, että
kantaluvulle $a$ pätee $a>0$ ja $a \neq 1$.

Eksponenttifunktioita käytetään kuvaamaan sellaista
jatkuvaa kasvua tai vähenemistä, jossa kullakin ajanhetkellä
funktion hetkellinen muutos on suoraan verrannollinen funktion sen
hetkiseen arvoon. Tähän palataan myöhemmillä matematiikan kursseilla.

Jos kysytään, millä $x$:n arvoilla eksponenttifunktio saavuttaa tietyn
arvon, muodostuu \emph{eksponenttiyhtälö}. 

\begin{esimerkki}
Millä muuttujan $x$ arvoilla eksponenttifunktio $f(x) = 2^x$ saa arvon
$f(x) = 64$?

Kirjoitetaan tehtävä yhtälöksi: $2^x = 64$.
Eksponenttifunktion kantalukuna on $2$, joten kyseessä on kasvava
eksponenttifunktio. Kokeillaan $x$:n eri arvoja: Kun $x = 3$,
$f(x) = 2^3 = 8$, joka on pienempi kuin $64$. Ratkaisu on siis
suurempi kuin kolme. Jatkamalla vastaavaa päättelyä löydetään ratkaisu
$x = 6$.

Varmistutaan vielä siitä, että yhtälöllä ei ole muita ratkaisuja:
koska eksponenttifunktio kasvaa kaikkialla, ei voi olla tilannetta, jossa
$f(x)$ saa uudelleen arvon $0$, kun $x > 6$. Kasvavuus perustelee
myös, miksi $f(x)$ ei voi olla $0$, kun $x < 6$.

Siis ainoa ratkaisu yhtälölle on $x = 6$.
\end{esimerkki}

\begin{esimerkki}
Millä muuttujan $x$ arvoilla eksponenttifunktio
$f(x) = \left( \frac{1}{2} \right)^{x}$ saa arvon
$f(x) = 1/5$?

Edellisen esimerkin tavoin kokeillaan $x$:n eri arvoja. Havaitaan,
että kun $x = 2$, funktio saa arvon $f(x) = \frac{1}{4}$, ja
kun $x = 3$, on $f(x) = \frac{1}{8}$. Ratkaisu on siis välillä
$2 < x < 3$.

Haarukointia voidaan jatkaa esimerkiksi $x$:n arvolla $x = 2,5$,
jolloin päästään lähemmäs ratkaisua. Yhtälön ratkaisu on kuitenkin
irrationaalinen, joten sen desimaalikehitelmä on äärettömän pitkä ja
jaksoton. Tarkkaa ratkaisua ei siis saada tällä menetelmällä.

Yleisen eksponenttiyhtälön tarkkaan ratkaisemiseen palataan myöhemmillä
matematiikan kursseilla.
\end{esimerkki}

\begin{tehtava}
Olkoon $f(x) = 4^x$. Laske
\begin{enumerate}[a)]
\item $f(0)$
\item $f(3)$
\item $f(\frac{1}{2})$
\end{enumerate}
\begin{vastaus}
\begin{enumerate}[a)]
\item $1$
\item $64$
\item $2$
\end{enumerate}
\end{vastaus}
\end{tehtava}

\begin{tehtava}
Olkoon $f(x) = 10^x$. Millä $x$:n arvoilla
\begin{enumerate}[a)]
\item $f(x) = 1000$
\item $f(x) = \frac{1}{100}$
\item $f(x) = -1$?
\end{enumerate}
\begin{vastaus}
\begin{enumerate}[a)]
\item $3$
\item $6$
\item Ei ratkaisua.
\end{enumerate}
\end{vastaus}
\end{tehtava}

\begin{tehtava}
Minkä kahden kokonaisluvun välissä yhtälön
$10^x = 500$ ratkaisu on?
\begin{vastaus}
Ratkaisu on lukujen $2$ ja $3$ välissä.
\end{vastaus}
\end{tehtava}
