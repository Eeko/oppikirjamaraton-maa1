\chapter{Kokonaisluvut}

Yksinkertaisimmat käyttämämme luvut ovat lukumäärien ilmaisemiseen käytetyt $0,
1, 2, 3, \ldots$. Näitä kutsutaan \emph{luonnollisiksi luvuiksi}, ja niiden
joukkoa eli kaikkia luonnollisia lukuja yhdessä merkitään symbolilla
$\mathbb{N}$. Edellä nolla määriteltiin luonnolliseksi luvuksi, mutta tästä
ei ole yhteistä sopimusta: jotkut pitävät nollaa luonnollisena lukuna ja
toiset eivät.

Luonnollisille luvuille $m$ ja $n$ on määritelty yhteenlasku $m + n$, esimerkiksi
$5 + 3 = 8$.
Luonnollisten lukujen $m$ ja $n$ kertolasku määritellään peräkkäisinä yhteenlaskuina
\[m \cdot n = \underbrace{m + m + \ldots + m}_{n\text{ kpl}} = \underbrace{n + n + \ldots + n}_{m\text{ kpl}}.\]
Nollalla kertomisen ajatellaan olevan "tyhjä yhteenlasku"\ eli nolla,
$0 \cdot m = 0$.

Luonnollisten lukujen $m$ ja $n$ erotus määritellään yhteenlaskun avulla:
$m-n$ on luku $k$, jolle $k + n = m$. Kahden luonnollisen luvun erotus
ei kuitenkaan aina ole luonnollinen luku, esimerkkinä $3 - 5$.
Ratkaisemme ongelman määrittelemällä kullekin luonnolliselle
luvulle $n$ vastaluvun $-n$, jolle $n + (-n) = 0$.

Luonnolliset luvut ja niiden vastaluvut muodostavat yhdessä
kokonaislukujen joukon
\[\mathbb{Z} = \{\ldots, -2, -1, 0, 1, 2, \ldots\}.\]
Kun käytämme kokonaislukuja, voidaan kahden luvun erotus määritellä
yhteenlaskun ja vastaluvun avulla yksinkertaisesti $m-n = m+(-n)$.