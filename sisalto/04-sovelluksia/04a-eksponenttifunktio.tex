\chapter{Eksponenttifunktio}

\laatikko{Muotoa $f(x) = a^x$ olevia funktioita kutsutaan eksponenttifunktioiksi.}

Eksponenttifunktioita on kahta päätyyppiä: kasvavia ja väheneviä.
Kasvavilla eksponenttifunktioilla $a>1$, esimerkiksi

\missingfigure{Kuva jossa joukko kuvaajia $f(x) = a^x$, joilla $a>1$.}

Vähenevillä eksponenttifunktioilla $0<a<1$, esimerkiksi

\missingfigure{Kuva jossa joukko kuvaajia $f(x) = a^x$, joilla $0<a<1$.}

Kun $a<0$, eksponenttifunktiota ei ole määritelty, koska negatiiviselle
kantaluvulle ei ole määritelty ei-kokonaislukupotenssia.

Kun $a=0$ tai $a=1$, eksponenttifunktio pelkistyy vakiofunktioksi.
Lisäksi $0^0$ ei ole määritelty, joten turvallisinta on vaatia, että
kantaluvulle $a$ pätee $a>0$ ja $a \neq 1$.