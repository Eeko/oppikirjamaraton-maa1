\section{Desimaaliluvut}

\emph{Desimaaliluvut} on \emph{kymmenjärjestelmään} perustava tapa merkitä rationaalilukuja. Niillä voi merkitä myös irrationaalilukujen \emph{likiarvoja}.

\laatikko{
$123,456$ on esimerkki desimaaliluvusta.

\begin{itemize}
	\item $123$ on sen \emph{kokonaisosa}.
	\item Kokonaisluku erotetaan loppuosasta \emph{desimaalierottimella}, joka on Suomessa pilkku (,).
	\item Osaa $,456$ kutsutaan desimaaliluvun \emph{loppuosaksi}.
\end{itemize}

Esimerkkinä annettu desimaaliluku tulkitaan seuraavasti:
\begin{equation}
123,456 = 1 \cdot 10^2 + 2 \cdot 10^1 + 3 \cdot 10^0 + 4 \cdot 10^{-1} + 5 \cdot 10^{-2} + 6 \cdot 10^{-3}
\end{equation}
}

Desimaaliluvut voidaan muuttaa murtoluvuiksi laskemalla ne auki ylläolevan tavan mukaan.

\begin{esimerkki}
$21,37 = 2 \cdot 10^1 + 1 \cdot 10^0 + 3 \cdot 10^{-1} + 7 \cdot 10^{-2} = 2 \cdot 10 + 1 \cdot 1 + 3 \cdot \frac{1}{10} + 7 \cdot \frac{1}{100} = 21 + \frac{3}{10} + \frac{7}{100} = \frac{21 \cdot 100}{100} + \frac{3 \cdot 10}{10 \cdot 10} + \frac{7}{100} = \frac{21 \cdot 100 + 3 \cdot 10 + 7}{100} = \frac{2100+30+7}{100} = \frac{2137}{100}$
\end{esimerkki}

Toisaalta päästään paljon helpommalla, kun huomataan, että voidaan vain kertoa ja jakaa koko luku "niin monella kympillä, kuin on numeroita desimaalipilkun jälkeen". Täsmällisemmin sanottuna $10^n$:llä, jossa $n$ on pilkun jälkeen tulevien numeroiden määrä.

\begin{esimerkki}
\end{esimerkki}