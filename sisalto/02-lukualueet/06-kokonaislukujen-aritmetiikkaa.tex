\section{Kokonaislukujen laskusäännöt}

\section{Tiivistelmä}

    $a-(-b)=a+b$
    
    $-a\cdot (-b)=a\cdot b=ab$
    
    $-a:(-b)=a: b=a:b$

\section{Yhteen- ja vähennyslasku}

    Kysymys: Mitä saadaan, kun luvusta $5$ vähennetään luku $-8$?
    
    \laatikko{
        Määritelmä
        
        Jokaisella luvulla $a$ on vastaluku $-a$, jolle pätee $a+(-a)=0$.
    }
    
    Esimerkiksi luvun $2$ vastalukua merkitään $-2$, ja sille pätee $2+(-2)=0$.
    Vastaavasti luvun $-2$ vastaluku on sellainen luku, joka laskettuna yhteen
    luvun $-2$ kanssa antaa luvun $0$. Tämä on tietysti $2$, koska $-2+2=0$.
    Näin voidaan huomata, että $-(-2)=2$.
    
    Negatiivisten ja positiivisten lukujen yhteen- ja vähennyslaskut voidaan myös
    tulkita lukusuoran avulla.
    
    % tässä on vähän kyseenalaista käyttää sekaisin sanallista ja numeerista esitystä
    
    $5+8$ "viiteen lisätään $8$"
    
    \missingfigure{$5+8$ lukusuoralla}
    
    $5+(+8)$ "viiteen lisätään $+8$"
    
    $+8$ tarkoittaa samaa kuin $8$. '$+$'-merkkiä käytetään luvun edessä silloin,
    kun halutaan korostaa, että kyseessä on nimenomaan positiivinen luku.
    
    \missingfigure{$5+(+8)$ lukusuoralla}
    
    $5-(+8)$ "viidestä vähennetään $+8$"
    
    Tämä tarkoittaa samaa kuin 5-8. Lukusuoralla siis liikutaan 8 pykälää taaksepäin.
    
    \missingfigure{$5-(+8)$ lukusuoralla}
    
    $5+(-8)$ "viiteen lisätään $-8$"
    
    Mitä tapahtuu, kun lisätään negatiivinen luku? Kun lukuun lisätään 1, se
    kasvaa yhdellä. Kun lukuun lisätään 0, se ei kasva lainkaan. Eikö tällöin
    ole luonnollista ajatella, että kun lisätään luku, joka on pienempi kuin
    nolla, täytyisi lopputuloksesta tulla vielää pienempi. Tällä logiikalla
    negatiivisen luvun lisäämisen pitäisi siis pienentää alkuperäistä lukua.
    Siksi on sovittu, että $5+(-8)$ on yhtä suuri kuin $5-8$.
    
    \missingfigure{$5+(-8)$ lukusuoralla}
    
    $5-(-8)$ "viidestä vähennetään $-8$"
    
    Negatiivisen luvun lisääminen on vastakohtainen positiivisen luvun lisäämiselle.
    Tällöin olisi luonnollista, että negatiivisen luvun vähentäminen olisi myös
    vastakohtaista positiivisen luvun vähentämiselle. Kun positiivisen luvun
    vähentäminen pienentää lukua, pitäisi negatiivisen luvun vähentämisen siis
    kasvattaa lukua. Tämän vuoksi onkin sovittu, että $5-(-8)$ tarkoittaa samaa
    kuin $5+8$.
    
    \missingfigure{$5-(-8)$ lukusuoralla}
    
    Samaan logiikkaan perustuen on sovittu myös merkkisäännöt positiivisten ja
    negatiivisten lukujen kertolaskuissa. Kun negatiivinen ja positiivinen luku
    kerrotaan keskenään, saadaan negatiivinen luku, mutta kun kaksi negatiivista
    lukua kerrotaan keskenään, saadaan positiivinen luku.

\section{Kertolasku}

    $3 \cdot 4$ "kolme kappaletta nelosia"
    
    \missingfigure{$3 \cdot 4$ lukusuoralla}
    
    $3 \cdot (-4)$ "kolme kappaletta miinus-nelosia"
    
    \missingfigure{$3 \cdot (-4)$ lukusuoralla}
    
    $-3 \cdot 4$ "miinus-kolme kappaletta nelosia"
    
    \missingfigure{$-3 \cdot 4$ lukusuoralla}
    
    $-3 \cdot (-4)$ "miinus-kolme kappaletta miinus-nelosia"
    
    \missingfigure{$-3 \cdot (-4)$ lukusuoralla}

\section{Jakolasku}

    Jakolaskua sanotaan kertolaskun käänteistoimitukseksi, koska se tekee tekee
    saman toimituksen kuin kertolasku, mutta vastakkaiseen suuntaan. Esimerkiksi
    $100:5\cdot 5=100$ ja $792\cdot 132:132=792$. Kun ensin kerrotaan jollain
    luvulla, ja sitten jaetaan samalla luvulla, päädytään takaisin samaan, mistä 
    lähdettiin. Luonnollisesti olisi mukavaa, jos tämä ominaisuus säilyisi myös
    silloin, kun jaetaan ja kerrotaan negatiivisia lukuja. Tämän vuoksi jakolaskulle
    on sovittu samat merkkisäännöt kuin kertolaskulle, eli kaksi miinusmerkkiä
    kumoavat toisensa.
    
    Esimerkiksi haluamme, että $(-12):(-3)\cdot (-3)=-12$. Nyt voimme kysyä, mitä
    laskun $(-12):(-3)$ tulokseksi pitäisi tulla, jotta jakolasku ja kertolasku
    säilyvät toisilleen käänteisinä, eli mikä luku kerrottuna $-3$:lla on $-12$.
    Kertolaskun merkkisäännöistä nähdään helposti, että tämän luvun täytyy olla
    $+4$ eli $4$. Niinpä on sovittu, että $(-12):(-3)=12:3=4$.
    
    
    \begin{multicols}{2}
        \begin{tehtava}
            \begin{enumerate}[a)]
                \item $1+1$
                \item $11+(-14)$
                \item $-8-(-4)$
                \item $-9-(+7)$
                \item $-(-8)+(5)-(-(-11))$
                \item $-8:(-4)$
                \item $(-8):(-4)$
                \item $(-5)\cdot 12$
            \end{enumerate}
            
            \begin{vastaus}
                a) $2$
                b) $-3$
                c) $-4$
                d) $-16$
                e) $2$
                f) $2$
                g) $2$
                h) $-60$
            \end{vastaus}
        \end{tehtava}
        
        \begin{tehtava}
            \begin{enumerate}[a)]
                \item $3+5$
                \item $10-5-6+1$
                \item $2 \cdot 2 - 1$
                \item $-9 - 5 \cdot (-2) + 3$ 
                \item $10 \cdot (5 - 2)$
                \item $(2-5)(5 - 1) + 1$
                \item $-9 - 2 \cdot ( 3 - 2 \cdot (3\cdot2 - 1))$
            \end{enumerate}
            
            \begin{vastaus} 
                a) $3$ \\
                b) $0$ \\
                c) $3$ \\
                d) $4$ \\
                e) $30$ \\
                f) $-11$ \\
                g) $5$ \\
            \end{vastaus}
        \end{tehtava}
    \end{multicols}