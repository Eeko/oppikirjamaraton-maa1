\chapter{Reaaliluvut}

Ville työstää tätä nyt.

Jo antiikin aikoina huomattiin, että kaikki luvut eivät ole rationaalilukuja, eli kaikkia lukuja ei voi esittää kokonaislukujen suhteena. Tällaisia lukuja kutsutaan \emph{irrationaaliluvuiksi}.

Esimerkiksi $\sqrt{2}$ on irrationaaliluku. (Tämä todistetaan
luvun lopussa.) Toinen tuttu irrationaaliluku on $\pi$.

Rationaaliluvut eivät siis täytä lukusuoraa kokonaan, vaikka
rationaalilukuja onkin lukusuoralla tiheässä.

\missingfigure{Kuva lukusuorasta, johon on merkitty pii ja neliöjuuri 2}

Kun rationaalilukuihin lisätään irrationaaliluvut, saadaan reaalilukujen joukko $\mathbb{R}$. Kaikki rationaalilukuja koskevat
laskusäännöt pätevät myös reaaliluvuille.

Koska kaikkien rationaalilukujen desimaaliesitykset ovat päättyviä tai jaksolisia (tästä puhuttiin luvussa \ref{rationaaliluvut}), on
kaikkien irrationaalilukujen desimaaliesitys päättymätön ja jaksoton. Esimerkiksi luvun
\[\sqrt{2}= 1,414213562373095048801688724209\ldots\]
desimaaliesityksessä on toki samanlaisia kohtia
(tässäkin pätkässä 88 esiintyy kahdesti), mutta desimaalit eivät koskaan ''ala alusta''.

Reaalilukujen ominaisuuksien tarkka todistaminen on yllättävän monimutkaista, ja se on tapana sivuuttaa lukiossa täysin. Jatkossa
(myös tulevissa kursseissa) tyydymme toteamaan ilman todistusta, että rationaalilukujen ominaisuudet yleistyvät myös reaaliluvuille.
Emme siis pysähdy pohtimaan, mitä $\pi \cdot \pi$ täsmälleen tarkoittaa. Likiarvon voimme kyseiselle luvulle kyllä laskea
halutulla tarkkuudella:
\[ \pi\cdot \pi =9,86960440\ldots \]
Lisää reaaliluvuista liiteessä \ref{aksioomat}.

Reaalilukujen myötä lukiossa käytettävät lukujoukot on nyt kaikki esitelty. Ne ovat seuraavat:


\missingfigure{Kaavio, jossa N, Z, Q, R sisäkkäin.
Kussakin esimerkkiluku: 5, -2, 3/4, sqrt 2}


