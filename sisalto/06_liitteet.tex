% Tähän tulee liitteitä
% Esimerkiksi loogiset symbolit, reaalilukujen aksioomat, kompleksilukuintro, ...
\part{Liitteet}
% Vaihda tähän kirjaimin kulkeva "numerointi"
\chapter{Potenssien laskusääntöjen todistuksia}
\label{pot_todistukset}

Tähän tulee potenssien laskusääntöjen todistuksia, kunhan joku ne laatia.
\chapter{Logiikka ja joukko-oppi}
\chapter{Reaalilukujen aksioomat}
Reaaliluvut ovat kunta, eräs algebrallinen rakenne. Myös esimerkiksi rationaaliluvut ja seuraavassa liitteessä esiteltävät kompleksiluvut muodostavat kunnan. Sen sijaan luonnolliset luvut ja kokonaisluvut eivät ole kuntia.

Reaalilukujen aksiomaattinen määritelmä muodostuu kolmesta osasta:

Aksioomat on tehty, mutta ne ovat toistaiseksi piilossa, koska purkkaviritys (by NVI) vaatii mathtools-paketin.

% Pitäisikö 'kunta-aksioomat' erottaa 'kunta-aksioomista reaalilukujen tapauksessa'

% Toimii nyt, jos mathtools-paketti vaaditaan. Kaikilla ei sitä ole, joten kommentoitu väliaikaisesti pois.
%\begin{flalign*}
%&\textbf{Kunta-aksioomat} &\\
%&\textbf{K1.} \, \forall x, y \in \mathbb{R}: & &x+(y+z) = (x+y)+z & &| \, \text{summan liitäntälaki} &\\
%&\textbf{K2.} \, \exists 0 \in \mathbb{R}: & &x+0 = x & &| \, \text{summan neutraalialkio} &\\
%&\textbf{K3.} \, \forall x \in \mathbb{R} & &\exists (-x) \in \mathbb{R}: \quad x+(-x)=0 & &| \, \text{vasta-alkio} &\\
%&\textbf{K4.} \, \forall x, y \in \mathbb{R}: & &x+y = y+x & &| \, \text{summan vaihdantalaki} &\\
%&\textbf{K5.} \, \forall x, y, z \in \mathbb{R}: & &x \cdot (y+z) = x \cdot y + x \cdot z & &| \, \text{osittelulaki} &\\
%&\textbf{K6.} \, \forall x, y, z \in \mathbb{R}: & &x \cdot (y \cdot z) = (x \cdot y) \cdot z & &| \, \text{tulon liitäntälaki} &\\
%&\textbf{K7.} \, \exists 1 \in \mathbb{R}: & &1 \cdot x = x & &| \, \text{tulon neutraalialkio} &\\
%&\textbf{K8.} \, \forall x \in \mathbb{R} \setminus \{0\} & &\exists x^{-1} \in \mathbb{R} \setminus \{0\}: \quad x \cdot x^{-1}=1 & &| \, \text{tulon käänteisalkio} &\\
%&\textbf{K9.} \, \forall x, y \in \mathbb{R}: & &x \cdot y = y \cdot x & &| \, \text{tulon vaihdantalaki} \\
%&\textbf{Järjestysaksioomat} &\\
%&\textbf{J1.} \, \forall x, y \in \mathbb{R}: & &\text{täsmälleen yksi seuraavista:} & \\
%& & &(x > y), \, (x = y), \, (x < y) & &\\
%&\textbf{J2.} \, \forall x, y, z \in \mathbb{R}: & &(x < y) \land (y < z) \Rightarrow (x < z) & &\\
%&\textbf{J3.} \, \forall x, y, z \in \mathbb{R}: & &(x < y) \Leftrightarrow (x + z < y + z) & &\\
%&\textbf{J4.} \, \forall x, y \in ]0,\infty[: & &x \cdot y \in ]0,\infty[ & &\\
%&\textbf{Täydellisyysaksiooma} &\\
%\shortintertext{\textbf{T1.} Jokaisella ylhäältä rajoitetulla epätyhjällä reaalilukujen osajoukolla on pienin yläraja.} 
%\end{flalign*}

\begin{tehtava}
Todista aksioomista lähtien:
\begin{enumerate}[(1)]
\item $\forall x \in \mathbb{R}: 0 \cdot x = 0$
\item $\forall x \in \mathbb{R}: -1 \cdot x = -x$
\item $\forall x, y \in ]-\infty,0[: x \cdot y \in ]0,\infty[$
% lisää
\end{enumerate}
\begin{vastaus}
\begin{enumerate}[(1)]
% ???
\end{enumerate}
\end{vastaus}
\end{tehtava}

% Hanatehtävä on nyt tässä:
\begin{tehtava}
Kylpyhuoneessa on kolme hanaa. Hana A täyttää kylpyammeen 60 minuutissa, hana B 30 minuutissa ja hana C 15 minuutissa. Kuinka kauan kylpyammeen täyttymisessä kestää, jos kaikki hanat ovat yhtäaikaa auki?
\begin{vastaus}
$8$ min $34$ s
\end{vastaus}
\end{tehtava}

\Closesolutionfile{ans} % Answers-ratkaisut
\chapter{Ratkaisut}
\begin{Vastaus}{1}
Esimerkkivastaus.
\end{Vastaus}
\begin{Vastaus}{2}
$3$
\end{Vastaus}
\begin{Vastaus}{3}
$x=-1$
\end{Vastaus}
\begin{Vastaus}{4}
\begin{enumerate}
\item
\item
\item

\end{enumerate}
\end{Vastaus}
\begin{Vastaus}{6}
        a) $\frac{18}{5}$
        b) $\frac{5}{8}$
        c) $3$
        d) $-\frac{41}{6}$
    
\end{Vastaus}
\begin{Vastaus}{7}
        a) $\frac{2}{5}$
        b) $\frac{5}{6}$
        c) $\frac{1}{2}$
        d) $\frac{1}{8}$
    
\end{Vastaus}
\begin{Vastaus}{8}
        a) $\frac{47}{28}$
        b) $\frac{1}{2}$
        c) $-\frac{1}{3}$
        d) $\frac{54}{5}$
    
\end{Vastaus}
\begin{Vastaus}{9}
        Muut saavat piirakasta kuudesosan.
    
\end{Vastaus}
\begin{Vastaus}{10}
        37,50 euroa
    
\end{Vastaus}
\begin{Vastaus}{11}
a) $a^3$ \qquad b) $a^3b^4$ \qquad c) $a^4b^3$
\end{Vastaus}
\begin{Vastaus}{12}
a) $a^5$ \qquad b) $a^5$ \qquad c) $a^3$ \qquad d) $a^4$ \qquad e) $a^6$
\end{Vastaus}
\begin{Vastaus}{13}
a) $1$ \quad ($a\neq0$, koska $0^0$ ei ole määritelty) \qquad b) $1$ \qquad c) $a$ \qquad d) $a^2$ \qquad
e) $a$
\end{Vastaus}
\begin{Vastaus}{14}
a) $ a^4$ \qquad b) $a^4$ \qquad c) $a^5b$ \qquad d) $a^2b^2$
\end{Vastaus}
\begin{Vastaus}{15}
a) $ -8$ \qquad b) $1$
\end{Vastaus}
\begin{Vastaus}{16}
a) $a^2$ \qquad b) $-a^2b$ \qquad c) $a^4$
\end{Vastaus}
\begin{Vastaus}{17}
a) $a^2$ \qquad b) $-a^2b$ \qquad c) $a^4$
\end{Vastaus}
\begin{Vastaus}{18}
a) $a^5$ \qquad b) $a^2b$ \qquad c) $-a^4$
\end{Vastaus}
\begin{Vastaus}{19}
a) $64$ \qquad b) $64$ \qquad c) $64$ \qquad d) $64$
\end{Vastaus}
\begin{Vastaus}{20}
a) $1$ \qquad b) $3$ \qquad c) $32$ \qquad d) $4$ \qquad e) $1$
\end{Vastaus}
\begin{Vastaus}{21}
a) $a^3$ \qquad b) $a^{12}$ \qquad c) $a^8$ \qquad d) $a^3$ \qquad e) $1$
\end{Vastaus}
\begin{Vastaus}{22}
a) $a^{10}$ \qquad b) $a^6$ \qquad c) $a^{20}$ \qquad d) $a^5$
\end{Vastaus}
\begin{Vastaus}{23}
a) $a^3$ \qquad b) $4a^2$ \qquad c) $-8a^3b^3c^3$ \qquad d) $91a^4$
\end{Vastaus}
\begin{Vastaus}{24}
a) $a^8b^2$ \qquad b) $a^2b^6$ \qquad c) $a^{15}b^{12}$ \qquad d) $a^2b^3$
\end{Vastaus}
\begin{Vastaus}{25}
a) $a^2b^3$ \qquad b) $1$ \qquad c) $a^{17}$ \qquad d) $a^8b^{10}$
\end{Vastaus}
\begin{Vastaus}{26}
a) $16$ \qquad b) $64$ \qquad c) $16a^8$ \qquad d) $27b^4$
\end{Vastaus}
\begin{Vastaus}{27}
a) $a^6b^4$ \qquad b) $a^9b^{12}$ \qquad c) $a^{20}b^{10}$ \qquad d) $8a^3b^7$
\end{Vastaus}
\begin{Vastaus}{28}
a) $2$ \qquad b) $4$ \qquad c) $4$ \qquad d) $8$ \qquad e) $\frac{1}{2}$ \qquad f) $\frac{1}{4}$
\end{Vastaus}
\begin{Vastaus}{29}
a) $a$ \qquad b) $a^2$ \qquad c) $a^2$ \qquad d) $a^3$ \qquad e) $a^{-1} = \frac{1}{a}$ \qquad f) $a^{-2} = \frac{1}{a^2}$
\end{Vastaus}
\begin{Vastaus}{30}
a) $ab$ \qquad b) $b$ \qquad c) $1$ \qquad d) $1$ \qquad e) $-\frac{a}{b}$
\end{Vastaus}
\begin{Vastaus}{31}
a) $\frac{1}{a^3}$ \qquad b) $\frac{1}{a^2}$ \qquad c) $a^3$ \qquad d) $\frac{}{a^4b^3}$ \qquad e) $a^8$
\end{Vastaus}
\begin{Vastaus}{32}
a) $\frac{1}{4}$ \qquad b) $\frac{1}{27}$ \qquad c) $\frac{a^4}{b^4}$ \qquad d) $\frac{a^4}{b^6}$ \qquad e) $\frac{a^2}{b^4}$
\end{Vastaus}
\begin{Vastaus}{33}
a) $\frac{1}{4}$ \qquad b) $-\frac{b^3}{a^3}$ \qquad c) $a^8b^8$ \qquad d) $\frac{a^8}{b^8}$
\end{Vastaus}
\begin{Vastaus}{34}
a) $2a^7$ \qquad b) $b^9$ \qquad c) $-\frac{1}{3}a = -\frac{a}{3}$ \qquad d) $\frac{10\ 000}{b^6}$
\end{Vastaus}
\begin{Vastaus}{35}
\begin{enumerate}
\item $2x^2$
\item $3+4y$
\item $ -\frac{x}{6}$
\item $ \frac{2}{3} x - \frac{17}{6}$
\end{enumerate}
\end{Vastaus}


\chapter{Tekijät}

Lauri Hellsten\\
Niko Ilomäki\\
Tero Keinänen\\
Vesa Linja-aho\\
Ossi Mauno\\
Joonas Mäkinen\\
Matti Pajunen\\
Pekka Peura\\
Annika Piiroinen\\
Kaisa Pohjonen\\
Juha Sointu\\
Tommi Sottinen\\
Topi Talvitie\\
Sampo Tiensuu\\
Ville Tilvis