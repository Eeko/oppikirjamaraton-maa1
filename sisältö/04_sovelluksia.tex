%
\part{Sovelluksia}
%
%(Pythagoraan lause)
%
\chapter{Verrannollisuus}
\chapter{Verrannollisuus: sovelluksia}
\chapter{Prosenttilaskentaa - perustilanteet}
Kun lukuja verrataan toisiinsa, voidaan laskea niiden suhde eli osamäärä. Tämä kertoo jaettavan suhteellisen osuuden jakajasta. Suhteellinen osuus ilmaistaan usein prosentteina. Yksi prosentti tarkoittaa yhtä sadasosaa. %Prosentin merkki on \%.

%1 prosentti $= 1 \% = \frac{1}{100} = 0,01$

%Leivän suolapitoisuus on $1,5 \%$. Siis 100 grammassa leipää on 1,5 grammaa suolaa.

%Pekka maksaa palkastaan veroa $20 \%$. Siis 100 eurosta palkkaa hän maksaa 20 euroa veroa.

%Työttömyysaste on $7 \%$. Siis jokaisesta 100 työikäisestä 7 on työttömänä.

\section{Peruslaskut}

\subsection{Prosenttiosuuden laskeminen}

Kuinka monta prosentti

\subsection{Prosenttiarvon laskeminen}

\subsection{Perusarvon laskeminen}

\chapter{Prosenttiyhtälöitä ja sovelluksia}
%
%(Eksponenttiaalinen malli)
%
\chapter{Kertaustiivistelmä}
