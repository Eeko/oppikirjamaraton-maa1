\chapter{Laskusäännöt ja lausekkeiden sieventäminen}

\section{laskujärjestys}

\laatikko{
    \begin{enumerate}
        \item Sulut
        \item Potenssilaskut
        \item Kerto- ja jakolaskut vasemmalta oikealle
        \item Yhteen- ja jakolaskut vasemmalta oikealle
    \end{enumerate}
}

\section{lausekkeiden sieventäminen}

Matemaattisia ongelmia ratkaistaessa kannattaa usein etsiä vaihtoehtoisia tapoja jonkin laskutoimituksen, lausekkeen tai luvun ilmaisemiseksi. Tällöin usein korvataan esimerkiksi jokin laskutoimitus toisella laskutoimituksella, josta tulee sama tulos. Näin lauseke saadaan sellaiseen muotoon, jonka avulla ratkaisussa päästään eteenpäin.

Matematiikassa on tapana ajatella niin, että saman luvun voi kirjoittaa monella eri tavalla. Esimerkiksi merkinnät $42$, $-(-42)$, $6\cdot 7$ ja $(50-29)\cdot 2$ tarkoittavat kaikki samaa lukua. Niinpä missä tahansa lausekkeessa voi luvun $42$ paikale kirjoittaa merkinnän $(50-29)\cdot 2$, sillä ne tarkoittavat samaa lukua. Tähän lukuun on koottu sääntöjä, joiden avulla laskutoimituksia voi vaihtaa niin, että lopputulos ei muutu.

\laatikko{
Yhteenlaskut voi laskea missä järjestyksessä tahansa

$a+b=b+a$ (vaihdantalaki)

$a+(b+c)=(a+b)+c=a+b+c$ (liitäntälaki)
}

Esimerkiksi laskemalla voidaan tarkistaa, että $5+7=7+5$ ja että $(2+3)+5=2+(3+5)$.

Nämä säännöt voi yhdistää yleiseksi säännöksi, jonka mukaan laskujärjestystä voi vaihtaa ihan miten vain niin kauan kuin lausekkeessa on pelkkää yhteenlaskua.

Tämän säännön voi yleistää koskemaan myös vähennyslaskua, kun muistetaan, että vähennyslasku tarkoittaa oikeastaan käänteisluvun lisäämistä. $5-8$ tarkoittaa siis samaa kuin $5+(-8)$, joka voidaan nyt kirjoittaa yhteenlaskun vaihdantalain perusteella muotoon $(-8)+5$ eli $-8+5$ ilman, että laskun lopputulos muuttuu. Tästä seuraa seuraava sääntö:

\laatikko{
Pelkästään yhteen- ja vähennyslaskua sisältävässä lausekkeessa laskujärjestystä voi vaihtaa vapaasi, kun ajattelee miinusmerkin liikkuvan kuuluvan sitä seuraavaan lukuun ja liikkuvan sen mukana.
}

Esim. $5-8+7-2=5+(-8)+7+(-2)=(-2)+(-8)+5+7=-2-8+5+7$

Vastaavat säännöt pätevät kerto- ja jakolaskulle samoista syistä.

\laatikko{
Kertolaskut voi laskea missä järjestyksessä tahansa

$a\cdot b=b\cdot a$ (vaihdantalaki)

$a\cdot (b\cdot c)=(a\cdot b)\cdot c=a\cdot b\cdot c$ (liitäntälaki)
}

Jakolaskun voi ajatella käänteisluvulla kertomisena, eli

\laatikko{
Pelkästään kerto- ja jakolaskua sisältävässä lausekkeessa laskujärjestystä voi vaihtaa vapaasti, kun ajattelee jakolaskun käänteisluvulla kertomisena.
}

Esim. $5:8\cdot 7:2=5\cdot\frac18\cdot 7\cdot\frac12=7\cdot \frac12\cdot\frac18\cdot 5=7:2:8\cdot 5$

$a\cdot b:b=a$

$a:b\cdot b=a$

$a:b\cdot b=a$

