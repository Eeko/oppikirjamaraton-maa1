\section{Neliöjuuri}

Ajatellaan, että neliön pinta-ala on $a$. Halutaan tietää, mikä on kyseisen neliön sivun pituus. Vastausta tähän kysymykseen kutsutaan luvun $a\ge 0$ neliöjuureksi ja merkitään $\sqrt{a}$. Luvun $a$ neliöjuuri on myös yhtälön $x^2 = a$ vastaus. Tällöin täytyy kuitenkin huomata, että myös luku $x=-\sqrt{a}$ toteutaa kyseisen yhtälön. Neliöjuurella tarkoitetaan kyseisen yhtälön epänegatiivista vastausta. Tämä on luonnollista, koska neliön sivun pituus ei voi olla negatiivinen luku.

\laatikko{Luvun $a$ neliöjuuri on epänegatiivinen luku, jonka neliö on $a$. Tämä voidaan ilmaista myös $(\sqrt{a})^2=a$.}

Neliöjuurta ei tällä kurssilla määritellä negatiivisille luvuille, koska neliön pinta-ala on aina positiivinen luku tai nolla. Käytännössä lukujen neliöjuuria lasketaan laskimella.


\begin{esimerkki}
Laske.
\begin{enumerate}[a)]
\item $\sqrt{4}$

\item $\sqrt{144}$

\item $\sqrt{3471}$.
\end{enumerate}

{\bf Ratkaisut.}

a)
Laskimella tai päässä laskemalla nähdään, että $\sqrt{4} = 2$, koska $2>0$ ja $2^2 =4$.

b) 
Laskimella saadaan $\sqrt{144}=12$. Tämä voidaan vielä tarkistaa laskemalla $12^2 = 12\cdot 12=144$.

c)
Laskimella saadaan $\sqrt{4471}\approx 66,9$. Vertailun vuoksi laskimella saadaan myös $67\cdot 67=4489$.

{\bf Vastaukset.}
a) $2$, b) $12$, c) $66,9$.

\end{esimerkki}

\begin{esimerkki}
Taulutelevision kooksi (halkaisijaksi) on ilmoitettu mainoksessa $46,0$ tuumaa ($116,8$ cm) ja kuvasuhteeksi 16:9. Kuinka leveä televisio on?

{\bf Ratkaisu.}

Taulutelevision halkaisija, alareuna ja toinen sivu muodostavat suorakulmaisen kolmion. Kolmion hypotenuusa on television halkaisija ja kateetit alareuna ja toinen sivu.

Kuvausuhteen perusteella kateettien pituuksia voidaan merkitä $16x$ ja $9x$. Pythagoraan lauseesta saadaan
\[
(116,8)^2 = (16x)^2 + (9x)^2
\]
eli
\[
13642,24 = (256+81)x^2.
\]
Siten
\[
x^2 = \frac{13642,24}{337}
\]
ja siis
\[
x= \sqrt{\frac{13642,24}{337}} \approx 6,36.
\]
Television leveys on noin $16x = 16\cdot 6,36\approx 102$ cm.

{\bf Vastaus.} Noin $102$ cm.


%$40,7$" \, ($103,4$ cm). Huom. Sain hieman eri tuloksen.
\end{esimerkki}


%Antti, lisää tämä
\begin{tehtava}
Laske seuraavat neliöjuuret laskimella:\\ 
a) $\sqrt{320},\ b) $\sqrt{15}$,\ c) $\sqrt{71}$
\begin{vastaus}
a) $17{,}89$ b) $3{,}87$ c) $8{,}43$
\end{vastaus}
\end{tehtava}


%Antti, lisää tämä 
\begin{tehtava}
Laske $\sqrt{8100}$ päässä ajattelemalla juurrettava luku sopivana tulona.
\begin{vastaus}
$\sqrt{8100}=\sqrt{81}\sqrt{100}=9\cdot 10$.
\end{vastaus}
\end{tehtava}

%Antti, lisää tämä
\begin{tehtava}
Etsi luku $a>0$ jolle $a^4=83521$.
\begin{vastaus}
$a=\sqrt{\sqrt{83521}}$.
\end{vastaus}
\end{tehtava}

%Antti, lisää tämä
\begin{tehtava}
Oletetaan että suorakaiteen leveyden suhde korkeuteen on $2$ ja suorakaiteen pinta-ala on $10$. Mikä on suorakaiteen 
leveys ja korkeus?
\begin{vastaus}
Suorakaide muodostuu kahdesta vierekkäisestä neliöstä, joiden pinta-ala on $5$. Tämän neliön sivun pituus on $\sqrt{5}$.
Siis suorakaiteen korkeus on $\sqrt{5}$ ja leveys $2\sqrt{5}$.
\end{vastaus}
\end{tehtava}

\begin{tehtava}
Ajatellaan suorakulmaista hiekkakenttää, jonka pinta-ala on aari ($100^2 m^2$). Lyhyemmän ja pidemmän sivujen 
pituuksien suhde on $4:3$. Laske pythagoraan lauseen avulla matka hiekkakentän kulmasta kauimmaisena olevaan kulmaan.
\begin{vastaus}
$\frac{4}{3}x^2=100$ joten $x = \sqrt{\frac{300}{4}}$. 
Hypotenuusa: $\sqrt{x^2 + (\frac{4}{3}x)^2}=\sqrt{\frac{300}{4}+\frac{16}{9}\cdot \frac{300}{4}}
=\frac{5}{3}\sqrt{\frac{300}{4}}=14{,}43$.
\end{vastaus}
\end{tehtava}

