\chapter{Yhtälö}
Monissa käytännön tilanteissa saamme samalle asialle kaksi erilaista esitystapaa.

\begin{esimerkki}
Meillä on orsivaaka, joka on tasapainossa. (kuva!) Toisessa vaakakupissa on kahden kilon siika ja toisessa puolen kilon ahven sekä tuntematon määrä lakritsia. Kuinka paljon vaakakupissa on lakritsia? (Ratkaistaan...) (Muita esimerkkejä, vähitellen vaikeutuvia (1. asteen) yhtälöitä)
\end{esimerkki}

\missingfigure{Kuva kaloista vaa'assa}

\laatikko{
\emph{Yhtälöksi} kutsutaan kahden lausekkeen merkittyä yhtäsuuruutta. Siis mielivaltaisille lausekkeille $A$ ja $B$ merkitään $A=B$. (Esim. $A=5x+\sqrt{x}$ ja $B=7x+7$). Jos yhtälön puolien lausekkeiden arvot ovat samat, sanotaan että \emph{yhtälö pätee}.
}

Yhtälössä voi esiintyä \emph{muuttujia}, eli symboleja joiden arvoa ei ole etukäteen määrätty. Muuttujia merkitään usein kirjaimilla $x$, $y$ ja $z$. Niitä muuttujien arvoja, joilla yhtälö pätee, kutsutaan \emph{yhtälön ratkaisuiksi}. Yhtälön ratkaisemisella tarkoitetaan kaikkien yhtälön ratkaisujen selvittämistä.

\laatikko{
Eräs tapa ratkaista yhtälöitä on muokata niitä niin, että muokattu yhtälö pätee täsmälleen silloin kun alkuperäinen yhtälö pätee. Tällaisia sallittuja muunnoksia ovat esimerkiksi:
\begin{itemize}
\item Yhtälön molemmat puolet voidaan kertoa nollasta poikkeavalla luvulla $m$. Muutos tehdään aina molemmille puolille. Tällöin saadaan yhtälö $mA = mB$.
\item Yhtälön molemmille puolille voidaan lisätä tai molemmilta puolilta vähentää luku $n$. Tällöin saadaan yhtälö $A+n = B+n$.
\end{itemize}
}

\missingfigure{Kuva orvivaa'asta, jossa on myös heliumpallo}
% Nuo pitää ehkä perustella.

Monet yhtälöt ratkeavat toistamalla tällaisia muunnoksia kunnes yhtälö on niin yksinkertaisessa muodossa, että ratkaisu on helppo nähdä. Koska jokaisessa muokkausjonon yhtälössä ratkaisut ovat samat, näin saadaan alkuperäisen yhtälön ratkaisut.

%esimerkki tulee 1. asteen yhtälön yhteydessä

\laatikko{
Yhtälöt voidaan ratkaisujensa perusteella jakaa kolmeen tyyppiin:
\begin{enumerate}
\item Yhtälö, joka on aina tosi. Esimerkiksi yhtälöt $8=8$ ja $x=x$.
\item Yhtälö, joka on joskus tosi. Esimerkiksi yhtälö $x+4=7$ on tosi jos ja vain jos $x=3$.
\item Yhtälö, joka ei ole koskaan tosi. Esimerkiksi yhtälö $0=1$.
\end{enumerate}
}

Tämän kurssin ja ylipäätään matematiikan kannalta selvästi tärkein yhtälötyyppi on 2. Siirrymme nyt tarkastelemaan tärkeää erikoistapausta yhtälöistä, ensimmäisen asteen yhtälöitä.

