\chapter{Potenssisäännöt}

Potenssilla $2^4$ tarkoitetaan tuloa $2\cdot 2\cdot 2\cdot 2$.
\begin{equation}
\text{Eli} 2^4=2\cdot 2\cdot 2\cdot 2=16.
\end{equation}
Lausekkeessa $2^4$ luku 2 on \textbf{kantaluku} ja luku 4 on \textbf{eksponentti}.

\begin{esimerkki}
\textbf{Esimerkki 1}
\begin{equation}
\text{a)} (-2)^3=(-2)\cdot (-2)\cdot (-2)=-8
\end{equation}

\begin{equation}
\text{b)} (-2)^4=(-2)\cdot (-2)\cdot (-2)\cdot (-2)=16
\end{equation}

\begin{equation}
\text{c)} -2^4=-2\cdot 2\cdot 2\cdot 2=-16
\end{equation}

\begin{equation}
\text{d)} 2^2\cdot 2^3=\underbrace{2\cdot 2}_{2 kpl}\cdot \underbrace{2\cdot 2\cdot 2}_{3 kpl}=2^5=32
\end{equation}

\begin{equation}
\text{e)}\frac{2^4\cdot 2^2}{2^3}=\frac{\overbrace{2\cdot 2\cdot 2\cdot \cancel{2}}\cdot \overbrace{\cancel{2}\cdot \cancel{2}}}{\cancel{2}\cdot \cancel{2}\cdot \cancel{2}}
\end{equation}
\end{esimerkki}

