\chapter{Numerot ja luvut}

Ihmisillä ja eläimillä on luonnostaan matemaattisia taitoja. Monet niistä, esimerkiksi lukumäärien laskeminen, ovat monimutkaisia ajatusprosesseja, jotka kehittyvät lapsuudessa – toisilla aiemmin, toisilla myöhemmin. Koulussa opeteltava peruslaskento ja myös matematiikka tieteenalana rakentuvat tämän biologisen osaamisen päälle. Laskeminen itsessään on vain eräs matematiikan osa-alue, eikä kaikki matematiikka ole laskemista.

\sivulaatikko{Huom.! laskea (lukumäärä) englanti count ruotsi \_ /n
laskea (laskutoimitus) englanti calculate , ruotsi \_}

Olennaisena kehitysaskeleena niin yksilön matemaattiselle ajattelulle kuin yhteiskunnallekin on ollut luonnollisen kielen tavoin kyky merkitä lukumäärien laskemista ja muuta matemaattista pohdintaa kirjalliseen muotoon. On olemassa
useita erilaisia tapoja merkitä lukumääriä. Helpoin ja yksinkertaisin tapa on käyttää vain yhtä merkkiä ja toistaa sitä:

\missingfigure{Piirrettynä "tukkimiehen kirjanpitoa" ja  vertaus arabialaisilla numeroilla}

Käytettävissä olevien merkkien määrää voidaan lisätä, jolloin suuria lukuja voidaan kirjoittaa lyhyemmin. Tämä vastaa myös luonnollisten kielten tilannetta: suomen kielen aakkosiin kuuluu 29 kirjainta, joista sanat muodostetaan. Sanat voivat olla periaatteessa kuinka pitkiä hyvänsä. Esimerkiksi mandariinikiinassa sen sijaan käytetään jokaiselle sanalle omaa piirrosmerkkiään. Merkkejä täytyy osata 29:n sijaan tuhansia, mutta jokaisen sanan voi kirjoittaa lyhyesti.

Matematiikassa erilaisista numeromerkeistä tai yksinkertaisesti numeroista muodostetaan lukuja yhdistelemällä niitä sopivasti erilaisten paikkajärjestelmien mukaan. Esimerkiksi antiikin Roomassa käytössä olivat numeromerkit I, V, X, L, C , D ja M. Niiden numeroiden vastaavuudet meidän käyttämiimme lukuarvoihin ovat seuraavat:
\begin{equation*}
\rm I=1\quad
V=5\quad
X=10\quad
L=50\quad
C=100\quad
D=500\quad
M=1 000
\end{equation*}
Huomaa, että suuri osa roomalaisista numeromerkeistä ovat jo itsessään arvoltaan niin suuria, että tarvitsemme niiden nykyilmaisuun monta merkkiä! Nollaa roomalaisissa numeroissa ei ole, ja tiettävästi tuhatta suurempia arvoja esittäviä numeromerkkejä merkkejä otettiin käyttöön vasta keskiajalla. 
Lukuja koostetaan näistä merkeistä siten, että merkit kirjoitetaan peräkkäin pääasiassa laskevassa järjestyksessä ja niiden numeroarvot lasketaan yhteen. Jos arvoltaan pienempi numeromerkki (korkeintaan yksi) edeltää suurempaa, pienempi vähennetään suuremmasta ennen yhteenlaskun jatkamista. 

Luvut $134$ ja $413$ eivät ole sama luku; saamme eri lukuja, kun numeroita yhdistellään eri tavoin.

\begin{esimerkki}
III=1+1+1=3
IX=10-1=9
XII=10+1+1=12
XIX=10-1+10=19
CDX=500-100+10=410
MCMD=1 000+1 000-100+500
\end{esimerkki}


\laatikko{Länsimaisessa traditiossa käytössämme on kymmenen numeromerkkiä: 0, 1, 2, 3, 4, 5, 6, 7, 8 ja 9. Näitä kutsutaan alkuperänsä mukaan hindu-arabialaisiksi numeroiksi.}

Yksittäisellä numeromerkillä ei kuitenkaan ole vielä matematiikassa tarvittavaa lukuarvoa, vaan luvut rakennetaan yhdistelemällä numeroita.

\begin{esimerkki}
Luku \[715531\] koostuu numeroista 7, 1, 5, 5, 3 ja 1.
\end{esimerkki}



\sivulaatikko{Englannin kielen sana \textit{number} voi viitata sekä numeroon että lukuun. Sana \textit{digit} tarkoittaa pelkästään yhtä numeromerkkiä. Ruotsiksi luku on \textit{tal}, lukumäärä \textit{antal} ja numeroa tai lukumäärää tarkoittamatonta numeroyhdistelmää kuvaa suomen kielen tapaan sana \textit{nummer}.}



postinumero, puhelinnumero!


\begin{esimerkki}
Selitys kymmenjärjestelmästä, kymmenet, sadat, tuhannet, ...
\[20661,43\]
\end{esimerkki}

\sivulaatikko{Kielioppihuomautuksia: 1) Tuhaterottimena käytetään välilyöntiä, ei pilkkua tai pistettä. 2) Suomessa on käytössä desimaalipilkku, ei -piste! Yhdysvaltalaiset ovat suunnitelleet laskimesi.}


\section{Jos tämä on matematiikkaa, miksi käytämme kirjaimia?}

Numeroita kuvaavat merkit ovat mielivaltaisia symboleita. Lukujakin edustamaan päädytään joskus käyttämään jotakin lyhennysmerkintää

yleisyys

suuruus, yhtäsuuruusmerkintä, eri suuret

$x=2$
-> $2=x$
symbolit, muuttujat, alkiot, kreikkalaisia...

\sivulaatikko{Painotekstissä kirjaimella merkityt muuttujat kirjoitetaan \textit{kursiivilla} ja aina saman arvon saavat matemaattiset vakiot kuten $\pi$ pystyyn.}

\section{Paikkalukujärjestelmät}





Erilaisilla luvuilla voidaan suorittaa erilaisia laskutoimituksia. Seuraavissa luvuissa esitellään ja käydään läpi lukiomatematiikassa ja mahdollisissa jatko-opinnoissa käytettäviä lukujoukkoja ja tavallisimmat laskutoimitukset.
