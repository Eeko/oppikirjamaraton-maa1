\chapter{Prosenttilaskentaa - perustilanteet}

Sana prosentti tulee latinan kielen sanoista pro centum, mikä tarkoittaa
kirjaimellisesti sataa kohden. Prosentteja käytetään ilmaisemaan suhteellista
osuutta. Lukua, josta suhde lasketaan, kutsutaan \emph{perusarvoksi}. Prosentin
merkki on \%. Esimerkiksi jos sadan euron hintaisesta tuotteesta on alennettu 25
prosenttia, niin tuotteen alennettu hinta on 75 euroa. Jos sen sijaan alkuperäinen
hinta nousee 15 prosenttia, niin tuotteen uusi hinta on 115 euroa. Perusarvo on
molemmissa tapauksissa 100 euroa.

\laatikko{1 prosentti $= 1 \% = \frac{1}{100} = 0,01$}

\laatikko{Esimerkki: \\$6 \% = \frac{6}{100} = 0,06$, $48,2 \% = \frac{48}{100} = 0,482$, $140 \% = \frac{140}{100} = 1,40$}

Suhdeluku muutetaan prosenteiksi kertomalla se luvulla 100 ja lisäämällä
lopputuloksen jälkeen prosenttimerkki.

\begin{esimerkki}
Vesa ansaitsee kuukaudessa 2300 euroa ja Antero 1700 euroa.
    Kuinka monta prosenttia Anteron tulot ovat Vesan tuloista? 
    
    {\textbf Ratkaisu.}
    
    Lasketaan
    \[
    \frac{1700}{2300} \cdot 100 \% \approx 0,739\cdot 100 \% = 73,9 \%.
    \]
    Laskuissa käytettävä perusarvo on Vesan palkka eli 2300 euruoa.
    
    {\textbf Vastaus.}
     $73,9 \%$
\end{esimerkki}


Vertailuprosentilla ilmaistaan, kuinka paljon toinen luku on suurempi kuin toinen. Vertailukohteena käytetään aina sitä lukua, johon verrataan. Jos siis halutaan tietää, kuinka monta prosenttia luku $a$ on suurempi kuin $b$, vertailuprosentti saadaan laskettua kaavalla
\[
\frac{a-b}{b} \cdot 100 \%.
\]

\begin{esimerkki}
    Vesa ansaitsee kuukaudessa 2300 euroa ja Antero 1700 euroa.
    Kuinka monta prosenttia enemmän Vesa ansaitsee kuin Antero?
    
    {\textbf Ratkaisu.}
    
    Lasketaan aluksi Vesan ja Anteron palkkojen erotus
    \[
    2300-1700 = 600.
    \]
    Sitten lasketaan kuinka monta prosenttia 600 euroa on Anteron palkasta:
    \[
    \frac{600}{1700} \cdot 100 \% \approx 0,353\cdot 100\% = 35,3 \%.
    \]
    
    {\textbf Vastaus.}
    $35,3 \%$
\end{esimerkki}

Prosentteja käytetään usein ilmaisemaan suureiden muutoksia. Muutosprosenttia laskettaessa perusarvona on alkuperäinen arvo, johon nähden muutos on tapahtunut.

\begin{esimerkki}
    Vesan paino on tammikuussa 68 kg ja kesäkuussa 64 kg. Kuinka monta prosenttia Vesa on laihtunut?

    {\textbf Ratkaisu.}

    Lasketaan 
    \[
    \frac{68-64}{68}\cdot 100\% = \frac{4}{68} \cdot 100\%=0,059\cdot 100\% =5,9\%.
    \]
    
    {\textbf Vastaus.}
    Vesa on laihtunut $5,9\%$.
\end{esimerkki}


Prosenttiyksikkö mittaa prosenttiosuuksien välisiä eroja. Jos prosenttiluku muuttuu, muutos voidaan ilmaista joko prosentteina tai prosenttiyksikköinä.


\begin{esimerkki}
    Tuotteen markkinaosuus on vuoden tammikuussa 10 \% ja kesäkuussa 15 \%. 
    \begin{enumerate}
    \item[a)]
    Kuinka monta prosenttia tuotteen markkinaosuus on noussut?
    
    \item[b)] Kuinka monta prosenttiyksikköä tuotteen markkinaosuus on noussut?
    \end{enumerate}
    
    {\textbf Ratkaisu.} 
    
    a) Tuotteen markkinaosuus on noussut
    \[
    \frac{15-10}{10} \cdot 100 \%= \frac{5}{10}\cdot 100\% = 50\%.
    \]
    
    b) Tuotteen markkinaosuus on noussut $15-10=5$ prosenttiyksikköä. 
    
    {\textbf Vastaus.}
    
    a) 50 prosenttia, b) 5 prosenttiyksíkköä.
\end{esimerkki}




\section{Perusprosenttilaskut}

\begin{itemize}
	\item Prosenttiluvun laskeminen
	\item Prosenttiarvon laskeminen
	\item Perusarvon laskeminen
\end{itemize}

\section{Vertailu prosenttien avulla}

\begin{itemize}
	\item Muutosprosentti, vertailuprosentti
	\item Prosentuaalinen muutos
	\item Prosenttiyksikkö
\end{itemize}
