%
\part{Sovelluksia}
%
%(Pythagoraan lause)
%
\chapter{Verrannollisuus}
\chapter{Verrannollisuus: sovelluksia}

% Lyhyt matikka 1, s. 72
Pohdi, kuinka toinen suure muuttuu, kun toinen suure kaksinkertaistuu, kolminkertaistuu, puolittuu jne. Ovatko suureet suoraan verrannolliset?
a) kuljettu matka ja kulunut aika, kun keskinopeus on 30 km/h
b) kananmunien lukumäärä ja niiden kovaksi keittämiseen tarvittava keittoaika
c) hedelmätiskiltä valitun vesimelonin paino ja hinta
d) neliön sivun pituus ja neliön pinta-ala
Vastaus:
a) Ovat.
b) Eivät ole.
c) Ovat.
d) Eivät ole, sillä esimerkiksi kun neliön sivun pituus kaksinkertaistuu 1 cm:stä 2 cm:iin, niin neliön pinta-ala nelinkertaistuu 1 cm$^2$:stä 4 cm$^2$:iin.

Isi ja lapset ovat ajamassa mökille Sotkamoon. Ollaan ajettu jo neljä viidennestä matkasta ja aikaa on kulunut kaksi tuntia. "Joko ollaan perillä?" kysyvät lapset takapenkiltä. Kuinka pitkään vielä arviolta kuluu, ennen kuin ollaan mökillä?
Vastaus: 1 h 15 min

Äidinkielen kurssilla annettiin tehtäväksi lukea eräs 300-sivuinen romaani. Eräs opiskelija otti aikaa ja selvitti lukevansa vartissa seitsemän sivua. Kuinka monta tuntia häneltä kuluu arviolta koko romaanin lukemiseen, jos taukoja ei lasketa?
Vastaus: 642 minuuttia eli 10 h 42 min.

\chapter{Prosenttilaskentaa - perustilanteet}

Kun lukuja verrataan toisiinsa, lasketaan niiden suhde eli osamäärä. Tämä kertoo jaettavan suhteellisen osuuden jakajasta. Suhteellinen osuus ilmaistaan usein prosentteina. Yksi prosentti tarkoittaa yhtä sadasosaa. Prosentin merkki on $\%$.

\laatikko{1 prosentti $= 1 \% = \frac{1}{100} = 0,01$}

\laatikko{$6 \% = \frac{6}{100} = 0,06$, $48,2 \% = \frac{48}{100} = 0,482$, $140 \% = \frac{140}{100} = 1,40$}

Minkä tahansa suhdeluvun voi muuttaa prosenteiksi laskemalla osamäärän desimaalilukuna ja ottamalla siitä sadasosat.
% Pitäisi muotoilla se, että tuhannesosat ovat sitten 0,1 prosenttia jne. eikä että katkaistaan desimaaliluku sadasosiin :)

Kahden prosenttiluvun välisen erotuksen yksikköä kutsutaan prosenttiyksiköksi.

\section{Perusprosenttilaskut}

\begin{itemize}
	\item Prosenttiluvun laskeminen
	\item Prosenttiarvon laskeminen
	\item Perusarvon laskeminen
\end{itemize}

\section{Vertailu prosenttien avulla}

\begin{itemize}
	\item Muutosprosentti, vertailuprosentti
	\item Prosentuaalinen muutos
	\item Prosenttiyksikkö
\end{itemize}

\chapter{Prosenttiyhtälöitä ja sovelluksia}

Laukku maksaa 225 \euro ja on 25\%:n alennuksessa. Paljonko alennettu hinta on?
Vastaus: 168,75 \euro

%Pyramidi 1, s. 80
Kirjan myyntihinta, joka sisältää arvolisäveron, on 8\% suurempi kuin kirjan veroton hinta. Laske kirjan veroton hinta, kun myyntihinta on 15\euro.
Vastaus: 13,89 \euro

Perussuomalaisten kannatus oli vuoden 2007 eduskuntavaaleissa 4,1\% ja vuoden 2011 eduskuntavaaleissa 19,1\%. Kuinka monta prosenttiyksikköä kannatus nousi? Kuinka monta prosenttia kannatus nousi?
Vastaus: Kannatus nousi 15 \%-yksikköä ja 365,9 \%.

Askartelukaupassa on alennusviikot, ja kaikki tavarat myydään 60\%n alennuksella. Viimeisenä päivänä kaikista hinnoista annetaan vielä lisäalennus, joka lasketaan aiemmin alennetusta hinnasta. Minkä suuruinen lisäalennus tulee antaa, jos lopullisen kokonaisalennuksen halutaan olevan 80\%?
Vastaus: 50\%.

Erään pankin myöntämä opintolaina nousee korkoa 2\% vuodessa. Kuinka monta prosenttia laina on noussut korkoa alkuperäiseen summaan verrattuna kymmenen vuoden kuluttua?
Vastaus: 22\%.

Ansiotuloverotus on Suomessa progressiivista: suuremmista tuloista maksetaan

%
%(Eksponentiaalinen malli)
%
\chapter{Kertaustiivistelmä}
