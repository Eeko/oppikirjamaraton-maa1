\section{Sovelluksia}

\begin{tehtava}
    % Lyhyt matikka 1, s. 72
    Pohdi, kuinka toinen suure muuttuu, kun toinen suure kaksinkertaistuu,
    kolminkertaistuu, puolittuu jne. Ovatko suureet suoraan verrannolliset?
    
    \begin{enumerate}
        \item kuljettu matka ja kulunut aika, kun keskinopeus on 30 km/h
        \item kananmunien lukumäärä ja niiden kovaksi keittämiseen tarvittava keittoaika
        \item hedelmätiskiltä valitun vesimelonin paino ja hinta
        \item neliön sivun pituus ja neliön pinta-ala
    \end{enumerate}
    
    \begin{vastaus}
        Vastaus:
        \begin{enumerate}
            \item Ovat.
            \item Eivät ole.
            \item Ovat.
            \item Eivät ole, sillä esimerkiksi kun neliön sivun pituus
                kaksinkertaistuu 1 cm:stä 2 cm:iin, niin neliön pinta-ala
                nelinkertaistuu 1 cm$^2$:stä 4 cm$^2$:iin.
        \end{enumerate}
    \end{vastaus}
\end{tehtava}

\begin{tehtava}
Ratkaise
\begin{enumerate}
\item $ \frac{x}{3} = 1$
\item $ \frac{8}{y} = 2$
\item $ \frac{7}{x} = \frac{16}{8}$
\item $ \frac{x}{3} = \frac{1}{7}$
\end{enumerate}
\begin{vastaus}
\begin{enumerate}
\item $x= \frac{1}{3}$
\item $y= \frac{1}{4}$
\item $x= \frac{7}{2}$
\item $x= \frac{3}{7}$
\end{enumerate}
\end{vastaus}
\end{tehtava}

\begin{tehtava}
Muodosta seuraavia tilanteita kuvaavat yhtälöt. Käytä vakion merkkinä vaikka $c$:tä.
\begin{enumerate}
\item Kultakimpaleen arvo ($x$) on suoraan verrannollinen sen massaan ($m$). Siis mitä painavampi pala kimpale, sitä enemmän siitä saa rahaa.
\item Aidan maalaamiseen osallistuvien ihmisten määrä {$x$} on kääntäen verrannollinen maalaamiseen kuluvaan aikaan ($t$). Siis mitä enemmän maalaajia, sitä nopeammin homma on valmis.
\item Planeettojen toisiinsa aiheuttama vetovoima ($F$) on suoraan verrannollinen planeettojen massoihin ($m_1$ ja $m_2$) ja kääntäen verrannollinen niiden välisen etäisyyden ($r$) neliöön.
\end{enumerate}
\begin{vastaus}
\begin{enumerate}
\item $ \frac{x}{m}=c$
\item $ xt=c $
\item $ \frac{Fr^2}{m_1+m_2}=c$
\end{enumerate}
\end{vastaus}
\end{tehtava}

\begin{tehtava}
Rento pyöräilyvauhti kaupunkiolosuhteissa on noin $20$ km/h. Lukiolta urheiluhallille on matkaa $7$ km. Kuinka monta minuuttia kestää arviolta pyöräillä lukiolta urheiluhallille?
\begin{vastaus}
Viiden minuutin tarkkuudella $20$ min.
\end{vastaus}
\end{tehtava}

\begin{tehtava}
    Isi ja lapset ovat ajamassa mökille Sotkamoon. Ollaan ajettu jo neljä
    viidennestä matkasta ja aikaa on kulunut kaksi tuntia. ''Joko ollaan perillä?''
    kysyvät lapset takapenkiltä. Kuinka pitkään vielä arviolta kuluu, ennen
    kuin ollaan mökillä?
    
    \begin{vastaus}
        Vastaus: 1 h 15 min
    \end{vastaus}
\end{tehtava}

\begin{tehtava}
    Äidinkielen kurssilla annettiin tehtäväksi lukea eräs 300-sivuinen romaani.
    Eräs opiskelija otti aikaa ja selvitti lukevansa vartissa seitsemän sivua.
    Kuinka monta tuntia häneltä kuluu arviolta koko romaanin lukemiseen, jos
    taukoja ei lasketa?
    
    \begin{vastaus}
        Vastaus: 642 minuuttia eli 10 h 42 min.
    \end{vastaus}
\end{tehtava}
