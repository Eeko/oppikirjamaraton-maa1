\chapter{Potenssi}
    
    Jos $a$ on reaaliluku ja $n$ on positiivinen kokonaisluku, potenssilla $a^n$ tarkoitetaan tuloa
    \[
        a^n = \underbrace{a\cdot \ldots \cdot a}_{n\text{ kpl}}. 
    \]
Lukua $a$ kutsutaan potenssin \emph{kantaluvuksi} ja lukua $n$ \emph{eksponentiksi}. Merkinnällä $2^4$ siis tarkoitetaan tuloa $2\cdot 2\cdot 2\cdot 2$. Siten
        \[
            2^4=2\cdot 2\cdot 2\cdot 2=16.
        \]
Luvun toista potenssia $a^2$ kutsutaan myös luvun $a$ \emph{neliöksi} ja kolmatta potenssia $a^3$ sen \emph{kuutioksi}. Luvun $a>0$ neliö on sellaisen neliön pinta-ala, jonka sivun pituus on $a$. Vastaavasti, jos kuution särmän pituus on $a$, niin $a^3$ on kyseisen kuution tilavuus.  
            
    \begin{esimerkki}
        Potenssien sieventäminen luvuilla.
        \begin{enumerate}[a)]
            \item $(-2)^3 = (-2)\cdot (-2)\cdot (-2) = -8$
            \item $(-2)^4 = (-2)\cdot (-2)\cdot (-2)\cdot (-2) = 16$
            \item $-2^4   = -(2^4) = -(2\cdot 2\cdot 2\cdot 2) = -16$
            \item $2^2\cdot 2^3 =
                \underbrace{2\cdot 2}_{\text{$2$ kpl}}\cdot \underbrace{2\cdot
                2\cdot 2}_{\text{$3$ kpl}} = \underbrace{2\cdot 2}_{\text{$5$ kpl}}=2^5 = 32$
            \item $\displaystyle\frac{2^4\cdot 2^2}{2^3} =
                \frac{2\cdot 2\cdot 2\cdot \cancel{2} \cdot \cancel{2}\cdot
                \cancel{2}}{\cancel{2}\cdot \cancel{2}\cdot \cancel{2}} = 2^3 = 8$
        \end{enumerate}
    \end{esimerkki}
    
\section*{Potenssien laskusääntöjen perusteluja}
    
    Samankantaisten potenssien kertolasku
	\[
a^3\cdot a^4=\underbrace{a\cdot a\cdot a}_{\text{3 kpl}}\cdot \underbrace{a\cdot a\cdot a\cdot a}_{\text{4kpl}}=a^{\mathbf{3+4}}=a^7
    	\]
    Samankantaisten potenssien jakolasku
	\[
\frac{a^7}{a^4}=\frac{a\cdot a\cdot a\cdot \cancel{a}\cdot \cancel{a}\cdot \cancel{a}\cdot \cancel{a}}	{\cancel{a}\cdot \cancel{a}\cdot \cancel{a}\cdot \cancel{a}}=a^{\mathbf{7-4}}=a^3
    	\]
    Potenssin potenssi
	\[
(a^2)^3=\underbrace{a^2\cdot a^2\cdot a^2}_{3\text{kpl}}=\underbrace{a\cdot a\cdot a\cdot a\cdot a\cdot a}_{2\cdot 3=6\text{ kpl}}=a^{\boldsymbol{{2\cdot 3}}}=a^6
\]
    Tulon potenssi
	\[
(ab^5)^3=ab^5\cdot ab^5\cdot ab^5=a\cdot a\cdot a\cdot b^5\cdot b^5\cdot b^5=a^{\mathbf{1\cdot 3}}\cdot b^{\mathbf{5\cdot 3}}=a^3b^{15}
	\]
     Osamäärän potenssi
	\[
	\left(\frac{a^9}{b^7}\right)^3=\frac{a^9}{b^7}\cdot \frac{a^9}{b^7}\cdot \frac{a^9}{b^7}=\frac{a^9\cdot a^9\cdot a^9}{b^7\cdot b^7\cdot b^7}=\frac{a^{\mathbf{9\cdot 3}}}{b^{\mathbf{7\cdot 3}}}=\frac{a^{27}}{a^{21}}
	\]

    Potensseille pätevät seuraavat laskusäännöt.
    
    \laatikko{
        \textbf{Potenssien laskusääntöjä}
    
        Olkoot $m$ ja $n$ kokonaislukuja ja $a$ reaaliluku.
        
        \begin{tabular}{ll}
            $a^m\cdot a^n            = a^{m+n}$ & Samakantaisten potenssien tulo\\
            $\displaystyle \frac{a^m}{a^n}= a^{m-n}$ & Samakantaisten potenssien osamäärä ($a\neq 0$)\\
            $(a\cdot b)^n            = a^n\cdot b^n$ & Tulon potenssi\\
            $\displaystyle \left (\frac{a}{b}\right)^n = \frac{a^n}{b^n}$ & Osamäärän potenssi ($b\neq 0$)\\ 
            $(a^m)^n                 = a^{m\cdot n}$ & Potenssin potenssi\\
        \end{tabular}
    }
    
\section*{Negatiivinen luku ja nolla eksponenttina}
    
    Sievennetään osamäärä $\frac{a^3}{a^5}$ kahdella eri tavalla. Supistamalla yhteiset tekijät lopputulos
    
    \begin{equation*}
        \frac{a^3}{a^5} =
        \frac{\cancel{a}\cdot \cancel{a}\cdot \cancel{a}}{a\cdot a\cdot
        \cancel{a}\cdot \cancel{a}\cdot \cancel{a}} = 
        \frac{1}{a\cdot a}=\boldsymbol{\frac{1}{a^2}}
    \end{equation*}
    
    on erinäköinen, kuin käyttämällä potenssien laskusääntöjä
    
    \begin{equation*}
        \frac{a^3}{a^5} = a^{3-5}=\boldsymbol {a^{-2}}{.}
    \end{equation*}
    
    Koska lähtötilanne on molemmissa tapauksissa sama, voidaan määritellä, että
    
    \begin{equation*}
        a^{-2} = \frac{1}{a^2}
    \end{equation*}
  
    \laatikko{
        \textbf {Negatiivinen eksponentti} (kun $a\neq 0$)
        \begin{equation*}
        	a^{-n} = \frac{1}{a^n}
        \end{equation*}
    }
    
    Kun eksponenttina on nolla, on tuloksena aina luku $1$ kaikilla nollasta poikkeavilla kantaluvuilla. 
    Esimerkiksi
    \[
        \frac{a^3}{a^3}=\frac{\cancel{a}\cdot \cancel{a}\cdot \cancel{a}}{\cancel{a}\cdot \cancel{a}\cdot \cancel{a}}=1,
    \]
    mutta sama lasku voidaan laskea myös toisella tavalla: $\frac{a^3}{a^3}=a^{3-3}=a^0$.
  
    \laatikko{
        \textbf{Eksponenttina nolla} (kun $a\neq 0$)
        \begin{equation*}
            a^{0}=1
        \end{equation*}
    }

Merkintää $0^0$ ei ole määritelty.    

%    \section{Laskujärjestys}
%    
%    \laatikko{
%        \begin{enumerate}
%            \item Sulut
%            \item Potenssilaskut
%            \item Kerto- ja jakolaskut vasemmalta oikealle
%            \item Yhteen- ja jakolaskut vasemmalta oikealle
%        \end{enumerate}
%    }
%   
\section*{Tehtäviä}

\subsection*{Opi perusteet}
    %perustehtäviä
	Sievennä.
    \begin{tehtava}%perteht
        %Sievennä
		\begin{enumerate}[a)]
        	\item $2^3 $ 
        	\item $aaaa$ 
        	\item $a^3a^2$ 
        	\item $0^4$
		\end{enumerate}        
        \begin{vastaus}
        \begin{enumerate}[a)]
            \item $8$ 
            \item $a^4$ 
            \item $a^5$ 
            \item $0$
        \end{enumerate}
        \end{vastaus}
    \end{tehtava}

    \begin{tehtava}%perteht
        %Sievennä
        \begin{enumerate}[a)]
        	\item $a^2a^5 $ 
        	\item $\frac{a^5}{a^3}$ 
        	\item $(a^3)^2$ 
        	\item $12^0$
		\end{enumerate}        
        \begin{vastaus}
        \begin{enumerate}[a)]
            \item $a^7$ 
            \item $a^2$ 
            \item $a^6$ 
            \item $1$
        \end{enumerate}
        \end{vastaus}
    \end{tehtava}    
    
    %soveltavia tehtäviä
        
    \begin{tehtava}%sovteht
        %Sievennä
        \begin{enumerate}[a)]
        	\item $a^2(-a^4) $ 
        	\item $(ab^2)^0$ 
        	\item $(3a)^3$ 
        	\item $(a^5b^3)^3$
		\end{enumerate}        
        \begin{vastaus}
        \begin{enumerate}[a)]
            \item $-a^6$ 
            \item $1$ 
            \item $27a^3$ 
            \item $a^{15}b^9$
        \end{enumerate}
        \end{vastaus}
    \end{tehtava} 
    
    \begin{tehtava}%sovteht
        %Sievennä
        \begin{enumerate}[a)]
        	\item $\frac{2^7}{2^9}$ 
        	\item $\frac{a^3}{a}$ 
        	\item $\left(\frac{1}{3}\right)^2$ 
        	\item $\left(\frac{a^{-2}}{ab^4}\right)^4$
		\end{enumerate}        
        \begin{vastaus}
        \begin{enumerate}[a)]
            \item $\frac{1}{4}$ 
            \item $a^2$ 
            \item $\frac{1}{9} $ 
            \item $ \left(\frac{1}{a^{12}b^{16}}\right)$ tai $a^{-12}b^{-16}$
        \end{enumerate}
        \end{vastaus}
    \end{tehtava}     
            
        
        
        
 
    Sievennä.
    \begin{tehtava}
        %Sievennä \quad
        a) $a\cdot a\cdot a$ \quad
        b) $a\cdot a\cdot a\cdot b\cdot b\cdot b\cdot b$ \quad
        c) $a\cdot b\cdot a\cdot b\cdot a\cdot b\cdot a$
        
        \begin{vastaus}
            a) $a^3$ \qquad
            b) $a^3b^4$ \qquad
            c) $a^4b^3$
        \end{vastaus}
    \end{tehtava}
    
    \begin{tehtava}
        %Sievennä \quad
        a) $a^2\cdot a^3$ \qquad
        b) $a^3a^2$ \qquad
        c) $a^2 a$ \qquad
        d) $a a^2 a$ \qquad
        e) $a^2a^1a^3$
        
        \begin{vastaus}
            a) $a^5$ \qquad
            b) $a^5$ \qquad
            c) $a^3$ \qquad
            d) $a^4$ \qquad
            e) $a^6$
        \end{vastaus}
    \end{tehtava}
    
    \begin{tehtava}
        %Sievennä \quad
        a) $a^0$ \qquad
        b) $a^0a^0$ \qquad
        c) $a a^1$ \qquad
        d) $aa^0$ \qquad
        e) $a^0a^1$
        
        \begin{vastaus}
            a) $1$ \quad ($a\neq0$, koska $0^0$ ei ole määritelty) \qquad
            b) $1$ \qquad
            c) $a$ \qquad
            d) $a^2$ \qquad
            e) $a$
        \end{vastaus}
    \end{tehtava}
    
    \begin{tehtava}
        %Sievennä \quad
        a) $a^1 a a^2$ \qquad
        b) $aaaa$ \qquad
        c) $a^3ba^2$ \qquad
        d) $aba^0ba^1$
        
        \begin{vastaus}
            a) $ a^4$ \qquad
            b) $a^4$ \qquad
            c) $a^5b$ \qquad
            d) $a^2b^2$
        \end{vastaus}
    \end{tehtava}
    % teht 5
    
    \begin{tehtava}
        %Sievennä \quad
        a) $(-2)\cdot(-2)\cdot(-2)$ \qquad
        b) $(-1)\cdot(-1)\cdot(-1)\cdot(-1)$
        
        \begin{vastaus}
            a) $ -8$ \qquad
            b) $1$
        \end{vastaus}
    \end{tehtava}
    
    \begin{tehtava}
        %Sievennä \quad
        a) $(-a)\cdot(-a)$ \qquad
        b) $(-a)\cdot(-a)\cdot(-b)$ \qquad
        c) $(-a^2)\cdot(-a^2)$

        \begin{vastaus}
            a) $a^2$ \qquad
            b) $-a^2b$ \qquad
            c) $a^4$
        \end{vastaus}
    \end{tehtava}

    \begin{tehtava}
        %Sievennä \quad
        a) $-a\cdot(-a)$ \qquad
        b) $-a\cdot(-a)\cdot(-b)$ \qquad
        c) $-a^2\cdot(-a^2)$
    
        \begin{vastaus}
            a) $a^2$ \qquad
            b) $-a^2b$ \qquad
            c) $a^4$
        \end{vastaus}
    \end{tehtava}

    \begin{tehtava}
        %Sievennä \quad
        a) $-a^3\cdot(-a^2)$ \qquad
        b) $a\cdot(-a)\cdot(-b)$ \qquad
        c) $a^2\cdot(-a^2)$
        
        \begin{vastaus}
            a) $a^5$ \qquad
            b) $a^2b$ \qquad
            c) $-a^4$
        \end{vastaus}
    \end{tehtava}

    \begin{tehtava}
        %Sievennä \quad
        a) $2^3\cdot2^3$ \qquad
        b) $4^3$ \qquad
        c) $(2^2)^3$ \qquad
        d) $2^{2+2+2}$

        \begin{vastaus}
            a) $64$ \qquad
            b) $64$ \qquad
            c) $64$ \qquad
            d) $64$
        \end{vastaus}
    \end{tehtava}

    %teht. 10
    \begin{tehtava}
        %Sievennä \quad
        a) $0^3\cdot0^3\cdot0^3$ \qquad
        b) $3^1$ \qquad
        c) $2^{2+3}$ \qquad
        d) $2^{6-4}$ \qquad
        e) $5^0$

        \begin{vastaus}
            a) $1$ \qquad
            b) $3$ \qquad
            c) $32$ \qquad
            d) $4$ \qquad
            e) $1$
        \end{vastaus}
    \end{tehtava}
    \begin{tehtava}
        %Sievennä \quad
        a) $(a^3)^1$ \qquad
        b) $(a^6)^2$ \qquad
        c) $(a^2)^4$ \qquad 
        d) $(a^1)^3$ \qquad
        e) $(a^0)^5$

        \begin{vastaus}
            a) $a^3$ \qquad
            b) $a^{12}$ \qquad
            c) $a^8$ \qquad
            d) $a^3$ \qquad
            e) $1$
        \end{vastaus}
    \end{tehtava}
    \begin{tehtava}
        %Sievennä \quad
        a) $a^3\cdot a^2\cdot a^5$ \qquad
        b) $(a^3)^2$ \qquad
        c) $(a^2a^3)^4$ \qquad
        d) $a^{7-2}$

        \begin{vastaus}
            a) $a^{10}$ \qquad
            b) $a^6$ \qquad
            c) $a^{20}$ \qquad
            d) $a^5$
        \end{vastaus}
    \end{tehtava}
    \begin{tehtava}
        %Sievennä \quad
        a) $(1\cdot a)^3$ \qquad
        b) $(a\cdot 2)^2$ \qquad
        c) $(-2abc)^3$ \qquad
        d) $(3a)^4$

        \begin{vastaus}
            a) $a^3$ \qquad
            b) $4a^2$ \qquad
            c) $-8a^3b^3c^3$ \qquad
            d) $91a^4$
        \end{vastaus}
    \end{tehtava}

    \begin{tehtava}
        %Sievennä \quad
        a) $a^3\cdot b^2\cdot a^5$ \qquad
        b) $(-ab^3)^2$ \qquad
        c) $(a^5a^4)^3$ \qquad
        d) $b(ab)^2$

        \begin{vastaus}
            a) $a^8b^2$ \qquad
            b) $a^2b^6$ \qquad
            c) $a^{15}b^{12}$ \qquad
            d) $a^2b^3$
        \end{vastaus}
    \end{tehtava}

    %teht.15
    \begin{tehtava}
        %Sievennä \quad
        a) $b^3(ab^0)^2$ \qquad
        b) $(ab^3)^0$ \qquad
        c) $(aa^4)^3a^2$ \qquad
        d) $a^3(b^2a)^5$

        \begin{vastaus}
            a) $a^2b^3$ \qquad
            b) $1$ \qquad
            c) $a^{17}$ \qquad
            d) $a^8b^{10}$
        \end{vastaus}
    \end{tehtava}

    \begin{tehtava}
        %Sievennä \quad
        a) $(1^3\cdot 2^2)^2$ \qquad
        b) $(1^2\cdot 2^3)^2$ \qquad
        c) $(2^2a^4)^2$ \qquad
        d) $b(3b)^3$

        \begin{vastaus}
            a) $16$ \qquad
            b) $64$ \qquad
            c) $16a^8$ \qquad
            d) $27b^4$
        \end{vastaus}
    \end{tehtava}
    
    \begin{tehtava}
        %Sievennä \quad
        a) $(a^3b^2)^2$ \qquad
        b) $a(a^2b^3)^4$ \qquad
        c) $(b^2a^4)^5$ \qquad
        d) $b(2ab^2)^3$
        
        \begin{vastaus}
            a) $a^6b^4$ \qquad
            b) $a^9b^{12}$ \qquad
            c) $a^{20}b^{10}$ \qquad
            d) $8a^3b^7$
        \end{vastaus}
    \end{tehtava}
    
    \begin{tehtava}
        %Sievennä \quad
        a) $\frac{2^3}{2^2}$ \qquad
        b) $\frac{2^4}{2^2}$ \qquad
        c) $\frac{2^3}{2^1}$ \qquad
        d) $\frac{2^3}{2^0}$ \qquad
        e) $\frac{2^3}{2^4}$ \qquad
        f) $\frac{2^3}{2^5}$
        
        \begin{vastaus}
            a) $2$ \qquad
            b) $4$ \qquad
            c) $4$ \qquad
            d) $8$ \qquad
            e) $\frac{1}{2}$ \qquad
            f) $\frac{1}{4}$
        \end{vastaus}
    \end{tehtava}
    
    \begin{tehtava}
        %Sievennä \quad
        a) $\frac{a^3}{a^2}$ \qquad
        b) $\frac{a^4}{a^2}$ \qquad
        c) $\frac{a^3}{a^1}$ \qquad
        d) $\frac{a^3}{a^0}$ \qquad
        e) $\frac{a^3}{a^4}$ \qquad
        f) $\frac{a^3}{a^5}$
        
        \begin{vastaus}
            a) $a$ \qquad
            b) $a^2$ \qquad
            c) $a^2$ \qquad
            d) $a^3$ \qquad
            e) $a^{-1} = \frac{1}{a}$ \qquad
            f) $a^{-2} = \frac{1}{a^2}$
        \end{vastaus}
    \end{tehtava}
    
    %teht. 20
    \begin{tehtava}
        %Sievennä \quad
        a) $\frac{a^2b^2}{ab}$ \qquad
        b) $\frac{a^2b}{a^2}$ \qquad
        c) $\frac{a^3}{a^3}$ \qquad
        d) $\frac{1}{a^0}$ \qquad
        e) $\frac{ab^3}{-b^4}$
        
        \begin{vastaus}
            a) $ab$ \qquad
            b) $b$ \qquad
            c) $1$ \qquad
            d) $1$ \qquad
            e) $-\frac{a}{b}$
        \end{vastaus}
    \end{tehtava}
    
    Sievennä ja kirjoita potenssiksi, jonka eksponentti on positiivinen.
    
    \begin{tehtava}
        %Sievennä \quad
        a) $a^{-3}$ \qquad
        b) $\frac{a}{a^3}$ \qquad
        c) $a^{-2}\cdot a^5$ \qquad
        d) $\frac{b}{a^4}b^{-4}$ \qquad
        e) $\frac{a^3}{a^{-5}}$
        
        \begin{vastaus}
            a) $\frac{1}{a^3}$ \qquad
            b) $\frac{1}{a^2}$ \qquad
            c) $a^3$ \qquad
            d) $\frac{}{a^4b^3}$ \qquad
            e) $a^8$
        \end{vastaus}
    \end{tehtava}
    
    Esitä ilman sulkuja ja sievennä.
    
    \begin{tehtava}
        %Esitä ilman sulkuja ja sievennä \quad
        a) $(\frac{1}{2})^2$ \qquad
        b) $(\frac{1}{3})^3$ \qquad
        c) $(\frac{a}{b})^4$ \qquad
        d) $(\frac{a^2}{b^3})^2$ \qquad
        e) $\left(\frac{a^2}{ab^2}\right)^2$
        
        \begin{vastaus}
            a) $\frac{1}{4}$ \qquad
            b) $\frac{1}{27}$ \qquad
            c) $\frac{a^4}{b^4}$ \qquad
            d) $\frac{a^4}{b^6}$ \qquad
            e) $\frac{a^2}{b^4}$
        \end{vastaus}
    \end{tehtava}
    
    \begin{tehtava}
        %Esitä ilman sulkuja ja sievennä \quad
        a) $(\frac{1}{2})\cdot(\frac{1}{2})$ \qquad
        b) $(-\frac{ab^2}{a^2b})^3$ \qquad
        c) $(-a^4b^4)^2$ \qquad
        d) $\left((\frac{a}{b})^4\right)^2$
        
        \begin{vastaus}
            a) $\frac{1}{4}$ \qquad
            b) $-\frac{b^3}{a^3}$ \qquad
            c) $a^8b^8$ \qquad
            d) $\frac{a^8}{b^8}$
        \end{vastaus}
    \end{tehtava}
    
    \begin{tehtava}
        %Esitä ilman sulkuja ja sievennä \quad
        a) $\frac{6a^9}{3a^2}$ \qquad
        b) $(\frac{a^2b^{-2}}{a^2b})^{-3}$ \qquad
        c) $\frac{5a^2}{-15a}$ \qquad
        d) $\left(-(\frac{10^3}{100b})^2 b^{-1} \right )^2$
        
        \begin{vastaus}
            a) $2a^7$ \qquad
            b) $b^9$ \qquad
            c) $-\frac{1}{3}a = -\frac{a}{3}$ \qquad
            d) $\frac{10\ 000}{b^6}$
        \end{vastaus}
    \end{tehtava}
