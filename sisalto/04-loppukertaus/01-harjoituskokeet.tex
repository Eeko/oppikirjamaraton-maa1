\chapter{Harjoituskokeita}

\section*{Harjoituskoe 1}

\begin{description}
	\item[1.] tyhjää
	\item[2.] Sievennä: (a) $\frac{a^2 b^2}{a}$ (b) $3(a^2+1)-2(a^2-1)$ (c) $ab(a+2a)$ (d) $(a^3 b^2 c)^2$
	\item[3.]
	\item[4.] Mitkä seuraavista luvuista ovat alkulukuja? (a) $11$ (b) $4$ (c) $29$ (d) $39$
	\item[5.]
	\item[6.] 
	\item[7.] 
	\item[8.] Funktio $f$ määritellään kaavalla $f(x) = x^2 + 2x + 3$. Ilmaise $f(f(x))$ muodossa, jossa ei ole termiä $f(x)$.
\end{description}
