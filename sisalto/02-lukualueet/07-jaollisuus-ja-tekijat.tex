\chapter{Jaollisuus \& tekijät}

\laatikko{
Määritelmä

Kokonaisluvun $a$ sanotaan olevan jaollinen kokonaisluvulla $b$, jos on olemassa sellainen kokonaisluku $c$, että $a=b\cdot c$. Tällöin sanotaan myös, että $b$ on $a$:n tekijä ja $c$ on $a$:n tekijä.
}

Esimerkiksi luku $-12$ on jaollinen luvulla $3$, koska $-12=3\cdot (-4)$. Toisaalta luku $-12$ ei ole jaollinen luvulla $5$, koska ei ole mitään kokonaislukua, joka kerrottuna viidellä olisi $12$.

Jaollisuuden voi ajatella myös jakolaskun avulla niin, että esim. luku $12$ on jaollinen luvulla $3$, koska luku $12$ voidaan jakaa $3$ yhtäsuureen osaan niin, että jokaisen osan koko on kokonaisluku -- tässä tapauksessa $4$.

Selvästi voidaan huomata, että kaikki luvut ovat jaollisia itsellään ja luvulla $1$. Esimerkiksi $37=37*1=1*37$, joten $37$ on jaollinen $1$:llä ja $37$:llä. Muita tekijöitä luvulla ei välttämättä ole.

\laatikko{
Määritelmä

Ykköstä suurempaa kokonaislukua sanotaan alkuluvuksi, jos sillä ei ole muita tekijöitä kuin $1$ ja luku itse.
}

Kokeilemalla havaitaan helposti, että esimerkiksi luvut 2, 3, 5, 7, 11, 13, 17 ja 19 ovat alkulukuja. 

\laatikko{
Aritmetiikan peruslause

Jokainen ykköstä suurempi kokonaisluku voidaan esittää yksikäsitteisesti alkulukujen tulona.
}

Esimerkiksi luku $84$ voidaan kirjoittaa muodossa $2\cdot 2\cdot 3\cdot 7$. Kokeilemalla huomataan helposti, että 2, 3, ja 7 ovat kaikki alkulukuja. Aritmetiikan peruslauseen nojalla tiedetään, että tämä on ainoa tapa kirjoittaa $84$ alkulukujen tulona - mahdollista kertolaskujärjestyksen vaihtoa lukuunottamatta. Kun luku $84$ esitetään muodossa $2\cdot 2\cdot 3\cdot 7$ on tapana sanoa, että se on jaettu alkutekijöihin. Alkutekjät esitetään yleensä kasvavassa numerojärjestyksessä. Jos sama luku esiintyy tekijöissä useampaan kertaan, on se yleensä yleensä tapana merkitä potenssina. Tällöin luku $84$ voitaisiin kirjoittaa tekijöihin jaettuna $2^2\cdot 3\cdot 7$ ja luku $96$ muodossa $2\cdot 2\cdot 2\cdot 2\cdot 2\cdot 3=2^5\cdot 3$