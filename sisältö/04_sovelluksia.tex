%
\part{Sovelluksia}
%
%(Pythagoraan lause)
%
\chapter{Verrannollisuus}

Verrannollisuudella tarkoitetaan tilannetta, jossa 

Suoraan verrannollisuus tarkoittaa, että kahden asian suhde pysyy vakiona. Jos toinen kaksinkertaistuu, kaksinkertaistuu toinenkin. Esimerkiksi kaupasta ostettujen hedelmien määrä ja kauppahinta ovat suoraan verrannollisia toisiinsa. Jos ostat kaksi kertaa enemmän banaaneja, joudut myös maksamaan kaksi kertaa enemmän. Hinnan ja ostettujen banaanien massan\footnote{Arkikielessä puhutaan yleensä painosta.} suhde on molemmissa tapauksissa vakio. Hedelmäesimerkin tapauksessa tätävakiota kutsutaan kilohinnaksi.

Matemaattisesti suoraan verrannollisuus merkitään seuraavasti. Jos suure $a$ on suoraan verrannollinen suureeseen $b$, merkitään
\begin{equation}
\frac{a}{b}=c,
\end{equation}
missä $c$ on vakio.

Kääntäen verrannollisuus tarkoittaa, että kahden asian tulo pysyy vakiona. Jos toinen kaksinkertaistuu, toinen puolittuu. Esimerkiksi nopeus ja matkaan tarvittava aika ovat kääntäen verrannollisia toisiinsa. Jos ajat koulumatkan kaksi kertaa nopeammin, matka-aika puolittuu.

Muita esimerkkejä kääntäen verrannollisuudesta ovat:
\begin{itemize}
\item .
\end{itemize}

Matemaattisesti kääntäen verrannollisuus merkitään seuraavasti. Jos suure $a$ on kääntäen verrannollinen suureeseen $b$, merkitään
\begin{equation}
ab=c,
\end{equation}
missä $c$ on vakio.


\chapter{Verrannollisuus: sovelluksia}

\begin{tehtava}
% Lyhyt matikka 1, s. 72
Pohdi, kuinka toinen suure muuttuu, kun toinen suure kaksinkertaistuu, kolminkertaistuu, puolittuu jne. Ovatko suureet suoraan verrannolliset?
\begin{enumerate}
\item kuljettu matka ja kulunut aika, kun keskinopeus on 30 km/h
\item kananmunien lukumäärä ja niiden kovaksi keittämiseen tarvittava keittoaika
\item hedelmätiskiltä valitun vesimelonin paino ja hinta
\item neliön sivun pituus ja neliön pinta-ala
\end{enumerate}
\begin{vastaus}
Vastaus:
\begin{enumerate}
\item Ovat.
\item Eivät ole.
\item Ovat.
\item Eivät ole, sillä esimerkiksi kun neliön sivun pituus kaksinkertaistuu 1 cm:stä 2 cm:iin, niin neliön pinta-ala nelinkertaistuu 1 cm$^2$:stä 4 cm$^2$:iin.
\end{enumerate}
\end{vastaus}
\end{tehtava}

\begin{tehtava}
Isi ja lapset ovat ajamassa mökille Sotkamoon. Ollaan ajettu jo neljä viidennestä matkasta ja aikaa on kulunut kaksi tuntia. "Joko ollaan perillä?" kysyvät lapset takapenkiltä. Kuinka pitkään vielä arviolta kuluu, ennen kuin ollaan mökillä?
\begin{vastaus}
Vastaus: 1 h 15 min
\end{vastaus}
\end{tehtava}

\begin{tehtava}
Äidinkielen kurssilla annettiin tehtäväksi lukea eräs 300-sivuinen romaani. Eräs opiskelija otti aikaa ja selvitti lukevansa vartissa seitsemän sivua. Kuinka monta tuntia häneltä kuluu arviolta koko romaanin lukemiseen, jos taukoja ei lasketa?
\begin{vastaus}
Vastaus: 642 minuuttia eli 10 h 42 min.
\end{vastaus}
\end{tehtava}

\chapter{Prosenttilaskentaa - perustilanteet}

Sana prosentti tulee latinan kielen sanoista pro centum, mikä tarkoittaa kirjaimellisesti sataa kohden. Prosentteja käytetään ilmaisemaan suhteellista osuutta. Lukua, josta suhde lasketaan, kutsutaan \emph{perusarvoksi}. Prosentin merkki on \%. Esimerkiksi jos sadan euron hintaisesta tuotteesta on alennettu 25 prosenttia, niin tuotteen alennettu hinta on 75 euroa. Jos sen sijaan alkuperäinen hinta nousee 15 prosenttia, niin tuotteen uusi hinta on 115 euroa. Perusarvo on molemmissa tapauksissa 100 euroa.

\laatikko{1 prosentti $= 1 \% = \frac{1}{100} = 0,01$}

\laatikko{Esimerkki: \\$6 \% = \frac{6}{100} = 0,06$, $48,2 \% = \frac{48}{100} = 0,482$, $140 \% = \frac{140}{100} = 1,40$}

Suhdeluku muutetaan prosenteiksi kertomalla se luvulla 100 ja lisäämällä lopputuloksen jälkeen prosenttimerkki.

\begin{esimerkki}
Vesa ansaitsee kuukaudessa 2300 euroa ja Antero 1700 euroa.
Kuinka monta prosenttia Anteron tulot ovat Vesan tuloista? 

{\bf Ratkaisu.}

Lasketaan
\[
\frac{1700}{2300} \cdot 100 \% \approx 0,739\cdot 100 \% = 73,9 \%.
\]
Laskuissa käytettävä perusarvo on Vesan palkka eli 2300 euruoa.

{\bf Vastaus.}
 $73,9 \%$
\end{esimerkki}


Vertailuprosentilla ilmaistaan, kuinka paljon toinen luku on suurempi kuin toinen. Vertailukohteena käytetään aina sitä lukua, johon verrataan. Jos siis halutaan tietää, kuinka monta prosenttia luku $a$ on suurempi kuin $b$, vertailuprosentti saadaan laskettua kaavalla
\[
\frac{a-b}{b} \cdot 100 \%.
\]

\begin{esimerkki}
Vesa ansaitsee kuukaudessa 2300 euroa ja Antero 1700 euroa.
Kuinka monta prosenttia enemmän Vesa ansaitsee kuin Antero?

{\bf Ratkaisu.}

Lasketaan aluksi Vesan ja Anteron palkkojen erotus
\[
2300-1700 = 600.
\]
Sitten lasketaan kuinka monta prosenttia 600 euroa on Anteron palkasta:
\[
\frac{600}{1700} \cdot 100 \% \approx 0,353\cdot 100\% = 35,3 \%.
\]

{\bf Vastaus.}
$35,3 \%$
\end{esimerkki}

Prosentteja käytetään usein ilmaisemaan suureiden muutoksia. Muutosprosenttia laskettaessa perusarvona on alkuperäinen arvo, johon nähden muutos on tapahtunut.

\begin{esimerkki}
Vesan paino on tammikuussa 68 kg ja kesäkuussa 64 kg. Kuinka monta prosenttia Vesa on laihtunut?

{\bf Ratkaisu.}

Lasketaan 
\[
\frac{68-64}{68}\cdot 100\% = \frac{4}{68} \cdot 100\%=0,059\cdot 100\% =5,9\%.
\]

{\bf Vastaus.}
Vesa on laihtunut $5,9\%$.

\end{esimerkki}


Prosenttiyksikkö mittaa prosenttiosuuksien välisiä eroja. Jos prosenttiluku muuttuu, muutos voidaan ilmaista joko prosentteina tai prosenttiyksikköinä.


\begin{esimerkki}
Tuotteen markkinaosuus on vuoden tammikuussa 10 \% ja kesäkuussa 15 \%. 
\begin{enumerate}
\item[a)]
Kuinka monta prosenttia tuotteen markkinaosuus on noussut?

\item[b)] Kuinka monta prosenttiyksikköä tuotteen markkinaosuus on noussut?
\end{enumerate}

{\bf Ratkaisu.} 

a) Tuotteen markkinaosuus on noussut
\[
\frac{15-10}{10} \cdot 100 \%= \frac{5}{10}\cdot 100\% = 50\%.
\]

b) Tuotteen markkinaosuus on noussut $15-10=5$ prosenttiyksikköä. 

{\bf Vastaus.}

a) 50 prosenttia, b) 5 prosenttiyksíkköä.
\end{esimerkki}





\section{Perusprosenttilaskut}

\begin{itemize}
	\item Prosenttiluvun laskeminen
	\item Prosenttiarvon laskeminen
	\item Perusarvon laskeminen
\end{itemize}

\section{Vertailu prosenttien avulla}

\begin{itemize}
	\item Muutosprosentti, vertailuprosentti
	\item Prosentuaalinen muutos
	\item Prosenttiyksikkö
\end{itemize}

\chapter{Prosenttiyhtälöitä ja sovelluksia}

\begin{tehtava}
Laukku maksaa 225 \euro ja on 25\%:n alennuksessa. Paljonko alennettu hinta on?
\begin{vastaus}
Vastaus: 168,75 \euro
\end{vastaus}
\end{tehtava}

\begin{tehtava}
%Pyramidi 1, s. 80
Kirjan myyntihinta, joka sisältää arvolisäveron, on 8\% suurempi kuin kirjan veroton hinta. Laske kirjan veroton hinta, kun myyntihinta on 15\euro.
\begin{vastaus}
Vastaus: 13,89 \euro
\end{vastaus}
\end{tehtava}

\begin{tehtava}
Perussuomalaisten kannatus oli vuoden 2007 eduskuntavaaleissa 4,1\% ja vuoden 2011 eduskuntavaaleissa 19,1\%. Kuinka monta prosenttiyksikköä kannatus nousi? Kuinka monta prosenttia kannatus nousi?
\begin{vastaus}
Vastaus: Kannatus nousi 15 \%-yksikköä ja 365,9 \%.
\end{vastaus}
\end{tehtava}

\begin{tehtava}
Askartelukaupassa on alennusviikot, ja kaikki tavarat myydään 60\%n alennuksella. Viimeisenä päivänä kaikista hinnoista annetaan vielä lisäalennus, joka lasketaan aiemmin alennetusta hinnasta. Minkä suuruinen lisäalennus tulee antaa, jos lopullisen kokonaisalennuksen halutaan olevan 80\%?
\begin{vastaus}
Vastaus: 50\%.
\end{vastaus}
\end{tehtava}

\begin{tehtava}
%tässä tehtävässä pitää tietää potenssi
Erään pankin myöntämä opintolaina nousee korkoa 2\% vuodessa. Kuinka monta prosenttia laina on noussut korkoa alkuperäiseen summaan verrattuna kymmenen vuoden kuluttua?
\begin{vastaus}
Vastaus: 22\%.
\end{vastaus}
\end{tehtava}

Ansiotuloverotus on Suomessa progressiivista: suuremmista tuloista maksetaan

\begin{tehtava}
Tuoreissa omenissa on vettä 80\% ja sokeria 4\%. Kuinka monta prosenttia sokeria on samoissa omenissa, kun ne on kuivattu siten, että kosteusprosentti on 20? [K2000, 4]
\begin{vastaus}
Vastaus: 16\%
\end{vastaus}
\end{tehtava}

%
%(Eksponentiaalinen malli)
%
\chapter{Kertaustiivistelmä}
