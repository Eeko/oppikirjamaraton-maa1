\chapter{Murtopotenssi}

Tämä on Villellä kesken!

Mitä voisi tarkoittaa $2^\frac{1}{3}$ ?

Potenssien laskusäätöjen mukaan $(2^4)^3 = 2^{3\cdot 4} = 2^{12}$, joten miksi ei voisi laskea $\left( 2^{\frac{1}{3}}\right)^3 = 2^{\frac{1}{3}\cdot 3} = 2^1=2$ ? Toisaalta $(\sqrt[3]{2})^3=2$, joten on luontevaa määritellä $2^{\frac{1}{3}} = \sqrt[3]{2}$. Tällanen merkintä on käytännöllinen, koska murtoluvuilla laskeminen on monesti mukavampaa kuin juurimerkinnän käyttö.

\laatikko{Murtopotenssimerkintä: $a^\frac{1}{n} = \sqrt[n]{a}$, kun $a\geq 0$.}

Vastaavasti voidaan määritellä, mitä tarkoittaa kun eksponenttinä on mikä tahansa murtoluku.

\laatikko{Murtopotenssimerkintä: $a^\frac{m}{n} = (\sqrt[n]{a})^m$, kun $a\geq 0$.}

{\bf Huomio määrittelyjoukosta}. Murtopotenssimerkintää käyttettäessä joudumme vaatimaan, että $a\geq 0$ myös silloin, kun $n$ on pariton. Esimerkiksi $\sqrt[3]{-1}=-1$ (koska $(-1)^3=-1$), mutta lauseketta $(-1)^\frac{1}{3}$ ei ole määritelty. Tähän on hyvä syy, sillä muuten seuraa yllättäviä ongelmia:

\[ -1 = \sqrt[3]{-1} = (-1)^\frac{1}{3} = (-1)^\frac{2}{6}
= ((-1)^2)^\frac{1}{6} = 1^\frac{1}{6} = \sqrt[6]{1} = 1. \]

Hupsis! Mikä meni pieleen? Lavennus $\frac{1}{3}$:sta lukuun $\frac{2}{6}$ on ongelman ydin, mutta olisi hyvin ikävää jos murtolukuja ei saisikaan aina laventaa. Tämän takia peli vihelletään poikki
heti toiseen yhtäsuuruusmerkin kohdalla. On siis sovittu, ettei
murtopotenssimerkintää käytetä, jos kantaluku on negatiivinen.

\begin{esimerkki}
Muuta lausekkeet $\sqrt[5]{3}$ ja $(\sqrt[4]{a})^7$ murtopotenssimuotoon. Ratkaisu: \\
$\sqrt[5]{3} = 3^\frac{1}{5}$, \\
$(\sqrt[4]{a})^7 = (a^\frac{1}{4})^7=a^\frac{7}{4}$
\end{esimerkki}

\begin{esimerkki}
Sievennä lauseke $8^\frac{2}{3}$. Ratkaisu: \\
 $8^\frac{2}{3} = (\sqrt[3]{8})^2 = 2^2 = 4.$
\end{esimerkki}