\chapter{Potenssifunktio}

Usein kahden muuttujan välinen riippuvuus ei ole suoraan
tai kääntäen verrannollista. Potenssifunktion käyttäminen on eräs
keino kuvata tällaisia riippuvuuksia.

\laatikko{Potenssifunktioita ovat muotoa
$ f(x) = a x^n $ olevat funktiot, jossa $a \neq 0$. }

Eksponenttia $n$ kutsutaan potenssifunktion \emph{asteeksi}.
Potenssifunktion eksponentti voi olla mikä tahansa reaaliluku, mutta
rajoitumme ensin käsittelemään tapauksia, jossa $n = 1, 2, 3\ldots $.

\begin{esimerkki}
Jos neliön sivun pituus on $x$, neliön pinta-ala voidaan laskea
funktiolla $A(x)=x^2$.
Esimerkiksi jos $x = 3$ cm, saadaan neliön pinta-alaksi $A(x) = 9$ cm$^2$.
Vastaavasti kuutiolle: jos $x$ kuvaa kuution särmän pituutta, funktiolla
$V(x)=x^3$ voidaan laskea kuution tilavuus. Sekä $A(x)$ että $V(x)$ ovat
esimerkkejä potenssifunktioista.
\end{esimerkki}

Potenssifunktion aste vaikuttaa funktion kuvaajan muotoon:
\begin{itemize}
  \item
Jos aste on parillinen, kuvaaja on U-kirjaimen muotoinen ja funktio
saa $a$:n merkistä riippuen joko pelkästään positiivisia tai pelkästään
negatiivisia arvoja.
  \item
Jos aste on pariton, kuvaaja muodostaa ''kaksoismutkan'' ja potenssifunktio
saa sekä positiivisia että negatiivisia arvoja.
\end{itemize}

\missingfigure{Potenssifunktioiden kuvaajat - yksi parillisilla potensseilla ja toinen parittomilla (kerroin a positiivinen).}

Tärkeitä erikoistapauksia potenssifunktioista saadaan, kun asetetaan $n = 1$ tai
$n = -1$. Kun $n = 1$, saadaan edellisessä luvussa esitelty suoraan verrannollinen
riippuvuus $x$:n ja $f(x)$:n välillä, ja kun $n = -1$, muuttuja $x$ ja
funktion arvo $f(x)$ ovat kääntäen verrannolliset.

Potenssifunktiota voidaan laajentaa sallimalla eksponentille $n$
myös negatiiviset arvot.
Tällöin funktion muoto muuttuu merkittävästi. Funktio ei myöskään ole
enää määritelty kohdassa $x = 0$, vaan funktion arvot näyttävät
''räjähtävän äärettömyyteen'', kun y-akselia lähestytään:

\missingfigure{Potenssifunktioiden kuvaajat - yksi potenssilla -1
ja toinen potenssilla -2.}

%Potenssifunktiota käsitellään samalla lailla riippumatta eksponentin
%etumerkistä. Huomaa kuitenkin, että $\frac{1}{x^n} \neq 0 $ kaikilla $x$:n %arvoilla.

%Tästä alaspäin on potenssiyhtälöitä, jotka varmaan menee jo päälle
%aiemmin käydyn asian kanssa. Sekaannus meikäläisen osalta... -Matti

\section*{Tehtäviä}

\begin{tehtava}
Mikä on seuraavien potenssifunktioiden aste?
\begin{enumerate}[a)]
\item $f(x) = x$
\item $f(x) = 5x^5$
\item $f(x) = \frac{1}{2x}$
\item $f(x) = x^{-2}$
\end{enumerate}
\begin{vastaus}
\begin{enumerate}
\item $1$
\item $5$
\item $-1$
\item $-2$
\end{enumerate}
\end{vastaus}
\end{tehtava}
