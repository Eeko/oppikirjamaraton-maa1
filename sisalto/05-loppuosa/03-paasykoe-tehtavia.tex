\chapter{''Näihin pystyt jo'' -pääsykoetehtäviä}

\section{Arkkitehtuuri}

\section{Kansantaloustiede}

\section{Kauppatieteellinen}

\section{Matematiikka ja tilastotiede}

\begin{description}
	\item[(2012/1a)] Ratkaise yhtälö $\frac{3}{2}x - \frac{2}{3} = \frac{2}{3}x - \frac{1}{4}$.
	\item[(2011/1)] Oletetaan, että polttoaineessa E05 on etanolia 5 \% ja bensiiniä 95 \% ja polttoaineessa E10 etanolia 			10 \% ja bensiiniä 90 \%. Oletetaan myös, että etanolin energiasisältö on 2=3 puhtaan bensiinin
		energiasisällöstä. Jos tietyllä autolla 100 km kulutus on 10 litraa polttoainetta E05, paljonko kulutus on 
		polttoainetta E10? Anna vastaus sievennettynä murto- tai sekalukuna. Jos polttoaineen 
		E10 hinta on 1,60 €/l ja polttoaineen E05 hinta on 1,65 €/l, kumpaa on edullisempaa käyttää?
	\item[(2008/1)] Matkailuauton nopeus on 80 km/h, mutta kolmasosalla matkasta Jyvaskylästä Heinolaan se laskee tietöiden takia 40 kilometriin tunnissa. Kuinka paljon tietyöt alentavat matkailuauton keskinopeutta valillä Jyväskylä-Heinola?
\end{description}

\section{Tekniikan ala (AMK)}

\section{Tekniikan ala (diplomi-insinööri)}
\begin{description}
	\item[(2012/3)] Asumistukea maksetaan 80 \% vuokran määrästä, siltä osin kuin vuokra ei ylitä 252 euroa. Vuokran määrää vähennettynä asumistuella kutsutaan omavastuuksi.
		\begin{enumerate}[(a)]
			\item Minka suuruinen vuokra on, kun omavastuu on puolet vuokrasta?
		\end{enumerate}
	
	\item[(2009/1)] Kokonaistuotanto jaetaan materian ja palveluiden tuotantoon. Verrataan tuotantoa tammikuussa 2008 tammikuuhun 2009. Tänä vuoden pituisen tarkastelujakson aikana materiatuotanto kasvoi 2,0 \% ja palvelutuotanto laski 7,0 \%.
	
	Kuinka suuri oli materiatuotannon osuus kokonaistuotannosta tammikuussa 2009,
	\begin{enumerate}[(a)]
		\item kun tammikuussa 2008 materia- ja palvelutuotanto olivat yhtäsuuret?
		\item kun vertailuaikana kokonaistuotanto laski 2,0 \%?
	\end{enumerate}
	Anna kummatkin vastaukset 0,1 \%-yksikön tarkkuuteen pyöristettynä.

	\item[(2008/2)] Yritys hankkii 5000 kg raaka-ainetta, josta on vettä 5,40 \% (painoprosenttia) ja väripigmenttiä 2,60 \%. Ennen käyttöä raaka-aine on laimennettava siten, että lisäyksen jälkeen sekoituksesta 6,60 \% on vettä.
	
	\begin{enumerate}[(a)]
		\item Miten paljon hankittuun raaka-aineeseen tulee lisätä vettä, jotta haluttu vesipitoisuus saavutetaan?
		\item Miten paljon vettä ja väripigmenttiä tulee lisätä hankittuun raaka-aineeseen, jotta haluttu vesipitoisuus saavutetaan, ja lisäksi väripigmentin suhteellinen osuus massasta säilyy alkuperäisenä 2,60 \%:na?
	\end{enumerate}
	Anna vastaukset sadan gramman tarkkuudella.

	\item[(2007/1)] Vaaleissa kaikkiaan 39 300 äänestäjästä 45 \% äänestää varmasti puoluetta A ja 47 \% puoluetta B. Loput ovat ns. liikkuvia äänestäjiä, jotka eivät ole vielä päättäneet kantaansa.
	
	\begin{enumerate}[(a)]
		\item Oletetaan, että kaikki äänioikeutetut äänestävät. Kuinka monta liikkuvien äänestäjien ääntä puolueen A täytyy tällöin kerätä saadakseen enemmistön, vähintään puolet annetuista äänistä?
		\item Oletetaan, että täsmälleen kolmasosa liikkuvista äänestäjistä jättää äänestämättä. Kuinka monta prosenttia liikkuvien äänestäjien annetuista äänistä puolueen A täytyy tällöin kerätä saadakseen enemmistön kaikista annetuista äänistä?
	\end{enumerate}	 	
	
\end{description}

