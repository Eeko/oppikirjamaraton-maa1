\chapter{Reaaliluvut}

Jo antiikin aikoina huomattiin, että kaikki luvut eivät ole rationaalilukuja, eli kaikkia lukuja ei voi esittää kokonaislukujen suhteena. Tällaisia lukuja kutsutaan \emph{irrationaaliluvuiksi}.

Esimerkiksi $\sqrt{2}$ on irrationaaliluku. (Tämä todistetaan
luvun lopussa.) Toinen tuttu irrationaaliluku on $\pi$.

Rationaaliluvut eivät siis täytä lukusuoraa kokonaan, vaikka
rationaalilukuja onkin lukusuoralla tiheässä.

\definecolor{ffqqqq}{rgb}{1,0,0}
\begin{tikzpicture}[line cap=round,line join=round,>=triangle 45,x=6.0cm,y=6.0cm]
\draw[->,color=black] (1.2,0) -- (3.3,0);
\foreach \x in {1.2,1.3,1.4,1.5,1.6,1.7,1.8,1.9,2,2.1,2.2,2.3,2.4,2.5,2.6,2.7,2.8,2.9,3,3.1,3.2,3.3}
\draw[shift={(\x,0)},color=black] (0pt,2pt) -- (0pt,-2pt) node[below] {\footnotesize $\x$};
\clip(1.2,-0.1) rectangle (3.3,0.3);
\draw (0.5,2.1) node[anchor=north west] {$\sqrt[]{2}$};
\draw (2.8,2.4) node[anchor=north west] {$\pi$};
\draw [->] (1.2,0) -- (3.3,0);
\draw [->,line width=2pt,color=ffqqqq] (1.414,0.1) -- (1.414,0);
\draw [->,line width=2pt,color=ffqqqq] (pi,0.1) -- (pi,0);
\draw [color=ffqqqq](1.4,0.2) node[anchor=north west] {$\sqrt[]{2}$};
\draw [color=ffqqqq](3.1,0.2) node[anchor=north west] {$\pi$};
\end{tikzpicture}

Kun rationaalilukuihin otetaan mukaan irrationaaliluvut, saadaan reaalilukujen joukko $\mathbb{R}$. Kaikki rationaalilukuja koskevat
laskusäännöt pätevät myös reaaliluvuille.

Siinä missä rationaalilukujen desimaaliesitykset ovat päättyviä tai jaksollisia (tästä puhuttiin luvussa \ref{rationaaliluvut}), ovat
irrationaalilukujen desimaaliesitykset päättymättömiä ja
jaksottomia. Esimerkiksi luvun
\[\sqrt{2} = 1,414213562373095048801688724209\ldots\]
desimaaliesityksessä on toistaan muistuttavia kohtia
(tässäkin pätkässä 88 esiintyy kahdesti), mutta desimaalit eivät koskaan ''ala alusta''.

Reaalilukujen ominaisuuksien tarkka todistaminen on yllättävän
monimutkaista, ja se on tapana sivuuttaa lukiossa. Jatkossa
tyydymme toteamaan ilman todistusta, että rationaalilukujen
ominaisuudet yleistyvät myös reaaliluvuille.
Emme siis pysähdy pohtimaan, mitä esimerkiksi
$\pi \cdot \pi$ täsmälleen tarkoittaa.
Voimme silti laskea kyseiselle luvulle likiarvon
halutulla tarkkuudella,
\[ \pi\cdot \pi =9,86960440\ldots \].
Lisää reaalilukujen ominaisuuksista liitteessä \ref{aksioomat}.

Reaalilukujen myötä kaikki lukiossa käytettävät lukujoukot on nyt esitelty.
Ne on lueteltu seuraavassa:
\begin{center}\begin{tabular}{l|c|l}
Joukko & Symboli & Mitä ne ovat\\
\hline
Luonnolliset luvut & $\mathbb{N}$ &
Luvut 0, 1, 2, 3, $\ldots$ \\
Kokonaisluvut & $\mathbb{Z}$ & Luvut $\ldots$ -2, -1, 0, 1, 2 $\ldots$ \\
Rationaaliluvut & $\mathbb{Q}$ & Luvut, jotka voidaan esittää
murtolukuina \\
Reaaliluvut & $\mathbb{R}$ & Kaikki lukusuoran luvut
\end{tabular} \end{center}

\missingfigure{Kaavio, jossa N, Z, Q, R sisäkkäin.
Kussakin esimerkkiluku: 5, -2, 3/4, sqrt 2}

Lukualueita voidaan vielä tästäkin laajentaa. Seuraava laajennus olisi \emph{kompleksilukujen joukko} $\mathbb{C}$. Kompleksilukujen joukosta löytyy reaalilukujen lisäksi
esimerkiksi imaginaariyksikkö $i = \sqrt{-1}$.
%, jolle pätee $i^2=-1$. Minkään
%reaaliluvun neliö ei ole negatiivinen.
Kompleksiluvut eivät nykyään kuulu lukion oppimäärään, mutta
niitä tarvitaan esimerkiksi insinöörialoilla.