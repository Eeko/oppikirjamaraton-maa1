


\part{Lukualueet}

\begin{align*}
2,05\ l = x\ dm^3  &\Leftrightarrow 1\ l = 1\ dm^3 \\
&\implies x = 2,05 \\
&\implies  2,05\ l = 2,05\ dm^3
\end{align*}

\chapter{Luonnolliset luvut}

laskeminen vs. counting/calculating

käyttötarkoituksia: lukumäärä, järjestys, indeksointi, …

numeromerkki, numero, luku, paikka- ja lukujärjestelmät
Joukko-oppia (siis merkintöjä!)

joukko, osajoukko, alkio, Venn-diagrammi
Logiikkaa (siis merkintöjä!) [ei mielestäni oleellista tässä kohtaa - Ville Tilvis]
konjunktio, disjunktio, negaatio, implikaatio, ekvivalenssi 

Kokonaisluvut

Vastaluku, visuaalinen esitys peilauksena; % $-x=-1*x & a-b=a+(-b)$

neutraalialkiot
Kokonaislukujen aritmetiikkaa

laskutoimitus

(merkintöjä, esim. 2*x=2x, xyz yleensä aakkosjärjestykseen jne.), vähennyslasku     lukusuoralla

potenssimerkintä

lauseke, sulkeet, laskujärjestys

vaihdannaisuus, liitännäisyys, osittelulaki -> osittelulain käyttöä, päässälaskukikkoja, ...
Jaollisuus \& tekijät

kokonaislukujen jaollisuudesta

alkuluvut ja lukujen jakaminen tekijöihin

tulomuoto, lausekkeen jakaminen tekijöihin osittelulain avulla

polynomi

joidenkin muistikaavojen johtaminen [eikö vasta kurssissa 2? -Ville Tilvis]
Rationaaliluvut ja laskusäännöt

Määritelmä ja esitysmuodot: murtoluku, sekaluku, desimaaliluku; käänteisluku

desimaalikehitelmän jaksollisuus, vinculum, kolme pistettä, pyöristäminen

suhde, osuus

laventaminen, supistaminen

yhteenlasku, vähennyslasku, kertolasku, jakolasku
Potenssisäännöt \& murtolausekkeiden sieventämistä

Tulon ja osamäärän potenssi

samankantaisten potenssien tulo ja osamäärä

potenssin potenssi

nollas potenssi

negatiivinen kokonaislukueksponentti

sievennysharjoituksia
Juuret (tässä nyt kannattaa miettiä, tekeekö ensin vaikka neliöjuuresta ja sitten muita. Tää teoria määrittelyjoukoiltaan ja muilta on aika haastavaa)
Murtopotenssi
(Yksiköt, kertoimet, SI)
Irrationaaliluvut

desimaalikehitelmän jaksottomuus, likiarvo;
Reaaliluvut

lukusuora, reaalilukujen aksioomat, välit, ...
Kompleksiluvut

Lyhyt, mutta rehellinen ja vakava intro mukaan lukien visuaalinen hahmottelu; painotus lukuna ja “ensimmäinen täysin matemaattinen kikka, jolla ei kontaktia reaalimaailmaan”
Kertaustiivistelmä






%%% Local Variables: 
%%% mode: latex
%%% End: 
