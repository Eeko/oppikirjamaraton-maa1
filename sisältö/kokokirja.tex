

% Kun käytät tätä, älä lataa pakettia stfloats tai fix2col. Tämä lataa
% myös paketin fixltx2e.
%% "When the new output routine for LaTeX3 is done, this package will
%% be obsolete. The sooner the better..."
\RequirePackage{dblfloatfix}
% Kun käytät tätä, et enää tarvitse paketteja type1cm ja type1ec:
\RequirePackage{fix-cm}

% Helpottaa pdf(la)tex:in läsnäolon tarkistusta if-lauseilla
\RequirePackage{ifpdf}

\documentclass[a4paper,onecolumn,12pt,finnish,oneside,final]{boek3}

\setcounter{secnumdepth}{7}

%%%%%%%%%%%%%%%%%%%%%%%%%%%%%%%%%%%%%%%%%%%%%%%%%%%%%%%%%%%%%%%%%%%%%%%%%
% Suometus ja fontit
\usepackage[utf8]{inputenc}

% cmap: "Make the PDF files generated by pdflatex "searchable and copyable"
% in Adobe (Acrobat) Reader and other compliant PDF viewers."
% (Pakko olla ennen fontenc-pakettia)
\ifpdf % Eräillä saattaa olla buginen cmap. Yritetään välttää sitä.
\usepackage{cmap}
\fi

\usepackage[T1]{fontenc}
% Suometusten asettaminen jatkuu myöhemmin

\usepackage{blindtext}
\usepackage{lipsum}

%%%%%%%%%%%%%%%%%%%%%%%%%%%%%%%%%%%%%%%%%%%%%%%%%%%%%%%%%%%%%%%%
% Fontit
%

% Yleisimmät erikoismerkit, kuten copyleft
\usepackage{textcomp}

%%% Matikkafontteja

% Matikka-symboleita
\usepackage{latexsym}

% AMS:sän matikka-paketteja:
% Lisätietoa paketista optiolla "?"
%\usepackage[?]{amsmath}
\RequirePackage{amsmath}
\RequirePackage{amsfonts}
\RequirePackage[psamsfonts]{amssymb}
\RequirePackage{amsxtra}
\RequirePackage{amscd}
\RequirePackage{amsthm}
\RequirePackage{mathrsfs}

% Math fonts 
\usepackage{arevmath}

% Text fonts 
% TODO: Poistettu jotta kääntyisi kaikilla.
%\usepackage{dejavu}

\renewcommand{\familydefault}{\sfdefault}
\fontfamily{\familydefault}

\usepackage{titlesec}

%\titleformat{\part}[display]
%  {\normalfont\Huge\filright\bfseries}{\MakeUppercase{Osa\ \thepart}}{\Huge\filright\MakeUppercase}

\titleformat{\part}
  {\normalfont\filright\Huge\bfseries\rmfamily}{Osa \thepart}{1em}{}

%\titleformat{\chapter}[display]
%  {\normalfont\huge\filright\bfseries}{\MakeUppercase{\chaptertitlename\ \thechapter}}{\Huge\filright\MakeUppercase}

\titleformat{\chapter}
  {\normalfont\filright\huge\bfseries\rmfamily}{\chaptertitlename\ \thechapter}{1em}{}

\titleformat{\section}
  {\normalfont\filright\LARGE\bfseries\rmfamily}{\thesection}{1em}{}

\titleformat{\subsection}
  {\normalfont\filright\Large\bfseries\rmfamily}{\thesubsection}{1em}{}

\titleformat{\subsubsection}
  {\normalfont\filright\large\bfseries\rmfamily}{\thesubsubsection}{1em}{}

\titleformat{\paragraph}[runin]
  {\normalfont\normalsize\bfseries\rmfamily}{\theparagraph}{1em}{}

\titleformat{\subparagraph}[runin]
  {\normalfont\normalsize\bfseries\rmfamily}{\thesubparagraph}{1em}{}




% Tukea pdf(la)texin microtypes-laajennoksille. Pakko olla
% fonttimäärityksien jälkeen.
\ifpdf
\usepackage[protrusion=true,expansion=true,verbose=true]{microtype}
\else
\usepackage[verbose=true]{microtype}
\fi


\makeatletter
\g@addto@macro\verbatim{\pdfprotrudechars=0 \pdfadjustspacing=0\relax}
\makeatother




% Fontit asetettu!

% Loput suometukset (mutta myös enkunkieltä saatetaan tarvita):
\usepackage[finnish,english]{babel}
%\usepackage[finnish]{babel}

% Suometukset asetettu!

% if-then-else-rakenteita helposti
\usepackage{ifthen}

% Lisää kokomäärityksiä:
\usepackage[12pt]{moresize}

% Tee jotain jokaisen sivun kohdalla:
% (totpages tarvii tätä)
\usepackage{everyshi}

% Värien tuki:
% (Mihinkähän mä tätä tarvitsinkaan?)
%\usepackage{color}

% Kuvien lisäys dokumenttiin:
% (Mihinkähän mä tätä tarvitsinkaan?)
%\usepackage{graphicx}

% Laskutehtävien suoritusta (geometry-paketti hyötyy tästä):
\usepackage{calc}


\usepackage[nomarginpar,includeheadfoot]{geometry}

% vasen marginaali
\geometry{lmargin=4.0cm}
% oikea marginaali
\geometry{rmargin=2.0cm}
% ylämarginaali
\geometry{tmargin=2.0cm}
% alamarginaali
\geometry{bmargin=2.0cm}


%%%%%%%%%%%%%%%%%%%%%%%%%%%%%%%%%%%%%%%%%%%%%%%%%%%%%%%%%%%%%%%%
% Ykkösvälike, puolitoistakertainen välike ja kakkosvälike yms.
\usepackage{setspace}
%\singlespacing
\onehalfspacing
%\spacing{1.7} % Tämä bugasi joskus aiheuttaen tällaista: (\end occurred inside a group at level 1) 
%\doublespacing

% URL:ien ladonta ja "tavutus"
\usepackage[obeyspaces,spaces,hyphens,T1]{url}
%\renewcommand\url{\begingroup \def\UrlLeft{<URL: }\def\UrlRight{>}\urlstyle{rm}\Url}
%\def\UrlBigBreakPenalty{120}
%\def\UrlBreakPenalty{200}

% Helpottaa TeX:iin liittyvien nimien ladontaa (LaTeX, BibTeX jne.)
\usepackage{texnames}

%%%%%%%%%%%%%%%%%%%%%%%%%%%%%
% Upeet headerit ja footerit





% Ei sisennetä kappaleen ekaa riviä
\setlength{\parindent}{0.0cm}


\ifpdf % We are running pdftex
\pdfcompresslevel=9
\pdfpkresolution=1200
\fi

%\theoremstyle{definition}
\newtheorem{theorem}{Teoreema}

% Infoboksi keltaisella taustalla
\usepackage{mdframed}
\newcommand{\laatikko}[1]{\begin{mdframed}[backgroundcolor=yellow] #1 \end{mdframed}}

\begin{document}

\selectlanguage{finnish}

\begin{titlepage}

  \begin{center}
    \begin{doublespace}
      \begin{LARGE}
        \textrm{Hellsten -- Linja-aho -- Mauno -- Mäkinen -- Piiroinen -- Sottinen \ldots} \\
      \end{LARGE}
      
      \vspace{0.5cm}
      \hrule height 2pt
      \vspace{1cm}
      \begin{Huge}
        \textbf{\textrm{Avoin matikka 1}\\\ \\Kirja on työn alla!}
      \end{Huge}
      
      \vfill
      
      \begin{huge}
        \textrm{MAA1 -- Funktiot ja yhtälöt}
      \end{huge}
      \vspace{1cm}
      \hrule height 2pt
    \end{doublespace}
  \end{center}
  
  \vfill
  \begin{flushright}
    \textbf{Oppikirjamaraton - tätä lukee kuin avointa kirjaa! \\
      Sisältö on lisensoitu avoimella CC-BY-lisenssillä. \\
    }
  \end{flushright}
  
\end{titlepage}

\tableofcontents




%\part{Alku}
\chapter{Esipuhe}

%%%%%%%%%%%%%%%%%%%%%%%%%%%%%%%%%%%%%%%%%%%%%%%%%%%%%%%%%%%%%%%%%%%%%%%%%%%%%%%%
%%%%  /usr/share/doc/texlive-fonts-extra-doc/fonts/arev/mathtesty.tex

% mathtesty.tex, by Stephen Hartke 20050522
% based on mathtestx.tex in the mathptmx package
% and symbols.tex by David Carlisle

Lorem ipsum\ldots

\laatikko{Tässä on ältsin hieno teoriaboksi. Tänne voi laittaa myös kaavoja
\begin{equation}
(a+b)^2=a^2+2ab+b^2
\end{equation}
ja toimii kuin junan vessa.
}

\begin{tehtava}
Esimerkkikysymys.
\begin{vastaus}	
Esimerkkivastaus.
\end{vastaus}
\end{tehtava}

\begin{tehtava}
Laske: $1+2$
\begin{vastaus}
$3$
\end{vastaus}
\end{tehtava}

\begin{tehtava}
Ratkaise:
\[
x=2x+1
\]


\begin{vastaus}
$x=-1$
\end{vastaus}
\end{tehtava}


\begin{theorem}[Residue Theorem]
Let $f$ be analytic in the region $G$ except for the isolated singularities $a_1,a_2,\ldots,a_m$. If $\gamma$ is a closed rectifiable curve in $G$ which does not pass through any of the points $a_k$ and if $\gamma\approx 0$ in $G$ then
\[
\frac{1}{2\pi i}\int_\gamma f = \sum_{k=1}^m n(\gamma;a_k) \text{Res}(f;a_k).
\]
\end{theorem}

\begin{esimerkki}[Leivän paino]
Leipä painaa kilon ja puolet leivästä. Painavako oli leipä?\\
{\bf Ratkaisu.} Merkitään leivän painoa $x$:llä. Puolet leivästä on matemaattisesti ilmaistuna $\frac{x}{2}$ ja kun siihen lisätään kilogramma, saadaan leivän paino, joten saamme yhtälön
\begin{equation}
\frac{x}{2}+1=x
\end{equation}
josta ratkeaa
\begin{equation}
x=2.
\end{equation}
Leipä painaa siis 2 kilogrammaa.
\end{esimerkki}

Another nice theorem from complex analysis is

\begin{theorem}[Maximum Modulus]
Let $G$ be a bounded open set in $\mathbb{C}$ and suppose that $f$ is a continuous function on $G^-$ which is analytic in $G$. Then
\[
\max\{|f(z)|:z\in G^-\}=\max \{|f(z)|:z\in \partial G \}.
\]
\end{theorem}

\newcommand{\abc}{abcdefgh\hbar\hslash i\imath j\jmath klmnopqrstuvwxyz}
\newcommand{\ABC}{ABCDEFGHIJKLMNOPQRSTUVWXYZ}
\newcommand{\alphabeta}{\alpha\beta\varbeta\gamma\delta\epsilon\varepsilon\zeta\eta\theta\vartheta\iota\kappa\varkappa\lambda\mu\nu\xi o\pi\varpi\rho\varrho\sigma\varsigma\tau\upsilon\phi\varphi\chi\psi\omega}
\newcommand{\AlphaBeta}{\Gamma\Delta\Theta\Lambda\Xi\Pi\Sigma\Upsilon\Phi\Psi\Omega}



%%%%%%%%%%%%%%%%%%%%%%%%%%%%%%%%%%%%%%%%%%%%%%%%%%%%%%%%%%%%%%%%%%%%%%%%%%%%%%%%
%%%% /usr/share/doc/texlive-doc-en/fonts/free-math-font-survey/source/textfragment.tex







%%% Local Variables: 
%%% mode: latex
%%% End: 


\part{Luvut ja laskutoimitukset}
\chapter{Lähtötasotesti}

(Tää tulee oikeasti ennen tätä chapteria ja osaa)

\begin{tehtava}
\begin{enumerate}
\item Laske $2^2+2 \cdot 2+2$
\item sasdas
\item 
\end{enumerate}

\begin{vastaus}
\begin{enumerate}
\item 
\item
\item

\end{enumerate}
\end{vastaus}
\end{tehtava}
\chapter{Numerot ja luvut}

(Joonas jatkaa tästä vielä!)

Matematiikka tarjoaa työkaluja asioiden jäsentämiseen, päättelyyn ja mallintamiseen. Alasta riippuen käsittelemme matematiikassa erilaisia \textbf{objekteja}: Geometriassa tarkastelemme tasokuvioita ja kolmiulotteisia rakenteita. Algebrassa tutkii lukujen ja funktioiden ominaisuuksia. Todennäköisyyslaskenta arvioi erilaisten tapausten ja tilanteiden mahdollisuuksia ja riskejä. Matemaattinen analyysi (kurssit 7,8 ja 10) tutkii funktioita ja niiden muuttumista.

Jokaiseen tarkastelukohteeseen liitetään myös niille ominaisia \textbf{operaatioita}. Tämä kurssi käsittelee lähinnä lukuja ja niiden operaatioita, joita \textbf{laskutoimituksiksi} kutsutaan. Aloitetaan yksinkertaisista määritelmistä: mitä tarkoittavat \textbf{numero} ja \textbf{luku}?

\laatikko{Länsimaisessa ... on käytössämme kymmenen numeromerkkiä: 0, 1, 2, 3, 4, 5, 6, 7, 8 ja 9. Näitä kutsutaan hindu-arabialaisiksi numeroiksi.  }

Sanalla numero voidaan siis viitata yksittäiseen kirjoitettuun merkkiin. 

\begin{esimerkki}
Luku \[715531\] koostuu numeroista 7, 1, 5, 5, 3 ja 1.
\end{esimerkki}



Olennaista on myös...
lukujärjestelmä, paikkajärjestelmä

MIKSI KÄYTÄMME KIRJAIMIA?

suuruus, yhtäsuuruus, eri suuret

, ja Erilaisilla luvuilla voidaan suorittaa erilaisia laskutoimituksia. Seuraavissa luvuissa esitellään ja käydään läpi lukiomatematiikassa ja mahdollisissa jatko-opinnoissa käytettäviä lukujoukkoja ja tavallisimmat laskutoimitukset.

\chapter{Luonnolliset luvut}

(Joonas jatkaa tästä vielä!)

Suomen kielen verbi 'laskea' voi tarkoittaa matematiikassa kahta eri asiaa: lukumäärien laskemista ja laskutoimitusten suorittamista.

\laatikko{laskea (lukumäärä) englanti count ruotsi \_ /n
laskea (laskutoimitus) englanti calculate , ruotsi \_}

Ihmisellä ja muilla eläimillä on luonnostaan matemaattisia taitoja. Monet niistä, esimerkiksi lukumäärien laskeminen, ovat yllättävän monimutkaisia kognitiivisia prosesseja, jotka kehittyvät lapsuudessa – toisilla aiemmin, toisilla myöhemmin. Kaikki koulussa opeteltava peruslaskento ja myös matematiikka tieteen alana rakentavat tämän biologisen osaamisen päälle. Laskeminen itsessään on vain yksi matematiikan osa-alue, eikä kaikki matematiikka ole laskemista. Huomaa, että suomen kielen verbillä laskea tarkoitetaan sekä lukumäärien laskemista (engl. counting) että lukujen laskutoimitusten suorittamista (engl. calculating).

Hyvin olennaisena kehitysaskeleena niin yksilön matemaattiselle ajattelulle kuin yhteiskunnallekin on ollut luonnollisen kielen tavoin kyky merkitä lukumäärien laskemista ja muuta matemaattista pohdintaa kirjalliseen muotoon.  On olemassa hyvin monia erilaisia tapoja merkitä lukumääriä. Helpoin tapa ja yksinkertaisin tapa on käyttää vain yhtä samaa merkkiä ja toistaa sitä. Jos

käytettävissä olevien merkintöjä määrää voidaan lisätä, jolloin suuria lukuja voidaan kirjoittaa lyhyemmin. Tämä vastaa myös luonnollisten kielten tilannetta: Suomen kielen aakkosiin kuuluu 29 kirjainta, joista sanat muodostetaan. Sanat voivat olla kuinka pitkiä vain kahdesta kirjaimesta ylöspäin. Kiinassa sen sijaan käytetään omaa piirrosmerkkiä jokaiselle sanalle. Merkkejä täytyy osata 29 sijaan tuhansia, mutta jokaisen sanan voi kirjoittaa lyhyesti. 
Matematiikassa erilaisista numeromerkeistä tai yksinkertaisesti numeroista muodostetaan lukuja yhdistelemällä niitä sopivasti erilaisten paikkajärjestelmien mukaan. Esimerkiksi antiikin Roomassa käytössä olivat numeromerkit I, V, X, L, C , D ja M. Niiden numeroiden vastaavuudet meidän käyttämiimme lukuarvoihin ovat seuraavat:
I=1
V=5
X=10
L=50
C=100
D=500
M=1 000
Huomaa, että suuri osa roomalaisista numeromerkeistä ovat jo itsessään arvoltaan niin suuria, että me tarvitsemme niiden nykyilmaisuun monta merkkiä! Nollaa roomalaisissa numeroissa ei ole, ja tiettävästi tuhatta suurempia arvoja esittäviä numeromerkkejä merkkejä otettiin käyttöön vasta keskiajalla. 
Lukuja koostetaan näistä merkeistä siten, että merkit kirjoitetaan peräkkäin pääasiassa laskevassa järjestyksessä ja niiden numeroarvot lasketaan yhteen. Jos arvoltaan pienempi numeromerkki (korkeintaan yksi) edeltää suurempaa, pienempi vähennetään suuremmasta ennen yhteenlaskun jatkamista. 

\chapter{Joukko-oppia}

(voisi integroida lukuoppiin) T: Joonas
\chapter{Logiikkaa}

(voisi integroida yhtälöiden teoriaan, sinne saa hyvin ekvivalenssin, implikaation ja disjunktion ja nepä ovat ne, mitä juuri tarvitaan) T: Joonas
\chapter{Kokonaisluvut}
\chapter{Kokonaislukujen aritmetiikkaa}
\chapter{Jaollisuus \& tekijät}
\chapter{Rationaaliluvut ja laskusäännöt}

Laske %aika randomit luvut
a) $\frac{6}{2} + \frac{3}{5}$
b) $\frac{7}{8} - \frac{1}{4}$
c) $2 \frac{1}{3} + \frac{4}{6}$
Vastaus:
a) $\frac{18}{5}$
b) $\frac{5}{8}$
c) $3$
\begin{enumerate}
\item $x^9$
\item $a^6$
\item $a^4$
\item $1$
\item $1000$
\end{enumerate}
\end{vastaus}
\end{tehtava}

\section{Murtolausekkeiden sieventäminen}
%tässä pitää opettaa binomin neliösäännöt ja ne (3kpl)
%Kaavojen johtaminen
\laatikko{
\begin{enumerate}
\item $(a+b)^2 = a \cdot a + a \cdot b + b \cdot a + b \cdot b = a^2 + ab + ba + b^2 = a^2 + 2ab + b^2 $
\item $(a-b)^2 = a \cdot a + a \cdot -b + (-b) \cdot a + (-b) \cdot (-b) = a^2 - ab - ba + b^2 = a^2 - 2ab + b^2 $
\item $(a+b)(a-b) = a \cdot a + a \cdot (-b) + b \cdot a + b \cdot (-b) = a^2 - ab + ba - b^2 = a^2 - b^2 $
\end{enumerate}}

\begin{tehtava}
%tää voi olla eka tehtävä
Sievennä
\begin{enumerate}
\item $\frac{a^2+2ab+b^2}{a+b}$
\item $\frac{a^2-2ab+b^2}{a-b}$
\item $\frac{a^2-b^2}{a+b}$
\end{enumerate}
\begin{vastaus}
Vastaus:
\begin{enumerate}
\item $a-b$
\item $a+b$
\item $a-b$
\end{enumerate}
\end{vastaus}
\end{tehtava}
>>>>>>> 5e28041e61842d3cd8b8fe172f1cb301f14ce64a

\chapter{Potenssisäännöt \& murtolausekkeiden sieventämistä}
\chapter{Juuret}

\section{Neliöjuuri}

\laatikko{Luvun $a$ neliöjuuri on ei-negatiivinen luku, jonka neliö on $a$. Tämä voidaan ilmaista lyhyemmin $\sqrt{b^2}=b$.}

Neliöjuuren määritteleminen $\sqrt{a}^2=a$ ei johda samaan lopputulokseen. Pohdi, miksi näin on.
%%vai parempi antaa suoraan $\sqrt{a}^2=a$, kun $a \ge 0$
Jatkossa tälaisia määritelmän pieniä muokkauksia ja niistä aiheutuvia muutoksia olisi aina hyvä pohdiskella -- saattavat jopa auttaa muistamaan määritelmän oikean muodon.
%%%%%%%%%%%%%%% ONKO ITSEISARVO KÄSITELTY!!!!! %%%%%%%%%%%%%%%%%%%%%%%%%

%Määritelmäksi ei kelpaisi tämäkään. $\sqrt{a^2}=|a|$ EI OLE KÄSITELTY. Tulee esimerkkinä funktiosta funktioaiheen jälkeen.

Neliöjuurta ei siis nyt määritelty ollenkaan negatiivisille luvuille.

%yhtälöt tulevat vasta myöhemin, siksi esimerkit köyhiä

Esimerkki
\begin{align*}
\sqrt{4} = 2\ qquad\textrm{, koska $2>0$ ja $2^2 =4$} 
\end{align*}

%pythagoraan lause on pitänyt käydä ennen tätä!!
Taulutelevision kooksi on ilmoitettu mainoksessa $46''$ ja kuvasuhteeksi 19:6. Kuinka leveä televisio on arviolta?
Vastaus: $44''$ tai 111 cm

\section{Kuutiojuuri}

\laatikko{Luvun $a$ kuutiojuuri on luku, jonka kuutio on $a$. Tämä voidaan ilmaista lyhyemmin $\sqrt[3]{b^3}=b$.
Määritelmäksi voisi ottaa myös $\sqrt[3]{b^3}=b$.}
%tämä jälkimmäinen on ehkä järkevämpi määritelmä.
%Olisi varmaan hyvä ottaa samanlainen määritelmä neliöjuuren tapauksessakin
%jolloin neliöjuureen määritelmään tulisi 2 ehtoa.
Kuutiojuuren voi siis ottaa mistä tahansa luvusta.
%vai reaaliluvusta?
%Reaaliluvuista puhutaan kuitenkin vasta myöhemmin, niin olkoon näin.


\section{n. juuri}
Kaikkia juuria ei kuitenkaan kannata määritellä yksitellen. Tehdään siis mahdollisimman paljon kerralla. Edeltä kuitenkin voi huomata, että kuutiojuuri on määritelty kaikille luvuille, mutta neliöjuuri vain ei-negatiivisille luvuille. Tämä toistuu myös muissa juurissa. Määritellään siis parilliset ja parittomat juuret erikseen.

%on ehkä parempi esittää nämä molemma samalla eikä kuten subsections
%sama määritelmä, mutta todetaan parillisilla vaadittavan >= 0.
%%%RISTIRIITA ED. KANSSA
Juurimerkinnällä $\sqrt[n]{a}=b$ (luetaan \emph{ännäs juuri aasta on bee} tarkoitetaan lukua, joka toteuttaa ehdon $b^n = a$. Jotta juuri olisi ykskäsitteinen, on parillisilla juurilla ($\sqrt{a}, \sqrt[4]{a}, \sqrt[6]{a}$\ldots) vaadittava, että $b\ge0$.

\subsection{parilliset juuret}

\laatikko{Luvun $a$ $n$.s juuri (luetaan \emph{ännäs juuri}) on ei-negatiivinen luku, jonka neliö on $a$. Tämä voidaan ilmaista lyhyemmin $\sqrt[n]{b^n}=b$.}

\subsection{parittomat juuret}
\laatikko{Luvun $a$ n.s juuri on ei-negatiivinen luku, jonka neliö on $a$. Tämä voidaan ilmaista lyhyemmin $\sqrt[n]{b^n}=b$.}

Nyt on paikallaan todeta, että toista juurta $\sqrt[2]{a}$ merkitään $\sqrt{a}$.

\begin{tabular}{c|c}
parillinen juuri & pariton juuri\\
\hline
$\sqrt[n]{a}^n=a$, $a\ge0$ & $\sqrt[n]{a}^n$, kaikilla $a$
\end{tabular}

Esimerkiksi $\sqrt[3]{-8}=-2$ koska $(-2)^2=-8$, mutta $\sqrt[4]{-8}$ ei ole määritelty, koska minkään luvun neljäs potenssi ei ole negatiivinen.

%Mitä näille kahdelle seuraavalle tehdään?
%$\sqrt[n]{ab}=\sqrt[n]{a}\sqrt[n]{b}$
%Jos n on parillinen, niin on lisäksi vaadittava, että $a\ge0$ ja $b\ge0$.
%
%$\sqrt[n]{\frac{a}{b}}=\frac{\sqrt[n]{a}}{\sqrt[n]{b}}$
%Jos n on parillinen, niin on lisäksi vaadittava, että $a\ge0$ ja $b\ge0$.

\input{Murtopotenssi}
\chapter{Irrationaaliluvut}
\chapter{Reaaliluvut}
\chapter{Kompleksiluvut}
\chapter{Kertaustiivistelmä}

%
\part{Yhtälöt}
\chapter{Yhtälöiden teoriaa}
Monissa käytännön tilanteissa saamme samalle asialle kaksi erilaista esitystapaa.

\begin{esimerkki}
Meillä on orsivaaka, joka on tasapainossa. (kuva!) Toisessa vaakakupissa on kahden kilon siika ja toisessa puolen kilon ahven sekä tuntematon määrä lakritsia. Kuinka paljon vaakakupissa on lakritsia? (Ratkaistaan...) (Muita esimerkkejä, vähitellen vaikeutuvia (1. asteen) yhtälöitä)
\end{esimerkki}

Määritelmä: Yhtälöksi kutsutaan kahden lausekkeen merkittyä yhtäsuuruutta. Siis mielivaltaisille lausekkeille $A$ ja $B$ merkitään $A=B$. (Esim. $A=3x+5$ ja $B=7x+7$). Jos yhtälön puolien lausekkeiden arvot ovat samat, sanotaan että yhtälö pätee. 

Yhtälössä voi esiintyä myös muuttujia, eli symboleja joiden arvoa ei ole etukäteen määritelty. Muuttujia merkitään usein kirjaimilla $x$, $y$ ja $z$. Niitä muuttujien arvoja, joilla yhtälö pätee, kutsutaan yhtälön ratkaisuiksi. Yhtälön ratkaisemisella tarkoitetaan kaikkien yhtälön ratkaisujen selvittämistä.

Eräs tapa ratkaista yhtälöitä on muokata niitä niin, että muokattu yhtälö pätee täsmälleen silloin kun alkuperäinen yhtälö pätee. Tällaisia sallittuja muunnoksia ovat esimerkiksi:
\begin{itemize}
\item Yhtälön molemmat puolet voidaan kertoa nollasta poikkeavalla luvulla $m$. Muutos tehdään aina molemmille puolille. Tällöin saadaan yhtälö $mA = mB$.
\item Yhtälön molemmille puolille voidaan lisätä tai molemmilta puolilta vähentää luku $n$.
Tällöin saadaan yhtälö $A+n = B+n$.
\end{itemize}
[pitäskö perustella?] JOOOO T. ANNIKA

Monet yhtälöt ratkeavat toistamalla tällaisia muunnoksia kunnes yhtälö on niin yksinkertaisessa muodossa, että ratkaisu on helppo nähdä. Koska jokaisessa muokkausjonon yhtälössä ratkaisut ovat samat, näin saadaan alkuperäisen yhtälön ratkaisut.

[joku esimerkki tähän?]

Yhtälöt voidaan ratkaisujensa perusteella jakaa kolmeen tyyppiin:
\begin{enumerate}
\item Yhtälö, joka on aina tosi. Esimerkiksi yhtälöt $8=8$ ja $x=x$.
\item Yhtälö, joka on joskus tosi. Esimerkiksi yhtälö $x+2=3$ on tosi jos ja vain jos $x=1$.
\item Yhtälö, joka ei ole koskaan tosi. Esimerkiksi yhtälö $0=1$.
\end{enumerate}
Tämän kurssin ja ylipäätään matematiikan kannalta selvästi tärkein yhtälötyyppi on 2. Siirrymme nyt tarkastelemaan tärkeää erikoistapausta yhtälöistä, ensimmäisen asteen yhtälöitä.

\chapter{Ensimmäisen asteen yhtälö}
Ensimmäisen asteen yhtälö on yhtälö, joka on esitettävissä muodossa $ax+b=0$, jossa $a \neq 0$.

\begin{theorem}
Kaikki muotoa $ax+b=cx+d$ olevat yhtälöt, joissa $a \neq c$, ovat ensimmäisen asteen yhtälöitä.
\end{theorem}

\begin{proof}
\begin{align*}
ax+b &= cx+d & &| \, \textbf{Vähennetään molemmilta puolilta $cx+d$.} \\
ax+b - (cx+d) &= 0 & &| \, 
\end{align*}
\end{proof}

\begin{theorem}
Yleinen lähemistymistapa muotoa $ax+b = cx+d$ olevien yhtälöiden ratkaisuun on: \\
(1) Vähennä molemmilta puolilta $cx$. Saat yhtälön $(a-c)x + b = d$. \\
(2) Vähennä molemmilta puolita $b$. Saat yhtälön $(a-c)x = d-b$. \\
(3) Jaa $(A-C)$:llä. Saat yhtälön ratkaistuun muotoon $x = \frac{d-b}{a-c}$.
\end{theorem}

Esimerkki. Yhtälön $7x+4=4x+7$ ratkaisu saadaan seuraavasti:
\begin{align*}
7x+4 &= 4x+7 & &| \, \text{Vähennetään molemmilta puolilta 4x.} \\
3x+4 &= 7 & &| \, \text{Vähennetään molemmilta puolilta 4.} \\
3x &= 3 & &| \, \text{Jaetaan molemmat puolet kolmella eli kerrotaan $\frac{1}{3}$:lla.} \\
x &= 1 & &| \, \text{Saimme yhtälön ratkaistuun muotoon. $x=1$ on siis yhtälön ratkaisu.} \\
\end{align*}

\begin{tehtava}
%
Ratkaise:
\begin{enumerate}
\item $x + 4 = 5$
\item $1 - x = -3$
\item $7x = 35$
\item $-2x = 4$
\item $10 - 2x = x$
\item $9x + 4 = 6 - x$
\item $\frac{2x}{5} = 4$
\item $\frac{x}{3} + 1 = \frac{5}{6} - x$
\end{enumerate}
\begin{vastaus}
\begin{enumerate}
\item $x=1$
\item $x=4$
\item $x=35/7$
\item $x=-2$
\item $x=10/3$
\item $x=1/5$
\item $x=10$
\item $x=-1/8$
\end{enumerate}
\end{vastaus}
\end{tehtava}

\begin{tehtava}
Ratkaise kysytty muuttuja yhtälöstä
\begin{enumerate}
\item $F=ma$, $m=?$ "voima on massa kertaa kiihtyvyys"
\item $p=\frac{F}{A}$, $F=?$ "paine on voima jaettuna alalla"
\item $A=\pi r^2$, $r=?$ "(pallon) pinta-ala on pii kertaa säde toiseen"
\item $V=\frac{1}{3} \pi r^2 h$, $h=?$ "(kartion) tilavuus on kolmasosa pii kertaa säde toiseen kertaa korkeus"
\end{enumerate}
\begin{vastaus}
\begin{enumerate}
\item $m=\frac{F}{a}$
\item $F=\frac{p}{a}$
\item $r=\sqrt{\frac{A}{\pi}}$
\item $h=\frac{V}{ \frac{1}{3} \pi r^2 h}$
\end{enumerate}
\end{vastaus}
\end{tehtava}

\chapter{Yhtälöpari}


\chapter{Yleinen potenssi ja potenssiyhtälö}
%pitää esitellä mitä on mega, milli, sentti jne.

\begin{tehtava}
Esitä luku ilman kymmenpotenssia.
\begin{enumerate}
\item $3,2 * 10^4$
\item $-7,03 * 10^{-5}$
\item $10,005 * 10^{-2}$
\end{enumerate}
\begin{vastaus}
Vastaus
\begin{enumerate}
\item $32000$
\item $-0,0000703$
\item $0,10005$
\end{enumerate}
\end{vastaus}
\end{tehtava}

\begin{tehtava}
Esitä luku ilman etuliitettä.
\begin{enumerate}
\item $0,5 dl$
\item $233 mm$
\item $33 cm$
\item $16 kg$
\item $2 MJ$
%\item %megatavu, mibitavu jne.
%\item 
\end{enumerate}
\begin{vastaus}
Vastaus:
\begin{enumerate}
\item $0,05 l$
\item $0,233 m$
\item $0,33 m$
\item $16 000 g$
\item $2 000 000 J$
%\item $ $
%\item $ $
\end{enumerate}
\end{vastaus}
\end{tehtava}


\chapter{Kertaustiivistelmä}

\part{Funktiot}
\chapter{Funktio}
Matematiikassa tutkitaan paljon suureiden välisiä riippuvuuksia. Tällaiset riippuvuudet voidaan muotoilla funktioiden avulla. Esimerkiksi tuotteen arvonlisäveroprosentti riippuu tuotteen tyypistä. Tämä riippuvuus voidaan kirjoittaa funktiona eri tuotetyyppien joukolta $A$ reaalilukujen joukolle $\mathbb{R}$, missä funktio liittää jokaiseen tuotteeseen sen arvonlisäveroprosentin.

[Esimerkki, kuva arvonlisäverofunktiosta, missä \[A = \{\text{ahvenfilee}, \text{AIV-rehu}, \text{auto}, \text{runokirja}, \text{ravintola-ateria}, \text{särkylääke}, \text{televisio}\},\]$f(\text{ahvenfilee}) = 13$, $f(\text{AIV-rehu}) = 13$, $f(\text{auto}) = 23$, $f(\text{runokirja}) = 9$, $f(\text{ravintola-ateria}) = 13$, $f(\text{särkylääke}) = 9$, $f(\text{televisio}) = 23$]
%\input{kuvat/funktio.tex}


\laatikko{Funktio $f$ joukosta $A$ joukkoon $B$ on sääntö, joka liittää $A$:n jokaiseen alkioon täsmälleen yhden $B$:n alkion. $A$ on tällöin $f$:n määrittelyjoukko ja $B$ sen maalijoukko. Funktion arvojoukko on kaikkien sen saamien arvojen joukko.}

Funktioita kutsutaan myös kuvauksiksi. Monesti funktion määrittely- ja maalijoukot jätetään merkitsemättä. Tällöin 

\chapter{Erilaisia funktioita}


\part{Luvut ja laskutoimitukset}
\chapter{Lähtötasotesti}

(Tää tulee oikeasti ennen tätä chapteria ja osaa)

\begin{tehtava}
\begin{enumerate}
\item Laske $2^2+2 \cdot 2+2$
\item sasdas
\item 
\end{enumerate}

\begin{vastaus}
\begin{enumerate}
\item 
\item
\item

\end{enumerate}
\end{vastaus}
\end{tehtava}
\chapter{Numerot ja luvut}

(Joonas jatkaa tästä vielä!)

Matematiikka tarjoaa työkaluja asioiden jäsentämiseen, päättelyyn ja mallintamiseen. Alasta riippuen käsittelemme matematiikassa erilaisia \textbf{objekteja}: Geometriassa tarkastelemme tasokuvioita ja kolmiulotteisia rakenteita. Algebrassa tutkii lukujen ja funktioiden ominaisuuksia. Todennäköisyyslaskenta arvioi erilaisten tapausten ja tilanteiden mahdollisuuksia ja riskejä. Matemaattinen analyysi (kurssit 7,8 ja 10) tutkii funktioita ja niiden muuttumista.

Jokaiseen tarkastelukohteeseen liitetään myös niille ominaisia \textbf{operaatioita}. Tämä kurssi käsittelee lähinnä lukuja ja niiden operaatioita, joita \textbf{laskutoimituksiksi} kutsutaan. Aloitetaan yksinkertaisista määritelmistä: mitä tarkoittavat \textbf{numero} ja \textbf{luku}?

\laatikko{Länsimaisessa ... on käytössämme kymmenen numeromerkkiä: 0, 1, 2, 3, 4, 5, 6, 7, 8 ja 9. Näitä kutsutaan hindu-arabialaisiksi numeroiksi.  }

Sanalla numero voidaan siis viitata yksittäiseen kirjoitettuun merkkiin. 

\begin{esimerkki}
Luku \[715531\] koostuu numeroista 7, 1, 5, 5, 3 ja 1.
\end{esimerkki}



Olennaista on myös...
lukujärjestelmä, paikkajärjestelmä

MIKSI KÄYTÄMME KIRJAIMIA?

suuruus, yhtäsuuruus, eri suuret

, ja Erilaisilla luvuilla voidaan suorittaa erilaisia laskutoimituksia. Seuraavissa luvuissa esitellään ja käydään läpi lukiomatematiikassa ja mahdollisissa jatko-opinnoissa käytettäviä lukujoukkoja ja tavallisimmat laskutoimitukset.

\chapter{Luonnolliset luvut}

(Joonas jatkaa tästä vielä!)

Suomen kielen verbi 'laskea' voi tarkoittaa matematiikassa kahta eri asiaa: lukumäärien laskemista ja laskutoimitusten suorittamista.

\laatikko{laskea (lukumäärä) englanti count ruotsi \_ /n
laskea (laskutoimitus) englanti calculate , ruotsi \_}

Ihmisellä ja muilla eläimillä on luonnostaan matemaattisia taitoja. Monet niistä, esimerkiksi lukumäärien laskeminen, ovat yllättävän monimutkaisia kognitiivisia prosesseja, jotka kehittyvät lapsuudessa – toisilla aiemmin, toisilla myöhemmin. Kaikki koulussa opeteltava peruslaskento ja myös matematiikka tieteen alana rakentavat tämän biologisen osaamisen päälle. Laskeminen itsessään on vain yksi matematiikan osa-alue, eikä kaikki matematiikka ole laskemista. Huomaa, että suomen kielen verbillä laskea tarkoitetaan sekä lukumäärien laskemista (engl. counting) että lukujen laskutoimitusten suorittamista (engl. calculating).

Hyvin olennaisena kehitysaskeleena niin yksilön matemaattiselle ajattelulle kuin yhteiskunnallekin on ollut luonnollisen kielen tavoin kyky merkitä lukumäärien laskemista ja muuta matemaattista pohdintaa kirjalliseen muotoon.  On olemassa hyvin monia erilaisia tapoja merkitä lukumääriä. Helpoin tapa ja yksinkertaisin tapa on käyttää vain yhtä samaa merkkiä ja toistaa sitä. Jos

käytettävissä olevien merkintöjä määrää voidaan lisätä, jolloin suuria lukuja voidaan kirjoittaa lyhyemmin. Tämä vastaa myös luonnollisten kielten tilannetta: Suomen kielen aakkosiin kuuluu 29 kirjainta, joista sanat muodostetaan. Sanat voivat olla kuinka pitkiä vain kahdesta kirjaimesta ylöspäin. Kiinassa sen sijaan käytetään omaa piirrosmerkkiä jokaiselle sanalle. Merkkejä täytyy osata 29 sijaan tuhansia, mutta jokaisen sanan voi kirjoittaa lyhyesti. 
Matematiikassa erilaisista numeromerkeistä tai yksinkertaisesti numeroista muodostetaan lukuja yhdistelemällä niitä sopivasti erilaisten paikkajärjestelmien mukaan. Esimerkiksi antiikin Roomassa käytössä olivat numeromerkit I, V, X, L, C , D ja M. Niiden numeroiden vastaavuudet meidän käyttämiimme lukuarvoihin ovat seuraavat:
I=1
V=5
X=10
L=50
C=100
D=500
M=1 000
Huomaa, että suuri osa roomalaisista numeromerkeistä ovat jo itsessään arvoltaan niin suuria, että me tarvitsemme niiden nykyilmaisuun monta merkkiä! Nollaa roomalaisissa numeroissa ei ole, ja tiettävästi tuhatta suurempia arvoja esittäviä numeromerkkejä merkkejä otettiin käyttöön vasta keskiajalla. 
Lukuja koostetaan näistä merkeistä siten, että merkit kirjoitetaan peräkkäin pääasiassa laskevassa järjestyksessä ja niiden numeroarvot lasketaan yhteen. Jos arvoltaan pienempi numeromerkki (korkeintaan yksi) edeltää suurempaa, pienempi vähennetään suuremmasta ennen yhteenlaskun jatkamista. 

\chapter{Joukko-oppia}

(voisi integroida lukuoppiin) T: Joonas
\chapter{Logiikkaa}

(voisi integroida yhtälöiden teoriaan, sinne saa hyvin ekvivalenssin, implikaation ja disjunktion ja nepä ovat ne, mitä juuri tarvitaan) T: Joonas
\chapter{Kokonaisluvut}
\chapter{Kokonaislukujen aritmetiikkaa}
\chapter{Jaollisuus \& tekijät}
\chapter{Rationaaliluvut ja laskusäännöt}

Laske %aika randomit luvut
a) $\frac{6}{2} + \frac{3}{5}$
b) $\frac{7}{8} - \frac{1}{4}$
c) $2 \frac{1}{3} + \frac{4}{6}$
Vastaus:
a) $\frac{18}{5}$
b) $\frac{5}{8}$
c) $3$
\begin{enumerate}
\item $x^9$
\item $a^6$
\item $a^4$
\item $1$
\item $1000$
\end{enumerate}
\end{vastaus}
\end{tehtava}

\section{Murtolausekkeiden sieventäminen}
%tässä pitää opettaa binomin neliösäännöt ja ne (3kpl)
%Kaavojen johtaminen
\laatikko{
\begin{enumerate}
\item $(a+b)^2 = a \cdot a + a \cdot b + b \cdot a + b \cdot b = a^2 + ab + ba + b^2 = a^2 + 2ab + b^2 $
\item $(a-b)^2 = a \cdot a + a \cdot -b + (-b) \cdot a + (-b) \cdot (-b) = a^2 - ab - ba + b^2 = a^2 - 2ab + b^2 $
\item $(a+b)(a-b) = a \cdot a + a \cdot (-b) + b \cdot a + b \cdot (-b) = a^2 - ab + ba - b^2 = a^2 - b^2 $
\end{enumerate}}

\begin{tehtava}
%tää voi olla eka tehtävä
Sievennä
\begin{enumerate}
\item $\frac{a^2+2ab+b^2}{a+b}$
\item $\frac{a^2-2ab+b^2}{a-b}$
\item $\frac{a^2-b^2}{a+b}$
\end{enumerate}
\begin{vastaus}
Vastaus:
\begin{enumerate}
\item $a-b$
\item $a+b$
\item $a-b$
\end{enumerate}
\end{vastaus}
\end{tehtava}
>>>>>>> 5e28041e61842d3cd8b8fe172f1cb301f14ce64a

\chapter{Potenssisäännöt \& murtolausekkeiden sieventämistä}
\chapter{Juuret}

\section{Neliöjuuri}

\laatikko{Luvun $a$ neliöjuuri on ei-negatiivinen luku, jonka neliö on $a$. Tämä voidaan ilmaista lyhyemmin $\sqrt{b^2}=b$.}

Neliöjuuren määritteleminen $\sqrt{a}^2=a$ ei johda samaan lopputulokseen. Pohdi, miksi näin on.
%%vai parempi antaa suoraan $\sqrt{a}^2=a$, kun $a \ge 0$
Jatkossa tälaisia määritelmän pieniä muokkauksia ja niistä aiheutuvia muutoksia olisi aina hyvä pohdiskella -- saattavat jopa auttaa muistamaan määritelmän oikean muodon.
%%%%%%%%%%%%%%% ONKO ITSEISARVO KÄSITELTY!!!!! %%%%%%%%%%%%%%%%%%%%%%%%%

%Määritelmäksi ei kelpaisi tämäkään. $\sqrt{a^2}=|a|$ EI OLE KÄSITELTY. Tulee esimerkkinä funktiosta funktioaiheen jälkeen.

Neliöjuurta ei siis nyt määritelty ollenkaan negatiivisille luvuille.

%yhtälöt tulevat vasta myöhemin, siksi esimerkit köyhiä

Esimerkki
\begin{align*}
\sqrt{4} = 2\ qquad\textrm{, koska $2>0$ ja $2^2 =4$} 
\end{align*}

%pythagoraan lause on pitänyt käydä ennen tätä!!
Taulutelevision kooksi on ilmoitettu mainoksessa $46''$ ja kuvasuhteeksi 19:6. Kuinka leveä televisio on arviolta?
Vastaus: $44''$ tai 111 cm

\section{Kuutiojuuri}

\laatikko{Luvun $a$ kuutiojuuri on luku, jonka kuutio on $a$. Tämä voidaan ilmaista lyhyemmin $\sqrt[3]{b^3}=b$.
Määritelmäksi voisi ottaa myös $\sqrt[3]{b^3}=b$.}
%tämä jälkimmäinen on ehkä järkevämpi määritelmä.
%Olisi varmaan hyvä ottaa samanlainen määritelmä neliöjuuren tapauksessakin
%jolloin neliöjuureen määritelmään tulisi 2 ehtoa.
Kuutiojuuren voi siis ottaa mistä tahansa luvusta.
%vai reaaliluvusta?
%Reaaliluvuista puhutaan kuitenkin vasta myöhemmin, niin olkoon näin.


\section{n. juuri}
Kaikkia juuria ei kuitenkaan kannata määritellä yksitellen. Tehdään siis mahdollisimman paljon kerralla. Edeltä kuitenkin voi huomata, että kuutiojuuri on määritelty kaikille luvuille, mutta neliöjuuri vain ei-negatiivisille luvuille. Tämä toistuu myös muissa juurissa. Määritellään siis parilliset ja parittomat juuret erikseen.

%on ehkä parempi esittää nämä molemma samalla eikä kuten subsections
%sama määritelmä, mutta todetaan parillisilla vaadittavan >= 0.
%%%RISTIRIITA ED. KANSSA
Juurimerkinnällä $\sqrt[n]{a}=b$ (luetaan \emph{ännäs juuri aasta on bee} tarkoitetaan lukua, joka toteuttaa ehdon $b^n = a$. Jotta juuri olisi ykskäsitteinen, on parillisilla juurilla ($\sqrt{a}, \sqrt[4]{a}, \sqrt[6]{a}$\ldots) vaadittava, että $b\ge0$.

\subsection{parilliset juuret}

\laatikko{Luvun $a$ $n$.s juuri (luetaan \emph{ännäs juuri}) on ei-negatiivinen luku, jonka neliö on $a$. Tämä voidaan ilmaista lyhyemmin $\sqrt[n]{b^n}=b$.}

\subsection{parittomat juuret}
\laatikko{Luvun $a$ n.s juuri on ei-negatiivinen luku, jonka neliö on $a$. Tämä voidaan ilmaista lyhyemmin $\sqrt[n]{b^n}=b$.}

Nyt on paikallaan todeta, että toista juurta $\sqrt[2]{a}$ merkitään $\sqrt{a}$.

\begin{tabular}{c|c}
parillinen juuri & pariton juuri\\
\hline
$\sqrt[n]{a}^n=a$, $a\ge0$ & $\sqrt[n]{a}^n$, kaikilla $a$
\end{tabular}

Esimerkiksi $\sqrt[3]{-8}=-2$ koska $(-2)^2=-8$, mutta $\sqrt[4]{-8}$ ei ole määritelty, koska minkään luvun neljäs potenssi ei ole negatiivinen.

%Mitä näille kahdelle seuraavalle tehdään?
%$\sqrt[n]{ab}=\sqrt[n]{a}\sqrt[n]{b}$
%Jos n on parillinen, niin on lisäksi vaadittava, että $a\ge0$ ja $b\ge0$.
%
%$\sqrt[n]{\frac{a}{b}}=\frac{\sqrt[n]{a}}{\sqrt[n]{b}}$
%Jos n on parillinen, niin on lisäksi vaadittava, että $a\ge0$ ja $b\ge0$.

\input{Murtopotenssi}
\chapter{Irrationaaliluvut}
\chapter{Reaaliluvut}
\chapter{Kompleksiluvut}
\chapter{Kertaustiivistelmä}

%
\part{Sovelluksia}
%
%(Pythagoraan lause)
%
\chapter{Verrannollisuus}

Verrannollisuudella tarkoitetaan tilannetta, jossa 

Suoraan verrannollisuus tarkoittaa, että kahden asian suhde pysyy vakiona. Jos toinen kaksinkertaistuu, kaksinkertaistuu toinenkin. Esimerkiksi kaupasta ostettujen hedelmien määrä ja kauppahinta ovat suoraan verrannollisia toisiinsa. Jos ostat kaksi kertaa enemmän banaaneja, joudut myös maksamaan kaksi kertaa enemmän. Hinnan ja ostettujen banaanien massan\footnote{Arkikielessä puhutaan yleensä painosta.} suhde on molemmissa tapauksissa vakio. Hedelmäesimerkin tapauksessa tätävakiota kutsutaan kilohinnaksi.

Matemaattisesti suoraan verrannollisuus merkitään seuraavasti. Jos suure $a$ on suoraan verrannollinen suureeseen $b$, merkitään
\begin{equation}
\frac{a}{b}=c,
\end{equation}
missä $c$ on vakio.

Kääntäen verrannollisuus tarkoittaa, että kahden asian tulo pysyy vakiona. Jos toinen kaksinkertaistuu, toinen puolittuu. Esimerkiksi nopeus ja matkaan tarvittava aika ovat kääntäen verrannollisia toisiinsa. Jos ajat koulumatkan kaksi kertaa nopeammin, matka-aika puolittuu.

Muita esimerkkejä kääntäen verrannollisuudesta ovat:
\begin{itemize}
\item .
\end{itemize}

Matemaattisesti kääntäen verrannollisuus merkitään seuraavasti. Jos suure $a$ on kääntäen verrannollinen suureeseen $b$, merkitään
\begin{equation}
ab=c,
\end{equation}
missä $c$ on vakio.


\chapter{Verrannollisuus: sovelluksia}

\begin{tehtava}
% Lyhyt matikka 1, s. 72
Pohdi, kuinka toinen suure muuttuu, kun toinen suure kaksinkertaistuu, kolminkertaistuu, puolittuu jne. Ovatko suureet suoraan verrannolliset?
\begin{enumerate}
\item kuljettu matka ja kulunut aika, kun keskinopeus on 30 km/h
\item kananmunien lukumäärä ja niiden kovaksi keittämiseen tarvittava keittoaika
\item hedelmätiskiltä valitun vesimelonin paino ja hinta
\item neliön sivun pituus ja neliön pinta-ala
\end{enumerate}
\begin{vastaus}
Vastaus:
\begin{enumerate}
\item Ovat.
\item Eivät ole.
\item Ovat.
\item Eivät ole, sillä esimerkiksi kun neliön sivun pituus kaksinkertaistuu 1 cm:stä 2 cm:iin, niin neliön pinta-ala nelinkertaistuu 1 cm$^2$:stä 4 cm$^2$:iin.
\end{enumerate}
\end{vastaus}
\end{tehtava}

\begin{tehtava}
Isi ja lapset ovat ajamassa mökille Sotkamoon. Ollaan ajettu jo neljä viidennestä matkasta ja aikaa on kulunut kaksi tuntia. "Joko ollaan perillä?" kysyvät lapset takapenkiltä. Kuinka pitkään vielä arviolta kuluu, ennen kuin ollaan mökillä?
\begin{vastaus}
Vastaus: 1 h 15 min
\end{vastaus}
\end{tehtava}

\begin{tehtava}
Äidinkielen kurssilla annettiin tehtäväksi lukea eräs 300-sivuinen romaani. Eräs opiskelija otti aikaa ja selvitti lukevansa vartissa seitsemän sivua. Kuinka monta tuntia häneltä kuluu arviolta koko romaanin lukemiseen, jos taukoja ei lasketa?
\begin{vastaus}
Vastaus: 642 minuuttia eli 10 h 42 min.
\end{vastaus}
\end{tehtava}

\chapter{Prosenttilaskentaa - perustilanteet}

Kun lukuja verrataan toisiinsa, lasketaan niiden suhde eli osamäärä. Tämä kertoo jaettavan suhteellisen osuuden jakajasta. Suhteellinen osuus ilmaistaan usein prosentteina. Yksi prosentti tarkoittaa yhtä sadasosaa. Prosentin merkki on $\%$.

\laatikko{1 prosentti $= 1 \% = \frac{1}{100} = 0,01$}

\laatikko{Esimerkki: \\$6 \% = \frac{6}{100} = 0,06$, $48,2 \% = \frac{48}{100} = 0,482$, $140 \% = \frac{140}{100} = 1,40$}

Minkä tahansa suhdeluvun voi muuttaa prosenteiksi laskemalla osamäärän desimaalilukuna ja ottamalla siitä sadasosat.
% Pitäisi muotoilla se, että tuhannesosat ovat sitten 0,1 prosenttia jne. eikä että katkaistaan desimaaliluku sadasosiin :)

Kahden prosenttiluvun välisen erotuksen yksikköä kutsutaan prosenttiyksiköksi.

\section{Perusprosenttilaskut}

\begin{itemize}
	\item Prosenttiluvun laskeminen
	\item Prosenttiarvon laskeminen
	\item Perusarvon laskeminen
\end{itemize}

\section{Vertailu prosenttien avulla}

\begin{itemize}
	\item Muutosprosentti, vertailuprosentti
	\item Prosentuaalinen muutos
	\item Prosenttiyksikkö
\end{itemize}

\chapter{Prosenttiyhtälöitä ja sovelluksia}

\begin{tehtava}
Laukku maksaa 225 \euro ja on 25\%:n alennuksessa. Paljonko alennettu hinta on?
\begin{vastaus}
Vastaus: 168,75 \euro
\end{vastaus}
\end{tehtava}

\begin{tehtava}
%Pyramidi 1, s. 80
Kirjan myyntihinta, joka sisältää arvolisäveron, on 8\% suurempi kuin kirjan veroton hinta. Laske kirjan veroton hinta, kun myyntihinta on 15\euro.
\begin{vastaus}
Vastaus: 13,89 \euro
\end{vastaus}
\end{tehtava}

\begin{tehtava}
Perussuomalaisten kannatus oli vuoden 2007 eduskuntavaaleissa 4,1\% ja vuoden 2011 eduskuntavaaleissa 19,1\%. Kuinka monta prosenttiyksikköä kannatus nousi? Kuinka monta prosenttia kannatus nousi?
\begin{vastaus}
Vastaus: Kannatus nousi 15 \%-yksikköä ja 365,9 \%.
\end{vastaus}
\end{tehtava}

\begin{tehtava}
Askartelukaupassa on alennusviikot, ja kaikki tavarat myydään 60\%n alennuksella. Viimeisenä päivänä kaikista hinnoista annetaan vielä lisäalennus, joka lasketaan aiemmin alennetusta hinnasta. Minkä suuruinen lisäalennus tulee antaa, jos lopullisen kokonaisalennuksen halutaan olevan 80\%?
\begin{vastaus}
Vastaus: 50\%.
\end{vastaus}
\end{tehtava}

\begin{tehtava}
%tässä tehtävässä pitää tietää potenssi
Erään pankin myöntämä opintolaina nousee korkoa 2\% vuodessa. Kuinka monta prosenttia laina on noussut korkoa alkuperäiseen summaan verrattuna kymmenen vuoden kuluttua?
\begin{vastaus}
Vastaus: 22\%.
\end{vastaus}
\end{tehtava}

Ansiotuloverotus on Suomessa progressiivista: suuremmista tuloista maksetaan

\begin{tehtava}
Tuoreissa omenissa on vettä 80\% ja sokeria 4\%. Kuinka monta prosenttia sokeria on samoissa omenissa, kun ne on kuivattu siten, että kosteusprosentti on 20? [K2000, 4]
\begin{vastaus}
Vastaus: 16\%
\end{vastaus}
\end{tehtava}

%
%(Eksponentiaalinen malli)
%
\chapter{Kertaustiivistelmä}

%
\part{Kertaus ja harjoituskokeita}
\chapter{Verrannollisuus}

    Kertausosio (teoria ja esimerkit)
    Kertaustehtäväsarjoja
    Harjoituskokeita
    “Näihin pystyt jo” -yo-tehtäviä (myös lyhyestä)
    “Näihin pystyt jo” -pääsykoetehtäviä (moooonilta eri     aloilta! kauppatieteellinen, tradenomi (jos löytyy), kansantaloustiede, arkkitehtuuri, DI-haku, AMK tekniikan alat, fysiikka, tilastotiede, ...)
    Vastauksia ja ratkaisuja
    Suomi-ruotsi-englanti-sanasto ja hakemisto
    symbolitaulukko
\end{document}

% -*- coding: utf-8 -*-

%%% Local Variables: 
%%% mode: latex
%%% TeX-master: t
%%% coding: utf-8
%%% End: 

