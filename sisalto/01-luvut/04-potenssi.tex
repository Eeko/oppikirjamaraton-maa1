\chapter{Potenssi}
    
    Jos $a$ on reaaliluku ja $n$ on positiivinen kokonaisluku, potenssilla $a^n$ tarkoitetaan tuloa
    \[
        a^n = \underbrace{a\cdot \ldots \cdot a}_{n\text{ kpl}}. 
    \]
    Lukua $a$ kutsutaan potenssin \emph{kantaluvuksi} ja lukua $n$ \emph{eksponentiksi}. Luvun toista potenssia $a^2$ kutsutaan myös luvun $a$ \emph{neliöksi} ja kolmatta potenssia $a^3$ sen \emph{kuutioksi}. Luvun $a>0$ neliö on sellaisen neliön pinta-ala, jonka sivun pituus on $a$. Vastaavasti, jos kuution särmän pituus on $a$, niin $a^3$ on kyseisen kuution tilavuus.
    
    \begin{esimerkki}
        Laske $2^4$.
        
        \textbf{Ratkaisu.}
        Potenssissa $2^4$ luku $2$ on kantaluku ja luku $4$ on eksponentti.
        Merkinnällä $2^4$ siis tarkoitetaan tuloa $2\cdot 2\cdot 2\cdot 2$. Siten
        \[
            2^4=2\cdot 2\cdot 2\cdot 2=16.
        \]
        
        \textbf{Vastaus.} $16$.
    \end{esimerkki}
            
    \begin{esimerkki}
        Potenssit luvuilla.
        \begin{enumerate}[a)]
            \item $(-2)^3 = (-2)\cdot (-2)\cdot (-2) = -8$
            \item $(-2)^4 = (-2)\cdot (-2)\cdot (-2)\cdot (-2) = 16$
            \item $-2^4   = -(2^4) = -(2\cdot 2\cdot 2\cdot 2) = -16$
            \item $2^2\cdot 2^3 =
                \underbrace{2\cdot 2}_{\text{$2$ kpl}}\cdot \underbrace{2\cdot
                2\cdot 2}_{\text{$3$ kpl}} = \underbrace{2\cdot 2}_{\text{$5$ kpl}}=2^5 = 32$
            \item $\frac{2^4\cdot 2^2}{2^3} =
                \frac{2\cdot 2\cdot 2\cdot \cancel{2} \cdot \cancel{2}\cdot
                \cancel{2}}{\cancel{2}\cdot \cancel{2}\cdot \cancel{2}} = 2^3 = 8$
        \end{enumerate}
    \end{esimerkki}
    
\section*{Potenssien laskusääntöjä}
    
    Samankantaisten potenssien kertolasku\\
    a) $a^3\cdot a^4=\underbrace{a\cdot a\cdot a}_{\text{$3$ kpl}}\cdot \underbrace{a\cdot a\cdot a\cdot a}_{\text{$4$ kpl}}=a^{\mathbf{3+4}}=a^7$
    
    Samankantaisten potenssien jakolasku\\
    b) $\frac{a^7}{a^4}=\frac{a\cdot a\cdot a\cdot \cancel{a}\cdot \cancel{a}\cdot \cancel{a}\cdot \cancel{a}}{\cancel{a}\cdot \cancel{a}\cdot \cancel{a}\cdot \cancel{a}}=a^{\mathbf{7-4}}=a^3$
    
    Potenssin potenssi\\
    c) $(a^2)^3=\underbrace{a^2\cdot a^2\cdot a^2}_{3\text{ kpl}}=\underbrace{a\cdot a\cdot a\cdot a\cdot a\cdot a}_{\text{$2\cdot 3=6$  kpl}}=a^{\mathbf{2\cdot 3}}=a^6$
    
    Tulon potenssi\\
    d) $(ab^5)^3=ab^5\cdot ab^5\cdot ab^5=a\cdot a\cdot a\cdot b^5\cdot b^5\cdot b^5=a^{\mathbf{1\cdot 3}}\cdot b^{\mathbf{5\cdot 3}}=a^3b^{15}$
    
    Osamäärän potenssi\\
    e) $\left(\frac{a^9}{b^7}\right)^3=\frac{a^9}{b^7}\cdot \frac{a^9}{b^7}\cdot \frac{a^9}{b^7}=\frac{a^9\cdot a^9\cdot a^9}{b^7\cdot b^7\cdot b^7}=\frac{a^{\mathbf{9\cdot 3}}}{b^{\mathbf{7\cdot 3}}}=\frac{a^{27}}{a^{21}}$

    Potensseille pätevät seuraavat laskusäännöt.
    
    \laatikko{
        \textbf{Potenssien laskusääntöjä}
    
        Olkoot $m$ ja $n$ kokonaislukuja ja $a$ reaaliluku.
        
        \begin{tabular}{ll}
            $a^m\cdot a^n            = a^{m+n}$ & Samakantaisten potenssien tulo\\
            $\frac{a^m}{a^n}         = a^{m-n}$, $a\neq 0,\,m>n$ &Samakantaisten potenssien osamäärä\\
            $(a\cdot b)^n            = a^n\cdot b^n$ & Tulon potenssi\\
            $\Big(\frac{a}{b}\Big)^n = \frac{a^n}{b^n}$, $b\neq 0$ & Osamäärän potenssi\\ 
            $(a^m)^n                 = a^{m\cdot n}$ & Potenssin potenssi\\
        \end{tabular}
    }
    
\section*{Negatiivinen luku ja nolla eksponenttina}
    
    Sievennetään osamäärä $\frac{a^3}{a^5}$ kahdella eri tavalla. Supistamalla yhteiset tekijät lopputulos
    
    \begin{equation*}
        \frac{a^3}{a^5} =
        \frac{\cancel{a}\cdot \cancel{a}\cdot \cancel{a}}{a\cdot a\cdot
        \cancel{a}\cdot \cancel{a}\cdot \cancel{a}} = 
        \frac{1}{a\cdot a}=\boldsymbol{\frac{1}{a^2}}
    \end{equation*}
    
    on erinäköinen, kuin käyttämällä potenssien laskusääntöjä
    
    \begin{equation*}
        \frac{a^3}{a^5} = a^{3-5}=\boldsymbol {a^{-2}}{.}
    \end{equation*}
    
    Koska lähtötilanne on molemmissa tapauksissa sama, voidaan määritellä, että
    
    \begin{equation*}
        a^{-2} = \frac{1}{a^2}
    \end{equation*}
    
    \laatikko{
        Yleisesti määritellään (kun $x\neq 0$)
        \begin{equation*}
            x^{-n} = \frac{1}{x^n}
        \end{equation*}
    }
    
    \laatikko{
        \textbf {Negatiivinen eksponentti} (kun $a\neq 0$)
        \begin{equation*}
        	a^{-n} = \frac{1}{a^n}
        \end{equation*}
    }
    
    Kun eksponenttina on nolla, on tuloksena aina luku $1$ kaikilla nollasta poikkeavilla kantaluvuilla. 
    Esimerkiksi
    \[
        \frac{a^3}{a^3}=\frac{\cancel{a}\cdot \cancel{a}\cdot \cancel{a}}{\cancel{a}\cdot \cancel{a}\cdot \cancel{a}}=1,
    \]
    mutta sama lasku voidaan laskea myös toisella tavalla: $\frac{a^3}{a^3}=a^{3-3}=a^0$.

Merkintää $0^0$ ei ole määritelty.
    
    \laatikko{
        \textbf{Eksponenttina nolla} (kun $a\neq 0$)
        \begin{equation*}
            a^{0}=1
        \end{equation*}
    }
    
    \section{laskujärjestys}
    
    \laatikko{
        \begin{enumerate}
            \item Sulut
            \item Potenssilaskut
            \item Kerto- ja jakolaskut vasemmalta oikealle
            \item Yhteen- ja jakolaskut vasemmalta oikealle
        \end{enumerate}
    }
   
    %perustehtäviä
	Sievennä.
    \begin{tehtava}%perteht
        %Sievennä
		\begin{enumerate}
        	\item $2^3 $ 
        	\item $aaaa$ 
        	\item $a^3a^2$ 
        	\item $0^4$
		\end{enumerate}        
        \begin{vastaus}
        \begin{enumerate}
            \item $8$ 
            \item $a^4$ 
            \item $a^5$ 
            \item $0$
        \end{enumerate}
        \end{vastaus}
    \end{tehtava}

    \begin{tehtava}%perteht
        %Sievennä
        \begin{enumerate}
        	\item $a^2a^5 $ 
        	\item $\frac{a^5}{a^3}$ 
        	\item $(a^3)^2$ 
        	\item $12^0$
		\end{enumerate}        
        \begin{vastaus}
        \begin{enumerate}
            \item $a^7$ 
            \item $a^2$ 
            \item $a^6$ 
            \item $1$
        \end{enumerate}
        \end{vastaus}
    \end{tehtava}    
    
    %soveltavia tehtäviä
        
    \begin{tehtava}%sovteht
        %Sievennä
        \begin{enumerate}
        	\item $a^2(-a^4) $ 
        	\item $(ab^2)^0$ 
        	\item $(3a)^3$ 
        	\item $(a^5b^3)^3$
		\end{enumerate}        
        \begin{vastaus}
        \begin{enumerate}
            \item $-a^6$ 
            \item $1$ 
            \item $27a^3$ 
            \item $a^{15}b^9$
        \end{enumerate}
        \end{vastaus}
    \end{tehtava} 
    
    \begin{tehtava}%sovteht
        %Sievennä
        \begin{enumerate}
        	\item $\frac{2^7}{2^9}$ 
        	\item $\frac{a^3}{a}$ 
        	\item $\left(\frac{1}{3}\right)^2$ 
        	\item $\left(\frac{a^{-2}}{ab^4}\right)^4$
		\end{enumerate}        
        \begin{vastaus}
        \begin{enumerate}
            \item $\frac{1}{4}$ 
            \item $a^2$ 
            \item $\frac{1}{9} $ 
            \item $ \left(\frac{1}{a^{12}b^{16}}\right)$ tai $a^{-12}b^{-16}$
        \end{enumerate}
        \end{vastaus}
    \end{tehtava}     
            
        
        
        
    %sekalaisia tehtäviä (tehtäväpankki)
    
    %Tehtävät ovat tarkoitettu todella heikoille
    %Näitä lienee liikaa. Karsitaanko vai siirretäänkö muualle?
    %Tarvitaan myös haastavia tehtäviä.
    Sievennä.
    \begin{tehtava}
        %Sievennä \quad
        a) $a\cdot a\cdot a$ \quad
        b) $a\cdot a\cdot a\cdot b\cdot b\cdot b\cdot b$ \quad
        c) $a\cdot b\cdot a\cdot b\cdot a\cdot b\cdot a$
        
        \begin{vastaus}
            a) $a^3$ \qquad
            b) $a^3b^4$ \qquad
            c) $a^4b^3$
        \end{vastaus}
    \end{tehtava}
    
    \begin{tehtava}
        %Sievennä \quad
        a) $a^2\cdot a^3$ \qquad
        b) $a^3a^2$ \qquad
        c) $a^2 a$ \qquad
        d) $a a^2 a$ \qquad
        e) $a^2a^1a^3$
        
        \begin{vastaus}
            a) $a^5$ \qquad
            b) $a^5$ \qquad
            c) $a^3$ \qquad
            d) $a^4$ \qquad
            e) $a^6$
        \end{vastaus}
    \end{tehtava}
    
    \begin{tehtava}
        %Sievennä \quad
        a) $a^0$ \qquad
        b) $a^0a^0$ \qquad
        c) $a a^1$ \qquad
        d) $aa^0$ \qquad
        e) $a^0a^1$
        
        \begin{vastaus}
            a) $1$ \quad ($a\neq0$, koska $0^0$ ei ole määritelty) \qquad
            b) $1$ \qquad
            c) $a$ \qquad
            d) $a^2$ \qquad
            e) $a$
        \end{vastaus}
    \end{tehtava}
    
    \begin{tehtava}
        %Sievennä \quad
        a) $a^1 a a^2$ \qquad
        b) $aaaa$ \qquad
        c) $a^3ba^2$ \qquad
        d) $aba^0ba^1$
        
        \begin{vastaus}
            a) $ a^4$ \qquad
            b) $a^4$ \qquad
            c) $a^5b$ \qquad
            d) $a^2b^2$
        \end{vastaus}
    \end{tehtava}
    % teht 5
    
    \begin{tehtava}
        %Sievennä \quad
        a) $(-2)\cdot(-2)\cdot(-2)$ \qquad
        b) $(-1)\cdot(-1)\cdot(-1)\cdot(-1)$
        
        \begin{vastaus}
            a) $ -8$ \qquad
            b) $1$
        \end{vastaus}
    \end{tehtava}
    
    \begin{tehtava}
        %Sievennä \quad
        a) $(-a)\cdot(-a)$ \qquad
        b) $(-a)\cdot(-a)\cdot(-b)$ \qquad
        c) $(-a^2)\cdot(-a^2)$

        \begin{vastaus}
            a) $a^2$ \qquad
            b) $-a^2b$ \qquad
            c) $a^4$
        \end{vastaus}
    \end{tehtava}

    \begin{tehtava}
        %Sievennä \quad
        a) $-a\cdot(-a)$ \qquad
        b) $-a\cdot(-a)\cdot(-b)$ \qquad
        c) $-a^2\cdot(-a^2)$
    
        \begin{vastaus}
            a) $a^2$ \qquad
            b) $-a^2b$ \qquad
            c) $a^4$
        \end{vastaus}
    \end{tehtava}

    \begin{tehtava}
        %Sievennä \quad
        a) $-a^3\cdot(-a^2)$ \qquad
        b) $a\cdot(-a)\cdot(-b)$ \qquad
        c) $a^2\cdot(-a^2)$
        
        \begin{vastaus}
            a) $a^5$ \qquad
            b) $a^2b$ \qquad
            c) $-a^4$
        \end{vastaus}
    \end{tehtava}

    \begin{tehtava}
        %Sievennä \quad
        a) $2^3\cdot2^3$ \qquad
        b) $4^3$ \qquad
        c) $(2^2)^3$ \qquad
        d) $2^{2+2+2}$

        \begin{vastaus}
            a) $64$ \qquad
            b) $64$ \qquad
            c) $64$ \qquad
            d) $64$
        \end{vastaus}
    \end{tehtava}

    %teht. 10
    \begin{tehtava}
        %Sievennä \quad
        a) $0^3\cdot0^3\cdot0^3$ \qquad
        b) $3^1$ \qquad
        c) $2^{2+3}$ \qquad
        d) $2^{6-4}$ \qquad
        e) $5^0$

        \begin{vastaus}
            a) $1$ \qquad
            b) $3$ \qquad
            c) $32$ \qquad
            d) $4$ \qquad
            e) $1$
        \end{vastaus}
    \end{tehtava}
    \begin{tehtava}
        %Sievennä \quad
        a) $(a^3)^1$ \qquad
        b) $(a^6)^2$ \qquad
        c) $(a^2)^4$ \qquad 
        d) $(a^1)^3$ \qquad
        e) $(a^0)^5$

        \begin{vastaus}
            a) $a^3$ \qquad
            b) $a^{12}$ \qquad
            c) $a^8$ \qquad
            d) $a^3$ \qquad
            e) $1$
        \end{vastaus}
    \end{tehtava}
    \begin{tehtava}
        %Sievennä \quad
        a) $a^3\cdot a^2\cdot a^5$ \qquad
        b) $(a^3)^2$ \qquad
        c) $(a^2a^3)^4$ \qquad
        d) $a^{7-2}$

        \begin{vastaus}
            a) $a^{10}$ \qquad
            b) $a^6$ \qquad
            c) $a^{20}$ \qquad
            d) $a^5$
        \end{vastaus}
    \end{tehtava}
    \begin{tehtava}
        %Sievennä \quad
        a) $(1\cdot a)^3$ \qquad
        b) $(a\cdot 2)^2$ \qquad
        c) $(-2abc)^3$ \qquad
        d) $(3a)^4$

        \begin{vastaus}
            a) $a^3$ \qquad
            b) $4a^2$ \qquad
            c) $-8a^3b^3c^3$ \qquad
            d) $91a^4$
        \end{vastaus}
    \end{tehtava}

    \begin{tehtava}
        %Sievennä \quad
        a) $a^3\cdot b^2\cdot a^5$ \qquad
        b) $(-ab^3)^2$ \qquad
        c) $(a^5a^4)^3$ \qquad
        d) $b(ab)^2$

        \begin{vastaus}
            a) $a^8b^2$ \qquad
            b) $a^2b^6$ \qquad
            c) $a^{15}b^{12}$ \qquad
            d) $a^2b^3$
        \end{vastaus}
    \end{tehtava}

    %teht.15
    \begin{tehtava}
        %Sievennä \quad
        a) $b^3(ab^0)^2$ \qquad
        b) $(ab^3)^0$ \qquad
        c) $(aa^4)^3a^2$ \qquad
        d) $a^3(b^2a)^5$

        \begin{vastaus}
            a) $a^2b^3$ \qquad
            b) $1$ \qquad
            c) $a^{17}$ \qquad
            d) $a^8b^{10}$
        \end{vastaus}
    \end{tehtava}

    \begin{tehtava}
        %Sievennä \quad
        a) $(1^3\cdot 2^2)^2$ \qquad
        b) $(1^2\cdot 2^3)^2$ \qquad
        c) $(2^2a^4)^2$ \qquad
        d) $b(3b)^3$

        \begin{vastaus}
            a) $16$ \qquad
            b) $64$ \qquad
            c) $16a^8$ \qquad
            d) $27b^4$
        \end{vastaus}
    \end{tehtava}
    
    \begin{tehtava}
        %Sievennä \quad
        a) $(a^3b^2)^2$ \qquad
        b) $a(a^2b^3)^4$ \qquad
        c) $(b^2a^4)^5$ \qquad
        d) $b(2ab^2)^3$
        
        \begin{vastaus}
            a) $a^6b^4$ \qquad
            b) $a^9b^{12}$ \qquad
            c) $a^{20}b^{10}$ \qquad
            d) $8a^3b^7$
        \end{vastaus}
    \end{tehtava}
    
    \begin{tehtava}
        %Sievennä \quad
        a) $\frac{2^3}{2^2}$ \qquad
        b) $\frac{2^4}{2^2}$ \qquad
        c) $\frac{2^3}{2^1}$ \qquad
        d) $\frac{2^3}{2^0}$ \qquad
        e) $\frac{2^3}{2^4}$ \qquad
        f) $\frac{2^3}{2^5}$
        
        \begin{vastaus}
            a) $2$ \qquad
            b) $4$ \qquad
            c) $4$ \qquad
            d) $8$ \qquad
            e) $\frac{1}{2}$ \qquad
            f) $\frac{1}{4}$
        \end{vastaus}
    \end{tehtava}
    
    \begin{tehtava}
        %Sievennä \quad
        a) $\frac{a^3}{a^2}$ \qquad
        b) $\frac{a^4}{a^2}$ \qquad
        c) $\frac{a^3}{a^1}$ \qquad
        d) $\frac{a^3}{a^0}$ \qquad
        e) $\frac{a^3}{a^4}$ \qquad
        f) $\frac{a^3}{a^5}$
        
        \begin{vastaus}
            a) $a$ \qquad
            b) $a^2$ \qquad
            c) $a^2$ \qquad
            d) $a^3$ \qquad
            e) $a^{-1} = \frac{1}{a}$ \qquad
            f) $a^{-2} = \frac{1}{a^2}$
        \end{vastaus}
    \end{tehtava}
    
    %teht. 20
    \begin{tehtava}
        %Sievennä \quad
        a) $\frac{a^2b^2}{ab}$ \qquad
        b) $\frac{a^2b}{a^2}$ \qquad
        c) $\frac{a^3}{a^3}$ \qquad
        d) $\frac{1}{a^0}$ \qquad
        e) $\frac{ab^3}{-b^4}$
        
        \begin{vastaus}
            a) $ab$ \qquad
            b) $b$ \qquad
            c) $1$ \qquad
            d) $1$ \qquad
            e) $-\frac{a}{b}$
        \end{vastaus}
    \end{tehtava}
    
    Sievennä ja kirjoita potenssiksi, jonka eksponentti on positiivinen.
    
    \begin{tehtava}
        %Sievennä \quad
        a) $a^{-3}$ \qquad
        b) $\frac{a}{a^3}$ \qquad
        c) $a^{-2}\cdot a^5$ \qquad
        d) $\frac{b}{a^4}b^{-4}$ \qquad
        e) $\frac{a^3}{a^{-5}}$
        
        \begin{vastaus}
            a) $\frac{1}{a^3}$ \qquad
            b) $\frac{1}{a^2}$ \qquad
            c) $a^3$ \qquad
            d) $\frac{}{a^4b^3}$ \qquad
            e) $a^8$
        \end{vastaus}
    \end{tehtava}
    
    Esitä ilman sulkuja ja sievennä.
    
    \begin{tehtava}
        %Esitä ilman sulkuja ja sievennä \quad
        a) $(\frac{1}{2})^2$ \qquad
        b) $(\frac{1}{3})^3$ \qquad
        c) $(\frac{a}{b})^4$ \qquad
        d) $(\frac{a^2}{b^3})^2$ \qquad
        e) $\left(\frac{a^2}{ab^2}\right)^2$
        
        \begin{vastaus}
            a) $\frac{1}{4}$ \qquad
            b) $\frac{1}{27}$ \qquad
            c) $\frac{a^4}{b^4}$ \qquad
            d) $\frac{a^4}{b^6}$ \qquad
            e) $\frac{a^2}{b^4}$
        \end{vastaus}
    \end{tehtava}
    
    \begin{tehtava}
        %Esitä ilman sulkuja ja sievennä \quad
        a) $(\frac{1}{2})\cdot(\frac{1}{2})$ \qquad
        b) $(-\frac{ab^2}{a^2b})^3$ \qquad
        c) $(-a^4b^4)^2$ \qquad
        d) $\left((\frac{a}{b})^4\right)^2$
        
        \begin{vastaus}
            a) $\frac{1}{4}$ \qquad
            b) $-\frac{b^3}{a^3}$ \qquad
            c) $a^8b^8$ \qquad
            d) $\frac{a^8}{b^8}$
        \end{vastaus}
    \end{tehtava}
    
    \begin{tehtava}
        %Esitä ilman sulkuja ja sievennä \quad
        a) $\frac{6a^9}{3a^2}$ \qquad
        b) $(\frac{a^2b^{-2}}{a^2b})^{-3}$ \qquad
        c) $\frac{5a^2}{-15a}$ \qquad
        d) $\left(-(\frac{10^3}{100b})^2 b^{-1} \right )^2$
        
        \begin{vastaus}
            a) $2a^7$ \qquad
            b) $b^9$ \qquad
            c) $-\frac{1}{3}a = -\frac{a}{3}$ \qquad
            d) $\frac{10\ 000}{b^6}$
        \end{vastaus}
    \end{tehtava}

\section{Murtolausekkeiden sieventäminen}

\laatikko{
Jos murtoluvun osoittajassa tai nimittäjässä on summa, jonka osilla on yhteinen tekijä, sen voi ottaa \emph{yhteiseksi tekijäksi} sulkujen eteen. Jos osoittajassa ja nimittäjässä on sen jälkeen sama kerroin, sen voi jakaa pois molemmista eli \emph{supistaa} pois.
\begin{equation}
\frac{ac+bc}{c} = \frac{ \cancel{c} (a+b)}{\cancel{c}} = a+b
\end{equation}


Joskus murtolauseke sieventyy, jos sen esittääkin kahden murtoluvun summana.
\begin{equation}
\frac{ca+b}{c} = \frac{ca}{c} + \frac{b}{c} = a + \frac{b}{c}
\end{equation}
}

Kun jakaa kolme erikokoista nallekarkkipussia ($a$, $b$ ja $c$) tasan kolmen ihmisen kesken, on sama, laittaako kaikki ensin samaan kulhoon ja jakaa ne sitten ($\frac{a+b+c}{3}$) vai jakaako jokaisen pussin erikseen ($ \frac{a}{3} + \frac{b}{3} + \frac{c}{3}$).

Jos taas samat kolme henkilöä jakavat keskenään pussin tikkareita ($6$ kpl) ja yhden pussin nallekarkkeja ($n$ kpl), niin saadaan seuraavanlainen lasku: $ \frac{6\text{ tikkaria}+n\text{ nallekarkkia}}{3} = \frac{6\text{ tikkaria}}{3} + \frac{n\text{ nallekarkkia}}{3} = \frac{\cancel{3} \cdot 2\text{ tikkaria}}{\cancel{3}} + \frac{n\text{ nallekarkkia}}{3} = 2\text{ tikkaria} + \frac{n\text{ nallekarkkia}}{3}$. Toisin sanoen, kukin saa kaksi tikkaria ja kuinka paljon ikinä onkaan kolmasosa kaikista nallekarkeista.

\laatikko{
Samantyyppiset asiat voidaan laskea yhteen tai \emph{ryhmitellä}.
\begin{equation}
ax^2 + bx + cx^2 + dy + ex = (a+c)x^2 + (b+e)x + dy
\end{equation}
}

\begin{esimerkki}

$ \frac{1}{6} + \frac{3}{2} = \frac{1}{2\cdot 3} + \frac{3}{2} = \frac{1}{2 \cdot 3} + \frac{3 \cdot 3}{2 \cdot 3} = \frac{1}{6} + \frac{9}{6} = \frac{10}{6} = \frac{\cancel{2} \cdot 5}{\cancel{2} \cdot 3} = \frac{5}{3}$

\end{esimerkki}

\begin{tehtava}
% Ryhmittely
Sievennä
	\begin{enumerate}[a)]
	\item $2x^2+3x+5x^2$
	\item $x^2+3x^3+x^2+x^3+2x^2$
	\item $ax^2+bx+cx$
	\item $ax^3+bx+cy^3+dx+ey^3+fx^3$
	\end{enumerate}

\begin{vastaus}
	\begin{enumerate}[a)]
	\item $7x^2+3x$
	\item $4(x^2+x^3)$ tai $4x^2+4x^3$
	\item $ax^2+(b+c)x$ tai $ax^2+bx+cx$
	\item $(a+f)x^3+(b+d)x+(c+e)y^3$
	\end{enumerate}
\end{vastaus}
\end{tehtava}

\begin{tehtava}
% Yksi termi osoittajassa
Sievennä
	\begin{enumerate}[a)]
	\item $\frac{2x^3}{x}$
	\item $\frac{3x^3y^2}{xy}$
	\item $\frac{x^2yz}{xy^2}$
	\item $\frac{6xy^3z^2}{2xz}$
	\end{enumerate}

\begin{vastaus}
	\begin{enumerate}[a)]
	\item $2x^2$
	\item $\frac{x}{y}$
	\item $\frac{xz}{y}$
	\item $3y^3z$
	\end{enumerate}
\end{vastaus}
\end{tehtava}

\begin{tehtava}
% Useampia termejä osoittajassa
Sievennä
	\begin{enumerate}[a)]
	\item $\frac{2x^5+3x^3}{x^2}$
	\item $\frac{6x^2+8y}{2x^2}$
	\item $\frac{3x-2x^2y^3}{xy}$
	\item $\frac{2x^2+3xy^2z-4xz}{2xy^2z}$
	\end{enumerate}

\begin{vastaus}
	\begin{enumerate}[a)]
	\item $2x^3+3x$
	\item $3+4 \frac{y}{x^2}$
	\item $\frac{3}{y} - 2xy^2$
	\item $\frac{x}{y^2z} + \frac{3}{2} + \frac{2}{y^2}$
	\end{enumerate}
\end{vastaus}
\end{tehtava}

\begin{tehtava}
% Useampia termejä
Sievennä.
	\begin{enumerate}[a)]
	\item $ \frac{1-x}{3} + \frac{x-2}{6}$
	\item $ \frac{5x-1}{3} - \frac{2x+5}{2}$
	\item $\frac{4x^2+3x}{x} + \frac{5x^3y-2x^2y}{x^2y}$
	\item $\frac{7x+5y}{y} - \frac{3x-2y}{x}$
	\end{enumerate}

\begin{vastaus}
	\begin{enumerate}[a)]
	\item $ -\frac{x}{6}$
	\item $ \frac{2}{3} x - \frac{17}{6}$
	\item $9x+1$
	\item $\frac{7x}{y} + \frac{2y}{x} +2$
	\end{enumerate}
\end{vastaus}
\end{tehtava}

\begin{tehtava}
% Tuloja
Sievennä.
	\begin{enumerate}[a)]
	\item $\frac{x}{6y} \cdot \frac{3y}{2}$
	\item $x \cdot \frac{x+y}{xy}$
	\end{enumerate}

\begin{vastaus}
	\begin{enumerate}[a)]
	\item $\frac{x}{4}$
	\item $\frac{x}{y} + 1$
	\end{enumerate}
\end{vastaus}
\end{tehtava}


%%%anonyymiltä lahjoittajalta
\todo{näiden tehtävien mielekkyys kannattaa tarkistaa sekä myös kappale, jonka alle ne laitetaan, jos pääsevät mukaan}

\begin{tehtava}
Lavenna samannimisiksi \quad
a) $\frac{2}{3}$ ja $\frac{4}{5}$ \quad b) $\frac{5}{6}$ ja $\frac{7}{9}$ \quad \\ c) $\frac{2}{3}$ ja $\frac{7}{2}$ 
\begin{vastaus}
a) $\frac{10}{15}$ ja $\frac{12}{15}$ \qquad b) $\frac{15}{18}$ ja $\frac{14}{18}$ \qquad c) $\frac{4}{6}$ ja $\frac{21}{6}$
\end{vastaus}
\end{tehtava}


\begin{tehtava}
Supista \quad
a) $\frac{15}{20}$ \qquad b) $\frac{14}{21}$ \qquad c) $\frac{12}{20}$
\begin{vastaus}
a) $\frac{3}{4}$ \qquad b) $\frac{2}{3}$\qquad c) $\frac{3}{5}$
\end{vastaus}
\end{tehtava}

\begin{tehtava}
Muuta sekamurtoluvuksi \quad
%täsmällisemmin sekamurtolukumuoton, mutta pienellä piirillä ajateltiin, että tämä epätäsmällinen muotoilu parempi
a) $\frac{15}{2}$ \qquad b) $\frac{9}{4}$ \qquad c) $\frac{23}{7}$
\begin{vastaus}
a) $7\frac{1}{2}$ \qquad b) $2\frac{1}{4}$ \qquad c) $3\frac{2}{7}$
\end{vastaus}
\end{tehtava}

\begin{tehtava}
Muunna murtoluvuksi \quad
a) $3\frac{2}{5}$ \qquad b) $4\frac{1}{3}$ \qquad c) $2\frac{6}{7}$
\begin{vastaus}
a) $\frac{17}{5}$ \qquad b) $\frac{13}{12}$ \qquad c) $\frac{20}{7}$
\end{vastaus}
\end{tehtava}

\begin{tehtava}
a) $\frac{3}{11}+\frac{5}{11}$ \qquad b) $\frac{4}{5}-\frac{1}{5}$ \qquad c) $\frac{2}{3}+\frac{1}{6}$ \qquad
d) $ \frac{11}{12}-\frac{5}{6}$
\begin{vastaus}
a) $\frac{8}{11}$ \qquad b) $\frac{3}{5}$ \qquad c) $\frac{5}{6}$ \qquad d) $\frac{1}{12}$
\end{vastaus}
\end{tehtava}

\begin{tehtava}
a) $1\frac{2}{9}+\frac{5}{9}$ \qquad b) $\frac{1}{3}+2\frac{1}{3}$ \qquad c) $2+\frac{5}{4}$ \qquad
d) $ \frac{3}{2}-\frac{5}{6}$
\begin{vastaus}
a) $\frac{7}{9}$ \qquad b) $\frac{8}{3}$ \qquad c) $\frac{9}{4}$ \qquad d) $\frac{2}{3}$
\end{vastaus}
\end{tehtava}


\begin{tehtava}
a) $\frac{4}{9} : \frac{1}{5}$ \qquad b) $\frac{2}{7}:\frac{5}{9}$ \qquad c) $\frac{2}{3}:\frac{4}{3}$
\begin{vastaus}
a) $\frac{20}{9}$ \qquad b) $\frac{18}{35}$ \qquad c) $\frac{1}{2}$
\end{vastaus}
\end{tehtava}

\begin{tehtava}
a) $\frac{2}{3} : \frac{7}{11}$ \qquad b) $\frac{4}{3}:(\frac{-13}{4})$ \qquad c) $\frac{7}{8}:4$
\begin{vastaus}
a) $1\frac{1}{21}$ \qquad b) $-\frac{16}{39}$ \qquad c) $\frac{7}{32}$
\end{vastaus}
\end{tehtava}

\begin{tehtava}
a) $\frac{5}{8}\cdot(\frac{3}{5}+\frac{2}{5})$ \qquad b) $\frac{1}{3}+\frac{1}{4}\cdot\frac{6}{5}$
\begin{vastaus}
a) $\frac{5}{8}$ \qquad b) $\frac{19}{30}$
\end{vastaus}
\end{tehtava}

\begin{tehtava}
a) $\dfrac{\frac{1}{2}:\frac{3}{2}}{\frac{3}{2}+\frac{1}{3}}$ \qquad b) $\dfrac{\frac{2}{3}+\frac{3}{4}}{\frac{5}{6}-\frac{7}{12}}$.
\begin{vastaus}
a) $\frac{2}{11}$ \qquad b) $5\frac{2}{3}$
\end{vastaus}
\end{tehtava}

\begin{tehtava}
Laske murtolukujen $\frac{5}{6}$ ja $-\frac{2}{15}$ \\ a) summa \qquad b) erotus \qquad c) tulo \qquad d) osamäärä.
\begin{vastaus}
a) $\frac{7}{10}$ \qquad b) $\frac{29}{30}$ \qquad c) $-\frac{1}{9}$ \qquad d) $-6\frac{1}{4}$
\end{vastaus}
\end{tehtava}

\begin{tehtava}
Laske lausekkeen $\frac{x}{2-3x}$ arvo, kun $x$ on \\ a) 4 \qquad b) $-\frac{1}{2}$ \qquad c) $\frac{7}{10}$.
\begin{vastaus}
a) $-\frac{2}{5}$ \qquad b) $-\frac{1}{7}$ \qquad c) $-7$
\end{vastaus}
\end{tehtava}

\begin{tehtava}
Laske lausekkeen $\frac{x+y}{2x-y}$ arvo, kun \\ a) $x=\frac{1}{2}$ ja $y= \frac{1}{4}$ \qquad b) $x=\frac{1}{4}$ ja $y= -\frac{3}{8}$ \qquad.
\begin{vastaus}
a) $1$ \qquad b) $-\frac{1}{7}$
\end{vastaus}
\end{tehtava}

\begin{tehtava}
Fibonaccin luvut 0, 1, 1, 2, 3, 5, 8, 13, 21, $\ldots$ määritellään seuraavasti: Kaksi ensimmäistä
Fibonaccin lukua ovat 0 ja 1, ja siitä seuraavat saadaan kahden
edellisen summana: 
\[ 0+1=1, \quad 1+1=2, \quad 1+2 = 3, \quad 2+3=5, \quad 
\textrm{ja niin edelleen.} \]
Tutki, miten Fibonaccin luvut liittyvät lukuihin
\[ \frac{1}{1+1}, \quad \frac{1}{1+\frac{1}{1+1}}, \quad
\frac{1}{1+\frac{1}{1+\frac{1}{1+1}}}, \quad 
\frac{1}{1+\frac{1}{1+\frac{1}{1+\frac{1}{1+1}}}}, \quad \ldots\]
\begin{vastaus}
Luvut ovat sievennettynä peräkkäisten Fibonaccin
lukujen osamääriä:
\[\frac{1}{2}, \ \frac{2}{3}, \ \frac{3}{5}, \frac{5}{8} \ldots  \]
\end{vastaus}
\end{tehtava}