


%\part{Alku}
\chapter{Esipuhe}

%%%%%%%%%%%%%%%%%%%%%%%%%%%%%%%%%%%%%%%%%%%%%%%%%%%%%%%%%%%%%%%%%%%%%%%%%%%%%%%%
%%%%  /usr/share/doc/texlive-fonts-extra-doc/fonts/arev/mathtesty.tex

% mathtesty.tex, by Stephen Hartke 20050522
% based on mathtestx.tex in the mathptmx package
% and symbols.tex by David Carlisle

Matematiikka tarjoaa työkaluja asioiden jäsentämiseen, päättelyyn ja mallintamiseen. Alasta riippuen käsittelemme matematiikassa erilaisia \textbf{objekteja}: Geometriassa tarkastelemme tasokuvioita ja kolmiulotteisia rakenteita. Algebra tutkii lukujen ja funktioiden ominaisuuksia. Todennäköisyyslaskenta arvioi erilaisten tapausten ja tilanteiden mahdollisuuksia ja riskejä. Matemaattinen analyysi (kurssit 7,8 ja 10) tutkii funktioita ja niiden muuttumista.

Jokaiseen tarkastelukohteeseen liitetään myös niille ominaisia \textbf{operaatioita}. Tämä kurssi käsittelee lähinnä lukuja ja niiden operaatioita, joita \textbf{laskutoimituksiksi} kutsutaan. Kirjan ensimmäisessä osassa käsittelemme luvun käsitteen, yleisimmät lukutyypit ja lukujen tavallisimman laskutoimitukset.

\section*{Sananen kirjasta}

Tulimme, kirjoitimme, voitimme.

\section*{Tekijöiden kommentit}

\todo{Miten olisi lyhyt kommentti tai lainaus jokaiselta tekijältä? Jotain yleviä mietteitä kirjasta, rohkaisevia tai nasevia kommentteja lukijalle, alku- tai loppukevennyksiä tai jotain randomia}

\begin{tabular}{cc} 
	\begin{tabular}{c}
	 \textbf{Lauri Hellsten}
	\\ 
	kommentti1 \end{tabular}
&
	\begin{tabular}{c}
	 \textbf{Niko Ilomäki}
	\\ 
	Michael Corleonea mukaillen "Make 'em an offer they can't refuse!" Olen varma, että Suomen lukiolaiset ottavat ilmaisen oppikirjan innolla vastaan. \end{tabular}
\\
	\begin{tabular}{c}
	 \textbf{Tero Keinänen}
	\\ 
	kommentti3 \end{tabular}
&
	\begin{tabular}{c}
	 \textbf{Vesa Linja-aho}
	\\ 
	kommentti4 \end{tabular}
\\
	\begin{tabular}{c}
	 \textbf{Ossi Mauno}
	\\ 
	kommentti1 \end{tabular}
&
	\begin{tabular}{c}
	 \textbf{Joonas Mäkinen}
	\\ 
	kommentti2 \end{tabular}
\\
	\begin{tabular}{c}
	 \textbf{Matti Pajunen}
	\\ 
	kommentti3 \end{tabular}
&
	\begin{tabular}{c}
	 \textbf{Pekka Peura}
	\\ 
	kommentti4 \end{tabular}
\\
	\begin{tabular}{c}
	 \textbf{Annika Piiroinen}
	\\ 
	kommentti1 \end{tabular}
&
	\begin{tabular}{c}
	 \textbf{Kaisa Pohjonen}
	\\ 
	kommentti2 \end{tabular}
\\
	\begin{tabular}{c}
	 \textbf{Antti Rasila}
	\\ 
	kommentti3 \end{tabular}
&
	\begin{tabular}{c}
	 \textbf{Johanna Rämö}
	\\ 
	kommentti4 \end{tabular}		
\\
	\begin{tabular}{c}
	 \textbf{Annika Piiroinen}
	\\ 
	kommentti1 \end{tabular}
&
	\begin{tabular}{c}
	 \textbf{Kaisa Pohjonen}
	\\ 
	kommentti2 \end{tabular}
\\
	\begin{tabular}{c}
	 \textbf{Antti Rasila}
	\\ 
	kommentti3 \end{tabular}
&
	\begin{tabular}{c}
	 \textbf{Johanna Rämö}
	\\ 
	kommentti4 \end{tabular}
	\\
	\begin{tabular}{c}
	 \textbf{Juha Sointu}
	\\ 
	kommentti1 \end{tabular}
&
	\begin{tabular}{c}
	 \textbf{Tommi Sottinen}
	\\ 
	kommentti2 \end{tabular}
\\
	\begin{tabular}{c}
	 \textbf{Jarno Talponen}
	\\ 
	kommentti3 \end{tabular}
&
	\begin{tabular}{c}
	 \textbf{Topi Talvitie}
	\\ 
	kommentti4 \end{tabular}
\\
	\begin{tabular}{c}
	 \textbf{Sampo Tiensuu}
	\\ 
	kommentti3 \end{tabular}
&
	\begin{tabular}{c}
	 \textbf{Ville Tilvis}
	\\ 
	kommentti4 \end{tabular}
	
	  
\end{tabular} 

\section*{Kiitämme}
\begin{itemize}
\item Metropolia
\item TEK
\item Senja Larsen
\item Kebab Pizza Service
\end{itemize}

%%%%%%%%%%%%%%%%%%%%%%%%%%%%%%%%%%%%%%%%%%%%%%%%%%%%%%%%%%%%%%%%%%%%%%%%%%%%%%%%
%%%% /usr/share/doc/texlive-doc-en/fonts/free-math-font-survey/source/textfragment.tex







%%% Local Variables: 
%%% mode: latex
%%% End: 
