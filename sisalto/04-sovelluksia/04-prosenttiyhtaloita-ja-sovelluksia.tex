\section{Prosenttiyhtälöitä ja sovelluksia}

\begin{tehtava}
    Laukku maksaa 225 euroa ja on 25~\%:n alennuksessa. Mikä on alennettu hinta?
    
    \begin{vastaus}
    Vastaus: 168,75 euroa
    \end{vastaus}
\end{tehtava}

\begin{tehtava}
    %Pyramidi 1, s. 80
    Kirjan myyntihinta, joka sisältää arvolisäveron, on 8~\% suurempi kuin kirjan
    veroton hinta. Laske kirjan veroton hinta, kun myyntihinta on 15 euroa.
    
    \begin{vastaus}
        Vastaus: Kirjan veroton hinta on 13,89 euroa
    \end{vastaus}
\end{tehtava}

\begin{tehtava}
    Perussuomalaisten kannatus oli vuoden 2007 eduskuntavaaleissa 4,1~\% ja
    vuoden 2011 eduskuntavaaleissa 19,1~\%. Kuinka monta prosenttiyksikköä kannatus nousi? Kuinka monta prosenttia kannatus nousi?
    \begin{vastaus}
    Vastaus: Kannatus nousi 15 prosenttiyksikköä. Prosentteina mitattuna
    kannatus nousi 366~\%.
    \end{vastaus}
\end{tehtava}

\begin{tehtava}
    Askartelukaupassa on alennusviikot, ja kaikki tavarat myydään 60~\%:n alennuksella. Viimeisenä päivänä kaikista hinnoista annetaan 
    vielä lisäalennus, joka lasketaan aiemmin alennetusta hinnasta. Minkä suuruinen lisäalennus tulee antaa, jos lopullisen 
    kokonaisalennuksen halutaan olevan 80~\%?

    \begin{vastaus}
        Vastaus: 50\%.
    \end{vastaus}
\end{tehtava}

\begin{tehtava}
    %tässä tehtävässä pitää tietää potenssi
    Erään pankin myöntämä opintolaina kasvaa korkoa 2~\% vuodessa. Kuinka monta prosenttia laina on kasvanut korkoa alkuperäiseen 
    verrattuna kymmenen vuoden kuluttua?

    \begin{vastaus}
        Vastaus: 22~\%.
    \end{vastaus}
\end{tehtava}

\begin{tehtava}
Yleinen arvonlisäveroprosentti oli Suomessa vuonna 2012 23 \% tuotteen verottomasta
hinnasta. Tuotteen hinta koostuu sen verottomasta hinnasta
ja tuotteesta maksettavasta arvonlisäverosta. Kuinka monta
prosenttia arvonlisävero on tuotteen myyntihinnasta?
\begin{vastaus}
18,0 \%
\end{vastaus}
\end{tehtava}

%Ansiotuloverotus on Suomessa progressiivista: suuremmista tuloista maksetaan

\begin{tehtava}
    Tuoreissa omenissa on vettä 80~\% ja sokeria 4~\%. Kuinka monta prosenttia sokeria on samoissa omenissa, kun ne on kuivattu siten, 
    että kosteusprosentti on 20? [K2000, 4]
    
    \begin{vastaus}
        Vastaus: 16~\%
    \end{vastaus}
\end{tehtava}

\begin{tehtava}
    Kappaleen putoamisen kesto korkeudelta $x$ maahan on kääntäen verrannollinen putoamiskiihtyvyyden $g$ neliöjuureen. Vakio $g$ on kullekin     
    taivaankappaleelle ominainen ja eri puolilla taivaankappaletta likimain sama. Empire State Buildingin katolta (korkeus 
    $381$ m) pudotetulla kuulalla kestää n. $6,2$ s osua maahan. Marsin putoamiskiihtyvyys on $37,6$ \% Maan putoamiskiihtyvyydestä. 
    Jos Empire State Building sijaitsisi Marsissa, kuinka monta prosenttia pitempi aika kuluisi kuulan maahan osumiseen?

    \begin{vastaus}
        Vastaus: $10$ s
    \end{vastaus}
\end{tehtava}
