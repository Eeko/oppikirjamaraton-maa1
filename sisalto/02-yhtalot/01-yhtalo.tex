\chapter{Yhtälö}

\laatikko{
\emph{Yhtälö} on kahden lausekkeen merkitty yhtäsuuruus.
}

\begin{esimerkki}
Merkitään lausekkeet $5x+\sqrt{x}$ ja $7x+7$ yhtäsuuriksi, jolloin saadaan
yhtälö $5x+\sqrt{x} = 7x+7$.
\end{esimerkki}

Jos yhtälön kummankin puolen lausekkeen arvo on sama, sanotaan että \emph{yhtälö pätee}.

\begin{esimerkki}
\begin{enumerate}[a)]
\item Yhtälö $3x + 2 = 0$ pätee, kun $x = - \frac{2}{3}$.
\item Yhtälö $5 = 3$ ei päde.
\end{enumerate}
\end{esimerkki}

Monissa käytännön tilanteissa jokin asia voidaan laskea kahdella eri tavalla.
Nämä laskutavat voidaan kirjoittaa lausekkeiksi, ja merkitsemällä lausekkeet yhtäsuuriksi saadaan yhtälö.

\begin{esimerkki}
Kuvassa oleva orsivaaka on tasapainossa. Toisessa vaakakupissa on kahden kilon siika ja toisessa puolen kilon ahven sekä tuntematon määrä lakritsia. Kuinka paljon vaakakupissa on lakritsia?
%(Ratkaistaan...) (Muita esimerkkejä, vähitellen vaikeutuvia (1. asteen) yhtälöitä)
\end{esimerkki}

\missingfigure{Kuva kaloista vaa'assa}

Yhtälössä esiintyy yleensä \emph{muuttujia} eli symboleja, joiden arvo voi
vaihdella. Jos muuttujia on vain yksi, sitä merkitään yleensä kirjaimella $x$.
Niitä muuttujan $x$ arvoja, joilla yhtälö pätee, kutsutaan yhtälön \emph{ratkaisuiksi}.
Yhtälön ratkaisemisella tarkoitetaan kaikkien yhtälön ratkaisujen selvittämistä.

\laatikko{
Tyypillinen tapa ratkaista yhtälöitä on muokata niitä niin, että muokattu yhtälö pätee silloin, kun alkuperäinen yhtälö pätee, ja toisaalta ei päde silloin, kun alkuperäinen yhtälö ei päde. Tällaisia sallittuja muunnoksia ovat esimerkiksi:
\begin{itemize}
\item Yhtälön molemmille puolille voidaan lisätä tai molemmilta puolilta voidaan vähentää luku. Esimerkiksi yhtälö $3x+5 = 3$ saadaan näin muotoon $3x = -2$.
\item Yhtälön molemmat puolet voidaan kertoa nollasta poikkeavalla luvulla.
Esimerkiksi kertomalla yhtälön $2x = 4$ molemmat puolet luvulla $\frac{1}{2}$
saadaan yhtälö $x = 2$.
\end{itemize}
}

\missingfigure{Kuva orvivaa'asta, jossa on myös heliumpallo}
% Nuo pitää ehkä perustella.

Monet yhtälöt ratkeavat tekemällä tällaisia muunnoksia, kunnes yhtälö on niin yksinkertaisessa muodossa, että ratkaisu voidaan nähdä siitä suoraan.
%Koska jokaisessa muokkausjonon yhtälössä ratkaisut ovat samat, näin saadaan %alkuperäisen yhtälön ratkaisut.

%%esimerkki tulee 1. asteen yhtälön yhteydessä

\laatikko{
Yhtälöt voidaan totuusarvonsa perusteella jakaa kolmeen tyyppiin:
\begin{enumerate}
\item Yhtälö, joka on aina tosi. Esimerkiksi yhtälöt $8=8$ ja $x=x$.
\item Yhtälö, joka on joskus tosi. Esimerkiksi yhtälö $x+4=7$ on tosi, kun $x=3$,
ja epätosi muulloin.
\item Yhtälö, joka ei ole koskaan tosi. Esimerkiksi yhtälö $0=1$.
\end{enumerate}
}

Tärkein näistä yhtälötyypeistä on 2. Siirrymme nyt tarkastelemaan yhtälöiden keskeistä erikoistapausta, ensimmäisen asteen yhtälöitä.