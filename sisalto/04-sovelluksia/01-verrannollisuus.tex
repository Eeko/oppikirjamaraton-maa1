\chapter{Verrannollisuus}

Verrannollisuudella tarkoitetaan tilannetta, jossa 

Suoraan verrannollisuus tarkoittaa, että kahden asian suhde pysyy vakiona. Jos toinen kaksinkertaistuu, kaksinkertaistuu toinenkin. Esimerkiksi kaupasta ostettujen hedelmien määrä ja kauppahinta ovat suoraan verrannollisia toisiinsa. Jos ostat kaksi kertaa enemmän banaaneja, joudut myös maksamaan kaksi kertaa enemmän. Hinnan ja ostettujen banaanien massan\footnote{Arkikielessä puhutaan yleensä painosta.} suhde on molemmissa tapauksissa vakio. Hedelmäesimerkin tapauksessa tätävakiota kutsutaan kilohinnaksi.

Matemaattisesti suoraan verrannollisuus merkitään seuraavasti.

\laatikko{
Jos suure $a$ on suoraan verrannollinen suureeseen $b$, merkitään

\begin{equation}
    \frac{a}{b}=c,
\end{equation}
missä $c$ on vakio.
}

\missingfigure{tähän vois tulla nousevan suoran kuva}

Kääntäen verrannollisuus tarkoittaa, että kahden asian tulo pysyy vakiona. Jos toinen kaksinkertaistuu, toinen puolittuu. Esimerkiksi nopeus ja matkaan tarvittava aika ovat kääntäen verrannollisia toisiinsa. Jos ajat koulumatkan kaksi kertaa nopeammin, matka-aika puolittuu.

Muita esimerkkejä kääntäen verrannollisuudesta ovat:
\begin{itemize}
    \item .
\end{itemize}

Matemaattisesti kääntäen verrannollisuus merkitään seuraavasti.
\laatikko{
Jos suure $a$ on kääntäen verrannollinen suureeseen $b$, merkitään
\begin{equation}
    ab=c,
\end{equation}
missä $c$ on vakio.
}

\missingfigure{tähän vois tulla laskevan suoran kuva}
