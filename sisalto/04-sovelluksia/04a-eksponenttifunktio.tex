\chapter{Eksponenttifunktio}

\laatikko{Muotoa $f(x) = a^x$ olevia funktioita kutsutaan eksponenttifunktioiksi.}

Eksponenttifunktioita on kahta tyyppiä: kasvavia ja väheneviä.
Kasvavilla eksponenttifunktioilla $a>1$, esimerkiksi

\missingfigure{Kuva, jossa joukko kuvaajia $f(x) = a^x$, joille $a>1$.}

Vähenevillä eksponenttifunktioilla $0<a<1$, esimerkiksi

\missingfigure{Kuva, jossa joukko kuvaajia $f(x) = a^x$, joille $0<a<1$.}

Kun $a<0$, eksponenttifunktiota ei ole määritelty, koska negatiiviselle
kantaluvulle ei ole määritelty ei-kokonaislukupotenssia.

Kun $a=0$ tai $a=1$, eksponenttifunktio pelkistyy vakiofunktioksi.
Lisäksi $0^0$ ei ole määritelty, joten on turvallisinta vaatia, että
kantaluvulle $a$ pätee $a>0$ ja $a \neq 1$.

Eksponenttifunktioita käytetään mallintamaan sellaista
jatkuvaa kasvua tai vähenemistä, jossa kullakin ajanhetkellä
funktion hetkellinen muutos on suoraan verrannollinen funktion sen
hetkiseen arvoon. Tähän palataan myöhemmillä matematiikan kursseilla.

Jos kysytään, millä $x$:n arvoilla eksponenttifunktio saavuttaa tietyn
arvon, muodostuu \emph{eksponenttiyhtälö}. 

\begin{esimerkki}
Millä muuttujan $x$ arvoilla eksponenttifunktio $f(x) = 2^x$ saa arvon
$f(x) = 64$?

Eksponenttifunktion kantalukuna on $2$, joten kyseessä on kasvava
eksponenttifunktio. Kokeillaan eri $x$:n arvoja: Kun $x = 3$,
$f(x) = 2^3 = 8$, joka on pienempi kuin $64$. Ratkaisu on siis
suurempi kuin kolme. Jatkamalla vastaavaa päättelyä löydetään ratkaisu
$x = 6$.

Varmistutaan vielä siitä, että yhtälöllä ei ole muita ratkaisuja:
koska eksponenttifunktio kasvaa kaikkialla, ei voi olla tilannetta, jossa
$f(x)$ saa uudelleen arvon $0$, kun $x > 6$. Toisaalta kasvavuus perustelee
myös, miksi $f(x)$ ei voi olla $0$, kun $x < 6$.

Siis ainoa ratkaisu yhtälölle on $x = 6$.
\end{esimerkki}

