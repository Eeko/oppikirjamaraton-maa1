\chapter{Murtopotenssi}

Mitä voisi tarkoittaa $2^\frac{1}{3}$ ?

Potenssien laskusäätöjen mukaan $(2^4)^3 = 2^{3\cdot 4} = 2^{12}$, joten miksi ei voisi laskea $\left( 2^{\frac{1}{3}}\right)^3 = 2^{\frac{1}{3}\cdot 3} = 2^1=2$ ? Toisaalta $(\sqrt[3]{2})^3=2$, joten on luontevaa määritellä $2^{\frac{1}{3}} = \sqrt[3]{2}$. Tällanen merkintä on käytännöllinen, koska murtoluvuilla laskeminen on monesti mukavampaa kuin juurimerkinnän käyttö.

\laatikko{Murtopotenssimerkintä: $a^\frac{1}{n} = \sqrt[n]{a}$, kun $a\geq 0$.}

Erityisesti $x^\frac{1}{2}=\sqrt{x}$.

Tämän jälkeen tuntuisi luontevalta merkitä potenssin potenssin kaavaa käyttäen $(2^{\frac{1}{3}})^2 = 2^{\frac{2}{3}}.$ (Huomaa, että tämä ei ole itsestään selvää, koska potenssien laskusäännöt on toistaiseksi todistettu vain kokonaislukueksponenteille.) Tämä otetaan murtolukueksponentin määritelmäksi.

\laatikko{Murtopotenssimerkintä: $a^\frac{m}{n} = (a^{\frac{1}{n}})^m = 
(\sqrt[n]{a})^m$, kun $a\geq 0$.}

Kun murtolukueksponentit määritellään tällä tavalla, kaikki potenssien
laskusäännöt ovat sellaisenaan voimassa myös niille, esimerkiksi kaavat
\[ a^q\cdot a^q = a^{p+q}, \quad (a^p)^q = a^{pq}, \quad (ab)^q=a^qb^q \]
pätevät kaikille rationaaliluvuille. Tämän todistaminen vaatii pitkähköjä
laskuja, joten todistukset on sijoitettu liitteeseen.

{\bf Huomio määrittelyjoukosta}. Murtopotenssimerkintää käyttettäessä joudumme vaatimaan, että $a\geq 0$ myös silloin, kun $n$ on pariton. Esimerkiksi $\sqrt[3]{-1}=-1$ (koska $(-1)^3=-1$), mutta lauseketta $(-1)^\frac{1}{3}$ ei ole määritelty. Tähän on hyvä syy, sillä muuten seuraa yllättäviä ongelmia:

\[ -1 = \sqrt[3]{-1} = (-1)^\frac{1}{3} = (-1)^\frac{2}{6}
= ((-1)^2)^\frac{1}{6} = 1^\frac{1}{6} = \sqrt[6]{1} = 1. \]

Hupsis! Mikä meni pieleen? Lavennus $\frac{1}{3}$:sta lukuun $\frac{2}{6}$ on ongelman ydin, mutta olisi hyvin ikävää jos murtolukuja ei saisikaan aina laventaa. Tämän takia peli vihelletään poikki
heti toisen yhtäsuuruusmerkin kohdalla. On siis sovittu, ettei
murtopotenssimerkintää käytetä, jos kantaluku on negatiivinen.

\begin{esimerkki}
Muuta lausekkeet $\sqrt[5]{3}$ ja $(\sqrt[4]{a})^7$ murtopotenssimuotoon. Ratkaisu: \\
$\sqrt[5]{3} = 3^\frac{1}{5}$, \\
$(\sqrt[4]{a})^7 = (a^\frac{1}{4})^7=a^\frac{7}{4}$
\end{esimerkki}

\begin{esimerkki}
Sievennä lauseke $8^\frac{2}{3}$. Ratkaisu: \\
 $8^\frac{2}{3} = (\sqrt[3]{8})^2 = 2^2 = 4.$
\end{esimerkki}

Muuta lausekkeet murtopotenssimuotoon:

\begin{tehtava}
a) $(\sqrt[3]{a})$ \qquad
b) $(\sqrt[6]{a})$ \qquad
c) $(\sqrt[n]{a})$ 
\begin{vastaus}	
a) $a^\frac{1}{3}$ \qquad
b) $a^\frac{1}{6}$ \qquad
c) $a^\frac{1}{n}$ \qquad
\end{vastaus}
\end{tehtava}

\begin{tehtava}
a) $(\sqrt[3]{b})^6$ \qquad
b) $(\sqrt[6]{b})^3$ \qquad
c) $(\sqrt[5]{b})^2$ \qquad
d) $(\sqrt[16]{ö})^4$
\begin{vastaus}	
a) $b^2$ \qquad
b) $b^\frac{1}{2}$ \qquad
c) $b^\frac{2}{5}$ \qquad
d) $ö^\frac{1}{4}$
\end{vastaus}
\end{tehtava}

Sievennä:
\begin{tehtava}
a) $x^\frac{1}{5}$ \qquad
b) $x^\frac{4}{3}$ \qquad
c) $x^\frac{50}{100}$ \qquad
\begin{vastaus}	
a) $(\sqrt[5]{x})$ \qquad
b) $(\sqrt[3]{x})^4$ \qquad
c) $(\sqrt[2]{x})$ 
\end{vastaus}
\end{tehtava}
