\chapter{Eksponenttifunktio}

\laatikko{Muotoa $f(x) = a^x$ olevia funktioita kutsutaan eksponenttifunktioiksi.}

Eksponenttifunktio eroaa aiemmin esitellystä potenssifunktiosta siinä, että
eksponenttifunktiossa muuttuja $x$ on eksponentissa, kun potenssifunktiossa
se on kantalukuna. Tästä seuraa se, että eksponenttifunktio eroaa luonteeltaan
oleellisesti potenssifunktiosta.

Eksponenttifunktioita on kahta tyyppiä: kasvavia ja väheneviä.
Kasvavilla eksponenttifunktioilla $a>1$, esimerkiksi

\missingfigure{Kuva, jossa joukko kuvaajia $f(x) = a^x$, joille $a>1$.}

Vähenevillä eksponenttifunktioilla $0<a<1$, esimerkiksi

\missingfigure{Kuva, jossa joukko kuvaajia $f(x) = a^x$, joille $0<a<1$.}

Kun $a<0$, eksponenttifunktiota ei ole määritelty, koska negatiiviselle
kantaluvulle ei ole määritelty ei-kokonaislukupotenssia.

Kun $a=0$ tai $a=1$, eksponenttifunktio pelkistyy vakiofunktioksi.
Lisäksi $0^0$ ei ole määritelty, joten on turvallisinta vaatia, että
kantaluvulle $a$ pätee $a>0$ ja $a \neq 1$.

\emph{Eksponenttiyhtälö} muodostuu, kun kysytään, millä $x$:n arvoilla eksponenttifunktio saavuttaa tietyn arvon.

\begin{esimerkki}
Millä muuttujan $x$ arvoilla eksponenttifunktio $f(x) = 2^x$ saa arvon
$f(x) = 64$?

Kirjoitetaan tehtävä yhtälöksi: $2^x = 64$.
Eksponenttifunktion kantalukuna on $2$, joten kyseessä on kasvava
eksponenttifunktio. Kokeillaan $x$:n eri arvoja: Kun $x = 3$,
$f(x) = 2^3 = 8$, joka on pienempi kuin $64$. Ratkaisu on siis
suurempi kuin kolme. Jatkamalla vastaavaa päättelyä löydetään ratkaisu
$x = 6$.

Varmistutaan vielä siitä, että yhtälöllä ei ole muita ratkaisuja:
koska eksponenttifunktio kasvaa kaikkialla, ei voi olla tilannetta, jossa
$f(x)$ saa uudelleen arvon $64$, kun $x > 6$. Kasvavuus perustelee
myös, miksi $f(x)$ ei voi olla $64$, kun $x < 6$.

Siis ainoa ratkaisu yhtälölle on $x = 6$.
\end{esimerkki}

\begin{esimerkki}
Millä muuttujan $x$ arvoilla eksponenttifunktio
$f(x) = \left( \frac{1}{2} \right)^{x}$ saa arvon
$f(x) = 1/5$?

Edellisen esimerkin tavoin kokeillaan $x$:n eri arvoja. Havaitaan,
että kun $x = 2$, funktio saa arvon $f(x) = \frac{1}{4}$, ja
kun $x = 3$, on $f(x) = \frac{1}{8}$. Ratkaisu on siis välillä
$2 < x < 3$.

Haarukointia voidaan jatkaa esimerkiksi $x$:n arvolla $x = 2,5$,
jolloin päästään lähemmäs ratkaisua. Yhtälön ratkaisu on kuitenkin
irrationaalinen, joten sen desimaalikehitelmä on äärettömän pitkä ja
jaksoton. Tarkkaa ratkaisua ei siis saada tällä menetelmällä.

Yleisen eksponenttiyhtälön tarkkaan ratkaisemiseen palataan myöhemmillä
matematiikan kursseilla.
\end{esimerkki}

\begin{tehtava}
Olkoon $f(x) = 4^x$. Laske
\begin{enumerate}[a)]
\item $f(0)$
\item $f(3)$
\item $f(\frac{1}{2})$
\end{enumerate}
\begin{vastaus}
\begin{enumerate}[a)]
\item $1$
\item $64$
\item $2$
\end{enumerate}
\end{vastaus}
\end{tehtava}

\begin{tehtava}
Olkoon $f(x) = 10^x$. Millä $x$:n arvoilla
\begin{enumerate}[a)]
\item $f(x) = 1000$
\item $f(x) = \frac{1}{100}$
\item $f(x) = -1$?
\end{enumerate}
\begin{vastaus}
\begin{enumerate}[a)]
\item $3$
\item $6$
\item Ei ratkaisua.
\end{enumerate}
\end{vastaus}
\end{tehtava}

\begin{tehtava}
Minkä kahden kokonaisluvun välissä yhtälön
$10^x = 500$ ratkaisu on?
\begin{vastaus}
Ratkaisu on lukujen $2$ ja $3$ välissä.
\end{vastaus}
\end{tehtava}

\section{Eksponentiaalinen malli}

Kun eksponenttifunktiota käytetään kuvaamaan jotakin reaalimaailman
ilmiötä, siitä käytetään nimeä \emph{eksponentiaalinen malli}.

Eksponentiaalinen malli on eräs yleisimmin käytetyistä matemaattisista
malleista. Sillä kuvataan sellaista kasvua tai vähenemistä, jossa
kullakin ajanhetkellä funktion hetkellinen muutos on suoraan
verrannollinen funktion sen hetkiseen arvoon. Tämä muotoillaan
täsmällisesti myöhemmillä matematiikan kursseilla.

\begin{esimerkki}
Soluviljelmässä olevien \emph{Escherichia coli} -bakteerien
määrää voidaan kuvata eksponentiaalisella mallilla: ajanhetkellä
$t = 0$ bakteerien lukumäärä on $1$, ja kullakin aika-askeleella
bakteerien lukumäärä tuplaantuu. Tämä voidaan
kirjoittaa malliksi
\[
f(t) = 2^t, t \ge 0
\]
Huomaa, että yllä olevassa esimerkissä funktion $f(t)$ arvot ovat
mielekkäitä vain, kun $t$ on kokonaisluku, koska
muussa tapauksessa bakteerien määrä saa ei-kokonaislukuarvon.
Yleensä tätä ei pidetä ongelmallisena, vaan funktiota voidaan käsitellä
ikään kuin bakteerien määrä olisi jatkuvasti kasvava suure.
\end{esimerkki}

\begin{tehtava}
Millä ajanhetkellä bakteerien lukumäärä ylittää sadan?
\begin{vastaus}
Ajanhetkellä $t = 7$.
\end{vastaus}
\end{tehtava}

\begin{tehtava}
Millainen funktio kuvaa bakteerien kasvua, jos bakteerien lukumäärä
ajanhetkellä $t = 0$ on $f(t) = 10$?
\begin{vastaus}
$f(t) = 10 \cdot 2^t$
\end{vastaus}
\end{tehtava}

\begin{esimerkki}
Radioaktiivisessa hajoamisessa atomiydinten lukumäärää voidaan
kuvata eksponentiaalisella mallilla. Jos ydinten määrä ajanhetkellä
$t = 0$ on $f(t) = k$, voidaan kirjoittaa malli
\[
f(t) = k \cdot \left( \frac{1}{2} \right)^t
\]
Mallissa oletetaan, että ydinten lukumäärä puolittuu kullakin ajanhetkellä.
\end{esimerkki}

\begin{tehtava}
Millä ajanhetkellä atomiydinten määrä on alle $1/200$ alkuperäisestä?
\begin{vastaus}
Ajanhetkellä $t = 8$.
\end{vastaus}
\end{tehtava}