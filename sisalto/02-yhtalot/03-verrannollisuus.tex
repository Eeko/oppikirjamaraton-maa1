\chapter{Suoraan ja kääntäen verrannollisuus}

Kaksi muuttujaa voivat riippua toisistaan monin eri tavoin. Tavallisia
riippuvuuden tyyppejä ovat suoraan ja kääntäen verrannollisuus.

\laatikko{
Kaksi muuttujaa $x$ ja $y$ ovat suoraan verrannolliset, jos toinen saadaan
toisesta kertomalla se jollakin vakiolla, eli $y = kx$. Vakiota $k$
kutsutaan \emph{verrannollisuuskertoimeksi}.}

Suoraan verrannollisuus voidaan tunnistaa esimerkiksi laskemalla muuttujien
suhde ja toteamalla, että se on muuttujista riippumaton vakio.

\begin{esimerkki}
Banaanien kilohinta on $2,00$ euroa. Seuraavassa taulukossa on
banaanien paino\footnote{Fysikaalisesti kyse on massasta, mutta
arkikielessä termi on paino.}, jonka saa ostettua tietyllä rahamäärällä:
\begin{center} 
\begin{tabular}{|l|r|r|}
\hline
Hinta (euroa) & Paino (kg) & Hinta/paino (euroa/kg) \\
\hline
$1,00$ & $0,50$ & $2,00$ \\
$2,00$ & $1,00$ & $2,00$ \\
$3,00$ & $1,50$ & $2,00$ \\
$4,00$ & $2,00$ & $2,00$ \\
\hline
\end{tabular}
\end{center}
Hinnan ja painon suhde on vakio, $2,00$ euroa/kg, joten ostettujen
banaanien paino ja niihin käytetty rahamäärä ovat suoraan verrannolliset.
\end{esimerkki}

Suoraan verrannollisuutta voidaan kuvata myös niin, että jos
toinen muuttujista kaksinkertaistuu, niin toinenkin kaksinkertaistuu.
Samoin jos toisen muuttujan arvo puolittuu, toisenkin arvo puolittuu.

Jos suoraan verrannollisista muuttujista piirretään kuvaaja, pisteet
asettuvat suoralle:

\missingfigure{Kuva, johon piirretty vaaka-akselille käytetty rahamäärä
ja pystyakselille ostettujen banaanien paino.}

Suoraan verrannollisia muuttujia ovat myös esimerkiksi
\begin{itemize}
    \item aika ja kuljettu matka, kun liikutaan vakionopeudella, tai
    \item kappaleen massa ja painovoiman kappaleeseen aiheuttama voima.
\end{itemize}

Suoraan verrannollisuutta monimutkaisempi riippuvuus on kääntäen
verrannollisuus.

\laatikko{
Muuttujat $x$ ja $y$ ovat kääntäen verrannolliset, jos toinen saadaan toisesta
jakamalla jokin vakio sillä, eli $y = \frac{a}{x}$.
}

Kääntäen verrannollisuus voidaan tunnistaa esimerkiksi
kertomalla muuttujien arvoja keskenään ja huomaamalla,
että tulo on muuttujista riippumaton vakio.

\begin{esimerkki}
Nopeus ja matkaan tarvittava aika ovat kääntäen verrannolliset.
Jos kuljettavana matkana on $80$ km, voidaan nopeudet ja matka-ajat
kirjoittaa taulukoksi:
\begin{center} 
\begin{tabular}{|l|r|r|}
\hline
Nopeus (km/h) & Matka-aika (h) & Nopeus$\cdot$matka-aika (km) \\
\hline
$40$ & $2$ & $80$ \\
$80$ & $1$ & $80$ \\
$100$ & $0,8$ & $80$ \\
\hline
\end{tabular}
\end{center}
Tulo on aina $80$ km, joten nopeus ja matka-aika ovat kääntäen verrannolliset.
\end{esimerkki}

Kääntäen verrannollisuutta voidaan kuvata myös niin, että jos
toinen muuttujista kaksinkertaistuu, toinen puolittuu.

Jos kääntäen verrannollisista muuttujista piirretään kuvaaja, pisteet
muodostavat laskevan käyrän:

\missingfigure{Kuva, johon on piirretty matka-aika ja nopeus (yllä olevan
taulukon mukaisesti).}

Kääntäen verrannollisia muuttujista ovat myös
\begin{itemize}
    \item kaivamistyön suorittamiseen kuluva aika ja työntekijöiden lukumäärä.
\end{itemize}

\section*{Tehtäviä}

\begin{tehtava}
    % Lyhyt matikka 1, s. 72
    Pohdi seuraavissa tapauksissa, kuinka toinen muuttuja muuttuu, kun toinen
    kaksinkertaistuu, kolminkertaistuu tai puolittuu. Ovatko muuttujat
    suoraan verrannolliset, kääntäen verrannolliset vai eivät kumpaakaan?
    
    \begin{enumerate}
        \item Kuljettu matka ja kulunut aika, kun keskinopeus on 30 km/h.
        \item Kananmunien lukumäärä ja niiden kovaksi keittämiseen tarvittava keittoaika.
        \item Hedelmätiskiltä valitun vesimelonin paino ja hinta.
        \item Neliön sivun pituus ja neliön pinta-ala.
    \end{enumerate}
    
    \begin{vastaus}
        Vastaus:
        \begin{enumerate}
            \item Ovat.
            \item Eivät ole.
            \item Ovat.
            \item Eivät ole, sillä esimerkiksi kun neliön sivun pituus
                kaksinkertaistuu 1 cm:stä 2 cm:iin, niin neliön pinta-ala
                nelinkertaistuu 1 cm$^2$:stä 4 cm$^2$:iin.
        \end{enumerate}
    \end{vastaus}
\end{tehtava}

\begin{tehtava}
Ratkaise
\begin{enumerate}
\item $ \frac{x}{3} = 1$
\item $ \frac{8}{y} = 2$
\item $ \frac{7}{x} = \frac{16}{8}$
\item $ \frac{x}{3} = \frac{1}{7}$
\end{enumerate}
\begin{vastaus}
\begin{enumerate}
\item $x= \frac{1}{3}$
\item $y= \frac{1}{4}$
\item $x= \frac{7}{2}$
\item $x= \frac{3}{7}$
\end{enumerate}
\end{vastaus}
\end{tehtava}

\begin{tehtava}
Muodosta seuraavia tilanteita kuvaavat yhtälöt. Voit käyttää vakion
merkkinä esimerkiksi $c$:tä.
\begin{enumerate}
\item Kultakimpaleen arvo ($x$) on suoraan verrannollinen sen massaan ($m$),
eli mitä painavampi kimpale on, sitä enemmän siitä saa rahaa.
\item Aidan maalaamiseen osallistuvien ihmisten määrä {$x$} on kääntäen verrannollinen maalaamiseen kuluvaan aikaan ($t$). Toisin sanoen, mitä
enemmän maalaajia, sitä nopeammin homma on valmis.
\item Planeettojen toisiinsa aiheuttama vetovoima ($F$) on suoraan verrannollinen planeettojen massoihin ($m_1$ ja $m_2$) ja kääntäen verrannollinen niiden välisen etäisyyden ($r$) neliöön.
\end{enumerate}
\begin{vastaus}
\begin{enumerate}
\item $ \frac{x}{m}=c$
\item $ xt=c $
\item $ \frac{Fr^2}{m_1+m_2}=c$
\end{enumerate}
\end{vastaus}
\end{tehtava}

\begin{tehtava}
Rento pyöräilyvauhti kaupunkiolosuhteissa on noin $20$ km/h. Lukiolta urheiluhallille on matkaa $7$ km. Kuinka monta minuuttia kestää arviolta pyöräillä lukiolta urheiluhallille?
\begin{vastaus}
Viiden minuutin tarkkuudella $20$ min.
\end{vastaus}
\end{tehtava}

\begin{tehtava}
    Isä ja lapset ovat ajamassa mökille Sotkamoon. On ajettu jo neljä
    viidesosaa matkasta, ja aikaa on kulunut kaksi tuntia. ''Joko ollaan perillä?''
    lapset kysyvät takapenkiltä. Kuinka pitkään vielä arviolta kuluu, ennen
    kuin ollaan mökillä?
    
    \begin{vastaus}
        Vastaus: 1 h 15 min
    \end{vastaus}
\end{tehtava}

\begin{tehtava}
    Äidinkielen kurssilla annettiin tehtäväksi lukea 300-sivuinen romaani.
    Eräs opiskelija otti aikaa ja selvitti lukevansa vartissa seitsemän sivua.
    Kuinka monta tuntia häneltä kuluu koko romaanin lukemiseen, jos
    taukoja ei lasketa?
    
    \begin{vastaus}
        Vastaus: 642 minuuttia eli 10 h 42 min.
    \end{vastaus}
\end{tehtava}
