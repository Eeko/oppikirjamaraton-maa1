


\part{Lukualueet}
\chapter{Luonnolliset luvut}

Tähän tekstiä luonnollisista luvuista.

\chapter{Joukko-oppia}
\chapter{Logiikkaa}
\chapter{Kokonaisluvut}
\chapter{Kokonaislukujen aritmetiikkaa}
\chapter{Jaollisuus \& tekijät}
\chapter{Rationaaliluvut ja laskusäännöt}

\begin{tehtava}
	Laske %aika randomit luvut
	\begin{enumerate}
	\item $\frac{6}{2} + \frac{3}{5}$
	\item $\frac{7}{8} - \frac{1}{4}$
	\item $2 \frac{1}{3} + \frac{4}{6}$	
	\end{enumerate}
	
	\begin{vastaus}
	Vastaus:
	\begin{enumerate}
		\item $\frac{18}{5}$
		\item $\frac{5}{8}$
		\item $3$
	\end{enumerate}			
	\end{vastaus}
\end{tehtava}

\chapter{Potenssisäännöt \& murtolausekkeiden sieventämistä}
Potenssilla $2^4$ tarkoitetaan tuloa $2\cdot 2\cdot 2\cdot 2$.
\begin{equation}
\text{Eli} 2^4=2\cdot 2\cdot 2\cdot 2=16.
\end{equation}
Lausekkeessa $2^4$ luku 2 on \textbf{kantaluku} ja luku 4 on \textbf{eksponentti}.

\begin{esimerkki}
\textbf{Esimerkki 1}
\begin{equation}
\text{a)} (-2)^3=(-2)\cdot (-2)\cdot (-2)=-8
\end{equation}

\begin{equation}
\text{b)} (-2)^4=(-2)\cdot (-2)\cdot (-2)\cdot (-2)=16
\end{equation}

\begin{equation}
\text{c)} -2^4=-2\cdot 2\cdot 2\cdot 2=-16
\end{equation}

\begin{equation}
\text{d)} 2^2\cdot 2^3=\underbrace{2\cdot 2}_{2 kpl}\cdot \underbrace{2\cdot 2\cdot 2}_{3 kpl}=2^5=32
\end{equation}

\begin{equation}
\text{e)}\frac{2^4\cdot 2^2}{2^3}=\frac{\overbrace{2\cdot 2\cdot 2\cdot \cancel{2}}\cdot \overbrace{\cancel{2}\cdot \cancel{2}}}{\cancel{2}\cdot \cancel{2}\cdot \cancel{2}}
\end{equation}

\section{Potenssisäännöt}

\begin{tehtava}
%Pitkä Sigma 1, s.76
Laske lausekkeen arvo. Muista ottaa huomioon sulut!
\begin{enumerate}
\item $-2\cdot 4^2$
\item $(-2\cdot 4)^2$
\item $\frac{3^2}{7}$
\item $\left( \frac{3}{7} \right)^2$
\item $-2^6$
\item $(-2)^6$
\end{enumerate}
\begin{vastaus}
Vastaus:
\begin{enumerate}
\item $-32$
\item $64$
\item $\frac{9}{7}$
\item $\frac{9}{49}$
\item $-64$
\item $64$
\end{enumerate}
\end{vastaus}
\end{tehtava}

\begin{tehtava}
Sievennä
\begin{enumerate}
\item $x^7\cdot x^2$
\item $(a^3)^2$
\item $\frac{a^8}{a^2}$
\item $(2y+5x)^0$
\item $\frac{1}{10^{-3}}$
\end{enumerate}
\begin{vastaus}
Vastaus:
\begin{enumerate}
\item $x^9$
\item $a^6$
\item $a^4$
\item $1$
\item $1000$
\end{enumerate}
\end{vastaus}
\end{tehtava}

\section{Murtolausekkeiden sieventäminen}
%tässä pitää opettaa binomin neliösäännöt ja ne (3kpl)

\begin{tehtava}
%tää voi olla eka tehtävä
Sievennä
\begin{enumerate}
\item $\frac{a^2+2ab+b^2}{a+b}$
\item $\frac{a^2-2ab+b^2}{a-b}$
\item $\frac{a^2-b^2}{a+b}$
\end{enumerate}
\begin{vastaus}
Vastaus:
\begin{enumerate}
\item $a-b$
\item $a+b$
\item $a-b$
\end{enumerate}
\end{vastaus}
\end{tehtava}

\chapter{Juuret}

\section{Neliöjuuri}

\laatikko{Luvun $a$ neliöjuuri on ei-negatiivinen luku, jonka neliö on $a$. Tämä voidaan ilmaista lyhyemmin $\sqrt{b^2}=b$.}

Neliöjuuren määritteleminen $\sqrt{a}^2=a$ ei johda samaan lopputulokseen. Pohdi, miksi näin on.
%%vai parempi antaa suoraan $\sqrt{a}^2=a$, kun $a \ge 0$
Jatkossa tälaisia määritelmän pieniä muokkauksia ja niistä aiheutuvia muutoksia olisi aina hyvä pohdiskella -- saattavat jopa auttaa muistamaan määritelmän oikean muodon.
%%%%%%%%%%%%%%% ONKO ITSEISARVO KÄSITELTY!!!!! %%%%%%%%%%%%%%%%%%%%%%%%%
%Määritelmäksi ei kelpaisi tämäkään. $\sqrt{a^2}=|a|$
%%%%%%%%%%%%%%% ONKO ITSEISARVO KÄSITELTY!!!!! %%%%%%%%%%%%%%%%%%%%%%%%%

Neliöjuurta ei siis nyt määritelty ollenkaan negatiivisille luvuille.

%yhtälöt tulevat vasta myöhemin, siksi esimerkit köyhiä

Esimerkki
\begin{align*}
\sqrt{4} = 2\quad \textrm{, koska $2>0$ ja $2^2 =4$} 
\end{align*}

%Pythagoraan lause on pitänyt käydä ennen tätä!! (Se on yläasteen jälkeen vasta lukion kolmoskurssissa.)
Taulutelevision kooksi on ilmoitettu mainoksessa $46''$ ja kuvasuhteeksi $16:9$. Kuinka leveä televisio on arviolta?
Vastaus: $40,7$'' tai $103,4$ cm

\section{Kuutiojuuri}

\laatikko{Luvun $a$ kuutiojuuri on luku, jonka kuutio on $a$. Tämä voidaan ilmaista lyhyemmin $\sqrt[3]{b^3}=b$.
Määritelmäksi voisi ottaa myös $\sqrt[3]{b^3}=b$.}
%tämä jälkimmäinen on ehkä järkevämpi määritelmä.
%Olisi varmaan hyvä ottaa samanlainen määritelmä neliöjuuren tapauksessakin
%jolloin neliöjuureen määritelmään tulisi 2 ehtoa.
Kuutiojuuren voi siis ottaa mistä tahansa luvusta.
%vai reaaliluvusta?
%Reaaliluvuista puhutaan kuitenkin vasta myöhemmin, niin olkoon näin.

\section{n. juuri}
Kaikkia juuria ei kuitenkaan kannata määritellä yksitellen. Tehdään siis mahdollisimman paljon kerralla. Edeltä kuitenkin voi huomata, että kuutiojuuri on määritelty kaikille luvuille, mutta neliöjuuri vain ei-negatiivisille luvuille. Tämä toistuu myös muissa juurissa. Määritellään siis parilliset ja parittomat juuret erikseen.

%on ehkä parempi esittää nämä molemma samalla eikä kuten subsections
%sama määritelmä, mutta todetaan parillisilla vaadittavan >= 0.
%%%RISTIRIITA ED. KANSSA
Juurimerkinnällä $\sqrt[n]{a}=b$ (luetaan \emph{ännäs juuri aasta on bee} tarkoitetaan lukua, joka toteuttaa ehdon $b^n = a$. Jotta juuri olisi ykskäsitteinen, on parillisilla juurilla ($\sqrt{a}, \sqrt[4]{a}, \sqrt[6]{a}$\ldots) vaadittava, että $b\ge0$.

\subsection{parilliset juuret}

\laatikko{Luvun $a$ $n$.s juuri (luetaan \emph{ännäs juuri}) on ei-negatiivinen luku, jonka neliö on $a$. Tämä voidaan ilmaista lyhyemmin $\sqrt[n]{b^n}=b$.}

\subsection{parittomat juuret}
\laatikko{Luvun $a$ n.s juuri on ei-negatiivinen luku, jonka neliö on $a$. Tämä voidaan ilmaista lyhyemmin $\sqrt[n]{b^n}=b$.}

Nyt on paikallaan todeta, että toista juurta $\sqrt[2]{a}$ merkitään $\sqrt{a}$.

\begin{tabular}{c|c}
parillinen juuri & pariton juuri\\
$\sqrt[n]{a}^n=a$, $a\ge0$ & $\sqrt[n]{a}^n$, kaikilla $a$
\end{tabular}

\laatikko{
$\sqrt[n]{ab}=\sqrt[n]{a}\sqrt[n]{b}$
Jos n on parillinen, niin on lisäksi vaadittava, että $a\ge0$ ja $b\ge0$.}

\laatikko{
$\sqrt[n]{\frac{a}{b}}=\frac{\sqrt[n]{a}}{\sqrt[n]{b}}$
Jos n on parillinen, niin on lisäksi vaadittava, että $a\ge0$ ja $b\ge0$.}

\chapter{Murtopotenssi}
\chapter{Irrationaaliluvut}
\chapter{Reaaliluvut}
\chapter{Kompleksiluvut}
\chapter{Kertaustiivistelmä}
