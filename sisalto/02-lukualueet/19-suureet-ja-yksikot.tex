\chapter{Suureista ja yksiköistä}

\laatikko{
Kerrannaisyksiköiden etuliitteet:

\begin{tabular}{c|c}
\begin{tabular}{c|c|c}
Nimi & Kerroin & Tunnus \\
\hline
deka & $10^{1}$ & da 	\\
hehto & $10^{2}$ & h 	\\
kilo & $10^{3}$ & k 	\\
mega & $10^{6}$ & M 	\\
giga & $10^{9}$ & G		\\
tera & $10^{12}$ & T 	
\end{tabular}
\begin{tabular}{c|c|c}
Nimi & Kerroin & Tunnus \\
\hline
desi & $10^{-1}$ & d 	\\
sentti & $10^{-2}$ & c 	\\
milli & $10^{-3}$ & m	\\
mikro & $10^{-6}$ & mu \\
nano & $10^{-9}$ & n 	\\
piko & $10^{-12}$ & p
\end{tabular}
\end{tabular}
}

\laatikko{
Yksi tunti on $60$ minuuttia. Yksi minuutti on $60$ sekuntia.
\begin{equation}
1 \text{h} = 60 \text{min}
\end{equation}
\begin{equation}
1 \text{min} = 60 \text{s}
\end{equation}
\begin{equation}
1 \text{h} = 60 \text{min} = 60 \cdot 60 \text{s}
\end{equation}
}

\begin{esimerkki}
Kuinka monta minuuttia on $1,25$ h? $1,25 \text{h} = 1,25 \cdot 60 \text{min} = 75 \text{min}$. $1,35$ h on siis $75$ minuuttia. Huomaa, että voit laskuissasi esittää desimaaliluvun $1,25$ yhdistettynä lukuna $1 \frac{25}{100}$ eli $1 \frac{1}{4}$, mikä saattaa helpottaa laskemista.
\end{esimerkki}

\todo{pitää esitellä mitä on mega, milli, sentti jne.}
\todo{kuinka monta metriä on tuuma jne.}
\todo{merkitsevät numerot}

\begin{tehtava}
Muuta minuuteiksi
\begin{enumerate}
\item $1$ h $17$ min
\item $2$ h $45$ min
\item $1,5$ h
\item $1,75$ h
\end{enumerate}
\begin{vastaus}
\begin{enumerate}
\item $77$ min
\item $165$ min
\item $90$ min
\item $105$ min
\end{enumerate}
\end{vastaus}
\end{tehtava}

\begin{tehtava}
Muuta sekunneiksi
\begin{enumerate}
\item $1$ h $42$ min
\item $3$ h $32$ min
\item $1,25$ h
\item $4,5$ h
\end{enumerate}
\begin{vastaus}
\begin{enumerate}
\item $6120$ s
\item $12720$ s
\item $4500$ s
\item $16200$ s
\end{enumerate}
\end{vastaus}
\end{tehtava}

\begin{tehtava}
Muuta tunneiksi ja minuuteiksi
\begin{enumerate}
\item $125$ min
\item $667$ min
\item $120$ min
\item $194$ min
\end{enumerate}
\begin{vastaus}
\begin{enumerate}
\item $2$ h $5$ min
\item $11$ h $7$ min
\item $2$ h
\item $3$ h $14$ min
\end{enumerate}
\end{vastaus}
\end{tehtava}


\begin{tehtava}
Esitä luku ilman kymmenpotenssia.
\begin{enumerate}
\item $3,2 \cdot 10^4$
\item $-7,03 \cdot 10^{-5}$
\item $10,005 \cdot 10^{-2}$
\end{enumerate}
\begin{vastaus}
Vastaus
\begin{enumerate}
\item $32000$
\item $-0,0000703$
\item $0,10005$
\end{enumerate}
\end{vastaus}
\end{tehtava}

\begin{tehtava}
Esitä luku ilman etuliitettä.
\begin{enumerate}
\item $0,5 dl$
\item $233 mm$
\item $33 cm$
\item $16 kg$
\item $2 MJ$
\item %megatavu, mibitavu jne.
\item
\end{enumerate}
\begin{vastaus}
Vastaus:
\begin{enumerate}
\item $0,05 l$
\item $0,233 m$
\item $0,33 m$
\item $16 000 g$
\item $2 000 000 J$
\item $ $
\item $ $
\end{enumerate}
\end{vastaus}
\end{tehtava}

\section*{Pyöristäminen}

Mikäli urheiluliikkeessä lumilaudan pituudeksi ilmoitetaan tarkan mittauksen jälkeen 167,9337 cm, tämä tuskin on asiakaalle kovin hyödyllistä tietoa. Epätarkempi arvo 168 cm antaa kaiken olleellisen informaation ja on mukavampi lukea.

Pyöristämisen ajatus on korvata luku sitä lähellä olevalla luvulla, jonka esitysmuoto on lyhyempi. Voidaan pyöristää
esimerkiksi tasakymmenien, kokonaisten tai vaikkapa tuhannesosien
tarkkuuteen.

Pyöristys tehdään aina lähimpään oikeaa tarkkutta olevaan lukuun. Siis esimerkiksi kokonaisluvuksi pyöristettäessä $2,8 \approx 3$, koska 3 on lähin kokonaisluku. Lisäksi on sovittu, että
puolikkaat (kuten 2,5) pyöristetään ylöspäin.

Se, pyöristetäänkö ylös vai alaspäin (eli suurempaan vai
pienempään lukuun) riippuu siis haluttua tarkkuutta
seuraavasta numerosta: pienet
0, 1, 2, 3, 4 pyöristetään alaspäin, suuret 5, 6, 7, 8, 9 ylöspäin.

\begin{esimerkki}
Pyöristetään luku 15,0768 sadasosien tarkkuuteen. Katkaistaan
luku sadasosien jälkeen ja katsotaan seuraavaa desimaalia:\\
$15,0768 = 15,07|68 \approx 15,08$.\\
Pyöristettiin ylöpäin, koska seuraava desimaali oli 6.
\end{esimerkki}


\subsection*{Merkitsevät numerot}

Mikä on tarkin mittaus, 23 cm, 230 mm vai 0,00023 km? Kaikki kolme tarkoittavat täsmälleen samaa, joten niitä tulisi pitää
yhtä tarkkoina. Luvun esityksessä esiintyvät kokoluokkaa ilmaisevat nollat eivät ole \emph{merkitseviä numeroita}, vain
2 ja 3 ovat.

\laatikko{Merkitseviä numeroita ovat kaikki luvussa esiintyvät numerot, paitsi nollat kokonaislukujen lopussa ja desimaalilukujen alussa.}

Jos esimerkiksi pöydän paksuudeksi on mitattu millin tuhannesosien
tarkkuudella 2\,cm, voidaan pituus ilmoittaa muodossa 2,0000\,cm, jolloin tarkkuus tulee näkyviin. 

Kokonaislukujen kohdalla on toisinaan epäselvyyttä merkitsevien numeroiden määrässä. Kasvimaalla asuvaa 100 citykania on tuskin laskettu ihan tarkasti,
mutta 100 m juoksuradan todellinen pituus ei varmasti ole todellisuudessa
esimerkiksi 113 m.

\begin{center}
\begin{tabular}{r|l}
Luku & Merkitsevät numerot \\
\hline
123 & 3 \\
12 000 & 2 (tai enemmän)\\
12,34 & 4 \\
0,00123 & 3
\end{tabular}
\end{center}

\subsection*{Vastausten pyöristäminen käytännön laskuissa}

Pääsääntö on, että vastaukset pyöristetään aina epätarkimman
lähtöarvon mukaan. Yhteen- ja vähennyslaskuissa epätarkkuutta
mitataan desimaalien lukumäärällä.

Jos esimerkiksi 175\,cm pituisen ihmisen
nousee seisomaan 2,15 cm korkuisen laudan päälle, olisi varsin
optimistista ilmoittaa kokonaiskorkeudeksi 177,15 cm. Kyseisen ihmisen pituus kun todellisuudessa on mitä tahansa arvojen
174,5\,cm ja 175,5\,cm väliltä. Lasketaan siis\\
\indent 175\,cm + 2,15\,cm = 177,15\,cm $\approx$ 177\,cm.
Epätarkempi lähtöarvo oli mitattu senttien tarkkuudella, joten pyöristettiin tasasentteihin.

Kerto-ja jakolaskussa tarkkuutta arvioidaan merkitsevien numeroiden mukaan. Jos esimerkiksi pitkän pöydän pituus karkeasti
mitattuna 5,9\,m ja pöydän leveydeksi saadaan tarkalla mittauksella
1,7861\,m, ei ole perusteltua olettaa pöydän pinta-alan olevan todella
\[ 5,9\,\textrm{m} \cdot 1,7861\,\textrm{m} = 10,53799\,\textrm{m}^2. \]
Pyöristys tehdään epätarkimman
lähtöarvon mukaisesti kahteen merkitsevään numeroon:
\[ 5,9\,\textrm{m} \cdot 1,7861\,\textrm{m} = 10,53799\,\textrm{m}^2 \approx 11 \textrm{m}^2.\]

%Tarkkuus ei ole aina hyvästä lukujen esittämisessä. Esimerkiksi
%\[ \pi = 3,141592653589793238462643383279 \ldots \]
%mutta käytännön laskuihin riittää usein $3,14$. 