% Tähän tulee liitteitä
% Esimerkiksi loogiset symbolit, reaalilukujen aksioomat, kompleksilukuintro, ...
\part{Liiteet}
% Vaihda tähän kirjaimin kulkeva "numerointi"
\chapter{Logiikka ja joukko-oppi}
\chapter{Reaalilukujen aksioomat}
Reaaliluvut ovat kunta, eräs algebrallinen rakenne. Myös esimerkiksi rationaaliluvut ja seuraavassa liitteessä esiteltävät kompleksiluvut muodostavat kunnan. Sen sijaan luonnolliset luvut ja kokonaisluvut eivät ole kuntia.

Reaalilukujen aksiomaattinen määritelmä muodostuu kolmesta osasta:

\textbf{1. Kunta-aksioomat reaalilukuihin sovellettuna} \\
\begin{align*}
&\text{K1.} \, \forall x, y \in \mathbb{R}: x+(y+z) = (x+y)+z & &| \, \text{summan liitäntälaki} \\
&\text{K2.} \, \exists 0 \in \mathbb{R}: x+0 = x & &| \, \text{summan neutraalialkio} \\
&\text{K3.} \, \forall x \in \mathbb{R} \, \exists (-x) \in \mathbb{R}: x+(-x)=0 & &| \, \text{vasta-alkio} \\
&\text{K4.} \, \forall x, y \in \mathbb{R}: x+y = y+x & &| \, \text{summan vaihdantalaki} \\
&\text{K5.} \, \forall x, y, z \in \mathbb{R}: x*(y+z) = x*y + x*z & &| \, \text{osittelulaki} \\
&\text{K6.} \, \forall x, y, z \in \mathbb{R}: x*(y*z) = (x*y)*z & &| \, \text{tulon liitäntälaki} \\
&\text{K7.} \, \exists 1 \in \mathbb{R}: 1*x = x & &| \, \text{tulon neutraalialkio} \\
&\text{K8.} \, \forall x \in \mathbb{R} \setminus \{0\} \, \exists x^{-1} \in \mathbb{R} \setminus \{0\}: x*x^{-1}=1 & &| \, \text{tulon käänteisalkio} \\
&\text{K9.} \, \forall x, y \in \mathbb{R}: x*y = y*x & &| \, \text{tulon vaihdantalaki}
\end{align*}