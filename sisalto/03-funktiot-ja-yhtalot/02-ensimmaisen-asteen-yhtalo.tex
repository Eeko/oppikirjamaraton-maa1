\chapter{Ensimmäisen asteen yhtälö}

\laatikko{
Ensimmäisen asteen yhtälö on yhtälö, joka on esitettävissä muodossa $ax+b=0$, jossa $a \neq 0$.
}

\begin{theorem}
Kaikki muotoa $ax+b=cx+d$ olevat yhtälöt, joissa $a \neq c$, ovat ensimmäisen asteen yhtälöitä.
\end{theorem}

\begin{proof}
\begin{align*}
ax+b &= cx+d & &| \, \textbf{Vähennetään molemmilta puolilta $cx+d$.} \\
ax+b - (cx+d) &= 0 & &| \, \textbf{Järjestellään termejä uudelleen.} \\
ax - cx + b - d &= 0 & &| \, \textbf{Otetaan yhteinen tekijä.} \\
(a-c)x + (b-d) &= 0 & &| \, \textbf{Tämä on määritelmän mukainen ensimmäisen asteen yhtälö, koska $a \neq c$.}
\end{align*}
\end{proof}

\begin{theorem}
Yleinen lähemistymistapa muotoa $ax+b = cx+d$ olevien yhtälöiden ratkaisuun on: \\
(1) Vähennä molemmilta puolilta $cx$. Saat yhtälön $(a-c)x + b = d$. \\
(2) Vähennä molemmilta puolita $b$. Saat yhtälön $(a-c)x = d-b$. \\
(3) Jaa $(a-c)$:llä. Saat yhtälön ratkaistuun muotoon $x = \frac{d-b}{a-c}$.
\end{theorem}

Esimerkki. Yhtälön $7x+4=4x+7$ ratkaisu saadaan seuraavasti:
\begin{align*}
7x+4 &= 4x+7 & &| \, \text{Vähennetään molemmilta puolilta 4x.} \\
3x+4 &= 7 & &| \, \text{Vähennetään molemmilta puolilta 4.} \\
3x &= 3 & &| \, \text{Jaetaan molemmat puolet kolmella eli kerrotaan $\frac{1}{3}$:lla.} \\
x &= 1 & &| \, \text{Saimme yhtälön ratkaistuun muotoon. $x=1$ on siis yhtälön ratkaisu.} \\
\end{align*}

\begin{tehtava}
%
Ratkaise:
\begin{enumerate}[a)]
\item $x + 4 = 5$
\item $1 - x = -3$
\item $7x = 35$
\item $-2x = 4$
\item $10 - 2x = x$
\item $9x + 4 = 6 - x$
\item $\frac{2x}{5} = 4$
\item $\frac{x}{3} + 1 = \frac{5}{6} - x$
\end{enumerate}
\begin{vastaus}
\begin{enumerate}[a)]
\item $x=1$
\item $x=4$
\item $x=35/7$
\item $x=-2$
\item $x=10/3$
\item $x=1/5$
\item $x=10$
\item $x=-1/8$
\end{enumerate}
\end{vastaus}
\end{tehtava}

\begin{tehtava}
Ratkaise kysytty muuttuja yhtälöstä
\begin{enumerate}[a)]
\item $F=ma$, $m=?$ "voima on massa kertaa kiihtyvyys"
\item $p=\frac{F}{A}$, $F=?$ "paine on voima jaettuna alalla"
\item $A=\pi r^2$, $r=?$ "(pallon) pinta-ala on pii kertaa säde toiseen"
\item $V=\frac{1}{3} \pi r^2 h$, $h=?$ "(kartion) tilavuus on kolmasosa pii kertaa säde toiseen kertaa korkeus"
\end{enumerate}
\begin{vastaus}
\begin{enumerate}[a)]
\item $m=\frac{F}{a}$
\item $F=p A$
\item $r=\sqrt{\frac{A}{\pi}}$
\item $h=\frac{V}{ \frac{1}{3} \pi r^2 h}$
\end{enumerate}
\end{vastaus}
\end{tehtava}

\begin{tehtava}
Kännykkäliittymän kuukausittainen perusmaksu on 2,90 euroa. Lisäksi jokainen puheminuutti ja tekstiviesti maksaa 0,69 senttiä. Pekan kännykkälasku kuukaudelta oli 27,05 euroa.

\begin{enumerate}[a)]
	\item Kuinka monta puheminuuttia/tekstiviestiä Pekka käytti kuukauden aikana?
	\item Pekka lähetti kaksi tekstiviestiä jokaista viittä puheminuuttia kohden. Kuinka monta tekstiviestiä Pekka lähetti?
\end{enumerate}

	\begin{vastaus}
		\begin{enumerate}[a)]
			\item 350 puheminuuttia/tekstiviestiä
			\item 100 tekstiviestiä
		\end{enumerate}
	\end{vastaus}
\end{tehtava}

\begin{tehtava}
Sadevesikeräin näyttää vesipatsaan korkeuden millimetreissä. Eräänä aamuna kello 9 keräimessä oli 5 mm vettä. Seuraavana aamuna samaan aikaan keräimessä oli 23 mm vettä. Muodosta yhtälö ja ratkaise, kuinka paljon vettä oli satanut keskimäärin tunnissa kuluneen vuorokauden aikana.
	\begin{vastaus}
	0,75 mm/tunti
	\end{vastaus}
\end{tehtava}