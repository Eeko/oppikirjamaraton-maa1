\chapter{Potenssifunktio}

\laatikko{Potenssifunktio: 
\begin{function} f(x) = a\cdot x^n} \end{function},
jossa $a \neq 0$. }

Eksponentti n on potenssin x^n aste. Se voi olla mikä tahansa reaaliluku. Rajoitumme tässä kirjassa kuitenkin käsittelemään tapauksia, jossa $n = 0, 1, 2, 3\ldots $

\begin{esimerkki}
Neliön sivun pituus särmän pituus on $x$ senttimetriä. Tällöin neliön pinta-ala on $A(x)=x^2$ senttimetriä. Jos neliön sijaan tarkasteltava esine olisi kuutio, jossa $x$ kuvaa kuution särmän pituutta, ilmaisisi funktio $V(x)=x^3$ kuution tilavuutta.
\end{esimerkki}

Potenssin asteella on suuri merkitys funktion kuvaajan muotoon. Funktiot, joiden aste on parillinen, muodostavat U:n muotoisia kuvaajia ja saavat (a:n merkistä riippuen) vain positiivisia tai negatiivisia arvoja. Parittomat potenssifunktiot muodostavat sen sijaan "kaksoismutkan", ja saavat sekä positiivisia että negatiivisia arvoja.

\todo{potenssifunktioiden kuvaajien kuvat - yksi parillisilla potensseilla ja toinen parittomilla (kerroin positiivinen)}

Yhtälöt, jotka ovat muotoa $a\cdot x^n = b$ (joskus myös muotoa  ($a\cdot x^n - b = 0$), ratkaistavuus riippuu useammasta seikasta. Mikäli n on pariton, yhtälö on aina ratkaistavissa:

\begin{align*}
a\cdot x^n &= b \\
x^n &= \frac{b}{a} 
x &= \sqrt[n]{\frac{b}{a}}
\end{align*}

