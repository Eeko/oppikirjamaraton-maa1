
\chapter{Esipuhe}

%%%%%%%%%%%%%%%%%%%%%%%%%%%%%%%%%%%%%%%%%%%%%%%%%%%%%%%%%%%%%%%%%%%%%%%%%%%%%%%%
%%%%  /usr/share/doc/texlive-fonts-extra-doc/fonts/arev/mathtesty.tex

% mathtesty.tex, by Stephen Hartke 20050522
% based on mathtestx.tex in the mathptmx package
% and symbols.tex by David Carlisle

Matematiikka tarjoaa työkaluja asioiden jäsentämiseen, päättelyyn ja mallintamiseen. Alasta riippuen käsittelemme matematiikassa erilaisia \textbf{objekteja}: Geometriassa tarkastelemme tasokuvioita ja kolmiulotteisia rakenteita. Algebra tutkii lukujen ja funktioiden ominaisuuksia. Todennäköisyyslaskenta arvioi erilaisten tapausten ja tilanteiden mahdollisuuksia ja riskejä. Matemaattinen analyysi (kurssit 7,8 ja 10) tutkii funktioita ja niiden muuttumista.

Jokaiseen tarkastelukohteeseen liitetään myös niille ominaisia \textbf{operaatioita}. Tämä kurssi käsittelee lähinnä lukuja ja niiden operaatioita, joita \textbf{laskutoimituksiksi} kutsutaan. Kirjan ensimmäisessä osassa käsittelemme luvun käsitteen, yleisimmät lukutyypit ja lukujen tavallisimman laskutoimitukset.

Lukion opetussuunnitelman (vm 2003) mukaan pitkän matematiikan ensimmäisen kurssin tavoitteena on, että opiskelija 
\begin{itemize}
\item vahvistaa yhtälön ratkaisemisen ja prosenttilaskennan taitojaan,
\item syventää verrannollisuuden, neliöjuuren ja potenssin käsitteiden ymmärtämistään,
\item tottuu käyttämään neliöjuuren ja potenssin laskusääntöjä,
\item syventää funktiokäsitteen ymmärtämistään tutkimalla potenssi- ja eksponenttifunktioita,
\item ja oppii ratkaisemaan potenssiyhtälöitä.
\end{itemize}
Kurssin keskeisin sisältö opetussuunnitelmissa on:
\begin{itemize}
\item potenssifunktio
\item potenssiyhtälön ratkaiseminen
\item juuret ja murtopotenssi
\item eksponenttifunktio.
\end{itemize}

\section*{Sananen kirjasta}

Tämä kirja on syntynyt noin kahdenkymmenen hengen työn tuloksena pitkänä viikonloppuna 28.--30.9.2012. Tulos on nyt kädessäsi. Kirjan \LaTeX-lähdekoodi on saatavilla Githubissa osoitteessa \url{https://github.com/linjaaho/oppikirjamaraton-maa1}.



\section*{Kiitämme}
\begin{itemize}
\item Metropolia AMK
\item Tekniikan akateemisten liitto TEK
\item Senja Larsen
\item Kebab Pizza Service
\item Juhapekka "Naula"\ Tolvanen
\item Onnibus
\item Arto Piironen
\item Johanna Rämö
\item Kaikki tärkeät tyypit joiden nimi unohtui mainita
\end{itemize}

%%%%%%%%%%%%%%%%%%%%%%%%%%%%%%%%%%%%%%%%%%%%%%%%%%%%%%%%%%%%%%%%%%%%%%%%%%%%%%%%
%%%% /usr/share/doc/texlive-doc-en/fonts/free-math-font-survey/source/textfragment.tex

%%% Local Variables: 
%%% mode: latex
%%% End: 
