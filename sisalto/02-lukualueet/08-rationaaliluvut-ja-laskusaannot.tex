\chapter{Rationaaliluvut ja niiden laskusäännöt}

\emph{Rationaaliluvulla} tarkoitetaan lukua $q$, joka voidaan esittää muodossa
\[
q=\frac{a}{b}, 
\]
missä $a$ ja $b$ ovat kokonaislukuja ja $b\neq 0$. Rationaalilukujen joukkoa merkitään symbolilla $\mathbb{Q}$. Tällaista esitystä kutsutaan \emph{murtoluvuksi}. Tässä esityksessä lukua $a$ kutsutaan \emph{osoittajaksi} ja lukua $b$ \emph{nimittäjäksi}. Kaikki rationaaliluvut voidaan esittää murtolukuina, mutta tämä ei ole ainoa tapa rationaalilukujen esittämiseksi. 

\laatikko{
<<<<<<< HEAD
Murtoluku on muotoa
\[
\frac{a}{b}
\]
missä $a,b\in \mathbb{Z}$ ja $a\neq 0$. Jokainen murtoluku on rationaaliluku ja jokainen rationaaliluku voidaan esittää murtolukuna.
}

Seuraavaksi tutkitaan laskutoimituksia murtoluvuilla. Muotoa $a/c$ ja $b/c$ olevien murtolukujen yhteenlaskussa lukujen osoittajat lasketaan yhteen:
\[
\frac{a}{c} + \frac{b}{c} = \frac{a+b}{c}.
\]
Tällaisia murtolukuja joilla on sama nimittäjä kutsutaan samannimisiksi. Jos lasketaan yhteen murtolukuja joilla on eri nimittäjät, ne täytyy kertoa samannimisiksi eli \emph{laventaa} ennen kuin ne voidaan laskea yhteen. Jos siis $a/b$ ja $c/d$ ovat murtolukuja ja $b\neq d$, lasketaan
\[
\frac{a}{b} + \frac{c}{d} = \frac{ad}{bd} + \frac{bc}{bd} = \frac{ad+bc}{bd}
\]
Tässä luku $a/b$ lavennetaan luvun $c/d$ nimittäjällä $d$ ja vastaavasti luku $c/d$ lavennetaan ensimmäisen yhteelaskettavan nimittäjällä $b$. Näin saadaan kaksi samannimistä murtolukua, joiden kummankin nimittäjä on yhteenlaskettavien nimittäjien tulo $bd$.

\begin{esimerkki}
Laske
\[
\frac{1}{2} + \frac{1}{6} + \frac{2}{6}.
\]

{\bf Ratkaisu.}
Aluksi lavennetaan luvut saman nimisiksi ja sen jälkeen lasketaan ne yhteen:
\begin{eqnarray*}
\frac{1}{2} + \frac{1}{6} + \frac{2}{6}
&=&
\frac{3\cdot 1}{\cdot 2} + \frac{1}{6} + \frac{2}{6}\\
&=&
\frac{3}{6} + \frac{1}{6} + \frac{2}{6}\\
&=& \frac{3+1+2}{6}\\ &=& \frac{6}{6} = 1.
\end{eqnarray*}

{\bf Vastaus.} $1$.

\end{esimerkki}

Murtolukujen $a/b$ ja $c/d$ tulo lasketaan kertomalla lukujen osoittajat ja nimittäjät keskenään:
\[
\frac{a}{b}\cdot \frac{c}{d} = \frac{a\cdot c}{b\cdot d} = \frac{ac}{cd}.
\]
Rationaaliluvun $q\neq 0$ \emph{käänteisluvulla} tarkoitetaan sellaista lukua $q^{-1}$, jolle pätee
\[
q\cdot q^{-1} = 1.
\]
Jos rationaaliluku $q\neq 0$ on esitetty murtolukumuodossa $q=a/b$, niin sen käänteisluku saadaan vaihtamalla osoittaja ja nimittäjä: $q^{-1} = b/a$.

Murtolukujen $p=a/b$ ja $q=c/d\neq 0$ \emph{osamäärä} $p/q$ saadaan, kun kerrotaan luku $p$ luvun $q$ käänteisluvulla:
\[
\frac{p}{q} = p\cdot q^{-1} = \frac{a}{b}\cdot\Big(\frac{c}{d}\Big)^{-1} = \frac{a}{b}\cdot \frac{d}{c}
= \frac{ad}{bc}.
\]



\laatikko{
<<<<<<< HEAD
{\bf Murtolukujen laskusääntöjä}

Yhteenlasku
=======
Rationaaliluvulla tarkoitetaan sellaista lukua, jonka voi esittää kahden kokonaisluvun osamääränä eli jakolaskuna.
=======
Rationaaliluvut ovat lukuja, jotka voidaan esittää kahden kokonaisluvun
välisenä jakolaskuna eli niiden osamääränä.
>>>>>>> 336abc03beeba6eb40ab6dbfd27eaff5118deeb6
}

Rationaaliluvut ovat siis muotoa $\frac{m}{n}$ olevia lukuja, jossa
$m$ ja $n$ ovat kokonaislukuja. Lisäksi $n$:lle esitetään vaatimus
$n \neq 0$, koska nollalla jakamista ei ole määritelty.

Eräs esimerkki rationaaliluvusta on $5$, joka voidaan esittää
kahden kokonaisluvun osamääränä muun muassa muodoissa $\frac{5}{1}$,
$\frac{15}{3}$ ja $\frac{-20}{-4}$. Yleisemmin kaikki kokonaisluvut
ovat rationaalilukuja. Rationaalilukuihin kuuluu myös ei-kokonaislukuja:
esimerkiksi $\frac{1}{2}$, $\frac{1}{3}$ ja $\frac{1}{4}$ ovat rationaalilukuja
mutteivät kokonaislukuja.

Rationaaliluvut ovat siis tiettyjen lukujen muodostama lukujoukko.
\emph{Murtoluvut} ovat lukujen esitystapa, jossa 

Jakoviivan yläpuolista osaa kutsutaan \emph{osoittajaksi} ja
alapuolista osaa \emph{nimittäjäksi}.

Esimerkiksi luku $5$ on rationaaliluku, koska se voidaan esittää
esimerkiksi .

Samoin luku $-7$ on rationaaliluku, koska se saadaan esimerkiksi laskutoimituksesta $\frac{-14}{2}$.

\laatikko{
Jos nimittäjässä on eri luku, murtoluvut pitää ensin kertoa samannimisiksi eli \emph{laventaa}, jotta ne voi laskea yhteen.
>>>>>>> dd7fef0584db0820d69d588112e00680a7da5abe
\begin{equation}
\frac{a}{b} + \frac{c}{d} = \frac{ad}{bd} + \frac{bc}{bd} = \frac{ad+bc}{bd}
\end{equation}
Kertolasku
\begin{equation}
\frac{a}{b}\cdot \frac{c}{d} = \frac{a\cdot c}{b\cdot d} = \frac{ac}{cd}.
\end{equation}
Jakolasku
\begin{equation}
\frac{a}{b}\Big/\frac{c}{d}= \frac{a}{b}\cdot \frac{d}{c}
= \frac{ad}{bc}.
\end{equation}
}

Kahta murtolukua vertailtaessa kannattaa ensin laventaa ne saman nimisiksi.

\begin{esimerkki}
Salamipizza jaetaan kuuteen ja tonnikalapizza neljään yhtä suureen siivuun. Vesa saa kaksi siivua salamipizzaa ja yhden siivun tonnikalapizzaa. Minttu saa kaksi siivua tonnikalapizzaa. Kumpi saa enemmän pizzaa, jos molemmat pizzat ovat saman kokoisia?

\missingfigure{tähän kuva pizzoista}

{\bf Ratkaisu.}

Huomataan, että $12 = 3\cdot 4 = 2\cdot 6$. Siten ratkaisussa esiintyvät luvut kannattaa laventaa niin, että nimittäjänä on luku $12$. Vesan saama määrä pizzaa on
\[
\frac{2}{6} + \frac{1}{4} = \frac{2\cdot 2}{2\cdot 6} + \frac{3\cdot 1}{3\cdot 4} 
=\frac{4}{12}+\frac{3}{12} = \frac{7}{12}.
\]
Mintun saama määrä pizzaa on
\[
\frac{2}{4} = \frac{3\cdot 2}{3\cdot 4} = \frac{6}{12}.
\]
Koska $6/12 < 7/12$, Vesa saa enemmän.

{\bf Vastaus.} Vesa saa enemmän.
\end{esimerkki}



\laatikko{
Kokonaisluvun voi esittää murtolukuna asettamalle sen nimittäjäksi luvun yksi.
}

\begin{esimerkki}
Laske
\[
2 + \frac{1}{3}.
\]

{\bf Ratkaisu.}

Kirjoitetaan aluksi
\[
2=\frac{2}{1}.
\]
Lavennetaan saman nimisiksi ja lasketaan yhteen
\[
2 + \frac{1}{3} = \frac{2}{1} + \frac{1}{3} = \frac{3 \cdot 2}{3 \cdot 1} + \frac{1}{3} = \frac{6+1}{3} = \frac{7}{3}.
\]

{\bf Vastaus.}
\[
\frac{7}{3}.
\]
\end{esimerkki}


Kaksi ja yksi kolmasosa karkkipussillista karkkia on sama määrä pahoinvointia kuin seitsemän kolmasosakarkkipussillista karkkia.

\todo{enemmän asiaa prosenteista}

Yksi prosentti vastaa yhtä sadasosaa: $1 \% = \frac{1}{100}$

Laske %aika randomit luvut

\begin{tehtava}
    \begin{enumerate}[a)]
	\item $\frac{3}{5} + \frac{1}{5}$
	\item $\frac{5}{7} + \frac{4}{7}$
	\item $2 + \frac{2}{3}$
	\item$3 + \frac{3}{5} + \frac{2}{5}$   
    \end{enumerate}
    \begin{vastaus}
		\begin{enumerate}[a)]
			\item $\frac{4}{5}$
			\item $\frac{9}{7} = 1 \frac{2}{7}$
			\item $2 \frac{2}{3} = \frac{8}{3}$
			\item $4$
		\end{enumerate}
    \end{vastaus}
\end{tehtava}

\begin{tehtava}

\begin{enumerate}[a)]
	\item $\frac{6}{2} + \frac{3}{5}$
	\item $\frac{7}{8} - \frac{1}{4}$
	\item $2 \frac{1}{3} + \frac{4}{6}$
	\item $4 \frac{7}{2} - 6 \frac{5}{4}$
\end{enumerate}
    \begin{vastaus}		
		\begin{enumerate}[a)]
			\item $\frac{18}{5}$
			\item $\frac{5}{8}$
			\item $3$
			\item $-\frac{41}{6}$ 
		\end{enumerate}
    \end{vastaus}
\end{tehtava}

\begin{tehtava}

\begin{enumerate}[a)]
	\item $2 \cdot \frac{2}{5}$
	\item $2 \cdot \frac{2}{3}$
	\item $\frac{5}{4} \cdot 2 \cdot 3$
	\item $\frac{\frac{3}{7}}{4}$ 
\end{enumerate}
    \begin{vastaus}
		\begin{enumerate}[a)]
			\item $\frac{4}{5}$
			\item $\frac{4}{3} = 1 \frac{1}{3}$
			\item $\frac{15}{2} = 7 \frac{1}{2}$
			\item $\frac{3}{28}$
		\end{enumerate}
    \end{vastaus}
\end{tehtava}

\begin{tehtava}

\begin{enumerate}[a)]
	\item $\frac{1}{3} \cdot \frac{6}{5}$
	\item $\frac{5}{4} \cdot (-\frac{2}{3})$ 
	\item $\frac{2}{5} (2 - \frac{3}{4})$
	\item $(\frac{5}{6} - \frac{1}{3})(\frac{7}{4} - \frac{3}{2})$
\end{enumerate}
    \begin{vastaus}		
		\begin{enumerate}[a)]
			\item $\frac{2}{5}$
			\item $-\frac{5}{6}$
			\item $\frac{1}{2}$
			\item $\frac{1}{8}$ 
		\end{enumerate}
    \end{vastaus}
\end{tehtava}

\begin{tehtava} %lisää kakkaa

\begin{enumerate}[a)]
	\item $ \frac{\frac{3}{7} + \frac{5}{4}}{3}$
	\item $ \frac{\frac{10}{8}}{\frac{5}{2}}$
	\item $ \frac{\frac{1}{3} - \frac{5}{10}}{\frac{3}{4} + \frac{1}{2}}$
	\item $ 3\frac{\frac{4}{2} + \frac{10}{4}}{\frac{3}{2} - \frac{2}{3}}$
\end{enumerate}
    \begin{vastaus}		
		\begin{enumerate}[a)]
			\item $\frac{47}{28}$
			\item $\frac{1}{2}$
			\item $-\frac{1}{3}$
			\item $\frac{54}{5}$
		\end{enumerate}
    \end{vastaus}
\end{tehtava}

\begin{tehtava} %lisää kakkaa
    Pontus, Viljami, Jarkko-Kaaleppi, Ahmed ja Milla leipoivat lanttuvompattipiirakkaa.
    Pontus kuitenkin söi piirakasta kolmanneksen ennen muita, ja loput piirakasta
    jaetaan muiden kanssa tasan. Kuinka suuren osan muut saavat?
    
    \begin{vastaus}
        Muut saavat piirakasta kuudesosan.
    \end{vastaus}
\end{tehtava}

\begin{tehtava} %ja lisää
    Huvipuiston sisäänpääsylippu maksaa 20 euroa, ja lapset pääsevät puoleen
    hintaan. Avajaispäivänä sisään pääsee 25\% halvemmalla. Kuinka paljon kolmen
    lapsen yksinhuoltajaperheelle maksaa päästä sisään avajaispäivänä?
    
    \begin{vastaus}
        37,50 euroa
    \end{vastaus}
\end{tehtava}
<<<<<<< HEAD

% Antti voitko lisätä tämän 
\begin{tehtava}
Kasassa palloja kolmannes on mustia, neljännes valkoisia, viidennes harmaita ja loput värikkäitä palloja.
Mikä osuus palloista on värikkäitä?
\begin{vastaus}
$1-(\frac{1}{3}+\frac{1}{4}+\frac{1}{5})=\frac{60}{60}-\frac{20}{60}-\frac{15}{60}-\frac{12}{60}
=\frac{60}{60}-\frac{47}{60}=\frac{13}{60}$.
\end{vastaus}
\end{tehtava}

% Antti voitko lisätä
\begin{tehtava}
Laske 
$\frac{8}{9}\cdot \frac{9}{7} \cdot \frac{7}{6} \cdot \frac{6}{5} \cdot \frac{5}{4} \cdot \frac{4}{3} \cdot \frac{3}{2}$.
\begin{vastaus}
$\frac{8}{2}=4$.
\end{vastaus}

\end{tehtava}


<<<<<<< HEAD
\begin{tehtava}
	Eräässä kaupassa on käynnissä loppuunmyynti ja kaikki tuotteet myydään puoleen hintaan. 
	Lisäksi kanta-asiakkaat saavat aina viidenneksen alennusta tuotteiden senhetkisestä hinnasta.
	Paljonko kanta-asiakas maksaa nyt tuotteesta joka normaalisti maksaisi 40 euroa?
	\begin{vastaus}
	$40\cdot \frac{1}{2} \cdot \frac{4}{5}=40\cdot \frac{4}{10}= 16$. 
	\end{vastaus}
\end{tehtava}

\begin{tehtava}
Kokonaisesta kakusta syödään maanantaina iltäpäivällä puolet ja jäljelle jääneestä palasesta syödään tiistaina 	iltapäivällä taas puolet. Jos kakun jakamista ja syömistä jatketaan samaan tapaan koko viikko, niin monesko 
osa kakkua on alkuperäisestä määrästä jäljellä seuraavana maanantaiaamuna?
\begin{vastaus}
Toisena päivänä aamulla kakkua on jäljellä puolet, kolmantena päivänä aamulla $1-(\frac{1}{2}+\frac{1}{4})=\frac{1}{4}$, 
neljäntenä päivänä $1-(\frac{1}{2}+\frac{1}{4}+\frac{1}{8})=\frac{1}{8}$, jne. Siis seitsemän päivän jälkeen kakkua 
on jäljellä $1-(\frac{1}{2}+\frac{1}{4}+\frac{1}{8}+\frac{1}{16}+\frac{1}{32}+\frac{1}{64}+\frac{1}{128})=\frac{1}{128}$.  
\end{vastaus}
\end{tehtava}

=======
\chapter{Laskusäännöt ja lausekkeiden sieventäminen}



$a\cdot b:b=a$

$a:b\cdot b=a$

$a:b\cdot b=a$
>>>>>>> dd7fef0584db0820d69d588112e00680a7da5abe
=======
>>>>>>> 336abc03beeba6eb40ab6dbfd27eaff5118deeb6
