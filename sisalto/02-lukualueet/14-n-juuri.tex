%\section{n. juuri}
Kaikkia juuria ei kuitenkaan kannata määritellä yksitellen. Tehdään siis mahdollisimman paljon kerralla. Edeltä kuitenkin voi huomata, että kuutiojuuri on määritelty kaikille luvuille, mutta neliöjuuri vain ei-negatiivisille luvuille. Tämä toistuu myös muissa juurissa. Määritellään siis parilliset ja parittomat juuret erikseen.

%on ehkä parempi esittää nämä molemma samalla eikä kuten subsections
%sama määritelmä, mutta todetaan parillisilla vaadittavan >= 0.
%%%RISTIRIITA ED. KANSSA
Juurimerkinnällä $\sqrt[n]{a}=b$ (luetaan \emph{ännäs juuri aasta on bee} tarkoitetaan lukua, joka toteuttaa ehdon $b^n = a$. Jotta juuri olisi ykskäsitteinen, on parillisilla juurilla ($\sqrt{a}, \sqrt[4]{a}, \sqrt[6]{a}$\ldots) vaadittava, että $b\ge0$.

%\subsection{parilliset juuret}

\laatikko{Luvun $a$ $n$.s juuri (luetaan \emph{ännäs juuri}) on ei-negatiivinen luku, jonka neliö on $a$. Tämä voidaan ilmaista lyhyemmin $\sqrt[n]{b^n}=b$.}

\subsection{parittomat juuret}
\laatikko{Luvun $a$ n.s juuri on ei-negatiivinen luku, jonka neliö on $a$. Tämä voidaan ilmaista lyhyemmin $\sqrt[n]{b^n}=b$.}

Nyt on paikallaan todeta, että toista juurta $\sqrt[2]{a}$ merkitään $\sqrt{a}$.

\begin{tabular}{c|c}
parillinen juuri & pariton juuri\\
\hline
$\sqrt[n]{a}^n=a$, $a\ge0$ & $\sqrt[n]{a}^n$, kaikilla $a$
\end{tabular}

Esimerkiksi $\sqrt[3]{-8}=-2$ koska $(-2)^2=-8$, mutta $\sqrt[4]{-8}$ ei ole määritelty, koska minkään luvun neljäs potenssi ei ole negatiivinen.

%Mitä näille kahdelle seuraavalle tehdään?
%$\sqrt[n]{ab}=\sqrt[n]{a}\sqrt[n]{b}$
%Jos n on parillinen, niin on lisäksi vaadittava, että $a\ge0$ ja $b\ge0$.
%
%$\sqrt[n]{\frac{a}{b}}=\frac{\sqrt[n]{a}}{\sqrt[n]{b}}$
%Jos n on parillinen, niin on lisäksi vaadittava, että $a\ge0$ ja $b\ge0$.

