\chapter{Rationaaliluvut ja laskusäännöt}

\laatikko{
Jos nimittäjässä on eri luku, murtoluvut pitää ensin kertoa samannimisiksi eli \emph{laventaa}, jotta ne voi laskea yhteen.
\begin{equation}
\frac{a}{b} + \frac{c}{d} = \frac{ad}{bd} + \frac{bc}{bd} = \frac{ad+bc}{bd}
\end{equation}
}

Kumpi lapsi saa enemmän pizzaa: tyttö, joka saa kaksi kolmasosasiivua ($ \frac{2}{3}$) vai poika, joka saa kolme neljäsosasiivua ($ \frac{3}{4}$)? Huomataan että $4*3=12$. Jos ajatellaankin kummankin siivuja kahdestatoistaosina, osuuksia on helpompi vertailla. Tyttö saa kahdeksan kahdestoistaosaa, koska $ \frac{2}{3} = \frac{2 \cdot 4}{3 \cdot 4} = \frac{8}{12}$. Poika saa yhdeksän kahdestatoistaosaa, koska $ \frac{3}{4} = \frac{3 \cdot 3}{3 \cdot 4} = \frac{9}{12}$. Poika saa siis enemmän.

\missingfigure{tähän kuva pizzoista}

\laatikko{
Kokonaisluvun voi esittää murtolukuna asettamalle sen nimittäjäksi luvun yksi.
\begin{equation}
2 + \frac{1}{3} = \frac{2}{1} + \frac{1}{3} = \frac{3 \cdot 2}{3 \cdot 1} + \frac{1}{3} = \frac{6+1}{3} = \frac{7}{3}
\end{equation}
}

Kaksi ja yksi kolmasosa karkkipussillista karkkia on sama määrä pahoinvointia kuin seitsemän kolmasosakarkkipussillista karkkia.

\todo{enemmän asiaa prosenteista}

Yksi prosentti vastaa yhtä sadasosaa: $1 \% = \frac{1}{100}$

Laske %aika randomit luvut

\begin{tehtava}
\begin{enumerate}[a)]
	\item $\frac{3}{5} + \frac{1}{5}$
	\item $\frac{5}{7} + \frac{4}{7}$
	\item $2 + \frac{2}{3}$
	\item$3 + \frac{3}{5} + \frac{2}{5}$   
\end{enumerate} 
    \begin{vastaus}
		\begin{enumerate}[a)]
			\item $\frac{4}{5}$
			\item $\frac{9}{7} = 1 \frac{2}{7}$
			\item $2 \frac{2}{3} = \frac{8}{3}$
			\item $4$
		\end{enumerate}
    \end{vastaus}
\end{tehtava}

\begin{tehtava}

\begin{enumerate}[a)]
	\item $\frac{6}{2} + \frac{3}{5}$
	\item $\frac{7}{8} - \frac{1}{4}$
	\item $2 \frac{1}{3} + \frac{4}{6}$
	\item $4 \frac{7}{2} - 6 \frac{5}{4}$
\end{enumerate}
    \begin{vastaus}		
		\begin{enumerate}[a)]
			\item $\frac{18}{5}$
			\item $\frac{5}{8}$
			\item $3$
			\item $-\frac{41}{6}$ 
		\end{enumerate}
    \end{vastaus}
\end{tehtava}

\begin{tehtava}

\begin{enumerate}[a)]
	\item $2 \cdot \frac{2}{5}$
	\item $2 \cdot \frac{2}{3}$
	\item $\frac{5}{4} \cdot 2 \cdot 3$
	\item $\frac{\frac{3}{7}}{4}$ 
\end{enumerate}
    \begin{vastaus}
		\begin{enumerate}[a)]
			\item $\frac{4}{5}$
			\item $\frac{4}{3} = 1 \frac{1}{3}$
			\item $\frac{15}{2} = 7 \frac{1}{2}$
			\item $\frac{3}{28}$
		\end{enumerate}
    \end{vastaus}
\end{tehtava}

\begin{tehtava}

\begin{enumerate}[a)]
	\item $\frac{1}{3} \cdot \frac{6}{5}$
	\item $\frac{5}{4} \cdot (-\frac{2}{3})$ 
	\item $\frac{2}{5} (2 - \frac{3}{4})$
	\item $(\frac{5}{6} - \frac{1}{3})(\frac{7}{4} - \frac{3}{2})$
\end{enumerate}
    \begin{vastaus}		
		\begin{enumerate}[a)]
			\item $\frac{2}{5}$
			\item $-\frac{5}{6}$
			\item $\frac{1}{2}$
			\item $\frac{1}{8}$ 
		\end{enumerate}
    \end{vastaus}
\end{tehtava}

\begin{tehtava} %lisää kakkaa

\begin{enumerate}[a)]
	\item $ \frac{\frac{3}{7} + \frac{5}{4}}{3}$
	\item $ \frac{\frac{10}{8}}{\frac{5}{2}}$
	\item $ \frac{\frac{1}{3} - \frac{5}{10}}{\frac{3}{4} + \frac{1}{2}}$
	\item $ 3\frac{\frac{4}{2} + \frac{10}{4}}{\frac{3}{2} - \frac{2}{3}}$
\end{enumerate}
    \begin{vastaus}		
		\begin{enumerate}[a)]
			\item $\frac{47}{28}$
			\item $\frac{1}{2}$
			\item $-\frac{1}{3}$
			\item $\frac{54}{5}$
		\end{enumerate}
    \end{vastaus}
\end{tehtava}

\begin{tehtava} %lisää kakkaa
    Pontus, Viljami, Jarkko-Kaaleppi, Ahmed ja Milla leipoivat lanttuvompattipiirakkaa.
    Pontus kuitenkin söi piirakasta kolmanneksen ennen muita, ja loput piirakasta
    jaetaan muiden kanssa tasan. Kuinka suuren osan muut saavat?
    
    \begin{vastaus}
        Muut saavat piirakasta kuudesosan.
    \end{vastaus}
\end{tehtava}

\begin{tehtava} %ja lisää
    Huvipuiston sisäänpääsylippu maksaa 20 euroa, ja lapset pääsevät puoleen
    hintaan. Avajaispäivänä sisään pääsee 25\% halvemmalla. Kuinka paljon kolmen
    lapsen yksinhuoltajaperheelle maksaa päästä sisään avajaispäivänä?
    
    \begin{vastaus}
        37,50 euroa
    \end{vastaus}
\end{tehtava}
