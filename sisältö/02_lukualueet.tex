


\part{Lukualueet}
\chapter{Luonnolliset luvut}

Tähän tekstiä luonnollisista luvuista.

\chapter{Joukko-oppia}
\chapter{Logiikkaa}
\chapter{Kokonaisluvut}
\chapter{Kokonaislukujen aritmetiikkaa}
\chapter{Jaollisuus \& tekijät}
\chapter{Rationaaliluvut ja laskusäännöt}

Laske %aika randomit luvut
a) $\frac{6}{2} + \frac{3}{5}$
b) $\frac{7}{8} - \frac{1}{4}$
c) $2 \frac{1}{3} + \frac{4}{6}$
Vastaus:
a) $\frac{18}{5}$
b) $\frac{5}{8}$
c) $3$

\chapter{Potenssisäännöt \& murtolausekkeiden sieventämistä}

\section{Potenssisäännöt}

%Pitkä Sigma 1, s.76
Laske lausekkeen arvo. Muista ottaa huomioon sulut!
a) $-2*4^2$
b) $(-2*4)^2$
c) $\frac{3^2}{7}$
d) $\left( \frac{3}{7} \right)^2$
e) $-2^6$
f) $(-2)^6$
Vastaus:
a) $-32$
b) 64
c) \frac{9}{7}
d) \frac{9}{49}
e) $-64$
f) 64

Sievennä
a) $x^7*x^2$
b) $(a^3)^2$
c) $\frac{a^8}{a^2}$
d) $(2y+5x)^0$
e) $\frac{1}{10*{-3}}$
Vastaus:
a) $x^9$
b) $a^6$
c) $a^4$
d) $1$
e) $1000$

\section{Murtolausekkeiden sieventäminen}

\chapter{Juuret}

\section{Neliöjuuri}

\laatikko{Luvun $a$ neliöjuuri on ei-negatiivinen luku, jonka neliö on $a$. Tämä voidaan ilmaista lyhyemmin $\sqrt{b^2}=b$.}

Neliöjuuren määritteleminen $\sqrt{a}^2=a$ ei johda samaan lopputulokseen. Pohdi, miksi näin on.
%%vai parempi antaa suoraan $\sqrt{a}^2=a$, kun $a \ge 0$
Jatkossa tälaisia määritelmän pieniä muokkauksia ja niistä aiheutuvia muutoksia olisi aina hyvä pohdiskella -- saattavat jopa auttaa muistamaan määritelmän oikean muodon.
%%%%%%%%%%%%%%% ONKO ITSEISARVO KÄSITELTY!!!!! %%%%%%%%%%%%%%%%%%%%%%%%%
%Määtitelmäksi ei kelpaisi tämäkään. $\sqrt{a^2}=|a|$
%%%%%%%%%%%%%%% ONKO ITSEISARVO KÄSITELTY!!!!! %%%%%%%%%%%%%%%%%%%%%%%%%

Neliöjuurta ei siis nyt määritelty ollenkaan negatiivisille luvuille.

%yhtälöt tulevat vasta myöhemin, siksi esimerkit köyhiä

Esimerkki
\begin{align*}
\sqrt{4} = 2\ qquad\textrm{, koska $2>0$ ja $2^2 =4$} 
\end{align*}

%pythagoraan lause on pitänyt käydä ennen tätä!!
%tässä tehtävässä pitää osata tehdä ensimmäisen asteen yhtälö
Taulutelevision kooksi on ilmoitettu mainoksessa $46''$ ja kuvasuhteeksi 19:6. Kuinka leveä televisio on arviolta? ($1''$ = 1 tuuma = 2,54 cm)
Vastaus: $44''$ tai 111 cm

\section{Kuutiojuuri}

\laatikko{Luvun $a$ kuutiojuuri on luku, jonka kuutio on $a$. Tämä voidaan ilmaista lyhyemmin $\sqrt[3]{b^3}=b$.}
Määritelmäksi voisi ottaa myös $\sqrt[3]{b^3}=b$.}

\section{n.s juuri}
Toista juurta $\sqrt[2]{a}$ merkitään $\sqrt{a}$


%\begin{Ex}
%Tehtävänanto
%\begin{solution}
%Ratkaisu
%\end{solution}
%\end{Ex}

\chapter{Murtopotenssi}
\chapter{Irrationaaliluvut}
\chapter{Reaaliluvut}
\chapter{Kompleksiluvut}
\chapter{Kertaustiivistelmä}
