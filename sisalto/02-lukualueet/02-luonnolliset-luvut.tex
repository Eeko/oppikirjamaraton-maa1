\chapter{Luonnolliset luvut}

(Joonas jatkaa tästä vielä!)

Suomen kielen verbi 'laskea' voi tarkoittaa matematiikassa kahta eri asiaa: lukumäärien laskemista ja laskutoimitusten suorittamista.

\laatikko{laskea (lukumäärä) englanti count ruotsi \_ /n
laskea (laskutoimitus) englanti calculate , ruotsi \_}

Ihmisellä ja muilla eläimillä on luonnostaan matemaattisia taitoja. Monet niistä, esimerkiksi lukumäärien laskeminen, ovat yllättävän monimutkaisia kognitiivisia prosesseja, jotka kehittyvät lapsuudessa – toisilla aiemmin, toisilla myöhemmin. Kaikki koulussa opeteltava peruslaskento ja myös matematiikka tieteen alana rakentavat tämän biologisen osaamisen päälle. Laskeminen itsessään on vain yksi matematiikan osa-alue, eikä kaikki matematiikka ole laskemista. Huomaa, että suomen kielen verbillä laskea tarkoitetaan sekä lukumäärien laskemista (engl. counting) että lukujen laskutoimitusten suorittamista (engl. calculating).

Hyvin olennaisena kehitysaskeleena niin yksilön matemaattiselle ajattelulle kuin yhteiskunnallekin on ollut luonnollisen kielen tavoin kyky merkitä lukumäärien laskemista ja muuta matemaattista pohdintaa kirjalliseen muotoon.  On olemassa hyvin monia erilaisia tapoja merkitä lukumääriä. Helpoin tapa ja yksinkertaisin tapa on käyttää vain yhtä samaa merkkiä ja toistaa sitä. Jos

käytettävissä olevien merkintöjä määrää voidaan lisätä, jolloin suuria lukuja voidaan kirjoittaa lyhyemmin. Tämä vastaa myös luonnollisten kielten tilannetta: Suomen kielen aakkosiin kuuluu 29 kirjainta, joista sanat muodostetaan. Sanat voivat olla kuinka pitkiä vain kahdesta kirjaimesta ylöspäin. Kiinassa sen sijaan käytetään omaa piirrosmerkkiä jokaiselle sanalle. Merkkejä täytyy osata 29 sijaan tuhansia, mutta jokaisen sanan voi kirjoittaa lyhyesti. 
Matematiikassa erilaisista numeromerkeistä tai yksinkertaisesti numeroista muodostetaan lukuja yhdistelemällä niitä sopivasti erilaisten paikkajärjestelmien mukaan. Esimerkiksi antiikin Roomassa käytössä olivat numeromerkit I, V, X, L, C , D ja M. Niiden numeroiden vastaavuudet meidän käyttämiimme lukuarvoihin ovat seuraavat:
I=1
V=5
X=10
L=50
C=100
D=500
M=1 000
Huomaa, että suuri osa roomalaisista numeromerkeistä ovat jo itsessään arvoltaan niin suuria, että me tarvitsemme niiden nykyilmaisuun monta merkkiä! Nollaa roomalaisissa numeroissa ei ole, ja tiettävästi tuhatta suurempia arvoja esittäviä numeromerkkejä merkkejä otettiin käyttöön vasta keskiajalla. 
Lukuja koostetaan näistä merkeistä siten, että merkit kirjoitetaan peräkkäin pääasiassa laskevassa järjestyksessä ja niiden numeroarvot lasketaan yhteen. Jos arvoltaan pienempi numeromerkki (korkeintaan yksi) edeltää suurempaa, pienempi vähennetään suuremmasta ennen yhteenlaskun jatkamista. 
