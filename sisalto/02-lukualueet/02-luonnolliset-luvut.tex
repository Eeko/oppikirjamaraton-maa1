\chapter{Luonnolliset luvut}

Edellisen luvun johdattelun mukaisesti yksinkertaisimmat luvut, joita käytämme, ovat niin sanotut luonnolliset luvut – ne luvut, joiden laskemme lukumääriä. Merkkaamme \textbf{luonnollisten lukujen joukkoa} seuraavasti:

\[\mathbb{N}={1, 2, 3, 4, 5, ...}\]

Luonnollisiin lukuihin luetellaan kuuluvaksi siis kaikki positiiviset kokonaisluvut, joita on ääretön määrä. Joillakin matematiikan aloilla luonnollisiin lukuihin lasketaan kuuluvaksi myös luku $0$. Jos 

\sivulaatikko{engl. \emph{natural numbers, counting numbers} ruots. \emph{naturliga tal}}


\laatikko{Luonnollisia lukuja käytetään kolmeen eri tarkoitukseen:

\begin{enumerate}
\item Lukumäärien ilmoittamiseen (kaardinaaliluvut)
\item Järjestyksen ilmoittamiseen (ordinaaliluvut)
\item Indeksointiin ja asioiden nimeämiseen
\end{enumerate}
}


\section{Tehtäviä}

\begin{tehtava}

Onko kaardinaali vai ordinaali vai indeksointi?

\end{tehtava}