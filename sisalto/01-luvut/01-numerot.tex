\chapter{Numerot ja luvut}

Ihmisillä ja eläimillä on luonnostaan matemaattisia taitoja. Monet niistä, esimerkiksi lukumäärien laskeminen, ovat monimutkaisia ajatusprosesseja, jotka kehittyvät lapsuudessa – toisilla aiemmin, toisilla myöhemmin. Koulussa opeteltava peruslaskento ja myös matematiikka tieteenalana rakentuvat tämän biologisen osaamisen päälle. Laskeminen itsessään on vain eräs matematiikan osa-alue, eikä kaikki matematiikka ole laskemista.

\sivulaatikko{Huom.! laskea (lukumäärä) englanti count ruotsi \_ /n laskea (laskutoimitus) englanti calculate , ruotsi \_}

\subsubsection*{Yleisesti lukujen merkitsemisestä}

Laskutoimitusten ja muun matemaattisen pohdinnan merkitseminen kirjalliseen muotoon on taito, jonka oppiminen oli merkkittävä askel ihmisen kehityshistoriassa. Aikojen saatossa eri kansat ovat käyttäneet erilaisia tapoja merkitä laskutoimituksia ja lukumääriä. Tunnettu esimerkki on roomalaiset numerot, jotka näyttävät hyvin erilaisilta nykyään käyttämiimme lukumerkintöihin.

Helpoin ja yksinkertaisin tapa merkitä lukumäärää on käyttää vain yhtä merkkiä ja toistaa sitä:

\missingfigure{Piirrettynä "tukkimiehen kirjanpitoa" ja  vertaus arabialaisilla numeroilla}

Kun käytettävissä olevien merkkien määrää lisää, suuria lukuja voi kirjoitaa lyhyempään muotoon. Tilanne on verrattavissa vaikkapa kiinan kieleen, jossa on käytössä satoja erilaisia kirjoitusmerkkejä. Merkeissä on paljon muistettavaa, mutta toisaalta kokonaisen lauseen voi kirjoittaa vain parilla kirjoitusmerkillä.



%%%ROOMALAISET LUVUT%%%%%%

\subsection*{Roomalaiset luvut}
Antiikin Roomassa käytössä olivat numeromerkit I, V, X, L, C , D ja M. Niiden numeroiden vastaavuudet meidän käyttämiimme lukuarvoihin ovat seuraavat:

\begin{equation*}
\rm I=1\quad
V=5\quad
X=10\quad
L=50\quad
C=100\quad
D=500\quad
M=1 000
\end{equation*}

%Huomaa, että suuri osa roomalaisista numeromerkeistä ovat jo itsessään arvoltaan niin suuria, että tarvitsemme niiden nykyilmaisuun monta merkkiä! Nollaa roomalaisissa numeroissa ei ole, ja tiettävästi tuhatta suurempia arvoja esittäviä numeromerkkejä merkkejä otettiin käyttöön vasta keskiajalla. 

Lukuja koostetaan näistä merkeistä siten, että merkit kirjoitetaan peräkkäin pääasiassa laskevassa järjestyksessä ja niiden numeroarvot lasketaan yhteen. Jos arvoltaan pienempi numeromerkki (korkeintaan yksi) edeltää suurempaa, pienempi vähennetään suuremmasta ennen yhteenlaskun jatkamista.

\begin{esimerkki}
$\text{XIV} = 10 + (5 - 1) = 14$
\end{esimerkki}

Roomalaisia numeroita käytetään usein nykyään merkitsemään lukumääriä. Niitä näkee käytettävän myös esimerkiksi kuninkaiden nimien jälkeen ilmaisemaan, kuinka mones samanniminen kuningas on kyseessä.

\subsection*{Paikkajärjestelmä}

Eri lukujärjestelmille on yhteistä se, että niissä lukumäärää merkitään numeromerkeillä, joiden kirjoitusjärjestyksellä on väliä. Esimerkiksi luvut $134$ ja $413$ eivät ole sama luku; saamme eri lukuja, kun numeroita yhdistellään eri tavoin.

\begin{esimerkki}
\begin{itemize}
\item III$=1+1+1=3$
\item IX$=10-1=9$
\item XII$=10+1+1=12$
\item XIX$=10-1+10=19$
\item CDX$=500-100+10=410$
\item MCMD$=1 000+1 000-100+500$
\end{itemize}
\end{esimerkki}

Länsimaisessa traditiossa käytössämme on kymmenen numeromerkkiä: 0, 1, 2, 3, 4, 5, 6, 7, 8 ja 9. Näitä kutsutaan alkuperänsä mukaan hindu-arabialaisiksi numeroiksi. Kirjoittamalla näitä numeromerkkejä peräkkäin ilmaisemme lukuja.

\begin{esimerkki}
Luku \[715531\] koostuu numeroista 7, 1, 5, 5, 3 ja 1.
\end{esimerkki}

Järjestelmässämme, jota kutsutaan \emph{kymmenjärjestelmäksi}, kunkin numeron paikka kertoo, kuinka monta kymmentä, sataa, tuhatta... jne. numero tarkoittaa. Esimerkiksi $562 = 5 \cdot 100 + 6 \cdot 10 + 2$.

\sivulaatikko{Englannin kielen sana \textit{number} voi viitata sekä numeroon että lukuun. Sana \textit{digit} tarkoittaa pelkästään yhtä numeromerkkiä. Ruotsiksi luku on \textit{tal}, lukumäärä \textit{antal} ja numeroa tai lukumäärää tarkoittamatonta numeroyhdistelmää kuvaa suomen kielen tapaan sana \textit{nummer}.}

\subsection*{Muut lukujärjestelmät}

Järjestelmäämme merkitä lukuja kutsutaan kymmenjärjestelmälliseksi tai desimaalijärjestelmäksi, koska siinä hyödynnetään kymmentä eri numeromerkkiä. Muita yleisesti käytössä olevia järjestelmiä ovat 2-järjestelmä eli binäärijärjestelmä ja 16-järjestelmä eli heksadesimaalijärjestelmä, joita käytetään digitaalisen informaation tallentamiseen ja käsittelemiseen. Binäärijärjestelmässä luvun muodostavia numeromerkkejä kutsutaan biteiksi. Bitti voi olla joko päällä (1) tai pois päältä (0) ja toteutus tietokoneessa vastaa esimerkiksi sitä, että johtimessa kulkee virta (1) tai ei (0). Useampaa järjestelmää käytettäessä merkitään kantaluku luvun jälkeen alaindeksinä. Esimerkiksi luku yhdeksäntoista voidaan merkitä $19_{10}$, $10011_{2}$ tai $13_{16}$ käyttäen desimaali, binääri tai heksadesimaalijärjestelmää.
\begin{align}
19_{10} &= 1 \cdot 10 + 9 \\
10011_{2} &= 1 \cdot (2 \cdot 2 \cdot 2 \cdot 2) + 0 \cdot (2 \cdot 2 \cdot 2) + 0 \cdot (2 \cdot 2) + 1 \cdot 2 + 1 \\
13_{16} &= 1 \cdot 16 + 3
\end{align}
Kuusitoistajärjestelmässä tarvitaan vielä kuusi uutta numeromerkkiä. Tavaksi on vakiintunut käyttää kirjainmerkkejä $\mathrm{A, B, C, D, E}$ ja $\mathrm{F}$. Ne merkitsevät lukuja $10, 11, 12, 13, 14$ ja $15$. Yleisesti $n$-järjestelmässä käytetään $n$:ää kappaletta eri merkkejä, jotka merkitsevät lukuja nollasta lukuun $n-1$.
\begin{esimerkki}
$F4B_{16} = F \cdot (16 \cdot 16) + 4 \cdot 16 + B = 15 \cdot (16 \cdot 16) + 4 \cdot 16 + 12 = 3916_{10}$
\end{esimerkki}

\subsection*{Muuta, keksi otsikko}

\todo{jotain selitystä näistä: postinumero, puhelinnumero}


\sivulaatikko{Kielioppihuomautuksia: 1) Tuhaterottimena käytetään välilyöntiä, ei pilkkua tai pistettä. 2) Suomessa on käytössä desimaalipilkku, ei -piste! Yhdysvaltalaiset ovat suunnitelleet laskimesi.}


\section{Jos tämä on matematiikkaa, miksi käytämme kirjaimia?}

Numeroita kuvaavat merkit ovat mielivaltaisia symboleita. Lukujakin edustamaan päädytään joskus käyttämään jotakin lyhennysmerkintää

yleisyys

suuruus, yhtäsuuruusmerkintä, eri suuret

Jos kaksi lukua, $x$ ja $2$ ovat yhtä suuret, merkataan
$x=2$. Huomaa, että tällainen yhtäsuuruusrelaatio pätee aivan hyvin myös toisin päin: $2=x$, joten kumpikin kirjoitustapa on tilanteesta riippumatta oikein.

symbolit, muuttujat, alkiot, kreikkalaisia...

\sivulaatikko{Painotekstissä kirjaimella merkityt muuttujat kirjoitetaan \textit{kursiivilla} ja aina saman arvon saavat matemaattiset vakiot kuten $\pi$ pystyyn.}


Erilaisilla luvuilla voidaan suorittaa erilaisia laskutoimituksia. Seuraavissa luvuissa esitellään ja käydään läpi lukiomatematiikassa ja mahdollisissa jatko-opinnoissa käytettäviä lukujoukkoja ja tavallisimmat laskutoimitukset.
