%
\part{Sovelluksia}
%
%(Pythagoraan lause)
%
\chapter{Verrannollisuus}

Verrannollisuudella tarkoitetaan tilannetta, jossa 

Suoraan verrannollisuus tarkoittaa, että kahden asian suhde pysyy vakiona. Jos toinen kaksinkertaistuu, kaksinkertaistuu toinenkin. Esimerkiksi kaupasta ostettujen hedelmien määrä ja kauppahinta ovat suoraan verrannollisia toisiinsa. Jos ostat kaksi kertaa enemmän banaaneja, joudut myös maksamaan kaksi kertaa enemmän. Hinnan ja ostettujen banaanien massan\footnote{Arkikielessä puhutaan yleensä painosta.} suhde on molemmissa tapauksissa vakio. Hedelmäesimerkin tapauksessa tätävakiota kutsutaan kilohinnaksi.

Matemaattisesti suoraan verrannollisuus merkitään seuraavasti. Jos suure $a$ on suoraan verrannollinen suureeseen $b$, merkitään
\begin{equation}
\frac{a}{b}=c,
\end{equation}
missä $c$ on vakio.

Kääntäen verrannollinen tarkoittaa, että kahden asian tulo pysyy vakiona. Jos toinen kaksinkertaistuu, toinen puolittuu. Esimerkiksi nopeus ja matkaan tarvittava aika ovat kääntäen verrannollisia toisiinsa. Jos ajat koulumatkan kaksi kertaa nopeammin, matka-aika puolittuu.

Matemaattisesti kääntäen verrannollisuus merkitään seuraavasti. Jos suure $a$ on kääntäen verrannollinen suureeseen $b$, merkitään
\begin{equation}
ab=c,
\end{equation}
missä $c$ on vakio.


\chapter{Verrannollisuus: sovelluksia}

\begin{tehtava}
% Lyhyt matikka 1, s. 72
Pohdi, kuinka toinen suure muuttuu, kun toinen suure kaksinkertaistuu, kolminkertaistuu, puolittuu jne. Ovatko suureet suoraan verrannolliset?
\begin{enumerate}
\item kuljettu matka ja kulunut aika, kun keskinopeus on 30 km/h
\item kananmunien lukumäärä ja niiden kovaksi keittämiseen tarvittava keittoaika
\item hedelmätiskiltä valitun vesimelonin paino ja hinta
\item neliön sivun pituus ja neliön pinta-ala
\end{enumerate}
\begin{vastaus}
Vastaus:
\begin{enumerate}
\item Ovat.
\item Eivät ole.
\item Ovat.
\item Eivät ole, sillä esimerkiksi kun neliön sivun pituus kaksinkertaistuu 1 cm:stä 2 cm:iin, niin neliön pinta-ala nelinkertaistuu 1 cm$^2$:stä 4 cm$^2$:iin.
\end{enumerate}
\end{vastaus}
\end{tehtava}

\begin{tehtava}
Isi ja lapset ovat ajamassa mökille Sotkamoon. Ollaan ajettu jo neljä viidennestä matkasta ja aikaa on kulunut kaksi tuntia. "Joko ollaan perillä?" kysyvät lapset takapenkiltä. Kuinka pitkään vielä arviolta kuluu, ennen kuin ollaan mökillä?
\begin{vastaus}
Vastaus: 1 h 15 min
\end{vastaus}
\end{tehtava}

\begin{tehtava}
Äidinkielen kurssilla annettiin tehtäväksi lukea eräs 300-sivuinen romaani. Eräs opiskelija otti aikaa ja selvitti lukevansa vartissa seitsemän sivua. Kuinka monta tuntia häneltä kuluu arviolta koko romaanin lukemiseen, jos taukoja ei lasketa?
\begin{vastaus}
Vastaus: 642 minuuttia eli 10 h 42 min.
\end{vastaus}
\end{tehtava}

\chapter{Prosenttilaskentaa - perustilanteet}

Kun lukuja verrataan toisiinsa, lasketaan niiden suhde eli osamäärä. Tämä kertoo jaettavan suhteellisen osuuden jakajasta. Suhteellinen osuus ilmaistaan usein prosentteina. Yksi prosentti tarkoittaa yhtä sadasosaa. Prosentin merkki on $\%$.

\laatikko{1 prosentti $= 1 \% = \frac{1}{100} = 0,01$}

\laatikko{Esimerkki: \\$6 \% = \frac{6}{100} = 0,06$, $48,2 \% = \frac{48}{100} = 0,482$, $140 \% = \frac{140}{100} = 1,40$}

Minkä tahansa suhdeluvun voi muuttaa prosenteiksi laskemalla osamäärän desimaalilukuna ja ottamalla siitä sadasosat.
% Pitäisi muotoilla se, että tuhannesosat ovat sitten 0,1 prosenttia jne. eikä että katkaistaan desimaaliluku sadasosiin :)

Kahden prosenttiluvun välisen erotuksen yksikköä kutsutaan prosenttiyksiköksi.

\section{Perusprosenttilaskut}

\begin{itemize}
	\item Prosenttiluvun laskeminen
	\item Prosenttiarvon laskeminen
	\item Perusarvon laskeminen
\end{itemize}

\section{Vertailu prosenttien avulla}

\begin{itemize}
	\item Muutosprosentti, vertailuprosentti
	\item Prosentuaalinen muutos
	\item Prosenttiyksikkö
\end{itemize}

\chapter{Prosenttiyhtälöitä ja sovelluksia}

\begin{tehtava}
Laukku maksaa 225 \euro ja on 25\%:n alennuksessa. Paljonko alennettu hinta on?
\begin{vastaus}
Vastaus: 168,75 \euro
\end{vastaus}
\end{tehtava}

\begin{tehtava}
%Pyramidi 1, s. 80
Kirjan myyntihinta, joka sisältää arvolisäveron, on 8\% suurempi kuin kirjan veroton hinta. Laske kirjan veroton hinta, kun myyntihinta on 15\euro.
\begin{vastaus}
Vastaus: 13,89 \euro
\end{vastaus}
\end{tehtava}

\begin{tehtava}
Perussuomalaisten kannatus oli vuoden 2007 eduskuntavaaleissa 4,1\% ja vuoden 2011 eduskuntavaaleissa 19,1\%. Kuinka monta prosenttiyksikköä kannatus nousi? Kuinka monta prosenttia kannatus nousi?
\begin{vastaus}
Vastaus: Kannatus nousi 15 \%-yksikköä ja 365,9 \%.
\end{vastaus}
\end{tehtava}

\begin{tehtava}
Askartelukaupassa on alennusviikot, ja kaikki tavarat myydään 60\%n alennuksella. Viimeisenä päivänä kaikista hinnoista annetaan vielä lisäalennus, joka lasketaan aiemmin alennetusta hinnasta. Minkä suuruinen lisäalennus tulee antaa, jos lopullisen kokonaisalennuksen halutaan olevan 80\%?
\begin{vastaus}
Vastaus: 50\%.
\end{vastaus}
\end{tehtava}

\begin{tehtava}
%tässä tehtävässä pitää tietää potenssi
Erään pankin myöntämä opintolaina nousee korkoa 2\% vuodessa. Kuinka monta prosenttia laina on noussut korkoa alkuperäiseen summaan verrattuna kymmenen vuoden kuluttua?
\begin{vastaus}
Vastaus: 22\%.
\end{vastaus}
\end{tehtava}

Ansiotuloverotus on Suomessa progressiivista: suuremmista tuloista maksetaan

\begin{tehtava}
Tuoreissa omenissa on vettä 80\% ja sokeria 4\%. Kuinka monta prosenttia sokeria on samoissa omenissa, kun ne on kuivattu siten, että kosteusprosentti on 20? [K2000, 4]
\begin{vastaus}
Vastaus: 16\%
\end{vastaus}
\end{tehtava}

\begin{tehtava}
Kappaleen vapaa pudotus korkeudelta $x$ maahan on kääntäen verrannollinen putoamiskiihtyvyyden $g$ neliöjuureen. $g$ on kullekin taivaankappaleelle ominainen ja eri puolilla samaa taivaankappaletta likimain vakio. Empire State Buildingin katolta (korkeus $381$ m) pudotetulla kuulalla kestää n. $6,2$ s osua maahan. Marsin putoamiskiihtyvyys on $37,6$\%  Maan putoamiskiihtyvyydestä. Jos Empire State Building sijaitsisi Marsissa, kuinka monta prosenttia pitempi aika kuluisi kuulan maahan osumiseen?
\begin{vastaus}
Vastaus: $10,1$ s
\end{vastaus}
\end{tehtava}

%
%(Eksponentiaalinen malli)
%
\chapter{Kertaustiivistelmä}
