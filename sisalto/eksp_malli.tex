\section{Eksponentiaalinen malli}

Kun eksponenttifunktiota käytetään kuvaamaan jotakin reaalimaailman
ilmiötä, siitä käytetään nimeä \emph{eksponentiaalinen malli}.

Eksponentiaalinen malli on eräs yleisimmin käytetyistä matemaattisista
malleista. Sillä kuvataan sellaista kasvua tai vähenemistä, jossa
kullakin ajanhetkellä funktion hetkellinen muutos on suoraan
verrannollinen funktion sen hetkiseen arvoon. Tämä muotoillaan
täsmällisesti myöhemmillä matematiikan kursseilla.

\begin{esimerkki}
Soluviljelmässä olevien \emph{Escherichia coli} -bakteerien
määrää voidaan kuvata eksponentiaalisella mallilla: ajanhetkellä
$t = 0$ bakteerien lukumäärä on $1$, ja kullakin aika-askeleella
bakteerien lukumäärä tuplaantuu. Tämä voidaan
kirjoittaa malliksi
\[
f(t) = 2^t, t \ge 0
\]
Huomaa, että yllä olevassa esimerkissä funktion $f(t)$ arvot ovat
mielekkäitä vain, kun $t$ on kokonaisluku, koska
muussa tapauksessa bakteerien määrä saa ei-kokonaislukuarvon.
Yleensä tätä ei pidetä ongelmallisena, vaan funktiota voidaan käsitellä
ikään kuin bakteerien määrä olisi jatkuvasti kasvava suure.
\end{esimerkki}

\begin{tehtava}
Millä ajanhetkellä bakteerien lukumäärä ylittää sadan?
\begin{vastaus}
Ajanhetkellä $t = 7$.
\end{vastaus}
\end{tehtava}

\begin{tehtava}
Millainen funktio kuvaa bakteerien kasvua, jos bakteerien lukumäärä
ajanhetkellä $t = 0$ on $f(t) = 10$?
\begin{vastaus}
$f(t) = 10 \cdot 2^t$
\end{vastaus}
\end{tehtava}

\begin{esimerkki}
Radioaktiivisessa hajoamisessa atomiydinten lukumäärää voidaan
kuvata eksponentiaalisella mallilla. Jos ydinten määrä ajanhetkellä
$t = 0$ on $f(t) = k$, voidaan kirjoittaa malli
\[
f(t) = k \cdot \left( \frac{1}{2} \right)^t
\]
Mallissa oletetaan, että ydinten lukumäärä puolittuu kullakin ajanhetkellä.
\end{esimerkki}

\begin{tehtava}
Millä ajanhetkellä atomiydinten määrä on alle $1/200$ alkuperäisestä?
\begin{vastaus}
Ajanhetkellä $t = 8$.
\end{vastaus}
\end{tehtava}