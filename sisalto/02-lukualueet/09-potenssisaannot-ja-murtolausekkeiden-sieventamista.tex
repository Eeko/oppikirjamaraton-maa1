\chapter{Potenssisäännöt \& murtolausekkeiden sieventämistä}
Potenssilla $2^4$ tarkoitetaan tuloa $2\cdot 2\cdot 2\cdot 2$. Eli
\begin{equation}
2^4=2\cdot 2\cdot 2\cdot 2=16.
\end{equation}
Lausekkeessa $2^4$ luku 2 on \textbf{kantaluku} ja luku 4 on \textbf{eksponentti}.

\begin{esimerkki}
Potenssit luvuilla.

a) $(-2)^3=(-2)\cdot (-2)\cdot (-2)=-8$

b) $(-2)^4=(-2)\cdot (-2)\cdot (-2)\cdot (-2)=16$

c) $-2^4=-2\cdot 2\cdot 2\cdot 2=-16$

d) $2^2\cdot 2^3=\underbrace{2\cdot 2}_{2 kpl}\cdot \underbrace{2\cdot 2\cdot 2}_{3 \text{kpl}}=2^5=32$

e) $\frac{2^4\cdot 2^2}{2^3}=\frac{\overbrace{2\cdot 2\cdot 2\cdot \cancel{2}}\cdot \overbrace{\cancel{2}\cdot \cancel{2}}}{\cancel{2}\cdot \cancel{2}\cdot \cancel{2}}=2^3=8$
\end{esimerkki}

\begin{esimerkki}
Potenssit kirjaimilla.

a) $a^3\cdot a^4=\underbrace{a\cdot a\cdot a}_{3 kpl}\cdot \underbrace{a\cdot a\cdot a\cdot a}_{4 kpl}=a^{3+4}=a^7$

b) $\frac{a^7}{a^4}=\frac{a\cdot a\cdot a\cdot \cancel{a}\cdot \cancel{a}\cdot \cancel{a}\cdot \cancel{a}}{\cancel{a}\cdot \cancel{a}\cdot \cancel{a}\cdot \cancel{a}}=a^{7-4}=a^3$

c) $(a^2)^3=\underbrace{a^2\cdot a^2\cdot a^2}_{3 kpl}=\underbrace{a\cdot a\cdot a\cdot a\cdot a\cdot a}_{2\cdot 3=6 kpl}=a^{2\cdot 3}=a^6$

d) $(ab^5)^3=ab^5\cdot ab^5\cdot ab^5=a\cdot a\cdot a\cdot b^5\cdot b^5\cdot b^5=a^3b^{15}$

e) $\left(\frac{a^9}{b}\right)^3=\frac{a^9}{b}\cdot \frac{a^9}{b}\cdot \frac{a^9}{b}=\frac{a^9\cdot a^9\cdot a^9}{b\cdot b\cdot b}=\frac{a^{27}}{b^3}$
\end{esimerkki}

\begin{esimerkki}
Negatiivinen eksponentti.

a) $\frac{x^3}{x^5}=\frac{\cancel{x}\cdot \cancel{x}\cdot \cancel{x}}{x\cdot x\cdot \cancel{x}\cdot \cancel{x}\cdot \cancel{x}}=\frac{1}{x^2}$

b) $\frac{x^3}{x^5}=x^{3-5}=x^{-2}$

Esimerkeistä a) ja b) voidaan havaita, että 
\begin{equation}
\frac{1}{x^2}=x^{-2}
\end{equation}

\laatikko{Yleisesti
\begin{equation}
x^{-n}=\frac{1}{x^n}
\end{equation}
kun $a\neq 0$
}

\end{esimerkki}

