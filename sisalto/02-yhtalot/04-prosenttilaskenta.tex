\chapter{Prosenttilaskenta}

Sana prosentti tulee latinan kielen sanoista pro centum, mikä tarkoittaa
kirjaimellisesti sataa kohden. Prosentteja käytetään ilmaisemaan suhteellista
osuutta. Prosentin merkki on \%. Lukua, josta suhde lasketaan, kutsutaan \emph{perusarvoksi}. Esimerkiksi jos sadan euron hintaisen tuotteen hintaa on alennettu 25 prosenttia, niin alennettu hinta on 75 euroa. Jos sen sijaan alkuperäinen hinta nousee 15 prosenttia, niin tuotteen uusi hinta on 115 euroa. Perusarvo on molemmissa tapauksissa 100 euroa.

\laatikko{1 prosentti $= 1 \% = \frac{1}{100} = 0,01$}

\laatikko{Esimerkki: \\$6 \% = \frac{6}{100} = 0,06$, $48,2 \% = \frac{48,2}{100} = 0,482$, $140 \% = \frac{140}{100} = 1,40$}

Suhdeluku muutetaan prosenteiksi kertomalla se luvulla 100 ja lisäämällä
lopputuloksen jälkeen prosenttimerkki.

\begin{esimerkki}
	Vesa ansaitsee kuukaudessa 2300 euroa ja Antero 1700 euroa.
    Kuinka monta prosenttia Anteron tulot ovat Vesan tuloista? 
    
    {\bf Ratkaisu.}
    
    Lasketaan
    \[
    \frac{1700}{2300} \cdot 100 \% \approx 0,74\cdot 100 \% = 74 \%.
    \]
    Laskuissa käytettävä perusarvo on Vesan palkka eli 2300 euroa.
    
    {\bf Vastaus.}
     $74 \%$
\end{esimerkki}


\emph{Vertailuprosentilla} ilmaistaan, kuinka paljon luku on toista lukua suurempi.
Vertailuprosentin laskennassa käytetään perusarvona sitä lukua, johon
verrataan. Jos halutaan tietää, kuinka monta prosenttia luku $a$ on suurempi kuin $b$, vertailuprosentti saadaan laskettua kaavalla
\[
\frac{a-b}{b} \cdot 100 \%.
\]

\begin{esimerkki}
    Vesa ansaitsee kuukaudessa 2300 euroa ja Antero 1700 euroa.
    Kuinka monta prosenttia enemmän Vesa ansaitsee kuin Antero?
    
    {\bf Ratkaisu.}
    
    Lasketaan aluksi Vesan ja Anteron palkkojen erotus
    \[
    2300-1700 = 600.
    \]
    Sitten lasketaan, kuinka monta prosenttia 600 euroa on Anteron palkasta:
    \[
    \frac{600}{1700} \cdot 100 \% \approx 0,35\cdot 100\% = 35 \%.
    \]
    
    {\bf Vastaus.}
    $35 \%$
\end{esimerkki}

Prosentteja käytetään usein ilmaisemaan suureiden muutoksia. \emph{Muutosprosenttia} laskettaessa perusarvona on alkuperäinen arvo, johon nähden muutos on tapahtunut.

\begin{esimerkki}
    Vesan paino on tammikuussa 68 kg ja kesäkuussa 64 kg. Kuinka monta prosenttia Vesa on laihtunut?

    {\bf Ratkaisu.}

    Lasketaan 
    \[
    \frac{68-64}{68}\cdot 100\% = \frac{4}{68} \cdot 100\%=0,06\cdot 100\% = 6\%.
    \]
    
    {\bf Vastaus.}
    Vesa on laihtunut $6\%$.
\end{esimerkki}


\emph{Prosenttiyksikkö} mittaa prosenttiosuuksien välisiä eroja. Jos prosenttiluku muuttuu, muutos voidaan ilmaista joko prosentteina tai prosenttiyksikköinä.


\begin{esimerkki}
    Tuotteen markkinaosuus on vuoden tammikuussa 10 \% ja kesäkuussa 15 \%. 
    \begin{enumerate}[a)]
    \item Kuinka monta prosenttia tuotteen markkinaosuus on noussut?
    
    \item Kuinka monta prosenttiyksikköä tuotteen markkinaosuus on noussut?
    \end{enumerate}
    
    {\bf Ratkaisu.} 
    
    \begin{enumerate}[a)]
    \item Tuotteen markkinaosuus on noussut
    \[
    \frac{15-10}{10} \cdot 100 \%= \frac{5}{10}\cdot 100\% = 50\%.
    \]
    
    \item Tuotteen markkinaosuus on noussut $15-10=5$ prosenttiyksikköä. 
    \end{enumerate}
    
    {\bf Vastaus.}
    
    \begin{enumerate}[a)]
    \item 50 prosenttia
    \item 5 prosenttiyksíkköä.
    \end{enumerate}
\end{esimerkki}




\section{Perusprosenttilaskut}

\begin{itemize}
	\item Prosenttiluvun laskeminen
	\item Prosenttiarvon laskeminen
	\item Perusarvon laskeminen
\end{itemize}

\section{Vertailu prosenttien avulla}

\begin{itemize}
	\item Muutosprosentti, vertailuprosentti
	\item Prosentuaalinen muutos
	\item Prosenttiyksikkö
\end{itemize}

\section{Prosenttiyhtälöitä ja sovelluksia}

\begin{tehtava}
    Laukku maksaa 225 euroa ja on 25~\%:n alennuksessa. Mikä on alennettu hinta?
    
    \begin{vastaus}
    Vastaus: 168,75 euroa
    \end{vastaus}
\end{tehtava}

\begin{tehtava}
    %Pyramidi 1, s. 80
    Kirjan myyntihinta, joka sisältää arvolisäveron, on 8~\% suurempi kuin kirjan
    veroton hinta. Laske kirjan veroton hinta, kun myyntihinta on 15 euroa.
    
    \begin{vastaus}
        Vastaus: Kirjan veroton hinta on 13,89 euroa
    \end{vastaus}
\end{tehtava}

\begin{tehtava}
    Perussuomalaisten kannatus oli vuoden 2007 eduskuntavaaleissa 4,1~\% ja
    vuoden 2011 eduskuntavaaleissa 19,1~\%. Kuinka monta prosenttiyksikköä kannatus nousi? Kuinka monta prosenttia kannatus nousi?
    \begin{vastaus}
    Vastaus: Kannatus nousi 15 prosenttiyksikköä. Prosentteina mitattuna
    kannatus nousi 366~\%.
    \end{vastaus}
\end{tehtava}

\begin{tehtava}
    Askartelukaupassa on alennusviikot, ja kaikki tavarat myydään 60~\%:n alennuksella. Viimeisenä päivänä kaikista hinnoista annetaan 
    vielä lisäalennus, joka lasketaan aiemmin alennetusta hinnasta. Minkä suuruinen lisäalennus tulee antaa, jos lopullisen 
    kokonaisalennuksen halutaan olevan 80~\%?

    \begin{vastaus}
        Vastaus: 50\%.
    \end{vastaus}
\end{tehtava}

\begin{tehtava}
    %tässä tehtävässä pitää tietää potenssi
    Erään pankin myöntämä opintolaina kasvaa korkoa 2~\% vuodessa. Kuinka monta prosenttia laina on kasvanut korkoa alkuperäiseen 
    verrattuna kymmenen vuoden kuluttua?

    \begin{vastaus}
        Vastaus: 22~\%.
    \end{vastaus}
\end{tehtava}

\begin{tehtava}
Yleinen arvonlisäveroprosentti oli Suomessa vuonna 2012 23 \% tuotteen verottomasta
hinnasta. Tuotteen hinta koostuu sen verottomasta hinnasta
ja tuotteesta maksettavasta arvonlisäverosta. Kuinka monta
prosenttia arvonlisävero on tuotteen myyntihinnasta?
\begin{vastaus}
18,0 \%
\end{vastaus}
\end{tehtava}

%Ansiotuloverotus on Suomessa progressiivista: suuremmista tuloista maksetaan

\begin{tehtava}
    Tuoreissa omenissa on vettä 80~\% ja sokeria 4~\%. Kuinka monta prosenttia sokeria on samoissa omenissa, kun ne on kuivattu siten, 
    että kosteusprosentti on 20? [K2000, 4]
    
    \begin{vastaus}
        Vastaus: 16~\%
    \end{vastaus}
\end{tehtava}

\begin{tehtava}
    Kappaleen putoamisen kesto maahan korkeudelta $x$ on kääntäen verrannollinen putoamiskiihtyvyyden $g$ neliöjuureen. Vakio $g$ on kullekin     
    taivaankappaleelle ominainen ja eri puolilla taivaankappaletta likimain sama. Empire State Buildingin katolta (korkeus 
    $381$ m) pudotetulla kuulalla kestää n. $6,2$ s osua maahan. Marsin putoamiskiihtyvyys on $37,6$ \% Maan putoamiskiihtyvyydestä. 
    Jos Empire State Building sijaitsisi Marsissa, kuinka monta prosenttia pitempi aika kuluisi kuulan maahan osumiseen?

    \begin{vastaus}
        Vastaus: $10$ s
    \end{vastaus}
\end{tehtava}
