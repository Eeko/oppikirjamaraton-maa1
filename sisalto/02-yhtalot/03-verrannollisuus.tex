\chapter{Suoraan ja kääntäen verrannollisuus}

Kaksi muuttujaa voivat riippua toisistaan eri tavoin. Tavallisia
riippuvuuden tyyppejä ovat suoraan ja kääntäen verrannollisuus.

\laatikko{
Kaksi muuttujaa ovat suoraan verrannolliset, jos toinen saadaan
toisesta kertomalla se jollakin vakiolla. Tätä vakiota kutsutaan \emph{verrannollisuuskertoimeksi}.
}

Suoraan verrannollisuus voidaan havaita myös laskemalla muuttujien
suhde ja huomaamalla, että se on vakio.

\begin{esimerkki}
Banaanien kilohinta on $2,00$ euroa. Seuraavassa taulukossa on
banaanien paino\footnote{Fysikaalisesti kyse on massasta, mutta
arkikielessä termi on paino.}, jonka saa ostettua tietyllä rahamäärällä:
\begin{center} 
\begin{tabular}{|l|r|r|}
\hline
Hinta (euroa) & Paino (kg) & Hinta/paino (euroa/kg) \\
\hline
$1,00$ & $0,50$ & $2,00$ \\
$2,00$ & $1,00$ & $2,00$ \\
$3,00$ & $1,50$ & $2,00$ \\
$4,00$ & $2,00$ & $2,00$ \\
\hline
\end{tabular}
\end{center}
Muuttujien hinta ja paino suhde on vakio $2,00$ euroa/kg, joten ostettujen
banaanien paino ja niihin käytetty rahamäärä ovat suoraan verrannolliset.
\end{esimerkki}

Suoraan verrannollisuutta voidaan kuvata esimerkiksi niin, että jos
toinen suureista kaksinkertaistuu, niin toinenkin kaksinkertaistuu.
Samoin jos toinen suureista puolittuu, toinenkin suure puolittuu.

Jos suoraan verrannollisista suureista piirretään kuvaaja, pisteet
asettuvat suoralle:

\missingfigure{Kuva, johon piirretty vaaka-akselille käytetty rahamäärä
ja pystyakselille banaanien paino.}

Muita esimerkkejä suoraan verrannollisista suureista ovat
\begin{itemize}
    \item kuljettu matka ja aika, kun kuljetaan vakionopeudella, sekä
    \item painovoiman kappaleeseen aiheuttama voima ja kappaleen massa.
\end{itemize}

\laatikko{
Kaksi muuttujaa ovat kääntäen verrannolliset, jos toinen saadaan toisesta
jakamalla jokin vakio sillä.
}

Kääntäen verrannollisuus voidaan havaita myös kertomalla suureet
keskenään ja huomaamalla, että tulo on vakio.

\begin{esimerkki}
Nopeus ja matkaan tarvittava aika ovat kääntäen verrannolliset suureet.
Jos kuljettavana matkana on $80$ km, voidaan nopeudet ja matka-ajat
kirjoittaa taulukoksi:
\begin{center} 
\begin{tabular}{|l|r|r|}
\hline
Nopeus (km/h) & Matka-aika (h) & Nopeus$\cdot$matka-aika (km) \\
\hline
$40$ & $2$ & $80$ \\
$80$ & $1$ & $80$ \\
$100$ & $0,8$ & $80$ \\
\hline
\end{tabular}
\end{center}
Tulo on aina $80$ km, joten nopeus ja matka-aika ovat kääntäen verrannolliset.
\end{esimerkki}

Kääntäen verrannollisuutta voidaan kuvata niin, että jos
toinen suureista kaksinkertaistuu, niin toinen puolittuu.

Jos kääntäen verrannollisista suureista piirretään kuvaaja, pisteet
muodostavat laskevan käyrän:

\missingfigure{Kuva, johon piirretty matka-aika ja nopeus (yllä olevan
taulukon mukaisesti)}

Muita esimerkkejä kääntäen verrannollisia suureista ovat
\begin{itemize}
    \item kaivamistyön suorittamiseen kuluva aika ja työntekijöiden lukumäärä.
\end{itemize}

\section*{Tehtäviä}

\begin{tehtava}
    % Lyhyt matikka 1, s. 72
    Pohdi seuraavissa tapauksissa, kuinka toinen suure muuttuu, kun toinen suure
    kaksinkertaistuu, kolminkertaistuu tai puolittuu. Ovatko suureet
    suoraan verrannolliset, kääntäen verrannolliset vai eivät kumpaakaan?
    
    \begin{enumerate}
        \item Kuljettu matka ja kulunut aika, kun keskinopeus on 30 km/h.
        \item Kananmunien lukumäärä ja niiden kovaksi keittämiseen tarvittava keittoaika.
        \item Hedelmätiskiltä valitun vesimelonin paino ja hinta.
        \item Neliön sivun pituus ja neliön pinta-ala.
    \end{enumerate}
    
    \begin{vastaus}
        Vastaus:
        \begin{enumerate}
            \item Ovat.
            \item Eivät ole.
            \item Ovat.
            \item Eivät ole, sillä esimerkiksi kun neliön sivun pituus
                kaksinkertaistuu 1 cm:stä 2 cm:iin, niin neliön pinta-ala
                nelinkertaistuu 1 cm$^2$:stä 4 cm$^2$:iin.
        \end{enumerate}
    \end{vastaus}
\end{tehtava}

\begin{tehtava}
Ratkaise
\begin{enumerate}
\item $ \frac{x}{3} = 1$
\item $ \frac{8}{y} = 2$
\item $ \frac{7}{x} = \frac{16}{8}$
\item $ \frac{x}{3} = \frac{1}{7}$
\end{enumerate}
\begin{vastaus}
\begin{enumerate}
\item $x= \frac{1}{3}$
\item $y= \frac{1}{4}$
\item $x= \frac{7}{2}$
\item $x= \frac{3}{7}$
\end{enumerate}
\end{vastaus}
\end{tehtava}

\begin{tehtava}
Muodosta seuraavia tilanteita kuvaavat yhtälöt. Voit käyttää vakion
merkkinä esimerkiksi $c$:tä.
\begin{enumerate}
\item Kultakimpaleen arvo ($x$) on suoraan verrannollinen sen massaan ($m$),
eli mitä painavampi kimpale on, sitä enemmän siitä saa rahaa.
\item Aidan maalaamiseen osallistuvien ihmisten määrä {$x$} on kääntäen verrannollinen maalaamiseen kuluvaan aikaan ($t$). Toisin sanoen, mitä
enemmän maalaajia, sitä nopeammin homma on valmis.
\item Planeettojen toisiinsa aiheuttama vetovoima ($F$) on suoraan verrannollinen planeettojen massoihin ($m_1$ ja $m_2$) ja kääntäen verrannollinen niiden välisen etäisyyden ($r$) neliöön.
\end{enumerate}
\begin{vastaus}
\begin{enumerate}
\item $ \frac{x}{m}=c$
\item $ xt=c $
\item $ \frac{Fr^2}{m_1+m_2}=c$
\end{enumerate}
\end{vastaus}
\end{tehtava}

\begin{tehtava}
Rento pyöräilyvauhti kaupunkiolosuhteissa on noin $20$ km/h. Lukiolta urheiluhallille on matkaa $7$ km. Kuinka monta minuuttia kestää arviolta pyöräillä lukiolta urheiluhallille?
\begin{vastaus}
Viiden minuutin tarkkuudella $20$ min.
\end{vastaus}
\end{tehtava}

\begin{tehtava}
    Isi ja lapset ovat ajamassa mökille Sotkamoon. On ajettu jo neljä
    viidesosaa matkasta, ja aikaa on kulunut kaksi tuntia. ''Joko ollaan perillä?''
    kysyvät lapset takapenkiltä. Kuinka pitkään vielä arviolta kuluu, ennen
    kuin ollaan mökillä?
    
    \begin{vastaus}
        Vastaus: 1 h 15 min
    \end{vastaus}
\end{tehtava}

\begin{tehtava}
    Äidinkielen kurssilla annettiin tehtäväksi lukea 300-sivuinen romaani.
    Eräs opiskelija otti aikaa ja selvitti lukevansa vartissa seitsemän sivua.
    Kuinka monta tuntia häneltä kuluu koko romaanin lukemiseen, jos
    taukoja ei lasketa?
    
    \begin{vastaus}
        Vastaus: 642 minuuttia eli 10 h 42 min.
    \end{vastaus}
\end{tehtava}

\begin{tehtava}
	Jättiomenan paistoaika on suoraan verrannollinen omenan massan neliöjuureen. Tiedetään, että 9 kilon omenan paistoaika on 1,5 tuntia. Miten pitkä on 4 kilon omenan paistoaika?
    \begin{vastaus}
        1 tunti
    \end{vastaus}
\end{tehtava}

\begin{tehtava}
	Pellon kyntämiseen kulunut aika on kääntäen verrannollinen kyntäjien määrään. Jos kuudelta henkilöltä kuluu 7 tuntia tietyn kokoisen pellon kyntämiseen, niin kuinka kauan yhdeksältä henkilöltä kuluisi samankokoisen pellon kyntämiseen?
    \begin{vastaus}
        4 h 40 min
    \end{vastaus}
\end{tehtava}

\begin{tehtava}
	Tarvittava maalin määrä on suoraan verrannollinen maalattavan seinän pinta-alaan. Emilialta kului 2,5 litran maalipurkki  loppuun, kun hän oli maalannut seinää 17,5 $m^2$. Montako 2,5 litran maalipurkkia on Emilian vielä ostettava, kun hän haluaa maalata loputkin kyseisestä seinästä, jonka kokonaispinta-ala on 102 $m^2$?
    \begin{vastaus}
        5 purkkia
    \end{vastaus}
\end{tehtava}
